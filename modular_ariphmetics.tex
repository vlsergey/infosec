\section{Модульная арифметика}\label{section-modular-arithmetic}
\selectlanguage{russian}

\subsection{Сложность модульных операций}

Криптосистемы с открытым ключом, как правило, построены в модульной арифметике с длиной модуля от сотни до нескольких тысяч разрядов. Сложность алгоритмов оценивают как количество битовых операций в зависимости от длины. В таблице~\ref{tab:mod-binary-complexity} приведены оценки (с точностью до порядка) сложности модульных операций\index{битовая сложность} для простых (или <<школьных>>) алгоритмов вычислений. На самом деле, для реализации арифметики длинных чисел (сотни или тысячи двоичных разрядов) следует применять существенно более эффективные (более <<хитрые>>) алгоритмы вычислений, использующие, например, специальный вид быстрого преобразования Фурье и другие приёмы.

\begin{table}[!ht]
    \centering
    \caption{Битовая сложность операций по модулю $n$ длиной $k= \log n$ бит\label{tab:mod-binary-complexity}}
    \begin{tabular}{| p{0.75\textwidth} | c |}
        \hline
        Операция, алгоритм & Сложность \\
        \hline
        1. $a \pm b \mod n$ & $O(k)$ \\
        2. $a \cdot b \mod n$ & $O(k^2)$ \\
        3. $\gcd(a, b)$, алгоритм Евклида & $O(k^2)$ \\
        4. $(a,b) \rightarrow (x,y,d) : ax + by = d = \gcd(a,b)$, расширенный алгоритм Евклида & $O(k^2)$ \\
        5. $a^{-1} \mod n$, расширенный алгоритм Евклида & $O(k^2)$ \\
        6. Китайская теорема об остатках & $O(k^2)$ \\
        7. $a^b \mod n$ & $O(k^3)$ \\
        \hline
    \end{tabular}

\end{table}

\subsection{Возведение в степень по модулю}

Рассмотрим два алгоритма возведения числа $a$ в степень $b$ по модулю $n$ (см.~\cite[9.3.1. Простые двоичные схемы]{Crandall:Pomerance:2011}). Оба этих алгоритма основываются на разложении показателя $b$ в двоичное представление:

\begin{equation}
	\begin{array}{l}
		b = \sum\limits_{i=1}^{k} b_i 2^i, \\
		b_i \in \{0, 1\}.
	\end{array}
	\label{eq:power-mod-b}
\end{equation}

\subsubsection{Схема <<слева направо>>}

Алгоритм~\ref{alg:power-mod-left-to-right} сводится к вычислению следующей формулы:
\[ c = \left( \left( \left( \left( \left( 1 \cdot a^{b_k} \right)^2 \cdot a^{b_{k-1}} \right)^2 \cdot a^{b_{k-2}} \right)^2 \dots \right)^2 \cdot a^{b_2} \right)^2 \cdot a^{b_1} \mod n.\]

\begin{algorithm}[ht]
	\caption{Простая двоичная схема возведения в степень типа <<слева направо>>\label{alg:power-mod-left-to-right}}
	\begin{algorithmic}
		\STATE $c := a$;
		\FOR{ $i := k-1$ \TO $1$}
			\STATE $c := c^2 \mod n$;
			\IF { $\left( b_i == 1 \right)$ }
				\STATE $c := c \cdot a \mod n$;
			\ENDIF
		\ENDFOR
		\RETURN $c$;
	\end{algorithmic}
\end{algorithm}

Алгоритм требует $k-1$ возведений в квадрат и $t-1$ умножений, где $t$ -- количество единиц в двоичном представлении показателя степени. Так как возведение в квадрат можно сделать примерно в два раза быстрее, чем умножение на произвольное число, то, например, в криптосистеме RSA\index{криптосистема!RSA} показатель степени стараются выбрать таким образом, чтобы в его двоичной записи было мало бит, отличных от нуля: $3_{10} = 11_{2}$ или $65537_{10} = 10000000000000001_{2}$.

\example Посчитаем с помощью простой двоичной схемы возведения в степень типа <<слева направо>> значение $175^{235} \mod 257$. Представим число $235$ в двоичном виде:
\[ 235_{10} = 11101011_{2}.\]
Полное выражение для вычисления имеет вид:
\[ c = (((((((1 \cdot a^1)^2 \cdot a^1)^2 \cdot a^1)^2 \cdot a^0)^2 \cdot a^1)^2 \cdot a^0)^2 \cdot a^1)^2 \cdot a^1 \mod 257.\]

\begin{enumerate}
	\item $c := 175$;
	\item $c := 175^2 \mod 257 = 42, $\\
		$c := 42 \times 175 \mod 257 = 154;$
	\item $c := 154^2 \mod 257 = 72, $\\
		$c := 72 \times 175 \mod 257 = 7;$
	\item $c := 7^2 \mod 257 = 49; $
	\item $c := 49^2 \mod 257 = 88, $\\
		$c := 88 \times 175 \mod 257 = 237;$
	\item $c := 237^2 \mod 257 = 143; $
	\item $c := 143^2 \mod 257 = 146, $\\
		$c := 146 \times 175 \mod 257 = 107;$
	\item $c := 107^2 \mod 257 = 141, $\\
		$c := 141 \times 175 \mod 257 = 3;$
	\item Ответ: 3. Потребовалось 7 возведений в квадрат и 5 умножений.
\end{enumerate}
\exampleend

Алгоритм можно обобщить на использование произвольного основания разложения степени. Например, использование основания, представляющего собой степень двойки, будет являться методом улучшения описанной выше схемы под названием <<просматривание>> (\langen{windowing}, см.~\cite[9.3.2. Улучшение схем возведение в степень]{Crandall:Pomerance:2011}). Если в качестве основания выбрать $s = 4$, то формула из примера выше для вычисления $175^{235} \mod 257$ принимает вид:
\[\begin{array}{l}
	235_{10} = 3223_{4}; \\
	c = \left(\left(\left( 1 \cdot 175^3 \right)^4 \cdot 175^2 \right)^4 \cdot 175^2 \right)^4 \cdot 175^3 \mod 257.
\end{array}\]

Для вычисления уже потребуется $3 \times 2 = 6$ возведений в квадрат и $3$ умножения. Но сначала потребуется вычислить значения $175^2 \mod 257$ и $175^3 \mod 257$. Для больших показателей степени $n$ выгода в количестве умножений будет очевидна.

\subsubsection{Схема <<справа налево>>}
Другим вариантом является схема типа <<справа налево>> (см. алгоритм~\ref{alg:power-mod-right-to-left}). Она также основывается на разложении показателя степени по степеням двойки~\ref{eq:power-mod-b}. Её можно представить следующей формулой:

\[\begin{array}{ll}
c & = a^b = \\
  & = a^{\sum b_i 2^{i-1}} = \\
  & = a^{b_1} \times a^{(b_2 2)} \times a^{(b_3 2^2)} \times a^{(b_4 2^3)} \times \dots \times a^{(b_k 2^{k-1})} = \\
  & = a^{b_1} \times \left(a^2\right)^{b_2} \times \left(a^4\right)^{b_3} \times \left(a^8\right)^{b_4} \times \dots \times \left(a^{2^{k-1}}\right)^{b_k} = \\
  & = \prod\limits_{i=1}^{k} \left(a^{2^{i-1}}\right)^{b_i}.
\end{array}\]

\begin{algorithm}[ht]
	\caption{Простая двоичная схема возведения в степень типа <<справа налево>>\label{alg:power-mod-right-to-left}}
	\begin{algorithmic}
		\STATE $c := 1$;
		\STATE $t := a$;
		\FOR{ $i := 1$ \TO $k$ }
			\IF { $\left( b_i == 1 \right)$ }
				\STATE $c := c \cdot t \mod n$;
			\ENDIF
			\STATE $t := t^2 \mod n$;
		\ENDFOR
		\RETURN $c$.
	\end{algorithmic}
\end{algorithm}

\example Посчитаем с помощью простой двоичной схемы возведения в степень типа <<справа налево>> значение $175^{235} \mod 257$. Представим число $235$ в двоичном виде:
\[ 235_{10} = 11101011_{2}.\]
\begin{enumerate}
	\item $ c := 1 \times 175 \mod 257 = 175$, \\
		$ t:= 175^2 \mod 257 = 42$;
	\item $ c := 175 \times 42 \mod 257 = 154$, \\
		$ t:= 42^2 \mod 257 = 222$;
	\item $ t:= 222^2 \mod 257 = 197$;
	\item $ c := 154 \times 197 \mod 257 = 12$, \\
		$ t:= 197^2 \mod 257 = 2$;
	\item $ t:= 2^2 \mod 257 = 4$;
	\item $ c := 12 \times 4 \mod 257 = 48$, \\
		$ t:= 4^2 \mod 257 = 16$;
	\item $ c := 48 \times 16 \mod 257 = 254$, \\
		$ t:= 16^2 \mod 257 = 256$;
	\item $ c := 254 \times 256 \mod 257 = 3$.
	\item Ответ: 3. Потребовалось 7 возведений в квадрат и 5 умножений.
\end{enumerate}
\exampleend

\subsection{Алгоритм Евклида}\index{алгоритм!Евклида}
\selectlanguage{russian}

Рекурсивная форма алгоритма Евклида вычисления $\gcd(a,b)$ имеет следующий вид:
    \[\set(a,b): a>b;  \gcd(a,b) = \gcd(b, a \mod b). \]
Редуцирование чисел продолжается, пока не получим
    \[ a \mod b = 0, \]
тогда $b$ и будет искомым НОД.

\example
Вычислим $\gcd(56, 35)$:
\[ \begin{array}{ll}
    \gcd(56, 35) & =~ \gcd(35, ~ 56 \mod 35 = 21) ~= \\
    & =~ \gcd(21, ~ 35 \mod 21 = 14) ~= \\
    & =~ \gcd(14, ~ 21 \mod 14 = 7) ~= \\
    & =~ \gcd(7, ~ 14 \mod 7 = 0) ~= \\
    & =~ 7. \\
\end{array} \]
\exampleend


\subsection{Расширенный алгоритм Евклида}\index{алгоритм!Евклида!расширенный}

\textbf{Расширенный алгоритм Евклида} (см., например,~\cite[8.8 Наибольшие общие делители и алгоритм Евклида]{Aho:1979}) для целых $a, b, a > b$ находит
    \[ x, y, d = \gcd(a,b): ax + by = d. \]

Введём обозначения: $g_i$ -- частное от деления, $r_i$ -- остаток от деления на $i$-ом шаге. Алгоритм:

\[\begin{array}{ll}
	r_{-1} & := a, \\
	r_0 & := b, \\
	y_0 & := x_{-1} := 1, \\
	y_{-1} & := x_0 := 0. \\
\end{array}\]

\[\begin{array}{ll}
	g_i & := \left\lfloor r_{i-2} / r_{i-1} \right\rfloor, \\
	r_i & := r_{i-2} - g_i \times r_{i-1}, \\
	y_i & := y_{i-2} - g_i \times y_{i-1} , \\
	x_i & := x_{i-2} - g_i \times x_{i-1} . \\
\end{array}\]

Алгоритм останавливается, когда $r_i = 0$.

%Вычисление осуществляется точно так же, как и в обычном алгоритме Евклида, только на каждой итерации дополнительно находится частное и остаток от деления.

\example
В табл.~\ref{tab:extended-euclid} приведён числовой пример алгоритма для $a=136, b=36$.
\begin{table}[!ht]
    \centering
    \caption{Пример расширенного алгоритма Евклида для \\ $a=136, b=36$\label{tab:extended-euclid}}
    \begin{tabular}{|r|r|r|r|r|rrr|}
        \hline
        $i$ & $g_i$ & $r_i$ & $x_i$ & $y_i$ & & & \\
        \hline
        $-1$ &  --- & $136$ &   $1$ &   $0$ & $136 =$ & $ 1 \cdot 136$ & $ + 0 \cdot 36$ \\
	 $0$ &  --- &  $36$ &   $0$ &   $1$ &  $36 =$ & $ 0 \cdot 136$ & $ + 1 \cdot 36$ \\
	 $1$ &  $3$ &  $28$ &  $+1$ &  $-3$ &  $28 =$ & $+1 \cdot 136$ & $ - 3 \cdot 36$ \\
	 $2$ &  $1$ &   $8$ &  $-1$ &  $+4$ &   $8 =$ & $-1 \cdot 136$ & $ + 4 \cdot 36$ \\
	 $3$ &  $3$ &   $4$ &  $+5$ & $-19$ &  $-4 =$ & $+5 \cdot 136$ & $- 19 \cdot 36$ \\
	 $4$ &  $2$ &   $0$ &   --- &   --- & & & --- \\
        \hline
    \end{tabular}
\end{table}
Найдено $x = 5, ~ y = -19, ~ d = 4$.
\exampleend

\subsection[Нахождение мультипликативного обратного]{Нахождение мультипликативного \protect\\ обратного по модулю}

Расширенный алгоритм Евклида можно использовать для вычисления обратного элемента\index{обратный элемент}: для заданных $a, n$ найти $x, y, d = \gcd(a,n): ax + ny = d$. Пусть $a,n$ -- взаимно простые, тогда
\[\begin{array}{l}
	ax + ny = 1, \\
	ax \equiv 1 \mod n, \\
	x \equiv a^{-1} \mod n. \\
\end{array}\]

\example
В табл.~\ref{tab:extended-euclid-inverse} приведён числовой пример вычисления расширенным алгоритмом Евклида для $a=142, b=33$ обратных элементов $33^{-1} \equiv -43 \mod 142$ и $142^{-1} \equiv 10 \mod 33$.

\begin{table}[!ht]
    \centering
    \caption{Пример вычисления обратных элементов $33^{-1} \equiv -43 \mod 142$ и $142^{-1} \equiv 10 \mod 33$ из уравнения $142 x + 33 y = 1$ расширенным алгоритмом Эвклида\label{tab:extended-euclid-inverse}}
    \begin{tabular}{|r|r|r|r|r|rrr|}
        \hline
        $i$ & $g_i$ & $r_i$ & $x_i$ & $y_i$ & & & \\
        \hline
        $-1$ &  --- & $142$ &   $1$ &   $0$ & $142 =$ & $  1 \cdot 142$ & $ + 0 \cdot 33$ \\
	 $0$ &  --- &  $33$ &   $0$ &   $1$ &  $33 =$ & $  0 \cdot 142$ & $ + 1 \cdot 33$ \\
	 $1$ &  $4$ &  $10$ &  $+1$ &  $-4$ &  $10 =$ & $ +1 \cdot 142$ & $ - 4 \cdot 33$ \\
	 $2$ &  $3$ &   $3$ &  $-3$ & $+13$ &   $3 =$ & $ -3 \cdot 142$ & $+ 13 \cdot 33$ \\
	 $3$ &  $3$ &   $1$ & $+10$ & $-43$ &   $1 =$ & $+10 \cdot 142$ & $- 43 \cdot 33$ \\
	 $4$ &  $3$ &   $0$ &   --- &   --- & & & --- \\
        \hline
    \end{tabular}
\end{table}
\exampleend

Битовая сложность вычисления обратного элемента для $k$-битового $n$ имеет порядок $O(k^2)$. Если известно разложение числа $n$ на множители, то по теореме Эйлера
    \[ a^{-1} = a^{\varphi(n) - 1} \mod n \]
и вычисление обратного элемента реализуется с битовой сложностью $O(k^3),~ k = \lceil \log_2 n \rceil$. Сложность вычислений по этому алгоритму можно уменьшить, если известно разложение на сомножители числа $\varphi(n) - 1$.


\subsection{Китайская теорема об остатках}
\selectlanguage{russian}

Пусть
    \[ n = \prod\limits_{i=1}^{r} n_i, ~ \gcd(n_i, n_j) = 1. \]

 \textbf{Китайская теорема об остатках}\index{теорема!китайская об остатках} (Chinese Remainder Theorem, CRT) для взаимно простых $\gcd(n_i,n_j) = 1, ~ i \neq j$, утверждает:
\begin{enumerate}
    \item Между
        \[ a \mod n \]
        и вектором
        \[ (a_1 = a \mod n_1, ~~ a_2 = a \mod n_2,  \dots,  a_r = a \mod n_r) \]
        существует взаимно однозначное соответствие.
    \item Сохраняются операции сложения и умножения:
        \[ (a \pm b \mod n) ~\Leftrightarrow~ (a_1 \pm b_1 \mod n_1,  \dots,  a_r \pm b_r \mod n_r), \]
        \[ (a b \mod n) ~\Leftrightarrow~ (a_1 b_1 \mod n_1,  \dots,  a_r b_r \mod n_r). \]
\end{enumerate}

Китайская теорема означает, что
    \[ a \mod n ~~\equiv~~ (a \mod n_1, ~ a \mod n_2, ~ \dots, ~ a \mod n_r), \]
и все вычисления по $\mod n$ (сложение, умножение, вычисление обратного элемента) эквиваленты вычислениям по $\mod n_i$.

Чтобы от вектора (остатков) перейти к числу, следует произвести следующие вычисления:
        \[ N_i = \frac{n}{n_i}, \]
        \[ M_i = N_i^{-1} \mod n_i, \]
        \[ a = \sum\limits_{i=1}^{r} a_i M_i N_i \mod n. \]
        Для проверки равенства достаточно найти вычет (остаток от деления) последнего уравнения по $\mod n_i$.

        Сложность перехода в векторную форму имеет порядок:
        \[ O(k^2), ~ k = \lceil \log_2 n \rceil. \]
Теорема используется для решения систем линейных модульных уравнений и для ускорения вычислений.

Пусть битовая длина $n$ равна $k$, и пусть все $n_i$ имеют одинаковую битовую длину $k / r$. Тогда операция умножения в векторном виде будет в
    \[ \frac{k^2}{r \left( k \middle/ r \right)^2 } = r \]
раз быстрее.

Операция $c = m^e \mod n$ занимает $O(k^3)$ битовых операций. Если перейти к вычислениям по модулям $n_i$, то возведение в степень можно вычислить в
    \[ \frac{k^3}{r \left( k \middle/ r \right)^3 } = r^2 \]
раз быстрее, коэффициенты результирующего вектора равны
    \[ c_i ~=~ \left( m \mod n_i \right)^{e \mod \varphi(n_i)} \mod n_i, ~ i = 1, \dots, r. \]
Заметим, однако, что для криптографических приложений модули с большим количеством сомножителей применять нельзя из-за катастрофического снижения криптостойкости (снижения сложности разложения модуля на множители).


\subsection[Решение систем линейных уравнений]{Решение систем линейных уравнений}

\example
Решим для примера систему линейных уравнений. Применим CRT и а) для разложения одного уравнения по составному модулю на систему по взаимно простым модулям, и б) для нахождения конечного решения по системе уравнений:
\[
    \begin{cases}
        9 x = 8 \mod 11, \\
        5 x = 7 \mod 12, \\
        x = 5 \mod 6, \\
        122 x = 118 \mod 240; \\
    \end{cases}
    \Rightarrow ~~
    \begin{cases}
        x = 8 \cdot 9^{-1} \mod 11, \\
        x = 7 \cdot 5^{-1} \mod 12, \\
        x = 5 \mod 6, \\
        x = 59 \cdot 61^{-1} \mod 120; \\
    \end{cases}
    \Rightarrow
\] \[
    \Rightarrow ~~
    \begin{cases}
        x = -4 \mod 11, \\
        x = -1 \mod 12, \\
        x = -1 \mod 6, \\
        x = -1 \mod 120; \\
    \end{cases}
    \Rightarrow ~~
    \begin{cases}
        x = -4 \mod 11, \\
        \begin{cases}
            x = -1 \mod 3, \\
            x = -1 \mod 4, \\
        \end{cases} \\
        \begin{cases}
            x = -1 \mod 3, \\
            x = -1 \mod 2, \\
        \end{cases} \\
        \begin{cases}
            x = -1 \mod 8, \\
            x = -1 \mod 3, \\
            x = -1 \mod 5; \\
        \end{cases} \\
    \end{cases}
    \Rightarrow
\] \[
    \Rightarrow ~~
    \begin{cases}
        x = -4 \mod 11, \\
        x = -1 \mod 3, \\
        x = -1 \mod 8, \\
        x = -1 \mod 5. \\
    \end{cases}
\]
Все модули попарно взаимно простые, поэтому применима китайская теорема об остатках:
    \[ n_1 = 11, ~ n_2 = 3, ~ n_3 = 8, ~ n_4 = 5, \]
    \[ n = n_1 n_2 n_3 n_4 = 1320, \]
    \[ N_i = \frac{n}{n_i} ~~ \Rightarrow ~~ N_1 = 120, ~ N_2 = 440, ~ N_3 = 165, ~ N_4 = 264, \]
    \[ M_i = N_i^{-1} \mod n_i, ~~ \Rightarrow ~~ M_1 = 10, ~ M_2 = 2, ~ M_3 = 5, ~ M_4 = 4, \]
    \[ a_1 = -4, ~ a_2 = -1, ~ a_3 = -1, ~ a_4 = -1, \]
    \[ x ~=~ \sum_{i=1}^4 a_i N_i M_i \mod n ~=~ 359 \mod 1320. \]
Ответ: $x ~=~ 359 \mod 1320$.
\exampleend

