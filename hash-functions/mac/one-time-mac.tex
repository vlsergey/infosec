\subsection{Одноразовая имитовставка}\label{sec:one-time-mac}
\selectlanguage{russian}
Одноразовый MAC (\langen{One-Time MAC}) можно рассматривать как аналог одноразового шифрблокнота для целей аутентификации сообщения. Если использовать один ключ для ровно одного сообщения, можно построить код аутентификации, который гарантированно не может быть подделан злоумышленником.

Если сообщение короткое, то можно считать ключом два числа $a$ и $b$, а в качестве значения MAC для сообщения $m \in M$ будет выступать
\[
h(m) = a \times m + b \bmod p.
\]

Если $p$ -- простое, то вероятность угадать значение MAC для некоторого сообщения $m_2$ (то есть подменить сообщение $m_1$ с одновременной генерацией нового MAC) без знания ключа $k=\overrightarrow{(a,b)}$, равна строго $1/p$.

Можно воспользоваться описанной ранее функцией хеширования для сообщений переменной длины и использовать чуть более сложную формулу для вычисления MAC:

\[ \begin{array}{l}
\vec{m} = \left \langle m_1, m_2, \dots, m_l \right \rangle, \\
h(\vec{m}) = b + \sum_{i=1}^{l} a^i m_i \mod p. \\
\end{array}  \]

Для этой формулы по прежнему сохраняется свойство, что если ключ $k=\overrightarrow{(a,b)}$ используется ровно один раз, злоумышленник не сможет подделать (угадать) MAC с вероятностью более $1/p$.
