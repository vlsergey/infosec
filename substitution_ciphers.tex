\subsubsection{Шифры замены}
\selectlanguage{russian}

В шифрах \textbf{замены} символы одного алфавита заменяются символами другого алфавита обратимым преобразованием. В последовательности открытого текста символы входного алфавита заменяются на символы выходного алфавита. Такие шифры применяются как в симметричных, так и в асимметричных криптосистемах. Если при преобразовании используются однозначные функции, то шифры замены называются \textbf{однозначными} шифрами замены. Если используются многозначные функции, то шифры называются \textbf{многозначными} шифрами замены (омофонами).

В \textbf{омофоне}\index{омофон} символам входного алфавита ставятся в соответствие непересекающиеся подмножества символов выходного алфавита. Количество символов в каждом подмножестве замены пропорционально частоте встречаемости символа открытого текста. Таким образом, омофон создаёт равномерное распределение символов шифротекста, и прямой частотный криптоанализ невозможен. При шифровании омофонами символ входного алфавита заменяется на случайно выбранный из подмножества замены.

Шифры бывают \textbf{моноалфавитные}, когда для шифрования используется одно отображение входного алфавита в выходной алфавит. Если алфавит на входе и выходе одинаков, и его размер (число символов) равен $D$, тогда количество всевозможных моноалфавитных шифров замены такого типа равно $D!$.

\textbf{Полиалфавитный} шифр задаётся множеством различных вариантов отображения входного алфавита на выходной алфавит. Шифры замены могут быть как потоковыми, так и блочными. Однозначный полиалфавитный потоковый шифр замены называется \textbf{шифром гаммирования}\index{шифр!гаммирования}. Символом алфавита может быть, например, 256-битовое слово, а размер алфавита~--- $2^{256}$ соответственно.
