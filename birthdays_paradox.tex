\section{Парадокс дней рождений}\label{section-birthday-padradox}
\selectlanguage{russian}

Парадокс дней рождений\index{парадокс дней рождений} связан с контринтуитивным ответом на следующую задачу: какой должен быть минимальный размер группы, чтобы вероятность совпадения дней рождения хотя бы у пары человек из этой группы была больше $1 / 2$? Первый возникающий в голове вариант ответа <<183 человека>> (т.е. $\left\lceil 365 / 2 \right\rceil$) является неверным.

Найдем вероятность $P(n)$ того, что в группе из $n$ человек хотя бы двое имеют день рождения в один день года. Вероятность того, что $n$ человек ($n < 365$) не имеют общего дня рождения, есть
\[
    \bar{P}(n) = 1 \cdot \left( 1 - \frac{1}{365} \right) \cdot \left(1 - \frac{2}{365} \right)  \dots  \left( 1 - \frac{n-1}{365} \right) = \prod\limits_{i=0}^{n-1} \left( 1 - \frac{i}{365} \right).
\]

Аппроксимируя $1-x \leq e^{-x}$, находим
    \[ \bar{P}(n) \approx \prod\limits_{i=0}^{n-1} e^{-\frac{i}{365}} = e^{-\frac{n(n-1)}{2} \cdot \frac{1}{365}} \approx e^{-\frac{n^2}{2} \cdot \frac{1}{365}}. \]

Вероятность того, что хотя бы 2 человека из $n$ имеют общий день рождения, есть
    \[ P(n) = 1 - \bar{P}(n) \approx 1 -  e^{-\frac{n^2}{2} \cdot \frac{1}{365}}. \]

Кроме того, найдем минимальный размер группы, в которой дни рождения совпадают хотя бы у двоих с вероятностью не менее $1/2$. То есть найдем такое число $n_{1/2}$, чтобы выполнялось условие $P(n_{1/2}) \geq \frac{1}{2}$. Подставляя это значение в формулу для вероятности, получим $\frac{1}{2} \geq e^{-\frac{n_{1/2}^2}{2} \cdot \frac{1}{365}}$. Следовательно,
	\[n_{1/2} \geq \sqrt{2 \ln 2 \cdot n} \approx 1,18 \sqrt{ n } \approx 22,5.\]
	
В криптографии при оценках стойкости алгоритмов часто опускают коэффициент $\sqrt{2 \ln 2}$, считая ответом на задачу <<округлённое>> значение $\sqrt{ n }$. Например, оценку числа операций хэширования для поиска коллизии идеальной криптографической хэш-функции с размером выхода k бит часто записывают как $2^{k/2}$.