\section{Шифр RC4}\label{rc4}\index{шифр!RC4|(}
\selectlanguage{russian}

Шифр RC4 был разработан Роном Ривестом (\langen{Ronald Linn Rivest}) в 1987 году для компании RSA Data Security. Описание алгоритма было впервые анонимно опубликовано в телеконференции Usenet sci.crypt в 1994 году.\footnote{См. раздел 17.1. <<Алгоритм RC4>> в ~\cite{Schneier:2002}}

Генератор, используемый в шифре, хранит своё состояние в массиве из 256 ячеек $S_0, S_1, \dots, S_{255}$, заполненных значениями от 0 до 255 (каждое значение только один раз), а также двух других переменных размером в 1 байт $i$ и $j$. Таким образом, количество различных внутренних состояний генератора равно $255! \times 255 \times 255 \approx 2.17 \times 10^{509} \approx 2^{1962}$

Процедура инициализации генератора.
\begin{itemize}
	\item Для заполнения байтового массива из 256 ячеек $K_0, K_1, \dots, K_{255}$ используется предоставленный ключ. При необходимости (если размер ключа менее 256 байтов) ключ используется несколько раз, пока массив $K$ не будет заполнен целиком.
	\item Начальное значение $j$ равно $0$.
	\item Далее, для значений $i$ от $0$ до $255$ выполняется:
	\begin{enumerate}
		\item $j:= (j + S_i + K_i) \mod 256$,
		\item поменять местами $S_i$ и $S_j$.
	\end{enumerate}
\end{itemize}

Процедура получения следующего псевдослучайного байта $result$ (следующего байта гаммы):
\begin{enumerate}
	\item $ i := (i + 1) \mod 256$,
	\item $ j := (j + S_i) \mod 256$,
	\item поменять местами $S_i$ и $S_j$,
	\item $ t := ( S_i + S_j ) \mod 256$,
	\item $ result := S_t$.
\end{enumerate}

По утверждению Брюса Шнайера алгоритм настолько прост, что большинство программистов могут закодировать его по памяти. Шифр RC4 использовался во многих программных продуктах, в том числе в IBM Lotus Notes, Apple AOCE, Oracle Secure SQL и Microsoft Office, а также в стандарте сотовой передачи цифровых данных CDPD. В настоящий момент шифр не рекомендуются к использованию~\cite{rfc7465}, в нём были найдены многочисленные, хотя и некритичные уязвимости~\cite{Fluhrer:Mantin:Shamir:2001,Mantin:Shamir:2002,Paul:Maitra:2007,Sepehrdad:Vaudenay:Vuagnoux:2011}.

\index{шифр!RC4|)}
