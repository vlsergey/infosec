\section{Симметричные протоколы}
\selectlanguage{russian}

К симметричным будем относить протоколы, которые используют примитивы только классической криптографии на секретных ключах. К таким относятся уже известные блочные шифры, криптографически стойкие генераторы псевдослучайных чисел (КСГПСЧ) и хэш-функции.

При рассмотрении протоколов будем использовать следующие обозначения.
\begin{itemize}
	\item \textit{Alice}, \textit{Bob} -- легальные абоненты сети, для которых формируется общий сеансовый ключ. Алиса является инициатором.
	\item \textit{Trent} -- доверенный центр сети.
	\item $A$, $B$ -- некоторые идентификаторы легальных абонентов Алисы и Боба соответственно.
	\item $E_A$, $E_B$ -- результат шифрования некоторого блока данных с использованием секретных ключей легальных абонентов сети Алисы и Боба соответственно. Такое шифрование могу осуществить либо сами легальные абоненты, либо доверенный центр, которому известны все секретные ключи.
	\item $R_A$, $R_B$, $R_T$ -- случайные числа, генерируемые Алисой, Бобом и Трентом соответственно.
	\item $T_A$, $T_B$, $T_T$ -- метки времени, генерируемые Алисой, Бобом и Трентом соответственно.
	\item $K$ -- секретный сеансовый ключ, получение которого и является одной из целью протоколов.
\end{itemize}

\input{protocols-wide-mouth_frog}

\input{protocols-needham-schroeder}

\input{protocols-kerberos}
