\subsection{Симметричные и асимметричные криптосистемы}
\selectlanguage{russian}

Криптографические системы и шифры можно разделить на две большие группы в зависимости от принципа использования ключей для шифрования и расшифрования.

Если для шифрования и расшифрования используется один и тот же ключ $K$, либо если получение ключа расшифрования $K_2$ из ключа шифрования $K_1$ является тривиальной операцией, то такая криптосистема называется \emph{симметричной}\index{криптосистема!симметричная}. В зависимости от объёма данных, обрабатываемых за одну операцию шифрования, симметричные шифры делятся на \emph{блочные}\index{шифр!блочный}, в которых за одну операцию шифрования происходит преобразование одного блока данных (32 бита, 64, 128 или больше), и \emph{потоковые}\index{шифр!потоковый}, в которых работают с каждым символом открытого текста по отдельности (например, с 1 битом или 1 байтом). Примеры блочных шифров рассмотрены в главе~\ref{chapter-block-ciphers}, а потоковых~-- в главе~\ref{chapter-stream-ciphers}.

Использование блочного шифра подразумевает разделение открытого текста на блоки одинаковой длины, к каждому из которых применяется функция шифрования. Кроме того, результат шифрования следующего блока может зависеть от предыдущего\footnote{Строго говоря, функция шифрования может применяться не только к самому блоку данных, но и к другим параметрам текущего отрывка открытого текста. Например, к его позиции в тексте (\langen{offset}) или даже к результату шифрования предыдущего блока.}. Данная возможность регулируется \emph{режимом работы блочного шифра}. Примеры нескольких таких режимов рассмотрены в разделе~\ref{section-block-chaining}.

Если ключ расшифрования получить из ключа шифрования вычислительно сложно, то такие криптосистемы называют криптосистемами \emph{с открытым ключом}\index{криптосистема!с открытым ключом} или \emph{асимметричными} криптосистемами\index{криптосистема!асимметричная}. Некоторые из них рассмотрены в главе~\ref{chapter-public-key}. Все используемые на сегодняшний день асимметричные криптосистемы работают с открытым текстом, составляющим несколько сотен или тысяч бит, поэтому классификация таких систем по объёму обрабатываемых за одну операцию данных не производится.

Алгоритм, который выполняет отображение аргумента произвольной длины в значение фиксированной длины, называется \emph{хэш-функцией}. Если для такой хэш-функции выполняются определённые свойства, например, устойчивость к поиску коллизий, то это уже \emph{криптографическая хэш-функция}. Такие функции рассмотрены в главе~\ref{chapter-hash-functions}.

Для проверки аутентичности сообщения с использованием общего секретного ключа отправителем и получателем используется \emph{код аутентификации [сообщения]} (другое название в русскоязычной литературе - \emph{имитовставка}, \langen{message authentication code, MAC}), рассмотренный в разделе~\ref{section-MAC}. Его аналогом в криптосистемах с открытым ключом является \emph{электронная подпись}, алгоритмы генерации и проверки которой рассмотрены в главе~\ref{chapter-public-key} вместе с алгоритмами асимметричного шифрования.
