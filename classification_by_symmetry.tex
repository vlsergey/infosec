\subsection{Симметричные и асимметричные криптосистемы}
\selectlanguage{russian}

Криптографические системы и шифры можно разделить на две большие группы, в зависимости от принципа использования ключей для шифрования и расшифрования.

Если для шифрования и расшифрования используется один и тот же ключ $K$, либо если получение ключа расшифрования $K_2$ из ключа шифрования $K_1$ является тривиальной операцией, то такая криптосистема называется \textbf{симметричной}\index{криптосистема!симметричная}. В зависимости от объёма обработки данных за одну операцию шифрования симметричные шифры делятся на \textbf{блочные}\index{шифр!блочный}, в которых за одну операцию шифрования происходит преобразование одного блока данных (32 бита, 64, 128 или больше) и \textbf{потоковые}\index{шифр!потоковый}, в которых работают с каждым символом открытого текста по отдельности (например, с 1 битом или 1 байтом). Примеры блочных шифров рассмотрены в главе~\ref{chapter-block-ciphers}, а потоковых -- в главе~\ref{chapter-stream-ciphers}.

Шифрование блочным шифром подразумевает разделение открытого текста на блоки одинаковой длины. Блоки шифруются последовательно, причём результат шифрования следующего блока может зависеть от предыдущего. Это регулируется \textbf{режимом сцепления блоков}. Примеры нескольких таких режимов рассмотрены в разделе~\ref{chapter-block-chaining}.

Если ключ расшифрования получить из ключа шифрования сложно (или невозможно), то такие криптосистемы называют криптосистемами \textbf{с открытым ключом}\index{криптосистема!с открытым ключом} или \textbf{асимметричными} криптосистемами\index{криптосистема!асимметричная}. Некоторые из них рассмотрены в главе~\ref{chapter-public-key}. Все используемые на сегодняшний день асимметричные криптосистемы работают с блоком данных открытого текста, представленным в виде числа длиной в несколько сот или тысяч бит, поэтому классификация таких систем по объёму обрабатываемых за одну операцию данных не производится.

Алгоритм, который выполняет отображение аргумента произвольной длины в значение фиксированной длины, называется \textbf{хеш-функцией}. Если для такой хеш-функции выполняются определённые свойства устойчивости к поиску коллизий, то это уже \textbf{криптографическая хеш-функция}. Такие функции рассмотрены в главе~\ref{chapter-hash-functions}. Криптографические хеш-функции используются для проверки целостности сообщений. Для проверки с использованием общего секретного ключа отправителя и получателя используется механизм \textbf{имитовставки}, рассмотренный в разделе~\ref{section-MAC}. Её аналогом в криптосистемах с открытым ключом является \textbf{электронная подпись}, алгоритмы генерации и проверки которой рассмотрены в главе~\ref{chapter-public-key} вместе с алгоритмами асимметричного шифрования.
