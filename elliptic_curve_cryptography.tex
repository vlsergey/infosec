\section{Эллиптические кривые}\label{section-elliptic-curve-cryptosystems}
\selectlanguage{russian}

Существуют аналоги криптосистемы Эль-Гамаля, в которых вместо проблемы дискретного логарифма в мультипликативных полях используется проблема дискретного логарифма в группах точек эллиптических кривых над конечными полями (обычно $GF(p)$ либо $GF(2^n)$). Математическое описание данных полей приведено в разделе~\ref{section-math-ec-groups}. Нас же интересует тот факт, что для группы точек эллиптической кривой над конечным полем $\group{E}$ существует быстро выполнимая операция -- умножение целого числа $x$ на точку $A$ (суммирование точки самой с собой целое число раз):
\[ \begin{array}{l}
	x \in \group{Z}, \\
	A, B \in \group{E}, \\
	B = x \times A. \\
\end{array} \]

И получение исходной точки $A$ при известных $B$ и $x$ (<<деление>> точки на целое число), и получение целого числа $x$ при известных $A$ и $B$ являются сложными задачами. На этом и основаны алгоритмы шифрования и электронной подписи с использованием эллиптических кривых над конечными полями.

\subsection{ECIES}
\selectlanguage{russian}

Схема ECIES (\langen{Elliptic Curve Integrated Encryption Scheme}) является частью сразу нескольких стандартов, в том числе ANSI X9.63, IEEE 1363a, ISO 18033-2 и SECG SEC 1. Эти стандарты по-разному описывают выбор параметров схемы~\cite{Martinez:Encinas:Avila:2010}:

\begin{itemize}
	\item ENC (\langen{Encryption}) -- блочный режим шифрования (в том числе простое гаммирование, 3DES\index{шифр!3DES}, AES\index{шифр!AES}, MISTY1\index{шифр!MISTY1}, CAST-128\index{шифр!CAST-128}, Camelia\index{шифр!Camelia}, SEED\index{шифр!SEED});
	\item KA (\langen{Key Agreement})~-- метод для генерации общего секрета двумя сторонами (оригинальный метод, описанный в протоколе Диффи~--~Хеллмана\index{протокол!Диффи~--~Хеллмана}~\cite{Diffie:Hellman:1976}, либо его модификации~\cite{Miller:1986});
	\item KDF (\langen{Key Derivation Function}) -- метод получения ключей из ключевой и дополнительной информации;
	\item HASH~--- криптографическая хэш-функция (SHA-1\index{хэш-функция!SHA-1}, SHA-2\index{хэш-функция!SHA-2}, RIPEMD\index{хэш-функция!RIPEMD}, SHA-1\index{хэш-функция!WHIRLPOOL});
	\item MAC (\langen{Message Authentication Code})~--- функция вычисления имитовставки\index{имитовставка} (DEA, ANSI X9.71, MAC1, HMAC-SHA-1, HMAC-SHA-2, HMAC-RIPEMD, CMAC-AES).
\end{itemize}

К параметрам относится выбор группы точек над эллиптической кривой $\group{E}$, а также некоторой большой циклической подгруппы $\group{G}$ в группе $\group{E}$, задаваемой точкой-генератором $G$. Мощность циклической группы обозначается $n$.
\[n = \left\| \group{G} \right\|.\]

Предположим, что в нашем сценарии Алиса хочет послать сообщение Бобу. У Алисы есть открытый ключ Боба $P_B$, а у Боба ~--- соответствующий ему закрытый ключ $p_B$. Для отправки сообщения Алиса также сгенерирует временную (\langen{ephemeral}) пару из открытого ($P_A$) и закрытого ($p_A$) ключей. Закрытыми ключами являются некоторые натуральные числа, меньшие n, а открытыми ключами является произведение закрытого на точку-генератор $G$:

\[ \begin{array}{ll}
	p_A \in \Z, & p_B \in \Z, \\
	1 < p_A < n, & 1 < p_B < n, \\
	P_A = p_A \times G, & P_B = p_B \times G, \\
	P_A \in \group{G} \in \group{E}, & P_B \in \group{G} \in \group{E}.\\
\end{array} \]

\begin{enumerate}
	\item С помощью метода генерации общего секрета KA, Алиса вычисляет общий секрет $s$. В случае использования оригинального протокола Диффи~--~Хеллмана\index{протокол!Диффи~--~Хеллмана} общим секретом будет является результат умножения закрытого ключа Алисы на открытый ключ Боба $s = p_a \times P_B$.
	\item Используя полученный общий секрет $s$ и метод получения ключей из ключевой и дополнительной информации KDF Алиса получает ключ шифрования $k_{ENC}$, а также ключ для вычисления имитовставки $k_{MAC}$.
	\item С помощью симметричного алгоритма шифрования ENC Алиса шифрует открытое сообщение $m$ ключом $k_{ENC}$ и получает шифротекст $c$.
	\item Взяв ключ $k_{MAC}$, зашифрованное сообщение $c$ и другие заранее обговоренные сторонами параметры, Алиса вычисляет тэг сообщения ($tag$) с помощью функции MAC.
	\item Алиса отсылает Бобу $\{P_A, tag, c\}$.
\end{enumerate}

В процессе дешифровки Боб последовательно получает общий секрет $s = p_b \times P_A$, ключи шифрования $k_{ENC}$ и имитовставки $k_{MAC}$, вычисляет тэг сообщения и сверяет его с полученным. В случае совпадения вычисленного и полученного тэгов Боб расшифровывает исходное сообщение $m$ из шифротекста $c$ с помощью ключа шифрования $k_{ENC}$.


\section[Российский стандарт ЭП ГОСТ Р 34.10-2001]{Российский стандарт ЭП \protect\\ ГОСТ Р 34.10-2001}
\selectlanguage{russian}

Российский стандарт цифровой подписи основан на криптосистеме типа Эль-Гамаля\index{криптосистема!Эль-Гамаля}, в которой в качестве группы используется группа точек эллиптической кривой над конечным полем (см. Приложение). Группа должна быть большой с количеством элементов порядка $2^{255}$.

Пусть имеются две стороны $A$ и $B$ и между ними канал связи. Сторона $A$ желает передать сообщение $M$ стороне $B$ и подписать его. Сторона $B$ должна проверить правильность подписи, то есть аутентифицировать сторону $A$.

$A$ формирует открытый ключ следующим образом.

\begin{enumerate}
    \item Выбирает простое число $p > 2^{255}$.
    \item Записывает уравнение эллиптической кривой
        \[ E: ~ y^2 = x^3 + a x + b \mod p, \]
        которое определяет группу точек эллиптической кривой $\E(\Z_p)$.
        Выбирает группу, задавая либо случайные числа $0 < a, b < p-1$, либо инвариант $J(E)$:
        \[ J(E) = 1728 \frac{4 a^3}{4 a^3 + 27 b^2} \mod p. \]
        Если кривая задается инвариантом $J(E) \in \Z_p$, то он выбирается случайно в интервале $0 < J(E) < 1728$. Для нахождения $a,b$ вычисляется
        \[ K = \frac{J(E)}{1728 - J(E)}, \]
        \[ \begin{array}{l}
            a = 3 K \mod p, \\
            b = 2 K \mod p. \\
        \end{array} \]
    \item Пусть $m$ -- порядок группы точек эллиптической кривой $\E(\Z_p)$. ~Пользователь $A$ подбирает число $n$ и простое число $q$ такие, что
        \[ m = n q, ~ 2^{254} < q < 2^{256}, ~ n \geq 1, \]
        где $q$ -- делитель порядка группы.

        В циклической подгруппе порядка $q$ выбирается точка
        \[ P \in \E(\Z_p): ~ q P \equiv 0. \]
    \item Случайно выбирает число $d$ и вычисляет точку $Q = d P$.
    \item Формирует секретный и открытый ключи:
        \[ \SK = (d), ~ \PK = (p, E, q, P, Q). \]
\end{enumerate}

Теперь сторона $A$ создает свою цифровую подпись $S(M)$ сообщения $M$, выполняя следующие действия.
\begin{enumerate}
    \item Вычисляет число $\alpha = h(M)$, где $h$ -- криптографическая хэш-функция, определенная стандартом ГОСТ Р 34.11-94. В российском стандарте длина $h(M)$ равна 256 бит.
    \item Вычисляет  $e = \alpha \mod q$.
    \item Случайно выбирает число $k$ и вычисляет точку
        \[ C = k P = (x_c, y_c). \]
    \item Вычисляет  $r = x_c \mod q$.
	Если $r = 0$, то выбирает другое $k$.
    \item Вычисляет  $s = k e + r d \mod q$.
	Если $s = 0$, то выбирает другое $k$.
    \item Формирует подпись
        \[ S(M) = (r, s). \]
\end{enumerate}
Сторона $A$ передает стороне $B$ сообщение с подписью
    \[ (M, ~ S(M)). \]

Сторона $B$ проверяет подпись $(r,s)$, выполняя процедуру проверки подписи.
\begin{enumerate}
    \item Вычисляет  $\alpha = h(M)$ и $e = \alpha \mod q$.
    \item Вычисляет  $e^{-1} \mod q$.
    \item Проверяет условия $r < q, ~ r < s$. Если эти условия не выполняются, то подпись отвергается. Если условия выполняются, то процедура продолжается.
    \item Вычисляет числа
        \[ \begin{array}{l}
            a = s e^{-1} \mod q, \\
            b = -r e^{-1} \mod q. \\
        \end{array} \]
    \item Вычисляет точку
        \[ \tilde{C} = a P + b Q = (\tilde{x}_c, \tilde{y}_c). \]
        Если подпись верна, должны получить исходную точку $C$.
    \item Проверяет условие $\tilde{x}_{c} \mod q = r$. Если условие выполняется, то подпись принимается, в противном случае --- отвергается.
\end{enumerate}

Рассмотрим вычислительную сложность вскрытия подписи. Предположим, что криптоаналитик ставит своей задачей определение секретного ключа $d$. Как известно,  эта  задача является трудной. Для подтверждения этого можно привести следующий факт. Был поставлен следующий эксперимент: было выбрано число $p = 2^{97},$ и 1200 персональных компьютеров с тактовой частотой процессоров 200 МГц в 16 странах мира работали, чтобы решить эту задачу. Задача была решена за 53 дня круглосуточной работы. Если взять $p = 2^{256}$, то на решение такой задачи при наличии одного компьютера с частотой процессора 2 ГГц потребуется $10^{22}$ лет.

