\subsection{Первичная аутентификация в OpenID}
\selectlanguage{russian}

Из-за большого числа различных логинов, которые приходится использовать для доступа к разным сервисам, постепенно происходит внедрение единых систем аутентификации для разных сервисов (single sign-on), например, OpenID. Одновременно происходит концентрация пользователей вокруг больших порталов с единой аутентификацией, например, Google Account. Яндекс Паспорт также уменьшает число используемых паролей для разных служб.

Принцип аутентификации состоит в следующем.
\begin{enumerate}
    \item Пользователи и интернет-сервисы доверяют аутентификацию третьей стороне, центру единой аутентификации.
    \item Когда пользователь заходит на интернет-ресурс, веб-приложение перенаправляет его на центр аутентификации с защитой SSL-соединением.
    \item Центр аутентифицирует пользователя и выдает ему токен аутентификации, который пользователь предъявляет интернет-сервису.
    \item Сервис по токену аутентификации определяет имя пользователя.
\end{enumerate}

\begin{figure}[h!]
	\centering
	\includegraphics[width=0.9\textwidth]{pic/openid}
	\caption{Схема аутентификации в OpenID\label{fig:openid}}
\end{figure}

На рис. \ref{fig:openid} показана схема аутентификации в OpenID версии 2 для доступа к стороннему интернет-сервису.

%Аутентификация в OpenID обязательно требует SSL соединения для защиты от атак <<человек-посередине>> и \emph{взаимной} аутентификации сервиса и центра.

Если сервис и центр вместе создают общий секретный ключ $K$ для кода аутентификации сообщений $\MAC_K$ выполняются шаги 4, 5 по протоколу Диффи~---~Хеллмана:
\[ \begin{array}{l}
    \text{5. Сервис} ~\rightarrow~ \text{центр: модуль}~ p ~\text{группы}~ \Z_p^\times, ~\text{генератор}~ g, \\
        ~~~~~~~~~~~~~~~\text{число}~ g^a \mod p, \\
    \text{6. Сервис} ~\leftarrow~ \text{центр: число}~ g^b \mod p, ~\text{гаммированный} \\
        ~~~~~~~~~~~~~~~\text{ключ}~ K \oplus (g^{ab} \mod p),
\end{array} \]
то аутентификатор содержит $\MAC_K$, проверяемый шагом 10, на выданном ключе $K$. \footnote{Более правильным подходом является использование в качестве ключа $K = g^{ab} \mod p$, так в этом случае ключ создается совместно, а не выдается центром.} Код аутентификации сообщений определяется как описанный ранее $\HMAC$ с хэш-функцией SHA-256.

Если сервис и центр не создают общий ключ (этапы 4, 5 не выполняются), сервис делает запрос на проверку аутентификатора в шагах 10, 11.

В OpenID аутентификатор состоит из следующих основных полей: имя пользователя, URL сервиса, результата аутентификации в OpenID, одноразовой метки и, возможно, кода аутентификации от полей аутентифкатора на секретном ключе $K$, если он был создан на этапах 4, 5. Одноразовая метка является \emph{одноразовым} псевдослучайным идентификатором результата аутентификации, который центр сохраняет в своей БД. По одноразовой метке сервис запрашивает центр о верности результата аутентификации на этапах 10, 11. Дополнительно, одноразовая метка защищает от атак воспроизведения.

Можно было бы исключить шаги 4, 5, 10, 11, но тогда сервису бы пришлось реализовывать и хранить в БД использованные одноразовые метки для защиты от атак воспроизведения. Цель OpenID -- предоставить аутентификацию с минимальными издержками на интеграцию. Поэтому в OpenID реализуется модель, в которой сервис все проверки делегирует запросами к центру.

Важно отметить, что в аутентификации через OpenID необходимо использовать SSL-соединения на всех взаимодействиях с центром, так как в самом протоколе OpenID не производится аутентификация сервиса и центра, конфиденциальность и целостность не поддерживаются.
