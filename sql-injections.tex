\section{russian}[SQL инъекции с исполнением кода вебсервером]{SQL инъекции с исполнением кода \protect\\ базой данных интернет-сервиса}

Второй классической уязвимостью веб-приложений являются SQL-инъекции, когда пользователь имеет возможность поменять смысл запроса к базе данных веб-сервера. Запрос делается в виде текстовой строки на скриптовом языке SQL.  Например, выражение
\begin{verbatim}
s = "SELECT * FROM Users WHERE Name = '" + username + "';"
\end{verbatim}
предназначено для получения информации о пользователе \texttt{username}. Однако если пользователь вместо имени введет строку вида
\begin{center} \begin{verbatim}
john';  DELETE * FROM Users;  SELECT * FROM Users WHERE
  Name = 'john,
\end{verbatim} \end{center}
то выражение превратится в три SQL-операции:
%{\color{red} Проверить в каких системах будет работать. В JDBC PrepareStatement требует одно выражение, а ExecutableStatement использовать нельзя, так-как он не возвращает значения.}
\begin{verbatim}
-- запрос о пользователе john
SELECT * FROM Users WHERE Name = 'john';
-- удаление всех пользователей
DELETE FROM Users;
-- запрос о пользователе john
SELECT * FROM Users WHERE Name = 'john';
\end{verbatim}
При исполнении этого SQL-выражения базой данных все записи пользователей будут удалены.

Уязвимости в SQL-выражениях связаны с отсутствием а) экранирования специальных символов \texttt{"},  \texttt{'},  \texttt{;}, б) контроля над типом введенных данных.

Метод защиты так же заключается в \emph{разделении} кода и данных. Для защиты от приведенных атак на базу данных следует использовать параметрические запросы к базе данных с \emph{фиксированным} SQL-выражением. Например, в JDBC
\begin{verbatim}
PreparedStatement p = conn.prepareStatement(
    "SELECT * FROM Users WHERE Name=?");
p.setString(1, username);
\end{verbatim}
