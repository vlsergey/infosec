\section{Конечные поля и операции в алгоритме AES}\index{шифр!AES|(}
\selectlanguage{russian}

В алгоритме блочного шифрования\index{шифр!блочный} AES преобразования над битами и байтами осуществляются специальными математическими операциями. Биты и байты понимаются как элементы поля.

\subsection{Операции с байтами в AES}

Чтобы определить операции сложения и умножения двух байтов, введём сначала представление байта в виде многочлена степени 7 или меньше. Байт
    \[ a =( a_7, a_6, a_5, a_4, a_3, a_2, a_1, a_0) \]
преобразуется в многочлен $a(x)$ с коэффициентами 0 или 1 по правилу
    \[ a(x) = a_{7}x^{7}+a_{6}x^{6}+a_{5}x^{5}+a_{4}x^{4}+a_{3}x^{3}+a_{2}x^{2}+a_{1}x+a_{0}. \]

Далее байт трактуется как элемент конечного поля $\GF{2^8}$, заданного неприводимым многочленом, например:
    \[ m(x) = x^{8}+x^{4}+x^{3}+x +1. \]

Произведение многочленов $a(x)$ и $b(x)$ по модулю многочлена $m(x)$ записывают как
    \[ c(x) = a(x) b(x) \mod m(x). \]
Остаток $c(x)$ представляет собой многочлен степени 7 или меньше. Его коэффициенты $(c_{7}, c_{6}, c_{5}, c_{4}, c_{3}, c_{2}, c_{1}, c_{0})$ образуют байт $c$, который и называется произведением байтов $a$ и $b$.

Сложение байтов осуществляется как $\oplus$ (исключающее ИЛИ), что является операцией сложения многочленов в двоичном поле.

\emph{Единичным} элементом поля является байт $\mathrm{'00000001'}$, или $\mathrm{'01'}$ в шестнадцатеричной записи. \emph{Нулевым} элементом поля является байт $\mathrm{'00000000'}$, или $\mathrm{'00'}$ в шестнадцатеричной записи. Одним из \emph{примитивных} элементов поля является байт $\mathrm{'00000010'}$, или $\mathrm{'02'}$ в шестнадцатеричной записи. Байты часто записывают в шестнадцатеричной форме, но при математических преобразованиях они должны интерпретироваться как элементы поля $\GF{2^8}$.

Для каждого ненулевого байта $a$ существует обратный байт $b$ такой, что их произведение является единичным байтом:
    \[ a b = 1 \mod m(x). \]
Обратный байт обозначается $b = a^{-1}$.

Для байта $a$, представленного многочленом $a(x)$, нахождение обратного байта $a^{-1}$ сводится к решению уравнения
    \[ m(x) d(x) + b(x) a(x) = 1. \]
Если такое решение найдено, то многочлен $b(x) \mod m(x)$ является представлением обратного байта $a^{-1}$. Обратный элемент (байт) может быть найден с помощью расширенного алгоритма Евклида для многочленов.

\example
Найти байт, обратный байту $a = \mathrm{'C1'}$, в шестнадцатеричной записи. Так как $a(x) = x^{7} + x^{6} + 1$, то с помощью расширенного алгоритма Евклида находим
    \[ (x^{8} + x^{4} + x^{3} + x + 1) (x^{4} + x^{3} + x^{2} + x + 1) + (x^{7} + x^{6} + 1) (x^{5} + x^{3}) = 1. \]
Таким образом, обратный элемент поля, или обратный байт $\mathrm{'C1'}$, равен
    \[ x^{5} + x^{3} = a^{-1} = \mathrm{'00101000'} = \mathrm{'28'}. \]
\exampleend

\example
В алгоритме блочного шифрования AES байты рассматриваются как элементы поля Галуа $\GF{2^8}$. Сложим байты $\mathrm{'57'}$ и $\mathrm{'83'}$. Представляя их многочленами, находим:
    \[ (x^6 + x^4 + x^2 + x + 1) + (x^7 + x + 1) = x^7 + x^6 + x^4 + x^2, \]
или в двоичной записи --
    \[ 01010111 \oplus 10000011 = 11010100 = \mathrm{'D4'}. \]
Получили $\mathrm{'57'} + \mathrm{'83'} = \mathrm{'D4'}$.
\exampleend

\example
Выполним в поле $\GF{2^8}$, заданном неприводимым многочленом
    \[ m(x) = x^8 + x^4 + x^3 + x + 1 \]
(из алгоритма AES), операции с байтами: $\mathrm{'FA'} \cdot \mathrm{'A9'} + \mathrm{'E0'}$, где
    \[ FA = 11111010, ~ A9 = 10101001, ~ E0 = 11100000, \]
    \[ (x^7 + x^6 + x^5 + x^4 + x^3  +x)(x^7 + x^5 + x^3 + 1) + (x^7 + x^6 + x^5) \mod m(x) = \]
    \[ = x^{14} + x^{13} + x^{10} + x^{8} + x^7 + x^3 + x \mod m(x) = \]
    \[ = (x^{14} + x^{13} + x^{10} + x^{8} + x^7 + x^3 + x) + x^6 \cdot m(x) \mod m(x) = \]
    \[ = x^{13} + x^9 + x^8 + x^6 + x^3 + x \mod m(x) = \]
    \[ = (x^{13} + x^9 + x^8 + x^6 + x^3 + x) + x^5 \cdot m(x) \mod m(x) = \]
    \[ = x^5 + x^3 + x \mod m(x) = \mathrm{'2A'}. \]
\exampleend


\subsection{Операции над вектором из байтов в AES}
%\subsection{Многочлены над полем в алгоритме AES}

Поле $\GF{2^{nk}}$ можно задать как расширение степени $nk$ базового поля $\GF{2}$:
    \[ \alpha \in \GF{2^{nk}}, ~ \alpha = \sum\limits_{i=0}^{nk-1} a_i x^i, ~ a_i \in \GF{2} \]
с неприводимым многочленом $r(x)$ степени $nk$ над полем $\GF{2}$,
    \[ r(x) = \sum\limits_{i=0}^{nk} a_i x^i, ~ a_i \in \GF{2}, ~ a_{nk} = 1. \]

Поле $\GF{2^{nk}}$ можно задать и как расширение степени $k$ базового поля $\GF{2^n}$:
    \[ \alpha \in \GF{(2^n}^k), ~ \alpha = \sum\limits_{i=0}^{k-1} a_i x^i, ~ a_i \in \GF{2^n} \]
с неприводимым многочленом $r(x)$ степени $k$ над полем $\GF{2^n}$,
    \[ r(x) = \sum\limits_{i=0}^{k} a_i x^i, ~ a_i \in \GF{2^n}, ~ a_k = 1. \]

\example
В таблице~\ref{tab:irreducible-gf8} приведены примеры приводимых и неприводимых многочленов над полем $\GF{2^8}$.
\begin{table}[!ht]
    \centering
    \caption{Примеры многочленов над полем $\GF{2^8}$\label{tab:irreducible-gf8}}
    \begin{tabular}{|c|c|}
        \hline
        Многочлен & Разложение \\
        \hline
        $\mathrm{'01'} x + \mathrm{'00'}$ & неприводимый \\
        $\mathrm{'01'} x + \mathrm{'01'}$ & неприводимый \\
        $\mathrm{'01'} x + \mathrm{'A9'}$ & неприводимый \\
        $\mathrm{'01'} x^2 + \mathrm{'00'} x + \mathrm{'00'}$ & $(\mathrm{'01'} x) \cdot (\mathrm{'01'} x)$ \\
        $\mathrm{'1D'} x^2 + \mathrm{'AF'} x + \mathrm{'52'}$ & $(\mathrm{'41'} x + \mathrm{'0A'}) \cdot (\mathrm{'E3'} x + \mathrm{'5A'})$ \\
        $\mathrm{'01'} x^4 + \mathrm{'01'}$ & $(\mathrm{'01'} x + \mathrm{'01'})^4$ \\
        \hline
    \end{tabular}
\end{table}
\exampleend

В алгоритме AES вектор-столбец $\mathbf{a}$ состоит из четырёх байтов $a_{0}, a_{1}, a_{2}, a_{3}$. Ему ставится в соответствие многочлен $\mathbf{a}(y)$ от переменной $y$ вида
    \[ \mathbf{a}(y) = a_{3}y^{3}+a_{2}y^{2}+a_{1}y+a_{0}, \]
причём коэффициенты многочлена (байты) интерпретируются как элементы поля $\GF{2^{8}}$. Это значит, что при сложении или умножении двух таких многочленов их коэффициенты складываются и перемножаются, как определено выше.

Многочлены $\mathbf{a}(y)$ и $\mathbf{b}(y)$ умножаются по модулю многочлена
    \[ \mathbf{M}(y)= \mathrm{'01'} y^4 + \mathrm{'01'} = y^4 + 1, ~ \mathrm{'01'} \in \GF{2^8}, \]
    \[ \mathbf{M}(y)= (\mathrm{'01'}, \mathrm{'00'},\mathrm{'00'}, \mathrm{'00'}, \mathrm{'01'}), \]
который \emph{не} является неприводимым над $\GF{2^8}$.
%Следовательно, многочлен $\mathbf{a}(y)$ задаёт многочлен третьей степени над полем $\GF{2^8}$, но не является элементом поля $\GF{2^{32}}$.

Операция умножения по модулю $\mathbf{M}(y)$ обозначается $\otimes$:
    \[ \mathbf{a}(y) ~ \mathbf{b}(y) \mod \mathbf{M}(y) ~\equiv~ \mathbf{a}(y) \otimes \mathbf{b}(y). \]

Операция <<перемешивание столбца>> в шифровании AES состоит в умножении многочлена столбца на
    \[ \mathbf{c}(y) = (03, 01, 01, 02) = \mathrm{'03'} y^3 + \mathrm{'01'} y^2 + \mathrm{'01'} y + \mathrm{'02'} \]
по модулю $\mathbf{M}(y)$. Многочлен $\mathbf{c}(y)$ имеет обратный многочлен
    \[ \mathbf{d}(y) = \mathbf{c}^{-1}(y) \mod \mathbf{M}(y) = (\mathrm{0B}, \mathrm{0D}, \mathrm{09}, \mathrm{0E}) = \]
        \[ = \mathrm{'0B'} y^3 + \mathrm{'0D'} y^2 + \mathrm{'09'} y + \mathrm{'0E'}, \]
    \[ \mathbf{c}(y) \otimes \mathbf{d}(y) = (00, 00, 00, 01) = 1. \]
При расшифровании выполняется умножение на $\mathbf{d}(y)$ вместо $\mathbf{c}(y)$.

Так как
    \[ y^j = y^{j \mod 4} \mod y^4+1, \]
где коэффициенты из поля $\GF{2^8}$, то произведение многочленов
    \[ \mathbf{a}(y) = a_{3}y^{3}+ a_{2}y^{2} + a_{1}y + a_{0} \]
и
    \[ \mathbf{b}(y) = b_{3}y^{3} + b_{2}y^{2} + b_{1}y + b_{0}, \]
обозначаемое как
    \[ \mathbf{f}(y) = \mathbf{a}(y) \otimes \mathbf{b}(y) = f_{3}y^{3} + f_{2}y^{2} + f_{1}y + f_{0}, \]
содержит коэффициенты
\[
    \begin{array}{ccccccccc}
        f_{0} & = & a_{0}b_{0} & + & a_{3}b_{1} & + & a_{2}b_{2} & + & a_{1}b_{3}, \\
        f_{1} & = & a_{1}b_{0} & + & a_{0}b_{1} & + & a_{3}b_{2} & + & a_{2}b_{3}, \\
        f_{2} & = & a_{2}b_{0} & + & a_{1}b_{1} & + & a_{0}b_{2} & + & a_{3}b_{3}, \\
        f_{3} & = & a_{3}b_{0} & + & a_{2}b_{1} & + & a_{1}b_{2} & + &  a_{0}b_{3}.
    \end{array}
\]

Эти соотношения можно переписать также в матричном виде:
\[
    \begin{array}{cccc}
        \left[ \begin{array}{c}
            f_{0} \\
            f_{1} \\
            f_{2} \\
            f_{3}
        \end{array} \right] &  = & \left[\begin{array}{cccc}
            a_{0} & a_{3} & a_{2} & a_{1} \\
            a_{1} & a_{0} & a_{3} & a_{2} \\
            a_{2} & a_{1} & a_{0} & a_{3} \\
            a_{3} & a_{2} & a_{1} & a_{0}
        \end{array}\right] & \left[\begin{array}{c}
            b_{0} \\
            b_{1} \\
            b_{2} \\
            b_{3}
        \end{array} \right]
    \end{array}.
\]

Умножение матриц производится в поле $\GF{2^8}$. Матричное представление полезно, если нужно умножать фиксированный вектор на несколько различных векторов.

\example
Вычислим для $\mathbf{a}(y) = (\mathrm{F2}, \mathrm{7E}, \mathrm{41}, \mathrm{0A})$ произведение $\mathbf{a}(y) \otimes \mathbf{c}(y)$:
\[
    \mathbf{c}(y) = (03, 01, 01, 02),
\] \[
    \mathbf{d}(y) = \mathbf{c}^{-1}(y) \mod \mathbf{M}(y) = (\mathrm{0B}, \mathrm{0D}, \mathrm{09}, \mathrm{0E}).
\] \[
    \mathbf{a}(y) \otimes \mathbf{c}(y) =
    \left[ \begin{array}{cccc}
        \mathrm{0A} & \mathrm{F2} & \mathrm{7E} & \mathrm{41} \\
        \mathrm{41} & \mathrm{0A} & \mathrm{F2} & \mathrm{7E} \\
        \mathrm{7E} & \mathrm{41} & \mathrm{0A} & \mathrm{F2} \\
        \mathrm{F2} & \mathrm{7E} & \mathrm{41} & \mathrm{0A} \\
    \end{array} \right] \cdot
    \left[ \begin{array}{c} \mathrm{02} \\ \mathrm{01} \\ \mathrm{01} \\ \mathrm{03} \end{array} \right] =
\] \[
    \left[ \begin{array}{ccccccc}
        \mathrm{0A} \cdot \mathrm{02} & \oplus & \mathrm{F2} & \oplus & \mathrm{7E} & \oplus & \mathrm{41} \cdot \mathrm{03} \\
        \mathrm{41} \cdot \mathrm{02} & \oplus & \mathrm{0A} & \oplus & \mathrm{F2} & \oplus & \mathrm{7E} \cdot \mathrm{03} \\
        \mathrm{7E} \cdot \mathrm{02} & \oplus & \mathrm{41} & \oplus & \mathrm{0A} & \oplus & \mathrm{F2} \cdot \mathrm{03} \\
        \mathrm{F2} \cdot \mathrm{02} & \oplus & \mathrm{7E} & \oplus & \mathrm{41} & \oplus & \mathrm{0A} \cdot \mathrm{03} \\
    \end{array} \right] =
    \left[ \begin{array}{c} \mathrm{5B} \\ \mathrm{F8} \\ \mathrm{BA} \\ \mathrm{DE} \end{array} \right];
\] \[
    \begin{array}{l}
        \mathbf{a}(y) \otimes \mathbf{c}(y) = \mathbf{b}(y), \\
        \mathbf{b}(y) \otimes \mathbf{d}(y) = \mathbf{a}(y); \\
    \end{array}
\] \[
    \begin{array}{ccccc}
        (\mathrm{F2}, \mathrm{7E}, \mathrm{41}, \mathrm{0A}) & \otimes & (\mathrm{03}, \mathrm{01}, \mathrm{01}, \mathrm{02}) & = & (\mathrm{DE}, \mathrm{BA}, \mathrm{F8}, \mathrm{5B}), \\
        (\mathrm{DE}, \mathrm{BA}, \mathrm{F8}, \mathrm{5B}) & \otimes & (\mathrm{0B}, \mathrm{0D}, \mathrm{09}, \mathrm{0E}) & = & (\mathrm{F2}, \mathrm{7E}, \mathrm{41}, \mathrm{0A}). \\
    \end{array}
\]
\exampleend

\index{шифр!AES|)}
