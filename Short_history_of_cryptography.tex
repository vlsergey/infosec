\section{Краткая история криптографии}

Вслед за возникновением письменности появилась задача обеспечения секретности и подлинности передаваемых сообщений путем, так называемой, тайнописи. Поскольку государства возникали почти одновременно с письменностью, дипломатия и военное управление требовали секретности.

Данные о первых способах тайнописи весьма обрывочны. Предполагается, что тайнопись была известна в древнем Египте и Вавилоне. До нашего времени дошли литературные свидетельства того, что секретное письмо использовалось в древней Греции. Наиболее известен метод шифрования, который использовался Гаем Юлием Цезарем (100--44~гг.~до~н.э.).

Первое известное исследование по анализу стойкости методов шифрования было сделано в <<Манускрипте о дешифровании криптографических сообщений>> Абу аль-Кинди (801–-873~гг.~н.э.). Он показал, что моноалфавитные шифры, в которых каждому символу кодируемого текста ставится в однозначное соответствие какой-то другой символ алфавита, легко поддаются частотному криптоанализу. Абу аль-Кинди был так же знаком с более сложными полиалфавитными шифрами.

В европейских странах полиалфавитные шифры были открыты в эпоху Возрождения. Итальянский архитектор Баттиста Альберти (1404--1472) изобрел полиалфавитный шифр, который впоследствии получил имя дипломата XVI века Блеза де Виженера. В истории развития полиалфавитных шифров до XX века, также, наиболее известны немецкий аббат XVI века Иоганн Трисемус и английский ученый начала XIX века Чарльз Витстон. Витстон изобрел простой и стойкий способ полиалфавитной замены, называемый шифром Плейфера по имени лорда Плейфера, способствовавшему внедрению шифра. Шифр Плейфера использовался вплоть до Первой мировой войны.

Прообразом современных шифров для электронно-вычислительных машин стали так называемые роторные машины XX века, которые позволяли создавать и реализовывать устойчивые к взлому полиалфавитные шифры. Примером такой машины является немецкая машина \emph{Enigma}\index{Enigma}, разработанная в конце Первой мировой войны. Период активного применения Enigma пришелся на Вторую мировую войну.

Появление в середине XX столетия первых ЭВМ кардинально изменило ситуацию. Вычислительные способности компьютеров подняли на совершенно новый уровень как возможности реализации шифров, недоступных ранее из-за их высокой сложности, так и возможности криптоаналитиков по их взлому. Следствием этого факта стало разделение шифров по области применения.

В 1976 году появился шифр DES (Data Encryption Standard), который был принят как стандарт США. DES широко использовался для шифрования пакетов данных при передаче в компьютерных сетях и системах хранения данных. С 90-х годов параллельно с традиционными шифрами, основой которых была булева алгебра, активно развиваются шифры, основанные на операциях в конечном поле. Широкое распространение персональных компьютеров и быстрый рост объема передаваемых данных в компьютерных сетях привели к замене в 2002 году стандарта DES на более стойкий и быстрый в программной реализации стандарт -- шифр AES (Advanced Encryption Standard). Окончательно, DES был выведен из эксплуатации как стандарт в 2005 году.

В беспроводных голосовых сетях передачи данных используются шифры с малой задержкой шифрования и расшифрования на основе посимвольных преобразований -- так называемые \emph{потоковые шифры}.

%Основным их преимуществом является сочетание помехоустойчивого кодирования с криптостойкостью шифра.

Параллельно с разработкой быстрых шифров в 1977 г. появился новый класс криптосистем, так называемые \emph{криптосистемы с открытым ключом}. Хотя эти новые криптосистемы намного медленнее (технически сложнее) симметричных, они открыли принципиально новые возможности --  \emph{электронная подпись}, \emph{аутентификация} и \emph{сертификация} составили основу современной защищенной связи в интернете.

В настоящее время типичное использование криптографии в информационных системах состоит в:
\begin{itemize}
\item цифровой аутентификации пользователей с помощью криптосистем с открытым ключом,
\item создании кратковременных сеансовых ключей и
\item применении быстрых шифров в процессах обмена данными.
\end{itemize}
