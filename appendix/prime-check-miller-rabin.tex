\subsection{Тест Миллера~---~Рабина}\label{section-prime-check-miller-rabin}\index{тест!Миллера~---~Рабина}
\selectlanguage{russian}

В 1980 году Рабин (\langen{Michael O. Rabin}, \cite{Rabin:1980}) обратил внимание на то, что у нечётного составного числа $n$ количество свидетелей простоты $1 < a < n$ по Миллеру не превышает $n/4$. Это означает, что если число $1 < a < n$ является свидетелем простоты числа $n$ по Миллеру, то число $n$ является простым с вероятностью ошибки не более чем $1/4$. Что приводит нас к вероятностному тесту Миллера~---~Рабина.

Тест Миллера~---~Рабина\index{тест!Миллера~---~Рабина} состоит в проверке $t$ случайно выбранных чисел $1 < a < n$. Если для всех $t$ чисел $a$ тест пройден, то $n$ называется псевдопростым\index{число!псевдопростое}, и вероятность того, что число $n$ не простое, имеет оценку:
    \[ P_{error} < \left( \frac{1}{4} \right)^t. \]
Если для какого-то числа $a$ тест не пройден, то число $n$ точно составное\index{число!составное}.

Описание теста приведено в алгоритме~\ref{miller-rabin}.

\begin{algorithm}[ht]
    \caption{Вероятностный тест Миллера~---~Рабина проверки числа на простоту\label{miller-rabin}}
    \begin{algorithmic}
        \STATE Вход: нечётное $n>1$ для проверки на простоту и $t$ -- параметр надёжности.
        \STATE Выход: \textsc{Составное} или \textsc{Псевдопростое}.
        \STATE $n - 1 = 2^s r, ~ r$ -- нечётное.
        \FOR{~$j = 1$ ~\textbf{to}~ $t$~}
            \STATE Выбрать случайное число $a \in [2, n-2]$.
            \IF{~$(a_0 = a^r ~\neq~ \pm 1 \mod n)$ ~\textbf{and} \\
            \indent ~~~~~~ $(\forall i \in [1, s-1] \rightarrow a_i = a_0^{2^i} ~\neq~ -1 \mod n)$~}
               \STATE \textbf{return} \textsc{Составное}.
           \ENDIF
        \ENDFOR
       \STATE \textbf{return} \textsc{Псевдопростое} с вероятностью ошибки $P_{error} < \left( \frac{1}{4} \right)^t$.
    \end{algorithmic}
\end{algorithm}

Сложность алгоритма Миллера~---~Рабина для $k$-битового числа $n$ имеет порядок
    \[ O(t k^3) \]
двоичных операций, где $t$ -- количество раундов.

\example
Пример выполнения теста Миллера~---~Рабина для $n = 169, ~ n-1 = 21 \cdot 2^3$.

Выберем следующие числа в качестве возможных кандидатов в свидетели простоты числа $n$: 2, 19, 22, 23.

Степени, в которые нужно возводить $a$: 21, 42, 84, 168.

\begin{itemize}
    \item $a = 2$
        \[\begin{matrix}
        a^{21} \bmod 169 & = & 2^{21} \bmod 169 & = & 31 \\ 
        a^{42} \bmod 169 & = & 31^2 \bmod 169 & = & 116 \\ 
        a^{84} \bmod 169 & = & 116^2 \bmod 169 & = & 116 \\ 
        a^{168} \bmod 169 & = & 116^2 \bmod 169 & = & 40
        \end{matrix}\]
    Получилась последовательность: 31, 116, 105, 40.
    \item $a = 19$
        \[\begin{matrix}
        a^{21} \bmod 169 & = & 19^{21} \bmod 169 & = & 70 \\ 
        a^{42} \bmod 169 & = & 70^2 \bmod 169 & = & -1 \\ 
        a^{84} \bmod 169 & = & -1^2 \bmod 169 & = & 1 \\ 
        a^{168} \bmod 169 & = & 1^2 \bmod 169 & = & 1
        \end{matrix}\]
    Получилась последовательность: 70, -1, 1, 1.
    \item $a = 22$
        \[\begin{matrix}
        a^{21} \bmod 169 & = & 22^{21} \bmod 169 & = & 1 \\ 
        a^{42} \bmod 169 & = & 1^2 \bmod 169 & = & 1 \\ 
        a^{84} \bmod 169 & = & 1^2 \bmod 169 & = & 1 \\ 
        a^{168} \bmod 169 & = & 1^2 \bmod 169 & = & 1
        \end{matrix}\]
    Получилась последовательность: 1, 1, 1, 1.
    \item $a = 23$
        \[\begin{matrix}
        a^{21} \bmod 169 & = & 23^{21} \bmod 169 & = & -1 \\ 
        a^{42} \bmod 169 & = & -1^2 \bmod 169 & = & 1 \\ 
        a^{84} \bmod 169 & = & 1^2 \bmod 169 & = & 1 \\ 
        a^{168} \bmod 169 & = & 1^2 \bmod 169 & = & 1
        \end{matrix}\]
    Получилась последовательность: -1, 1, 1, 1.
\end{itemize}

Согласно определению выше, числа 19, 22 и 23 являются свидетелями простоты числа $n=169$ по Миллеру. Если бы мы рассматривали только эти числа в качестве кандидатов в свидетели, то результатом работы алгоритма Миллера~---~Рабина был бы вывод, что число $n=169$ является псевдопростым с вероятностью ошибки $e = 1 / 4^3 = 0{,}015625 \approx 1{,}6\%$. Однако так как в результате проверки числа $a = 2$ было обнаружено, что оно не является свидетелем простоты, то результатом работы алгоритма является вывод, что число $n=169$ составное.
\exampleend

Тест Миллера~---~Рабина не основан на гипотезе Римана или других недоказанных утверждениях. Он является доказанным, полиномиальным, но вероятностным тестом простоты. Также он является наиболее используемым тестом простоты на сегодняшний день.
