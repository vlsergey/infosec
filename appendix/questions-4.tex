\section{Курс <<Защита информации>>}
\selectlanguage{russian}

Список вопросов по курсу <<Защита информации>> кафедры радиотехники и систем управления МФТИ для 4-го курса.

\begin{enumerate}
    \item Цели, задачи и методы защиты информации. Примеры выполнения целей по защите информации без использования криптографических средств. Идентификация, аутентификация, авторизация, аудит, компрометация.
    \item Криптология, криптоанализ, криптография. Криптографические примитивы. Основные определения и примеры использования. Код, шифр, ключ, хеш-функция, криптографический протокол, цифровая подпись, etc. Принцип Керкгоффса.
    \item Применение основ теории информации в криптографии. Абсолютно защищённые шифры. Критерии и свойства. Латинский квадрат и шифроблокнот.
    \item Расстояние единственности. Вывод для линеаризованной корректной криптосистемы.
    \item Моноалфавитные и полиалфавитные шифры. Описание и криптоанализ.
    \item Введение в блочные шифры. SP-сети и ячейка Фейстеля, их плюсы и минусы с точки зрения криптографа (автора шифра). Раундовые шифры, раундовые ключи, процедура их получения и использования. Общий вид блочного раундового шифра, от потока открытого текста до получения шифротекста. На примере (в сравнении) шифров Lucifer, DES и ГОСТ 28147-89.
    \item Режимы сцепления блоков. Описание, плюсы и минусы каждого из режимов. Возможность самовосстановления, оценки потерь в расшифрованном тексте из-за ошибок канала. Возможности параллелизации шифрования и расшифрования.
    \item Имитовставка. Свойства и процесс выработки на примере ГОСТ 28147-89.
    \item Блочные шифры стандартов DES и ГОСТ 28147-89 (подробно).
    \item Блочный шифр стандарта AES (подробно).
    \item Генераторы случайных последовательностей. Генераторы псевдослучайных последовательностей. Принцип Дирихле. Период генератора. Линейно-конгруэнтный генератор, генератор на основе единственного регистра с линейной обратной связью. Оценка возможности использования в криптографии.
    \item Криптографически стойкие генераторы псевдослучайной последовательности. Поточные шифры и требования к ним. Возможность создания поточных шифров из блочных. Плюсы и минусы подобного подхода. Объединение генераторов на основе РСЛОС для создания криптографически стойкого генератора псевдослучайной последовательности.
    \item Терморектальный криптоанализ. Формулировка основной теоремы, идея доказательства. Свойства, примеры использования.
    \item Современные потоковые шифры на примере A5/1. Общий вид, требования, характеристики и анализ защищённости.
    \item Современные потоковые шифры на примере RC4. Общий вид, требования, характеристики и анализ защищённости.
    \item Задача о словаре и хеш-функции. Коллизии в хеш-фунциях. Криптографически стойкие хеш-функции. Свойства, принципы построения криптографически стойких хеш-функций (стандарта США или ГОСТ Р 34.11-2012 <<СТРИБОГ>>). Структуры Меркла~---~Дамгора\index{структура!Меркла~---~Дамгора}, Миагучи~---~Пренеля\index{структура!Миагучи~---~Пренеля}. Использование хеш-функций в криптографии.
    \item База данных на основе Echo-сети. Blockchain. Доказательство работы (proof-of-work, proof-of-share). BitCoin.
    \item Односторонние функции с потайной дверцей. Пример, не связанный с задачами из области теории чисел (т.е. не факторизация, не дискретный логарифм, etc.) Возможность использования односторонних функций в криптографии. Общие идеи использования криптографии с открытым ключом для шифрования. Проблемы криптографической стойкости, производительности.
    \item Цифровые подписи. Цели, основные принципы получения и использования. Конкретные примеры использования цифровых подписей в современных информационных системах.
    \item RSA. Доказательство корректности, использование для шифрования и электронной подписи. Проблемы, лежащие в основе криптографической стойкости RSA. Проблемы <<ванильной>> реализации RSA.
    \item El Gamal. Доказательство корректности, использование для шифрования и электронной подписи.
    \item Шифрование с открытым ключом с использованием эллиптических кривых. Схемы DLIES и ECIES.
    \item Цифровые подписи, требования к ним и характеристики на примере стандарта ГОСТ Р 34.10-2001.
    \item PKI. Централизованная и децентрализованная схема реализации. Использование на примере программ серии PGP, протокола HTTPS, для защиты ПО.
    \item Протоколы аутентификации и идентификации сторон на основе систем симметричного шифрования. Построение, плюсы и минусы, криптографическая стойкость на примере протоколов Yahalom и Нидхема~---~Шрёдера.
    \item Протоколы аутентификации и идентификации сторон на основе систем асимметричного шифрования. Построение, плюсы и минусы, криптографическая стойкость на примере протокола DASS, Деннинг~---~Сакко или Ву~---~Лама.
    \item Протоколы распространения ключей. Протокол Диффи~---~Хеллмана для мультипликативной группы и для группы точек эллиптической кривой.
    \item Протоколы распространения ключей. Протоколы MTI, STS, Жиро (два на выбор).
    \item Квантовый протокол распространения ключей BB84.
    \item Разделение секрета. Пороговые схемы разделения секрета Шамира и Блэкли (подробно).
    \item Протокол распространения ключей на схеме Блома. Анализ криптостойкости, примеры использования и взлома.
    \item Атака на переполнение буфера. Причины и последствия. Детальное описание (без примера на ассемблере), программные и аппаратные способы защиты: безопасные функции, security cookies, DEP, ASLR, etc.
    \item Атаки на плохие указатели. Причины и последствия. Детальное описание (без примера на ассемблере).
    \item Атаки на ошибки контроля данных, примеры с printf, \foreignlanguage{english}{SQL injection, JavaScript injection. Directory traversal}, альтернативные имена файлов в NTFS.
    \item Атаки на некорректное применение криптографических алгоритмов и нестрогое следование стандартам. Примеры с вектором инициализации в CBC, с многоразовыми блокнотами, с проверкой длины хеша.
    \item Атаки на плохие генераторы псевдослучайной последовательности. Примеры с Netscape SSL, WinZIP, PHP.
    \item Доверенная среда исполнения программного обеспечения. Средства реализации для настольных и мобильных систем, методы обхода.
    \item Протокол Kerberos. Математическое описание, описание реализации (v4 или v5 на выбор).
    \item Протокол IPsec (подробно).
    \item Порядок разработки технических и криптографических средств защиты информации в Российской Федерации.
    \item Порядок разработки системы управления защитой информации по ISO2700x.
    \item Китайская теорема об остатках. Доказательство, использование для защиты информации. Использование КТО при доказательстве корректности криптосистемы RSA.
    \item <<Длинная>> модульная арифметика, использование в криптографии. Быстрое <<левое>> и <<правое>> возведение в степень, расширенный алгоритм Евклида.
    \item Проверка чисел на простоту с использованием тестов Ферма, Миллера, Миллера~---~Рабина (подробно).
    \item Группы, подгруппы, генераторы, циклические группы. Примеры, построение, операции, свойства, использование в криптографии.
    \item Поля Галуа вида $\mathbb{GF}(p)$ и $\mathbb{GF}(2^n)$. Построение, операции, свойства, использование в криптографии.
    \item Группы точек эллиптической кривой над множеством рациональных чисел и над конечными полями. Построение, операции, свойства. Теорема Хассе. Использование в криптографии.
    \item Безопасная разработка программного обеспечения. Стандарты, подходы.
\end{enumerate}
