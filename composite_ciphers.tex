\subsubsection{Композиционные шифры}
\selectlanguage{russian}

Почти все современные шифры являются \emph{композиционными}~\cite{AlZKCh:2001}. Также распространено название \emph{составные шифры}, впервые введенное в работе Клода Шеннона (\langen{Claude Elwood Shannon},~\cite{Shannon:1949:CTS}). В них применяются несколько различных методов шифрования к одному и тому же открытому тексту. В современных криптосистемах используется,например, композиция шифра замены и шифра перестановок, применяемая многократно.
