\subsection{Алгоритм Евклида}\index{алгоритм!Евклида}
\selectlanguage{russian}

Рекурсивная форма алгоритма Евклида вычисления $\gcd(a,b)$ имеет следующий вид:
    \[\set(a,b): a>b;  \gcd(a,b) = \gcd(b, a \mod b). \]
Исполнитель алгоримта продолжает редуцирование чисел, пока не получает
    \[ a \mod b = 0, \]
тогда $b$ и будет искомым НОД.

\example
Вычислим $\gcd(56, 35)$:
\[ \begin{array}{ll}
    \gcd(56, 35) & =~ \gcd(35, ~ 56 \mod 35 = 21) ~= \\
    & =~ \gcd(21, ~ 35 \mod 21 = 14) ~= \\
    & =~ \gcd(14, ~ 21 \mod 14 = 7) ~= \\
    & =~ \gcd(7, ~ 14 \mod 7 = 0) ~= \\
    & =~ 7. \\
\end{array} \]
\exampleend


\subsection{Расширенный алгоритм Евклида}\index{алгоритм!Евклида!расширенный}

\emph{Расширенный алгоритм Евклида} (см., например,~\cite[8.8 Наибольшие общие делители и алгоритм Евклида]{Aho:1979}) для целых $a, b, a > b$ находит
    \[ x, y, d = \gcd(a,b): ax + by = d. \]

Введём обозначения: $g_i$ -- частное от деления, $r_i$ -- остаток от деления на $i$-ом шаге. Алгоритм:

\[\begin{array}{ll}
	r_{-1} & := a, \\
	r_0 & := b, \\
	y_0 & := x_{-1} := 1, \\
	y_{-1} & := x_0 := 0. \\
\end{array}\]

\[\begin{array}{ll}
	g_i & := \left\lfloor r_{i-2} / r_{i-1} \right\rfloor, \\
	r_i & := r_{i-2} - g_i \cdot r_{i-1}, \\
	y_i & := y_{i-2} - g_i \cdot y_{i-1} , \\
	x_i & := x_{i-2} - g_i \cdot x_{i-1} . \\
\end{array}\]

Алгоритм останавливается, когда $r_i = 0$.

%Вычисление осуществляется точно так же, как и в обычном алгоритме Евклида, только на каждой итерации дополнительно находится частное и остаток от деления.

\example
В таблице~\ref{tab:extended-euclid} приведён числовой пример алгоритма для $a=136, b=36$.
\begin{table}[!ht]
    \centering
    \caption{Пример расширенного алгоритма Евклида для \\ $a=136, b=36$\label{tab:extended-euclid}}
    \begin{tabular}{|r|r|r|r|r|rrr|}
        \hline
        $i$ & $g_i$ & $r_i$ & $x_i$ & $y_i$ & & & \\
        \hline
        $-1$ &  --- & $136$ &   $1$ &   $0$ & $136 =$ & $ 1 \cdot 136$ & $ + 0 \cdot 36$ \\
	 $0$ &  --- &  $36$ &   $0$ &   $1$ &  $36 =$ & $ 0 \cdot 136$ & $ + 1 \cdot 36$ \\
	 $1$ &  $3$ &  $28$ &  $+1$ &  $-3$ &  $28 =$ & $+1 \cdot 136$ & $ - 3 \cdot 36$ \\
	 $2$ &  $1$ &   $8$ &  $-1$ &  $+4$ &  $8 =$ & $-1 \cdot 136$ & $ + 4 \cdot 36$ \\
	 $3$ &  $3$ &   $4$ &  $+4$ & $-15$ &  $4 =$ & $+4 \cdot 136$ & $ -15 \cdot 36$ \\
	 $4$ &  $2$ &   $0$ &   --- &   --- & & & --- \\
        \hline
    \end{tabular}
\end{table}
Найдено $x = 4, ~ y = -15, ~ d = 4$.
\exampleend

\subsection[Нахождение мультипликативного обратного]{Нахождение мультипликативного \protect\\ обратного по модулю}

Расширенный алгоритм Евклида можно использовать для вычисления обратного элемента\index{обратный элемент}: для заданных $a, n$ найти $x, y, d = \gcd(a,n): ax + ny = d$. Пусть $a,n$ -- взаимно простые, тогда
\[\begin{array}{l}
	ax + ny = 1, \\
	ax \equiv 1 \mod n, \\
	x \equiv a^{-1} \mod n. \\
\end{array}\]

\example
В таблице~\ref{tab:extended-euclid-inverse} приведён числовой пример вычисления расширенным алгоритмом Евклида для $a=142, b=33$ обратных элементов $33^{-1} \equiv -43 \mod 142$ и $142^{-1} \equiv 10 \mod 33$.

\begin{table}[!ht]
    \centering
    \caption{Пример вычисления обратных элементов $33^{-1} \equiv -43 \mod 142$ и $142^{-1} \equiv 10 \mod 33$ из уравнения $142 x + 33 y = 1$ расширенным алгоритмом Евклида\label{tab:extended-euclid-inverse}}
    \begin{tabular}{|r|r|r|r|r|rrr|}
        \hline
        $i$ & $g_i$ & $r_i$ & $x_i$ & $y_i$ & & & \\
        \hline
        $-1$ &  --- & $142$ &   $1$ &   $0$ & $142 =$ & $  1 \cdot 142$ & $ + 0 \cdot 33$ \\
	 $0$ &  --- &  $33$ &   $0$ &   $1$ &  $33 =$ & $  0 \cdot 142$ & $ + 1 \cdot 33$ \\
	 $1$ &  $4$ &  $10$ &  $+1$ &  $-4$ &  $10 =$ & $ +1 \cdot 142$ & $ - 4 \cdot 33$ \\
	 $2$ &  $3$ &   $3$ &  $-3$ & $+13$ &   $3 =$ & $ -3 \cdot 142$ & $+ 13 \cdot 33$ \\
	 $3$ &  $3$ &   $1$ & $+10$ & $-43$ &   $1 =$ & $+10 \cdot 142$ & $- 43 \cdot 33$ \\
	 $4$ &  $3$ &   $0$ &   --- &   --- & & & --- \\
        \hline
    \end{tabular}
\end{table}
\exampleend

Для $k$-битового числа $n$-битовая сложность вычисления обратного элемента имеет порядок $O(k^2)$. Если известно разложение числа $n$ на множители, то по теореме Эйлера
    \[ a^{-1} = a^{\varphi(n) - 1} \mod n, \]
и вычисление обратного элемента реализуется с битовой сложностью $O(k^3),~ k = \lceil \log_2 n \rceil$. Сложность вычислений по этому алгоритму можно уменьшить, если известно разложение на сомножители числа $\varphi(n) - 1$.
