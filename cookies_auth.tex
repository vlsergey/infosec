\subsection{Вторичная аутентификация по cookie}
\selectlanguage{russian}

Если сервер использует первичную аутентификацию по паролю, который передаётся в виде данных POST-запроса, то осуществлять подобную передачу данных при каждом обращении к неудобно. Клиент должен иметь возможность доказать серверу, что он уже прошёл первичную аутентификацию. Должен быть предусмотрен механизм вторичной аутентификации. Для этого используется случайный токен, который уникален для каждого пользователя (обычно -- для каждого сеансса работы пользователя), который сервер передаёт пользователю после первичной аутентификации. Данный токен должен передаваться клиентом на сервис при каждом обращении к страницам, которые относятся к защищённой области сервиса. На практике применяются следующие механизмы для передачи данного токена при каждом запросе:

\begin{itemize}
	\item Первым способом является модификация вывода страницы клиенту, которая добавляет ко всем URL в HTML-коде страницы этот токен. В результате, переходя по ссылкам на HTML-странице (а также заполняя формы и отправляя их на сервер) клиент будет автоматически отправлять токен как часть запроса в URL-адресе страницы:

\texttt{http://tempuri.org/page.html?token=12345}
	\item Вторым способом является использование механизма cookie (<<куки>>, <<кукиз>>, на русский обычно не переводится, подробнее см.~\cite[Client Identification and Cookies]{Totty:2002}). Данный механизм позволяет серверу передать пользователю некоторую строку, которая будет отправляться на сервер при каждом последующем запросе.
\end{itemize}

Основным механизмом для вторичной аутентификации пользователей в веб-сервисах является механизм cookie, а токены как часть URL используются для распределённых системах вроде уже рассмотренной OpenID, так как сервисы, находящиеся в разных доменах, не имеют доступа к общим cookie. Далее рассмотрим подробнее механизм использования cookie.

Когда браузер в первый раз делает HTTP-запрос
\begin{center} \begin{verbatim}
GET /index.html HTTP/1.1
Host: www.wikipedia.org
Accept: */*
\end{verbatim} \end{center}

в заголовке ответа сервера веб-приложение может добавить заголовок \texttt{Set-Cookie}, который содержит новые значения cookie:
\begin{center} \begin{verbatim}
HTTP/1.1 200 OK
Content-type: text/html
Set-Cookie: name1=value1; name2=value2; ...

...далее HTML-страница...
\end{verbatim} \end{center}

Браузер, если это разрешено настройками, при последующих запросах к веб-серверу автоматически будет отсылать cookie назад веб-приложению:
\begin{center} \begin{verbatim}
GET /wiki/HTTP_cookie HTTP/1.1
Host: www.wikipedia.org
Cookie: name1=value1; name2=value2; ...
Accept: */*
\end{verbatim} \end{center}

Далее веб-приложение может создать новый cookie, изменить значение старого и т.д. Браузер хранит cookie на устройстве клиента. То есть cookie -- это возможность хранить переменные на устройстве клиента, отсылать сохраненные значения, получать новые переменные. В результате создается передача состояний, что дает возможность не вводить логин и пароль каждый раз при входе в интернет-сервис, использовать несколько окон для одного сеанса работы в интернет-магазине и т.д. При создании cookie может указываться его конечное время действия, после которого браузер удалит устаревший cookie.

Для вторичной аутентификации в cookie веб-приложение записывает токен в виде текстовой строки. В качестве токена можно использовать \emph{псевдослучайную} текстовую строку достаточной длины, созданную веб-приложением. Например:
\begin{center} \begin{verbatim}
Cookie: auth=B35NMVNASUY26MMWNVZ87
\end{verbatim} \end{center}

В этом случае веб-сервис должен вести журнал выданных токенов пользователям и их сроков действия. Если информационная система небольшого размера (один или десятки серверов), то вместо журнала может использоваться механизм session storage.
\begin{itemize}
	\item При первом заходе на сайт сервер приложений (платформа исполнения веб-приложения) <<назначает>> клиенту сессию, отправляя ему через механизм cookie новый (псевдо)случайный токен сессии, а в памяти сервера выделяя структуру, которая недоступна самому клиенту, но которая соответствует данной конкретной сессии.
	\item При каждом последующем обращении клиент передаёт токен (идентификатор) сессии с помощью механизма cookie. Сервер приложений берёт из памяти соответствующую структуру сессии и передаёт её приложению вместе с параметрами запроса.
	\item В момент прохождения первичной аутентификации приложение добавляет в указанную область памяти ссылку на информацию о пользователе.
	\item При последующих обращениях приложение использует информацию о пользователе, записанной в области памяти сессии клиента.
	\item Сессия автоматически стирается из памяти после прохождения некоторого времени неактивности клиента (что контролируется настройками сервера), либо если приложение явно вызвало функцию инвалидации сессии (англ. \textit{invalidate}).
\end{itemize}

Плюсом использования session storage является то, что этот механизм уже реализован в большинстве платформ для построения веб-приложений (см, например, \cite[Controlling sessions]{Brittain:Darwin:2007}). Его минусом является сложность синхронизации структур сессий в памяти серверов для распределённых информационных систем большого размера.

Вторым способом вторичной аутентификации с использованием cookie является непосредственное включение аутентификационных данных (идентификатор пользователя, срок действия) в cookie вместо случайного токена. К данным в обязательном порядке добавляется имитовставка\index{имитовставка} по ключу, который известен только сервису. Данный подход может значительно увеличить размер передаваемых cookie. С другой стороны, он облегчает вторичную аутентификацию в распределённых системах, так как промежуточным сервисом для хранения факта прошедшей аутентификации является только клиент, а не сервер.

Конечно, беспокоиться об аутентификации в веб-сервисах при использовании обычного HTTP-протокола\index{протокол!HTTP} без зашифрованного SSL-соединения\index{протокол!SSL/TLS} имеет смысл только по отношению к угадыванию токенов аутентификации другими пользователями, которые не имеют доступа к маршрутизаторам и сети, через которые клиент общается с сервисом. Кража компьютера или одного cookie-файла, перехват незащищенного трафика протокола HTTP\index{протокол!HTTP} приводят к доступу в систему под именем взломанного пользователя.
