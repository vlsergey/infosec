\subsubsection{Композиционные шифры}
\selectlanguage{russian}

Почти все современные шифры \textbf{композиционные}~\cite{AlZKCh:2001}, в них применяется несколько различных методов шифрования к одному и тому же открытому тексту.  Другое их название -- \textbf{составные шифры}. Впервые понятие составных шифров введено в работе Клода Шеннона.

В современных криптосистемах шифры замены и шифры перестановки используются многократно, образуя составные (композиционные) шифры.
