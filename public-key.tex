\chapter{Криптосистемы с открытым ключом}\label{chapter-public-key}
\selectlanguage{russian}

\emph{Криптосистемой с открытым ключом} (\langen{public-key cryptosystem, PKC}) называется криптографическое преобразование, использующее два ключа -- открытый и закрытый. Пара из \emph{закрытого}\index{ключ!закрытый} (\langen{private key, secret key, SK})\footnote{В контексте криптосистем с открытым ключом можно ещё встретить использование термина <<секретный ключ>>. Мы не рекомендуем использовать данный термин, чтобы не путать с секретным ключом\index{ключ!секретный}, используемым в симметричных криптосистемах} и \emph{открытого}\index{ключ!открытый} (\langen{public key, PK}) ключей создаётся пользователем, который свой закрытый ключ держит в секрете, а открытый ключ делает общедоступным для всех пользователей. Криптографическое преобразование в одну сторону (шифрование) можно выполнить зная только открытый ключ, а в другую (расшифрование) -- только зная закрытый ключ. Во многих криптосистемах из закрытого ключа теоретически можно вычислить открытый ключ, однако это является сложной вычислительной задачей.

Если прямое преобразование выполняется открытым ключом, а обратное -- закрытым, то криптосистема называется \emph{схемой шифрования с открытым ключом}. Все пользователи, зная открытый ключ получателя, могут зашифровать для него сообщение, которое может расшифровать только владелец закрытого ключа.

Если прямое преобразование выполняется закрытым ключом, а обратное -- открытым, то криптосистема называется \emph{схемой электронной подписи (ЭП)}. Владелец закрытого ключа может \emph{подписать} сообщение, а все пользователи, зная открытый ключ, могут проверить, что подпись была создана только владельцем закрытого ключа и никем другим.

Криптосистемы с открытым ключом снижают требования к каналам связи, которые требуются для передачи данных. В симметричных криптосистемах перед началом связи (перед шифрованием сообщения и его передачей) требуется по защищённому каналу связи передать или согласовать секретный ключ шифрования. Злоумышленник не должен иметь возможности ни прослушать данный канал связи, ни подменить передаваемую информацию (ключ). Для надёжной работы криптосистем с открытым ключом необходимо, чтобы злоумышленник не имел возможности подменить открытый ключ легального пользователя. Другими словами, криптосистема с открытым ключом, в случае использования открытых и незащищённых каналов связи, устойчива к пассивному криптоаналитику\index{криптоаналитик!пассивный}, но всё ещё должна предпринимать меры по защите от активного криптоаналитика\index{криптоаналитик!активный}.

Для предотвращения атак <<человек посередине>> (man-in-the-middle attack)\index{атака!<<человек посередине>>} с активным криптоаналитиком\index{криптоаналитик!активный}, который бы подменял открытый ключ получателя во время его передачи будущему отправителю сообщений, используют \emph{сертификаты открытых ключей}\index{сертификат открытого ключа}. Сертификат представляет собой информацию о соответствии открытого ключа и его владельца, подписанную электронной подписью третьего лица. В корпоративных информационных системах достаточно, если на всю организацию такое лицо, подписывающее сертификаты, будет одно. В этом случае его называют \emph{доверенным центром сертификации} или \emph{удостоверяющим центром}. В глобальной сети Интернет для защиты распространения программного обеспечения (например, защиты от подделок в ПО) и проверок сертификатов в протоколах на базе SSL/TLS\index{протокол!SSL/TLS} используется иерархия удостоверяющих центров, рассмотренная в разделе~\ref{section-CAs}. При обмене личными сообщениями и при распространении программного обеспечения с открытым кодом вместо жёсткой иерархии может использоваться \emph{сеть доверия}\index{сеть доверия}. В сети доверия каждый участник может подписать сертификат любого другого участника. Предполагается, что подписывающий знает лично владельца сертификата и удостоверился о соответствии сертификата владельцу при личной встрече.

Криптосистемы с открытым ключом построены на основе односторонних (однонаправленных) функций c потайным входом. Под \emph{односторонней} функцией понимают \emph{вычислительную} невозможность вычисления её обращения: вычисление значения функции $y = f(x)$ при заданном аргументе $x$ является лёгкой задачей, вычисление аргумента $x$ при заданном значении функции $y$ -- трудной задачей.

Односторонняя функция $y = f(x,K)$ с \emph{потайным входом}\index{функция!с потайным входом} $K$ определяется как функция, которая легко вычисляется при заданном $x$, и аргумент $x$ которой можно легко вычислить из $y$, если известен <<секретный>> параметр $K$, и вычислить невозможно, если параметр $K$ неизвестен.

Примером подобной функции является возведение в степень по модулю составного числа $n$:
	\[ c = f \left( m \right) = m ^ e \mod n.\]

Для того, чтобы быстро вычислить обратную функцию
	\[ m = f^{-1} \left( c \right) = \sqrt[e]{c} \mod n, \]
её можно представить в виде
	\[ m = c^{d} \mod n,\]
где
	\[ d = e^{-1} \mod \varphi \left( n \right). \]

В последнем выражении $\varphi \left( n \right)$ -- это функция Эйлера\index{функция!Эйлера}. В качестве <<потайной дверцы>> или секрета можно рассматривать или непосредственно само число <<$d$>>, или значение $\varphi \left( b \right)$. Последнее можно быстро найти только в том случае, если известно разложение числа $n$ на простые сомножители. Именно эта функция с потайной дверцей лежит в основе криптосистемы RSA\index{криптосистема!RSA}.

Необходимые математические основы модульной арифметики, групп, полей и простых чисел приведены в Приложении~\ref{chap:discrete-math}.

\section{Криптосистемы RSA}
\selectlanguage{russian}
\index{криптосистема!RSA}

\subsection[Шифрование]{Шифрование RSA}

В 1978 г. Рональд Рив\'{е}ст, Ади Шамир и Леонард Адлеман  (R. Rivest, A. Shamir, L. Adleman) предложили алгоритм, обладающий рядом интересных для криптографии свойств. На его основе была построена первая система шифрования с открытым ключом, получившая название по первым буквам фамилий авторов -- система RSA.

Рассмотрим принцип построения криптосистемы шифрования RSA с открытым ключом.

\begin{enumerate}
    \item \textbf{Создание пары из секретного и открытого ключей.}
        \begin{enumerate}
            \item Случайно выбрать большие простые различные числа $p,q$, для которых $\log_2 p \simeq \log_2 q > 512$ бит.
            \item Вычислить произведение $n = pq$.
            \item Вычислить функцию Эйлера $\varphi(n) = (p-1)(q-1)$.
            \item Выбрать случайное целое число $e \in [2, \varphi(n)-1]$ взаимно простое с $\varphi(n)$: $~ \gcd(e, \varphi(n)) = 1$. Свойство проверяют с помощью алгоритма Евклида.
            \item Вычислить число $d$, такое, что  $d e= 1 \mod \varphi(n)$. Для вычисления используется расширенный алгоритм Евклида.
            \item Секретный ключ -- $\SK$, открытый ключ -- $\PK$
                \[ \SK = (d), ~ \PK = (n, e). \]

        \end{enumerate}

Генерация модуля $n = pq$ RSA системы является трудной задачей. Действительно, количество нечетных целых длиной точно 500 бит равно $2^{(500-2)}$. Среди них имеется примерно
$(2^{500})/500 - (2^{499})/499 \approx (2^{500})/1000$ простых 500-разрядных чисел. Вероятность случайного выбора простого числа составляет примерно $1/250 $.
Поиск случайных больших простых чисел $p,q$ состоит в генерации случайного нечетного целого числа и проверке его по критериям простоты. Самый распространенный критерий -- вероятностный тест Миллера--Рабина\index{тест!Миллера-Рабина}. Все вероятностные тесты либо \emph{точно} определяют, что данное число составное, либо что оно \emph{возможно} простое. При $t$-кратной проверке тестом Миллера--Рабина со всеми положительными ответами <<возможно простое>> существует вероятность ошибки $P < \left( \frac{1}{4} \right)^t$, т.е. ненулевая вероятность того, что число окажется, на самом деле, составным. Существуют и многие другие детерминированные и вероятностные тесты на простоту числа.

Криптостойкость RSA системы определяется сложностью разложения на сомножители целого $n$-разрядного числа и отсутствием <<лишних>> делителей.

    \item \textbf{Шифрование на открытом ключе $\PK$.}
        \begin{enumerate}
            \item Сообщение представляют целым числом $m \in [1, n-1]$.
            \item Шифротекст вычисляется как
                \[ c = m^e \mod n. \]
                Шифротекст -- тоже целое число из диапазона $[1, n-1]$.
        \end{enumerate}
    \item \textbf{Расшифрование на секретном ключе $\SK$.}
        \begin{enumerate}
            \item Владелец секретного ключа вычисляет
                \[ m = c^d \mod n. \]
            \item Покажем верность расшифрования. Пусть
                \[ ed = 1 + a \varphi(n). \]
                Если $m$ и $n$ взаимно простые, то по теореме Эйлера (по модулю $n$):
                \[ c^d = m ^{ed} = m^1 m^{a\varphi(n)} = m \cdot 1^a = m \mod n. \]

                В общем случае $m$ и $n$ могут иметь общие делители, но расшифрование тоже оказывается верным. Пусть $m = 0 \mod p$. По китайской теореме об остатках:
                \[
                     m = c^d \mod n ~\Leftrightarrow~
                     \left\{ \begin{array}{l}
                        m = c^d \mod p, \\
                        m = c^d \mod q. \\
                     \end{array} \right..
                \]
                Подставляя $c=m^e$, получаем тождество
                \[ \left\{ \begin{array}{l}
                    m^{ed} = 0 = m \mod p, \\
                    m^{ed} = m  \left( m^{q-1} \right)^{a(p-1)} = m \cdot 1^{a(p-1)} = m \mod q. \\
                \end{array} \right. \]
                Следовательно, $m^{ed} = m \mod pq$.
        \end{enumerate}
\end{enumerate}


Что касается вычислительной сложности других операций, то применение алгоритма Евклида для проверки, является ли число $e$  взаимно простым с числами $p-1, q-1$, а также вычисление обратного элемента $d$, считается легкой задачей (задачей с квадратичной сложностью, не более).
Возведение числа в заданную степень $d$ выполняется с помощью последовательного \emph{возведения в квадрат и перемножения}. Пусть
    \[ d = d_0 + d_1 2^1 + d_2 2^2 + \ldots + d_{k-1} 2^{k-1} \]
двоичное представление с коэффициентами $d_{i} \in \{ 0, 1 \}$. Степень $c^d$ вычисляется рекуррентным образом:
  \[ c^d =((... (((c^ {d_{k-1}})^2  (c^{d_{k-2}}))^2)\dots(c^{d_2}))^2 (c^{d_1}))^2 (c^{d_0}).\]

%    \[ c^d = c^ {d_0} \cdot (c^2)^{d_1} \cdot (c^{2^2})^{d_2} \dots  (c^{2^{k-1}})^{d_{k-1}}, \]
Всего выполняется  $k-1$ операций возведения в квадрат и не более $k-1$ умножений, что считается легкой задачей.


\subsubsection{Пример схемы}

%\example
%Схема шифрования RSA.
\begin{enumerate}
    \item Генерирование параметров.
        \begin{enumerate}
            \item Выберем числа $p=13, q=11, n = 143$.
            \item Вычислим $\varphi(n) = (p-1)(q-1) = 12 \cdot 10 = 120$.
            \item Выберем $e=23: ~ \gcd(e, \varphi(n))=1, ~ e \in [2, 119]$.
            \item Найдем $d = e^{-1} \mod \varphi(n) = 23^{-1} \mod 120 = 47$.
            \item Открытый и секретные ключи:
                \[ \PK = (e:23, n:143), ~ \SK = (d:47). \]
        \end{enumerate}
    \item Шифрование.
        \begin{enumerate}
            \item Пусть сообщение $m = 22 \in [1, n-1]$.
            \item Вычислим шифротекст
                \[ c = m^e = 22^{23} = 55 \mod 143. \]
        \end{enumerate}
    \item Расшифрование.
        \begin{enumerate}
            \item Полученный шифротекст $c = 55$.
            \item Вычислим открытый текст
                \[ m = c^d = 55^{47} = 22 \mod 143. \]
        \end{enumerate}
\end{enumerate}

%Рассмотрим ее основные положения на примере криптосистемы с открытым ключом.
%Приведем общую схему алгоритма RSA.
%$C_i=M_{i}^{E_k}(mod N_j)$
%$N_j=P_{j}Q_{j}$
%$M_i=C_{i}^{D_k}(mod N_j)$
%$E_k\neq D_k$
%Вычислить $E_k$ из $D_k$  при длине блока сообщения  $L_{блока} > L_{дополнения}$ можно только с экспоненциальной сложностью. $E_k D_K=1(mod \varphi(N_j))$
%Данное сравнение не дает единственного решения. Решение данного сравнения и можно свести к следующему уравнению:
%$ax+by=1$
%$E_k D_k=k \varphi(N_j)+1$
%$1\leq E_k D_k <\varphi(N_j)$
%$\varphi(N_j)(-k)+ E_k D_k=1$
%Стандарт ISO X.509 определяет требования по реализации алгоритма RSA, в частности, требования к общесистемным параметрам и ключам, методы распространения сертификатов ключей и ключевых параметров, а также порядок ввода их в действие и многое другое.


\subsection[Электронная подпись]{Электронная подпись RSA}

Предположим, что пользователь $A$ сообщения не шифрует, но хочет посылать свои сообщения в виде открытых текстов с подписью. Для этого надо создать электронную подпись (ЭП). Это можно сделать, используя систему RSA. При этом должны быть выполнены следующие требования:
\begin{itemize}
    \item вычисление подписи от сообщения является вычислительно легкой задачей;
    \item фальсификация подписи при неизвестном секретном ключе -- вычислительно трудная задача;
    \item подпись должна быть проверяемой открытым ключом.
\end{itemize}

Создание параметров ЭП RSA производится так же, как и для схемы шифрования RSA. Пусть  $A$ имеет секретный ключ $\SK = (d)$, а получатель (проверяющий) $B$ -- открытый ключ $\PK = (e,n)$ пользователя $A$.

\begin{enumerate}
    \item $A$ вычисляет подпись сообщения $m \in [1,n-1]$ как
        \[ s = m^{d} \mod n \]
        на своем секретном ключе $\SK$.
    \item $A$ посылает $B$ сообщение в виде $(m, s)$, где $m$ -- открытый текст, $s$ -- подпись.
    \item $B$ принимает сообщение $(m, s)$, возводит $s$ в степень $e$ по модулю $n$ ($e, n$ -- часть открытого ключа). В результате вычислений $B$ получает открытый текст
        \[ \left( m^{d} \mod n \right)^{e} \mod n = m. \]
    \item Сравнивает полученное значение с первой частью сообщения. При полном совпадении подпись принимается.
\end{enumerate}
Недостаток этой системы создания ЭП состоит в том, что подпись $m^{d} \mod n$ имеет большую длину, равную длине открытого сообщения $m$.

Для уменьшения длины подписи применяется другой вариант процедуры: вместо сообщения $m$ отправитель подписывает $h(m)$, где $h(x)$ -- известная криптографическая хэш-функция. Модифицированная процедура состоит в следующем.

\begin{enumerate}
    \item $A$ посылает $B$ сообщение в виде $(m, s)$, где $m$ -- открытый текст,
        \[ s = h(m)^d \mod n \]
        подпись.
    \item $B$ принимает сообщение $(m, s)$, вычисляет хэш $h(m)$ и возводит подпись в степень
        \[ h_1 = s^e \mod n. \]
    \item $B$ сравнивает значения $h(m)$ и $h_1$. При равенстве
        \[ h(m) = h_1 \]
        подпись считается подлинной, при неравенстве -- фальсифицированной.
\end{enumerate}


\subsubsection{Пример схемы}

\begin{enumerate}
    \item Генерирование параметров.
        \begin{enumerate}
            \item Выберем $p=13, q=17, n = 221$.
            \item Вычислим $\varphi(n) = (p-1)(q-1) = 12 \cdot 16 = 192$.
            \item Выберем $e=25: ~ \gcd(e = 25, \varphi(n) = 192) = 1, \\
                e \in [2, \varphi(n) - 1 = 191]$.
            \item Найдем $d = e^{-1} \mod \varphi(n) = 25^{-1} \mod 192 = 169$.
            \item Открытый и секретные ключи:
                \[ \PK = (e:25, n:221), ~ \SK = (d:169). \]
        \end{enumerate}
    \item Подписание.
        \begin{enumerate}
            \item Пусть хэш сообщения $h(m) = 12 \in [1, n-1]$.
            \item Вычислим ЭП
                \[ s = h^d = 12^{169} = 90 \mod 221. \]
        \end{enumerate}
    \item Проверка подписи.
        \begin{enumerate}
            \item Пусть хэш полученного сообщения $h(m) = 12$, полученная подпись $s = 90$.
            \item Выполним проверку
                \[ h_1 = s^e = 90^{25} = 12 \mod 221, ~~ h_1 = h, \]
                подпись верна.
        \end{enumerate}
\end{enumerate}


\subsection[Рандомизация шифрования и ЭП]{Рандомизация шифрования и \protect\\ подписания RSA}

\textbf{Семантически безопасной}\index{криптосистема!семантически-безопасная} называется криптосистема, для которой вычислительно невозможно извлечь любую информацию из шифротекстов, кроме длины шифротекста. Алгоритм RSA не является семантически безопасным. Одинаковые сообщения шифруются одинаково и, следовательно, применима атака на различение сообщений.

Кроме того, сообщения длиной менее $\frac{k}{3}$ бит, зашифрованные на малой экспоненте $e=3$, \emph{дешифруются} нелегальным пользователем извлечением обычного кубического корня.

В приложениях RSA используется только в сочетании с рандомизацией\index{рандомизация шифрования}. В стандарте PKCS\#1 RSA Laboratories описана схема рандомизации перед шифрованием OAEP-RSA (Optimal Asymmetric Encryption Padding). Примерная схема:
\begin{enumerate}
    \item Выбирается случайное $r$.
    \item Для открытого текста $m$ вычисляется
        \[ x = m \oplus H_1(r), ~ y = r \oplus H_2(x), \]
        где $H_1$ и $H_2$ -- криптографические хэш-функции.
    \item Сообщение $M = x \| y$ далее шифруется RSA.
\end{enumerate}
Восстановление $m$ из $M$ при расшифровании:
    \[ r = y \oplus H_2(x), ~ m = x \oplus H_1(r). \]

В модификации OAEP+ $x$ вычисляется как
    \[ x = (m \oplus H_1(r)) \| H_3(m \| r). \]

В описанной выше схеме ЭП под $m$ понимается хэш открытого текста, вместо шифрования выполняется подписание, вместо расшифрования -- проверка подписи.


\subsection{Выбор параметров и оптимизация}

\subsubsection{Выбор экспоненты $e$}

В случайно выбранной экспоненте $e$ c битовой длиной $k = \lceil \log_2 e \rceil$ половина бит в среднем равна 0, половина -- 1. При возведении в степень $m^e \mod n$ по методу <<возводи в квадрат и перемножай>> получится $k-1$ возведений в квадрат и, в среднем,
 $\frac{1}{2}(k-1)$ умножений.

Если выбрать $e$, содержащим малое число единиц в двоичной записи, то число умножений уменьшится до числа единиц в $e$.

Часто экспонента $e$ выбирается  \emph{малым} \emph{простым} числом и/или содержащим малое число единиц в битовой записи, для ускорения шифрования или проверки подписи, например:
\[
    \begin{array}{l}
        3 = [11]_2, \\
        17 = 2^4+1 = [10001]_2, \\
        257 = 2^8+1 = [100000001]_2, \\
        65537 = 2^{16}+1 = [10000000000000001]_2.
    \end{array}
\]

%Время шифрования или проверки подписи для малых экспонент становится $O(k^2)$ вместо $O(k^3)$, то есть в сотни раз быстрее для 1000-битовых чисел.


\subsubsection[Ускорение шифрования]{Ускорение шифрования по китайской \protect\\ теореме об остатках}

Возводя $m$ в степень $e$ отдельно по $\mod p$ и $\mod q$ и применяя китайскую теорему об остатках (Chinese remainder theorem, CRT), можно шифрование выполнить быстрее.

Однако ускорение шифрования в криптосистеме RSA через CRT может привести к уязвимостям в некоторых применениях, например, в смарт-картах.

\example
Пусть $c = m^e \mod n$ передается на расшифрование на смарт-карту, где вычисляется
\[ \begin{array}{c}
    m_p = c^d \mod p, \\
    m_q = c^d \mod q, \\
    m = m_p q (q^{-1} \mod p) + m_q p (p^{-1} \mod q) \mod n. \\
\end{array} \]
Криптоаналитик внешним воздействием может вызвать сбой во время вычисления $m_p$ (или $m_q$), в результате получится $m_p'$ и $m'$ вместо $m$. Зная $m_p'$ и $m'$ криптоаналитик находит разложение числа $n$ на множители $p,q$:
    \[ \gcd(m' - m, ~ n) = \gcd( (m_p' - m) q (q^{-1} \mod p), ~ pq) = q. \]
\exampleend


\subsubsection{Длина ключей}

В 2005 году было разложено 663-битовое число вида RSA. Время разложения в эквиваленте составило 75 лет вычислений одного ПК. Самые быстрые алгоритмы факторизации -- субэкспоненциальные\index{задача!факторизации}. Минимальная рекомендуемая длина модуля $n$ -- 1024 бит, но лучше использовать 2048 или 4096 бит.

В приложении показано, что битовая сложность (количество битовых операций) вычисления произвольной степени $a^b \mod n$ является кубической $O(k^3)$, а возведения в квадрат $a^2 \mod n$ и умножения $a b \mod n$ -- квадратичными $O(k^2)$, где $k$ -- битовая длина чисел $a,b,n$.

%Увеличение длины модуля $n$ в 2 раза увеличивает время возведения в степень в $2^3$ раз для большой экспоненты $e$, а для маленькой экспоненты -- в $2^2$ раза.


\section{Криптосистема Эль-Гамаля}\index{криптосистема!Эль-Гамаля|(}
\selectlanguage{russian}

Эта система шифрования с открытым ключом опубликована в 1985 году Эль-Гамалем (Taher El Gamal, \cite{ElGamal:1985}). Рассмотрим принципы ее построения.

Пусть имеется мультипликативная группа $\Z_p^* = \{1, 2, \dots, p-1\}$, где $p$ -- большое простое\index{число!простое} число, содержащее не менее 1024 двоичных разряда. В  группе $\Z_p^*$ существует $\varphi( \varphi( p ) ) = \varphi( p - 1 )$ элементов, которые порождают все элементы группы. Такие элементы называются генераторами.\footnote{Подробнее см. раздел~\ref{section-groups} в приложении.}

Выберем один из таких генераторов $g$ и целое число $x$ в интервале $1 \le x \le p-1$. Вычислим:
    \[ y = g^x \mod p. \]

Хотя элементы $x$ и $y$ группы $\Z_p^*$ задают друг друга однозначным образом, найти $y$ зная $x$ просто, а вот эффективного алгоритма для получения $x$ по $y$ неизвестно. Говорят, что задача вычисления дискретного логарифма
	\[ x = \log_g y \mod p \]
является вычислительно сложной задачей. На сложности вычисления дискретного логарифма для больших простых $p$ основывается криптосистема Эль-Гамаля.

\subsection{Шифрование}\index{шифр!Эль-Гамаля|(}

Процедура шифрования в криптосистеме Эль-Гамаля состоит из следующих операций.

\begin{enumerate}
    \item \textbf{Создание пары из закрытого и открытого ключей стороной $A$.}
        \begin{enumerate}
            \item $A$ выбирает простое\index{число!простое} случайное число $p$.
            \item Выбирает генератор $g$ (в программных реализациях алгоритма генератор часто выбирается малым числом, например $g = 2 \mod p$).
            \item Выбирает $x \in [2, p - 1]$ с помощью генератора случайных чисел.
            \item Вычисляет $y=g^{x}\mod p$.
            \item Создаёт закрытый и открытые ключи $\SK$ и $\PK$:
                \[ \SK = (p, g, x), ~ \PK = (p, g, y). \]
                Криптостойкость задаётся битовой длиной параметра $p$.
        \end{enumerate}
    \item \textbf{Шифрование на открытом ключе стороной $B$.}
        \begin{enumerate}
            \item Стороне $B$ известен открытый ключ $\text{PK} = (p, g, y)$ стороны $A$.
            \item Сообщение представляется числом $m \in [0, p-1]$.
            \item Выбирает случайное число $r \in [1, p-1]$ и вычисляет
                \[ \begin{array}{l}
                    a = g^r \mod p, \\
                    b = m \cdot y^r \mod p.
                \end{array} \]
            \item Создаёт шифрованное сообщение в виде
                \[ c = (a, b) \]
                и посылает стороне $A$.
        \end{enumerate}
    \item \textbf{Расшифрование на закрытом ключе стороной $A$.}

	Получив сообщение $(a, b)$ и владея закрытым ключом $\text{SK} = (p, g, x)$, $A$ вычисляет
                \[ m = \frac{b}{a^x} \mod p. \]
\end{enumerate}

Шифрование корректно, так как 
\[ \begin{array}{l}
    m' = \frac{b}{a^x} = \frac{m y^r}{g^{rx}} = m \mod p, \\
    m' \equiv m \mod p.
\end{array} \]

Чтобы криптоаналитику получить исходное сообщение $m$ из шифротекста $(a, b)$, зная только открытый ключ получателя $\text{PK} = (p, g, y)$, нужно вычислить значение $m = b \cdot y^{-r} \mod p$. Для этого криптоаналику нужно найти случайный параметр $r = \log_g a \mod p$, то есть вычислить дискретный логарифм. Такая задача является вычислительно сложной.

\example Создание ключей, шифрование и расшифрование в криптосистеме Эль-Гамаля.

\begin{enumerate}
    \item Генерирование параметров.
        \begin{enumerate}
            \item Выберем $p=41$.
            \item Группа $\Z_p^*$ циклическая, найдём генератор (примитивный элемент). Порядок группы
                \[ |\Z_p^*| = \varphi(p) = p-1 = 40. \]
                Делители 40: 2, 4, 5, 8, 10, 20. Элемент группы является примитивным, если все его степени, соответствующие делителям порядка группы, не сравнимы с 1. Из табл.~\ref{tab:elgamal-generator-search} видно, что число $g = 6$ является генератором всей группы.
                \begin{table}[!ht]
                    \centering
                    \caption{Поиск генератора в циклической группе $\Z_{41}^*$. Элемент 6 -- генератор\label{tab:elgamal-generator-search}}
                    \resizebox{\textwidth}{!}{ \begin{tabular}{|c|c|c|c|c|c|c|c|c|}
                        \hline
                        \multirow{2}{*}{Элемент} & \multicolumn{7}{|c|}{Степени} & \multirow{2}{*}{Порядок элемента} \\
                        \cline{2-8}
                                & 2   & 4   & 5   & 8  & 10 & 20 & 40 & \\
                        \hline
                        2       & 4   & 16  & -9  & 10 & -1 & 1  &    & 20 \\
                        3       & 9   & -1  & -3  & 1  &    &    &    & 8 \\
                        5       & -16 & 10  & 9   & 18 & -1 & 1  &    & 20 \\
                        6       & -5  & -16 & -14 & 10 & -9 & -1 & 1  & 40 \\
                        \hline
                    \end{tabular} }
                \end{table}
            \item Выберем случайное $x = 19 \in [1, p-1]$.
            \item Вычислим
                \[ \begin{array}{ll}
                    y & = g^x \mod p = \\
                    & = 6^{19} \mod 41 = \\
                    & = 6^{1 + 2 + 4 \cdot 0 + 8 \cdot 0 + 16} \mod 41 = \\
                    & = 6^1 \cdot 6^2 \cdot 6^{4 \cdot 0} \cdot 6^{8 \cdot 0} \cdot 6^{16} \mod 41 = \\
                    & = 6 \cdot (-5) \cdot (-16)^0 \cdot 10^0 \cdot 18 \mod 41 = \\
                    & = -7 \mod 41.
                \end{array} \]
            \item Открытый и закрытый ключи:
                \[ \PK = (p:41, g:6, y:-7), ~ \SK = (p:41, g:6, x:19). \]
        \end{enumerate}
    \item Шифрование.
        \begin{enumerate}
            \item Пусть сообщением является число $m = 3 \in \Z_p^*$.
            \item Выберем случайное число $r = 25 \in [1, p-1]$.
            \item Вычислим
                \[ \begin{array}{l}
                    a = g^r \mod p = 6^{25} \mod 41 = 14 \mod 41, \\
                    b = m y^r \mod p = 3 \cdot (-7)^{25} \mod 41 = -9 \mod 41.
                \end{array} \]
            \item Шифротекстом является пара чисел
                \[ c = (a:14, ~ b:-9). \]
        \end{enumerate}
    \item Расшифрование.
        \begin{enumerate}
            \item Пусть получен шифротекст
                \[ c = (a:14, ~ b:-9). \]
            \item Вычислим открытый текст как
                \[ \begin{array}{ll}
                    m & = \frac{b}{a^x} \mod p = \\
                    & = -9 \cdot (14^{-1})^{19} \mod 41 = \\
                    & = -9 \cdot 3^{19} \mod 41 = \\
                    & = -9 \cdot (-14) \mod 41 = \\
                    & = 3 \mod 41. \\
                \end{array} \]
        \end{enumerate}
\end{enumerate}

\exampleend
\index{шифр!Эль-Гамаля|)}

\subsection{Электронная подпись}\index{электронная подпись!Эль-Гамаля|(}

Криптосистема Эль-Гамаля, как и криптосистема RSA\index{криптосистема!RSA}, может быть использована для создания электронной подписи.

По-прежнему имеются два пользователя $A$ и $B$ и незащищённый канал связи между ними. Пользователь $A$  хочет подписать свое открытое сообщение $m$  для того, чтобы пользователь $B$ мог убедиться, что именно $A$ подписал сообщение.

Пусть $A$ имеет закрытый ключ $\SK = (p, g, x)$, открытый ключ $\PK = (p, g, y)$ (полученные так же, как и в системе шифрования Эль-Гамаля) и хочет подписать открытое сообщение. Обозначим подпись $S(m)$.

Для создания подписи $S(m)$ пользователь $A$ выполняет следующие операции:
\begin{itemize}
    \item вычисляет значение криптографической хэш-функции  $h(m) \in [0,p-2]$ от своего открытого сообщения $m$;
    \item выбирает случайное число $r, ~ \gcd(r, p-1)=1$;
    \item используя закрытый ключ, вычисляет
        \[ \begin{array}{l}
            a = g^r \mod p, \\
            b = \frac{h(m) - xa}{r} \mod (p-1); \\
        \end{array} \]
    \item создаёт подпись в виде двух чисел
        \[ S(m) = (a, b) \]
        и посылает сообщение с подписью $(m, S(m))$.
\end{itemize}

Получив сообщение, $B$ осуществляет проверку подписи, выполняя следующие операции:
\begin{itemize}
    \item по известному сообщению $m$ вычисляет значение хэш-функции $h(m)$;
    \item вычисляет
        \[ \begin{array}{l}
            f_1 = g^{h(m)} \mod p, \\
            f_2 = y^a a^b \mod p; \\
        \end{array} \]
    \item сравнивает значения $f_1$ и $f_2$, если
        \[ f_1 = f_2, \]
        то подпись подлинная, в противном случае -- фальсифицированная (или случайно испорченная).
\end{itemize}

Покажем, что проверка подписи корректна. По малой теореме Ферма получаем
\[ \begin{array}{ll}
    f_1 & = g^{h(m)} \mod p = \\
    & \\
    & = g^{h(m) \mod (p-1)} \mod p; \\
\end{array} \] \[ \begin{array}{ll}
    f_2 & = y^a a^b \mod p = \\
    & = \underbrace{\left( g^x \right)^a}_{y^a} \cdot
        \underbrace{\left( g^r \mod p \right)^{\frac{h(m) - xa}{r} \mod (p-1)}}_{a^b} \mod p = \\
    & \\
    & = g^{xa \mod (p-1)} ~\cdot~ g^{h(m) - xa \mod (p-1)} \mod p = \\
    & = g^{h(m) \mod (p-1)} \mod p = \\
    & = f_1.
\end{array} \]

\example Создание и валидация электронной подписи в криптосистеме Эль-Гамаля.

\begin{enumerate}
    \item Генерирование параметров.
        \begin{enumerate}
            \item Выберем $p=41$.
            \item Выберем генератор $g=6$ в группе $\Z_{41}^*$.
            \item Выберем случайное $x = 19 \in [1, p-1]$.%, ~ \gcd(x, p-1) = 1$.
            \item Вычислим
                \[ \begin{array}{ll}
                    y & = g^x \mod p = \\
                    & = 6^{19} \mod 41 = \\
                    & = 6^{1 + 2 + 4 \cdot 0 + 8 \cdot 0 + 16} \mod 41 = \\
                    & = 6 \cdot (-5) \cdot (-16)^0 \cdot 10^0 \cdot 18 \mod 41 = \\
                    & = -7 \mod 41. \\
                \end{array} \]
            \item Открытый и закрытый ключи:
                \[ \PK = (p:41, g:6, y:-7), ~ \SK = (p:41, g:6, x:19). \]
        \end{enumerate}
    \item Подписание.
        \begin{enumerate}
            \item От сообщения $m$ вычисляется хэш $h = H(m)$. Пусть хэш $h  = 3 \in [0, p-2]$.
            \item Выберем случайное $r = 9 \in [1, p-2]$: \\
                $\gcd(r=9, p-1 = 40) = 1$.
            \item Вычислим
                \[ \begin{array}{ll}
                    a & = g^r \mod p = \\
                      & = 6^9 \mod 41 = 19 \mod 41, \\
                    b & = \frac{h - xa}{r} \mod (p-1) = \\
                      & = (3 - 19 \cdot 19) \cdot 9^{-1} \mod 40 = \\
                      & = 2 \cdot 9 \mod 40 = 18 \mod 40. \\
                \end{array} \]
            \item Подпись
                \[ s = (a:19, b:18). \]
        \end{enumerate}
    \item Проверка подписи.
        \begin{enumerate}
            \item Для полученного сообщения находится хэш $h = H(m) = 3$. Пусть полученная подпись к нему имеет вид
                \[ s = (a:19, b:18). \]
            \item Вычислим
                \[ \begin{array}{ll}
                    f_1 & = g^h \mod p = \\
                        & = 6^3 \mod 41 = 11 \mod 41, \\
                    f_2 & = y^a a^b \mod p = \\
                        & = (-7)^{19} \cdot 19^{18} \mod 41 = 11 \mod 41. \\
                \end{array} \]
            \item Проверим равенство $f_1$ и $f_2$. Подпись верна, так как
                \[ f_1 = f_2 = 11. \]
        \end{enumerate}
\end{enumerate}

\exampleend
\index{электронная подпись!Эль-Гамаля|)}

\subsection{Криптостойкость}

Пусть дано уравнение $y=g^{x} \mod p$, требуется определить $x$ в интервале $0 < x < p-1$. Задача называется \textbf{дискретным логарифмированием}\index{задача!дискретного логарифмирования}.

Рассмотрим возможные способы нахождения неизвестного числа $x$. Начнем с перебора различных значений $x$ из интервала $0<x<p-1$ и проверки равенства $y=g^{x} \mod p$. Число попыток в среднем равно $\frac{p}{2}$, при $p=2^{1024}$ это число равно $2^{1023}$, что на практике не осуществимо.

Другой подход предложен советским математиком Гельфондом\index{алгоритм!Гельфонда} безотносительно к криптографии. Он состоит в следующем.
Вычислим $S=\lceil\sqrt{p-1}\rceil $, где скобки означают ближайшее (сверху) целое от числа $\sqrt{p-1} $.

Представим искомое число $x$   в следующем виде

\begin{equation}
    x=x_{1} S+x_{2},
    \label{S}
\end{equation}

где $x_{1}$ и $x_{2}$ -- целые неотрицательные числа,
    \[ x_{1} \le S-1, ~ x_{2} \le S-1. \]
Такое представление является однозначным.

Вычислим и занесем в таблицу следующие $S$  чисел:
    \[ g^{0} \mod p, ~~ g^{1} \mod p, ~~ g^{2} \mod p, ~~ \dots, ~~ g^{S-1} \mod p. \]
Вычислим $g^{-S} \mod p$ и также занесем в таблицу.

\begin{center} \begin{tabular}{|l|c|c|c|c|c|c|}
    \hline
    $\lambda $ & 0 & 1 & 2 & \dots & $S-1$ & $-S$ \\
    \hline
    $g^{\lambda} \mod p$ & $g^{0}$ & $g^{1}$ & $g^{2}$ & \dots & $g^{S-1}$ & $g^{-S}$ \\
    \hline
\end{tabular} \end{center}

Для решения уравнения~\ref{S} используем перебор значений $x_{1}$.
\begin{enumerate}
    \item Предположим, что $x_{1} = 0$. Тогда $x = x_{2}$. Если число $y = g^{x_{2}} \mod p$ содержится в таблице, то находим его и выдаём результат: $x=x_{2} $. Задача решена. В противном случае переходим к пункту 2.
    \item Предположим, что $x_{1} =1$. Тогда $x=S+x_{2} $ и $y=g^{S+x_{2}} \mod p$. Вычисляем $yg^{-S} \mod p=g^{x_{2}} \mod p$. Задача сведена к предыдущей: если $g^{x_{2} } \mod p$ содержится в таблице, то в таблице находим число $x_{2} $ и выдаём результат $x$: $x=S+x_{2} $.
    \item Предположим, что $x_{1} =2$. Тогда $x=2S+x_{2} $ и $y=g^{2S+x_{2} } \mod p$. Если число $yg^{-2S} \mod p=g^{x_{2} } \mod p$ содержится в таблице, то находим число $x_{2}$ и выдаём результат: $x = 2S + x_{2}$.
     \item Пробегая все возможные значения, доберемся, в худшем случае, до $x_{1} =S-1$. Тогда $x=(S-1)S+x_{2} $ и $y = g^{(S-1)S+x_{2} } \mod p$. Если число $yg^{-(S-1)S} \mod p=g^{x_{2}} \mod p$ содержится в таблице, то находим его и выдаём результат: $x=(S-1)S+x_{2}$.
\end{enumerate}

Легко проверить, что с помощью построенной таблицы мы проверили все возможные значения $x$. Максимальное число умножений равно $2S \approx 2\sqrt{p-1} =2\times 2^{512} $, что для практики очень велико. Тем самым проблему достаточной криптостойкости этой системы можно было бы считать решенной. Однако, это не верно, так как числа $p-1$ являются составными. Если  $p-1$ можно разложить на маленькие множители, то криптоаналитик может применить процедуру, подобную процедуре Гельфонда, по взаимно простым делителям  $p-1$  и найти секрет. Пусть  $p-1=st$. Тогда элемент $g^s$ образует подгруппу порядка $t$ и наоборот. Теперь, решая уравнение $y^s=(g^s)^a\mod p$, находим вычет $x=a\mod t$. Поступая аналогично, находим $x=b\mod s$ и по Китайской теореме об остатках находим $x$.

Несколько позже подобный метод ускоренного решения уравнения~\ref{S} был предложен Шенксом (Daniel Shanks, \cite{Shanks:1971})\index{алгоритм!Шенкса}.

Пусть $k = \lceil \log_2 p \rceil$ -- битовая длина числа $p$. Алгоритм Гельфонда имеет экспоненциальную сложность (число двоичных операций)
    \[ O(\sqrt{p}) = O(e^{\frac{1}{2} \frac{1}{\log_2 e} k}). \]

Наилучшие из известных алгоритмов решения задачи дискретного логарифмирования имеют экспоненциальную сложность порядка
    \[ O(e^{\sqrt{k}}). \]

\index{криптосистема!Эль-Гамаля|)}


\section[Российский стандарт ЭП ГОСТ Р 34.10-2001]{Российский стандарт ЭП \protect\\ ГОСТ Р 34.10-2001}
\selectlanguage{russian}

Российский стандарт цифровой подписи основан на криптосистеме типа Эль-Гамаля\index{криптосистема!Эль-Гамаля}, в которой в качестве группы используется группа точек эллиптической кривой над конечным полем (см. Приложение). Группа должна быть большой с количеством элементов порядка $2^{255}$.

Пусть имеются две стороны $A$ и $B$ и между ними канал связи. Сторона $A$ желает передать сообщение $M$ стороне $B$ и подписать его. Сторона $B$ должна проверить правильность подписи, то есть аутентифицировать сторону $A$.

$A$ формирует открытый ключ следующим образом.

\begin{enumerate}
    \item Выбирает простое число $p > 2^{255}$.
    \item Записывает уравнение эллиптической кривой
        \[ E: ~ y^2 = x^3 + a x + b \mod p, \]
        которое определяет группу точек эллиптической кривой $\E(\Z_p)$.
        Выбирает группу, задавая либо случайные числа $0 < a, b < p-1$, либо инвариант $J(E)$:
        \[ J(E) = 1728 \frac{4 a^3}{4 a^3 + 27 b^2} \mod p. \]
        Если кривая задается инвариантом $J(E) \in \Z_p$, то он выбирается случайно в интервале $0 < J(E) < 1728$. Для нахождения $a,b$ вычисляется
        \[ K = \frac{J(E)}{1728 - J(E)}, \]
        \[ \begin{array}{l}
            a = 3 K \mod p, \\
            b = 2 K \mod p. \\
        \end{array} \]
    \item Пусть $m$ -- порядок группы точек эллиптической кривой $\E(\Z_p)$. ~Пользователь $A$ подбирает число $n$ и простое число $q$ такие, что
        \[ m = n q, ~ 2^{254} < q < 2^{256}, ~ n \geq 1, \]
        где $q$ -- делитель порядка группы.

        В циклической подгруппе порядка $q$ выбирается точка
        \[ P \in \E(\Z_p): ~ q P \equiv 0. \]
    \item Случайно выбирает число $d$ и вычисляет точку $Q = d P$.
    \item Формирует секретный и открытый ключи:
        \[ \SK = (d), ~ \PK = (p, E, q, P, Q). \]
\end{enumerate}

Теперь сторона $A$ создает свою цифровую подпись $S(M)$ сообщения $M$, выполняя следующие действия.
\begin{enumerate}
    \item Вычисляет число $\alpha = h(M)$, где $h$ -- криптографическая хэш-функция, определенная стандартом ГОСТ Р 34.11-94. В российском стандарте длина $h(M)$ равна 256 бит.
    \item Вычисляет  $e = \alpha \mod q$.
    \item Случайно выбирает число $k$ и вычисляет точку
        \[ C = k P = (x_c, y_c). \]
    \item Вычисляет  $r = x_c \mod q$.
	Если $r = 0$, то выбирает другое $k$.
    \item Вычисляет  $s = k e + r d \mod q$.
	Если $s = 0$, то выбирает другое $k$.
    \item Формирует подпись
        \[ S(M) = (r, s). \]
\end{enumerate}
Сторона $A$ передает стороне $B$ сообщение с подписью
    \[ (M, ~ S(M)). \]

Сторона $B$ проверяет подпись $(r,s)$, выполняя процедуру проверки подписи.
\begin{enumerate}
    \item Вычисляет  $\alpha = h(M)$ и $e = \alpha \mod q$.
    \item Вычисляет  $e^{-1} \mod q$.
    \item Проверяет условия $r < q, ~ r < s$. Если эти условия не выполняются, то подпись отвергается. Если условия выполняются, то процедура продолжается.
    \item Вычисляет числа
        \[ \begin{array}{l}
            a = s e^{-1} \mod q, \\
            b = -r e^{-1} \mod q. \\
        \end{array} \]
    \item Вычисляет точку
        \[ \tilde{C} = a P + b Q = (\tilde{x}_c, \tilde{y}_c). \]
        Если подпись верна, должны получить исходную точку $C$.
    \item Проверяет условие $\tilde{x}_{c} \mod q = r$. Если условие выполняется, то подпись принимается, в противном случае --- отвергается.
\end{enumerate}

Рассмотрим вычислительную сложность вскрытия подписи. Предположим, что криптоаналитик ставит своей задачей определение секретного ключа $d$. Как известно,  эта  задача является трудной. Для подтверждения этого можно привести следующий факт. Был поставлен следующий эксперимент: было выбрано число $p = 2^{97},$ и 1200 персональных компьютеров с тактовой частотой процессоров 200 МГц в 16 странах мира работали, чтобы решить эту задачу. Задача была решена за 53 дня круглосуточной работы. Если взять $p = 2^{256}$, то на решение такой задачи при наличии одного компьютера с частотой процессора 2 ГГц потребуется $10^{22}$ лет.


\section{Длины ключей}
\selectlanguage{russian}

В таблице~\ref{tab:recommended-key-lengths} приведены битовые длины ключей для криптосистем.
%Традиционные рекомендации основаны на аппроксимации существующих алгоритмов для взлома на 10-30 лет вперед.

\begin{table}[!ht]
    \centering
    \caption{Минимальные длины ключей в битах по стандартам России и США\label{tab:recommended-key-lengths}}
    \resizebox{\textwidth}{!}{ \begin{tabular}{|l|c|c|c|c|}
        \hline
        & \multirow{2}{*}{\parbox{1.5cm}{\medskip \centering Блочные шифры, $K$}} & \multicolumn{3}{|c|}{Схема ЭП} \\
        \cline{3-5}
        & & \parbox{1.5cm}{\centering RSA\index{криптосистема!RSA}, $n$} & \parbox{2.4cm}{\centering Эллипт. кривые, порядок точки} & \parbox{3.5cm}{\centering Эль-Гамаль\index{криптосистема!Эль-Гамаля} $\operatorname{mod} p$: модуль / порядок (под)группы} \\
        \hline \hline
        \multicolumn{5}{|c|}{Взломано} \\
        \hline
        Биты & 56 & 768 & 109 & 503  \\
        Конкурс & \textsc{DesChal} & RSA-768 & ECC2K-108 &  \\
        Год & 1997 & 2009 & 2000 &  \\
        \hline \hline
        \multicolumn{5}{|c|}{Стандарт России} \\
        \hline
        Биты & 256 &  & 255 & \\
        ГОСТ & 28147"---89 &~--- & 34.10-2001 &~--- \\
        Год & 1989 & & 2001 & \\
%       \hline
%       \multicolumn{2}{|l|}{\parbox{4cm}{Россия: нелицензируемая деятельность}} & \multicolumn{4}{c|}{40} \\
        \hline \hline
        \multicolumn{5}{|c|}{Стандарт США} \\
        \hline
        Биты & 128-256 & 1024-3072 & 151-480 & 1024-3072/160-256 \\
        FIPS \No & 197 & draft 186-3 & draft 186-3 & draft 186-3 \\
        Год & 2001 & 2006 & 2006 & 2006 \\
%       \hline
%       \multicolumn{2}{|l|}{\parbox{4cm}{США: экспортные ограничения до 2001 г.}} & 56 & 512 & 112 & 512/112 \\
%       \hline \hline
%       \multicolumn{2}{|l|}{Традиционные} & 80 & 1024 & 160 & 1024/160 \\
%       \cline{3-6}
%       \multicolumn{2}{|l|}{рекомендации} & 112 & 2048 & 224 & 2048/224 \\
%       \hline
%       \multicolumn{2}{|l|}{\parbox{4cm}{Рекомендация Lenstra, Verheul для 2010 г.}} & 78 & 1369 & 146-160 & 1369/138 \\
        \hline
    \end{tabular} }
\end{table}
%}\end{center}


\subsection*{Скорость вычисления ЭП}

Сравним производительность схем ЭП, чтобы продемонстрировать преимущества ЭП вида Эль-Гамаля\index{криптосистема!Эль-Гамаля} перед RSA\index{криптосистема!RSA} для больших ключей. В приложении показано, что в модульной арифметике по модулю числа $n$ с битовой длиной $k \simeq \log_2 n$ операции имеют битовую сложность:
\[ \begin{array}{lcl}
    a^b \mod n & - & O(k^3), \\
    ab \mod n, ~ a^{-1} \mod n & - & O(k^2), \\
    a+b \mod n & - & O(k). \\
\end{array} \]

Так как все описанные схемы ЭП используют возведение в степень по модулю, то битовая сложность~-- $O(k^3)$. Оценки количества целочисленных $t$-разрядных умножений при вычислении ЭП имеют вид:
\begin{enumerate}
    \item RSA\index{электронная подпись!RSA}:
        \[ (2 \log_2 n) \cdot \left( \frac{\log_2 n}{t} \right)^2; \]
    \item DSA\index{электронная подпись!DSA} (\langen{Digital Signature Algorithm}, стандарт США~\cite{FIPS-PUB-186-4})~-- электронная подпись, вычисляемая по принципу Эль-Гамаля\index{криптосистема!Эль-Гамаля} по модулю $p$ и с порядком циклической подгруппы $q$:
        \[ (2 \log_2 q) \cdot \left( \frac{\log_2 p}{t} \right)^2; \]
    \item ГОСТ Р 34.10-2001\index{электронная подпись!ГОСТ Р 34.10-2001} (стандарт России~\cite{GOST-2001}) и ECDSA\index{электронная подпись!ECDSA} (\langen{Elliptic Curve Digital Signature Algorithm}, стандарт США~\cite{FIPS-PUB-186-4}), вычисляемые по принципу Эль-Гамаля\index{криптосистема!Эль-Гамаля} в группе точек эллиптической кривой по модулю $p$:
        \[ (2 \log_2 p) \cdot 4 \cdot \left( \frac{\log_2 p}{t} \right)^2. \]
\end{enumerate}

В таблице~\ref{tab:signature-rate} приведены оценки скорости вычисления ЭП (оценки числа умножений 64-битовых слов).

\begin{table}[!ht]
    \centering
    \caption{Оценочное число 64-битовых умножений для вычисления ЭП\label{tab:signature-rate}}
    \begin{tabular}{|c|l|c|}
        \hline
        ЭП & Оценочное число 64-битовых умножений \\
        \hline \hline
        RSA\index{электронная подпись!RSA} 1024 & $(2 \cdot 1024) \cdot \left( \frac{1024}{64} \right)^2 \approx$ 500 000 \\
        RSA\index{электронная подпись!RSA} 2048 & 4 000 000 \\
        RSA\index{электронная подпись!RSA} 3072 & 14 000 000 \\
        RSA\index{электронная подпись!RSA} 4096 & 34 000 000 \\
        \hline \hline
        DSA\index{электронная подпись!DSA} 1024/160 & $(2 \cdot 160) \cdot \left( \frac{1024}{64} \right)^2 \approx$ 82 000 \\
        DSA\index{электронная подпись!DSA} 3072/256 & 1 200 000 \\
        \hline \hline
        ECDSA\index{электронная подпись!ECDSA} 160 & $(2 \cdot 160) \cdot 4 \cdot \left( \frac{160}{64} \right)^2 \approx$ 8 000 \\
        ECDSA\index{электронная подпись!ECDSA} 512 & 260 000 \\
        \hline \hline
        ГОСТ Р 34.10-2001\index{электронная подпись!ГОСТ Р 34.10-2001} & $(2 \cdot 256) \cdot 4 \cdot \left( \frac{256}{64} \right)^2 \approx$ 33 000 \\
        \hline
    \end{tabular}
\end{table}


\section{Инфраструктура открытых ключей}\label{chapter-public-key-infrastructure}

\subsection{Иерархия удостоверяющих центров}\label{section-CAs}
\selectlanguage{russian}

Проблему аутентификации и распределения сеансовых симметричных ключей шифрования в Интернете, а также в больших локальных и виртуальных сетях решают с помощью построения иерархии открытых ключей криптосистем с открытым ключом.

\begin{enumerate}
    \item Существует удостоверяющий центр (УЦ) верхнего уровня\index{удостоверяющий центр!верхнего уровня}, корневой УЦ\index{удостоверяющий центр!корневой} (root certificate authority, root CA), обладающий парой из закрытого и открытого ключей. Открытый ключ УЦ верхнего уровня распространяется среди всех пользователей, причем все пользователи \emph{доверяют УЦ}. Это означает, что:
        \begin{itemize}
            \item УЦ -- <<хороший>>, то есть обеспечивает надёжное хранение закрытого ключа, не пытается фальсифицировать и скомпрометировать свои ключи;
            \item имеющийся у пользователей открытый ключ УЦ действительно принадлежит УЦ.
        \end{itemize}
        В массовых информационных и интернет-системах открытые ключи многих корневых УЦ встроены в дистрибутивы и пакеты обновлений ПО. Доверие пользователей неявно проявляется в их уверенности в том, что открытые ключи корневых УЦ, включенные в ПО, не фальсифицированы и не скомпрометированы. \emph{Де-факто пользователи доверяют а) распространителям ПО и обновлений, б) корневому УЦ.}\index{доверие}

        Назначение УЦ верхнего уровня -- проверка принадлежности и подписание открытых ключей других удостоверяющих центров второго уровня, а также организаций и сервисов. УЦ подписывает своим закрытым ключом следующее сообщение:
        \begin{itemize}
            \item название и URI УЦ нижележащего уровня или организации/сервиса,
            \item значение сгенерированного открытого ключа и название алгоритма соответствующей криптосистемы с открытым ключом,
            \item время выдачи и срок действия открытого ключа.
        \end{itemize}

    \item УЦ второго уровня\index{удостоверяющий центр} (certificate authority, certification authority, CA) имеют свои пары открытых и закрытых ключей, сгенерированных и подписанных корневым УЦ. Причем перед подписанием корневой УЦ убеждается в <<надёжности>> УЦ второго уровня, производит юридические проверки. Корневой УЦ не имеет доступа к закрытым ключам УЦ второго уровня.

        Пользователи, имея в своей базе открытых ключей доверенные открытые ключи корневого УЦ, могут проверить ЭП открытых ключей УЦ 2-го уровня и убедиться, что предъявленный открытый ключ действительно принадлежит данному УЦ. Таким образом:
        \begin{itemize}
            \item Пользователи полностью доверяют корневому УЦ и его открытому ключу, который у них хранится. Пользователи верят, что корневой УЦ не подписывает небезопасные ключи и гарантирует, что подписанные им ключи действительно принадлежат УЦ 2-го уровня.
            \item Проверив ЭП открытого ключа УЦ 2-го уровня с помощью доверенного открытого ключа УЦ 1-го уровня, пользователь верит, что открытый ключ УЦ 2-го уровня действительно принадлежит данному УЦ и не был скомпрометирован.
        \end{itemize}

        Аутентификация в протоколе защищённого интернет-соединения SSL/TLS\index{протокол!SSL/TLS} достигается в результате проверки пользователями совпадения URI-адреса сервера из ЭП с фактическим адресом.

        УЦ второго уровня в свою очередь тоже подписывает открытые ключи УЦ третьего уровня, а также организаций. И так далее по уровням.

    \item В результате построена \emph{иерархия} подписанных открытых ключей.

    \item Открытый ключ с идентификационной информацией (название организации, URI-адрес веб-ресурса, дата выдачи, срок действия и др.) и подписью УЦ вышележащего уровня, заверяющей ключ и идентифицирующие реквизиты, называются \textbf{сертификатом открытого ключа},\index{сертификат открытого ключа} на который существует международный стандарт X.509\index{сертификат!X509}, последняя версия 3. В сертификате указывается его область применения: подписание других сертификатов, аутентификация для веба, аутентификация для электронной почты и т.~д.
\end{enumerate}

\begin{figure}[!ht]
	\centering
	\includegraphics[width=0.8\textwidth]{pic/X509-hierarchy}
	\caption{Иерархия сертификатов\label{fig:x509-hierarchy}}
\end{figure}

На рис.~\ref{fig:x509-hierarchy} приведены пример иерархии сертификатов и путь подписания сертификата X.509\index{сертификат!X509} интернет-сервиса Google Mail.

Система распределения, хранения и управления сертификатами открытых ключей называется \textbf{инфраструктурой открытых ключей}\index{инфраструктура открытых ключей} (public key infrastructure, PKI). PKI применяется для аутентификации в системах SSL/TLS\index{протокол!SSL/TLS}, IPsec\index{протокол!IPsec}, PGP и т.~д. Помимо процедур выдачи и распределения открытых ключей PKI также определяет процедуру отзыва скомпрометированных или устаревших сертификатов.


\subsection{Структура сертификата X.509}
\selectlanguage{russian}

Ниже приведён пример сертификата X.509\index{сертификат!X509} интернет-сервиса mail.google.com, использовавшийся для защищённого SSL-соединения в 2009 г. Сертификат напечатан командой \texttt{openssl x509 -in file.crt -noout -text}:

{\small \begin{verbatim}
Certificate:
Data:
  Version: 3 (0x2)
  Serial Number:
    6e:df:0d:94:99:fd:45:33:dd:12:97:fc:42:a9:3b:e1
  Signature Algorithm: sha1WithRSAEncryption
  Issuer: C=ZA, O=Thawte Consulting (Pty) Ltd.,
    CN=Thawte SGC CA
  Validity
    Not Before: Mar 25 16:49:29 2009 GMT
    Not After : Mar 25 16:49:29 2010 GMT
  Subject: C=US, ST=California, L=Mountain View, O=Google Inc,
    CN=mail.google.com
  Subject Public Key Info:
    Public Key Algorithm: rsaEncryption
    RSA Public Key: (1024 bit)
      Modulus (1024 bit):
        00:c5:d6:f8:92:fc:ca:f5:61:4b:06:41:49:e8:0a:
        2c:95:81:a2:18:ef:41:ec:35:bd:7a:58:12:5a:e7:
        6f:9e:a5:4d:dc:89:3a:bb:eb:02:9f:6b:73:61:6b:
        f0:ff:d8:68:79:1f:ba:7a:f9:c4:ae:bf:37:06:ba:
        3e:ea:ee:d2:74:35:b4:dd:cf:b1:57:c0:5f:35:1d:
        66:aa:87:fe:e0:de:07:2d:66:d7:73:af:fb:d3:6a:
        b7:8b:ef:09:0e:0c:c8:61:a9:03:ac:90:dd:98:b5:
        1c:9c:41:56:6c:01:7f:0b:ee:c3:bf:f3:91:05:1f:
        fb:a0:f5:cc:68:50:ad:2a:59
      Exponent: 65537 (0x10001)
  X509v3 extensions:
    X509v3 Extended Key Usage: TLS Web Server
      Authentication, TLS Web Client Authentication,
      Netscape Server Gated Crypto
    X509v3 CRL Distribution Points:
    URI:http://crl.thawte.com/ThawteSGCCA.crl
    Authority Information Access:
    OCSP - URI:http://ocsp.thawte.com
    CA Issuers - URI:http://www.thawte.com/repository/
        Thawte_SGC_CA.crt
    X509v3 Basic Constraints: critical
    CA:FALSE
Signature Algorithm: sha1WithRSAEncryption
  62:f1:f3:05:0e:bc:10:5e:49:7c:7a:ed:f8:7e:24:d2:f4:a9:
  86:bb:3b:83:7b:d1:9b:91:eb:ca:d9:8b:06:59:92:f6:bd:2b:
  49:b7:d6:d3:cb:2e:42:7a:99:d6:06:c7:b1:d4:63:52:52:7f:
  ac:39:e6:a8:b6:72:6d:e5:bf:70:21:2a:52:cb:a0:76:34:a5:
  e3:32:01:1b:d1:86:8e:78:eb:5e:3c:93:cf:03:07:22:76:78:
  6f:20:74:94:fe:aa:0e:d9:d5:3b:21:10:a7:65:71:f9:02:09:
  cd:ae:88:43:85:c8:82:58:70:30:ee:15:f3:3d:76:1e:2e:45:
  a6:bc
\end{verbatim}}

Как видно, сертификат действителен с 26.03.2009 до 25.03.2010, открытый ключ представляет собой ключ RSA\index{криптосистема!RSA} с длиной модуля $n =$ 1024 бит и экспонентой $e = 65537$ и принадлежит компании Google Inc. Открытый ключ предназначен для взаимной аутентификации веб-сервера mail.google.com и веб-клиента в протоколе SSL/TLS. Сертификат подписан ключом удостоверяющего центра Thawte SGC CA, подпись вычислена с помощью криптографического хэша SHA-1\index{хэш-функция!SHA-1} и алгоритма RSA\index{электронная подпись!RSA}. В свою очередь, сертификат с открытым ключом Thawte SGC CA для проверки значения ЭП данного сертификата расположен по адресу \url{http://www.thawte.com/repository/Thawte\_SGC\_CA.crt}.

Электронная подпись вычисляется от всех полей сертификата, кроме самого значения подписи.

