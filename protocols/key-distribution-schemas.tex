\section{Схемы с доверенным центром}\label{key-distribution-schemas}\index{схема!распределения ключей|(}
\selectlanguage{russian}

Схемы распределения ключей с доверенным центром состоят из трёх этапов.

\begin{enumerate}
    \item На первом этапе доверенный центр создаёт некоторый секрет, известный только ему. Это может быть некоторая секретная матрица с особыми свойствами, как в схеме Блома\index{схема!Блома} из раздела~\ref{section-bloms-scheme}, или пара из закрытого и открытого ключей, как в схеме Жиро\index{схема!Жиро} из раздела~\ref{section-girault-scheme}.
    \item Для каждого нового легального участника сети доверенный центр, используя свою секретную информацию, вырабатывает некоторый отпечаток или сертификат, который позволяет новому участнику вырабатывать сеансовые ключи с другими легальными участниками.
    \item Наконец, на третьем этапе, когда начинается протокол общения двух легальных участников, они предъявляют друг-другу идентификаторы и/или дополнительную информацию от доверенного центра. Используя её, без дополнительного обращения к центру, они могут сгенерировать секретный сеансовый ключ для общения между собой.
\end{enumerate}

\subsection{Схема Жиро}\label{section-girault-scheme}\index{схема!Жиро|(}
\selectlanguage{russian}

В схеме Жиро (\langfr{Marc Girault},~\cite{Girault:1990, Girault:1991}) надёжность строится на стойкости криптосистемы RSA (сложности факторизации больших чисел и вычисления дискретного корня).

Предварительно:
\begin{itemize}
    \item Доверенный центр (Трент, $T$):
    \begin{itemize}
        \item выбирает общий модуль $n = p \times q$, где $p$ и $q$ -- большие простые числа;
        \item выбирает пару из закрытого и открытого ключей $K_{T, \text{public}}: (e, n)$ и $K_{T, \text{private}}: (d, n)$;
        \item выбирает элемент $g$ поля $\mathbb{Z}_n^{\times}$ максимального порядка;
        \item публикует в общедоступном месте параметры схемы $n$, $e$ и $g$.
    \end{itemize}
    \item Каждый из легальных участников:
    \begin{itemize}
        \item выбирает себе закрытый ключ $s_i$ и идентификатор $I_i$;
        \item вычисляет и отправляет доверенному центру $v_i = g^{-s_i} \bmod n$;
        \item используя протокол аутентификации сторон (см. ниже) легальный участник доказывает доверенному центру, что владеет закрытым ключом, не раскрывая его значение;
        \item получает от доверенного центр свой открытый ключ:
            \[ P_i = (v_i - I_i)^d = (g^{-s_i} - I_i)^d \mod n; \]
    \end{itemize}
    В результате для каждого участника, например, Алисы, которая владеет $P_A, I_A, s_a$ будет выполняться утверждение:
        \[ P_A^e + I_A = g^{-s_A} \mod n. \]
\end{itemize}

Протокол аутентификации сторон в общем случае выглядит следующим образом (рис.~\ref{fig:key_distribution-girault-auth}).

\begin{figure}
    \centering
    \includegraphics[width=0.5\textwidth]{pic/key_distribution-girault-auth}
    \caption{Взаимодействия участников в протоколе идентификации Жиро\label{fig:key_distribution-girault-auth}}
\end{figure}

\begin{protocol}
    \item[(1)] Алиса выбирает случайное $R_A$.
    \item[{}] $Alice \to \left\{ I_A, P_A, t = g^{R_A} \right\} \to Bob$
    \item[(2)] Боб выбирает случайное $R_B$.
    \item[{}] $Bob \to \left\{ R_B \right\} \to Alice$
    \item[(3)] $Alice \to \left\{ y = R_A + s_A \times R_B \right\} \to Bob$
    \item[(4)] Боб вычисляет $v_A = P_A^e + I_A$;
    \item[{}] Боб проверяет, что $g^ y v_A^{R_B} = t$.
\end{protocol}

Протокол генерации сессионного ключа, либо просто \emph{схема Жиро}, как и другие схемы, состоит из проходов обмена открытой информацией и вычисления ключа (рис.~\ref{fig:key_distribution-girault-scheme}).

\begin{figure}
    \centering
    \includegraphics[width=0.5\textwidth]{pic/key_distribution-girault-scheme}
    \caption{Взаимодействие участников в схеме Жиро\label{fig:key_distribution-girault-scheme}}
\end{figure}

\begin{protocol}
    \item[(1)] $Alice \to \left\{ P_A, I_A \right\} \to Bob$
    \item[(2)] Боб вычисляет $K_{BA} = (P_A^e + I_A)^{s_B} \bmod n$.
    \item[{}] $Bob \to \left\{ P_B, I_B \right\} \to Alice$
    \item[(3)] Алиса вычисляет $K_{AB} = (P_B^e + I_B)^{s_A} \bmod n$.
\end{protocol}

В результате работы схемы стороны сгенерировали одинаковый общий сеансовый ключ.
\[ K_{AB} = (P_A^e + I_A)^{s_B} = (g^{-s_A})^{s_B} = g^{-s_As_B} \mod n; \]
\[ K_{BA} = (P_B^e + I_B)^{s_A} = (g^{-s_B})^{s_A} = g^{-s_As_B} \mod n; \]
            \[ K = K_{AB} = K_{BA} = g^{-s_As_B} \mod n. \]

Схема обеспечивает аутентификацию ключа (цель G7), так как только легальные пользователи смогут вычислить корректное значение общего сессионного ключа.

\index{схема!Жиро|)}

\subsection{Схема Блома}\index{схема!Блома}
\selectlanguage{russian}

Рассмотрим распределение ключей по \emph{схеме Блома} (Rolf Blom,~\cite{Blom:1984, Blom:1985}), в котором каждые два пользователя из общего числа $N$ пользователей могут создать общий секретный ключ, причём секретные ключи каждой пары различны. Данная схема используется в протоколе HDCP\index{протокол!HDCP} (\langen{High-bandwidth Digital Content Protection}) для предотвращения копирования высококачественного видеосигнала.

На этапе инициализации доверенный центр выбирает симметричную матрицу $D_{m,m}$ над конечным полем $\GF p$. Для присоединения к сети распространения ключей новый участник либо самостоятельно, либо с помощью доверенного центра выбирает новый открытый ключ (идентификатор) $I$, представляющий собой вектор длины $k$ над $\GF p$. Доверенный центр вычисляет для нового участника закрытый ключ $K$:

\begin{equation}
	K = D_{m,m} I.
	\label{eq:blom_center_matrix}
\end{equation}

Симметричность матрицы $D_{m,m}$ доверенного центра позволяет любым двум участникам сети создать общий сеансовый ключ. Пусть Алиса и Боб -- легальные пользователи сети, то есть они обладают открытыми ключами $I_A$ и $I_B$ соответственно, а их закрытые ключи $K_A$ и $K_B$ были вычислены одним и тем же доверенным центром по формуле~\ref{eq:blom_center_matrix}. Тогда протокол выработки общего секретного ключа выглядит следующим образом.

\begin{enumerate}
	\item Алиса отправляет Бобу свой открытый ключ $I_A$.
	\item Боб отправляет Алисе свой открытый ключ $I_B$.
	\item Алиса вычисляет значение $s_{AB} = K^t_A I_B = I^t_A D_{m,m} I_B$.
	\item Боб вычисляет значение $s_{BA} = K^t_B I_A = I^t_B D_{m,m} I_A$.
\end{enumerate}

Из симметричности матрицы $D_{m,m}$ следует, что значения $s_{AB}$ и $s_{BA}$ совпадут, они же и будут являться общим секретным ключом для Алисы и Боба. Этот секретный ключ будет свой для каждой пары легальных пользователей сети.

Присоединение новых участников к схеме строго контролируется доверенным центром, что позволяет защитить сеть от нелегальных пользователей. Надёжность данной схемы основывается на невозможности восстановить исходную матрицу. Однако для восстановления матрицы доверенного центра размера $m \times m$ необходимо и достаточно всего $m$ пар линейно независимых открытых и закрытых ключей. В 2010 году компания Intel, которая является <<доверенным центром>> для пользователей системы защиты HDCP, подтвердила, что криптоаналитикам удалось найти секретную матрицу (точнее, аналогичную ей), используемую для генерации ключей в упомянутой системе предотвращения копирования высококачественного видеосигнала.


\index{схема!распределения ключей|)}