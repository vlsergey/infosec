\subsection{Схема Блома}\index{схема!Блома}
\selectlanguage{russian}

Рассмотрим распределение ключей по \emph{схеме Блома} (\langen{Rolf Blom},~\cite{Blom:1984, Blom:1985}), в которой каждые два  из общего числа $N$ пользователей могут создать общий секретный ключ, причём секретные ключи каждой пары различны. Данная схема используется в протоколе HDCP\index{протокол!HDCP} (\langen{High-bandwidth Digital Content Protection}) для предотвращения копирования высококачественного видеосигнала.

На этапе инициализации доверенный центр выбирает симметричную матрицу $D_{m,m}$ над конечным полем $\GF p$. Для присоединения к сети распространения ключей новый участник либо самостоятельно, либо с помощью доверенного центра выбирает новый открытый ключ (идентификатор) $I$, представляющий собой вектор длины $m$ над $\GF p$. Доверенный центр вычисляет для нового участника закрытый ключ $K$:

\begin{equation}
	K = D_{m,m} I.
	\label{eq:blom_center_matrix}
\end{equation}

Симметричность матрицы $D_{m,m}$ доверенного центра позволяет любым двум участникам сети создать общий сеансовый ключ. Пусть Алиса и Боб -- легальные пользователи сети, значит они обладают открытыми ключами $I_A$ и $I_B$ соответственно, а их закрытые ключи $K_A$ и $K_B$ были вычислены одним и тем же доверенным центром по формуле~\ref{eq:blom_center_matrix}. Тогда протокол выработки общего секретного ключа выглядит следующим образом.

\begin{enumerate}
	\item Алиса отправляет Бобу свой открытый ключ $I_A$.
	\item Боб отправляет Алисе свой открытый ключ $I_B$.
	\item Алиса вычисляет значение $s_{AB} = K^T_A I_B = I^T_A D_{m,m} I_B$.
	\item Боб вычисляет значение $s_{BA} = K^T_B I_A = I^T_B D_{m,m} I_A$.
\end{enumerate}

Из симметричности матрицы $D_{m,m}$ следует, что значения $s_{AB}$ и $s_{BA}$ совпадут, они же и будут являться общим секретным ключом для Алисы и Боба. Этот секретный ключ будет свой для каждой пары легальных пользователей сети.

Присоединение новых участников к схеме строго контролируется доверенным центром, что позволяет защитить сеть от нелегальных пользователей. Надёжность данной схемы основывается на невозможности восстановить исходную матрицу. Однако для восстановления матрицы доверенного центра размера $m \times m$ необходимо и достаточно всего $m$ пар линейно независимых открытых и закрытых ключей. В 2010 году компания Intel, которая является <<доверенным центром>> для пользователей системы защиты HDCP, подтвердила, что криптоаналитикам удалось найти секретную матрицу (точнее, аналогичную ей), используемую для генерации ключей в упомянутой системе предотвращения копирования высококачественного видеосигнала.
